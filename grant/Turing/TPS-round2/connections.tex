\section*{Connections to other activities (300 words):}

We see important synergies with at least the following 3 TPS project proposals:

\subsection*{1. Evolving Wratter into the Turing Tools Platform (Tomas Petricek)}

Integrating our framework into Wrattler as a new cell type will significantly
lower the barriers to entry for a data scientist wishing to experiment with our
visualisations, since they will be able to add them easily to an existing a
Wrattler-based notebook written in R or Python. The synergy should flow in the
other direction too: while Wrattler already has excellent support for
coarse-grained dependency-tracking and provenance, our toolkit will show the
advantages of fine-grained dependency tracking, as well as fully transparent
visualisations. This should also make Wrattler more appealing to potential
users.

\subsection*{2. The Turing Way (Kirstie Whitaker)}

Our project is motivated by concerns surrounding trust and traceability in data
visualisation, important methodological issues for science communication. While
developing new visualisation methodology is not an explicit goal of the project,
we are keen to be informed by current methodological thinking. Moreover, in the
longer term, we hope to influence developing methodologies, not only by
delivering tools that enable more transparency out of the box, but also by
developing a suite of visual design patterns, libraries and programming
practices that take advantage of these new tools.

\subsection*{3. Explainability and Interpretability in AI (Emmanouil Benetos)}

Benetos' project aims to consolidate and develop explainability techniques for
machine learning, such as guided backpropagation, and is highly complementary to
ours. They too are concerned with tracking input-output relationships in a
fine-grained way, but for neural networks rather than symbolic programs. By
composing their approach with ours, it might be possible to provide end-to-end
explainability for pipelines with a mixture of computational styles (say
conventional visualisation libraries mixed with CNNs). We believe the two
projects would benefit from a certain amount of temporal overlap.

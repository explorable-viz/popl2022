\section*{Abstract (500 words):}

Data visualisation is essential to data science and science communication, but
is open to both misinterpretation and misuse: patterns in raw data can be
obscured, statistical assumptions hidden, and effect sizes misrepresented
\cite{weissgerber15}. These concerns can be addressed in part through improved
statistical practices and better plot and chart designs \cite{allen19}, but also
by making visualisations themselves more open and explorable
\cite{dragicevic19}.

Understanding a visualisation requires grasping how it relates to the underlying
data and other visualisations. For example, geoscientists often work with
multiple layered views. To show how these are related, spatial analytics
applications like GeoDa \cite{anselin06} can automatically select the relevant
part of one view as the user changes the selection in a related view, say a
choropleth map. However, this feature is available only if it was specifically
anticipated by the application or library developer; if the geoscientist uses
custom libraries or wants other views linked that the developer did not
consider, they are out of luck.

Our project will deliver a framework for authoring visualisations where support
for linking, between data, code, and visualisations is built in, making this
powerful comprehension feature automatic. To allow us to focus on the needs of
SPF projects working within the Urban Analytics theme, we have assembled a team
with expertise in geocomputation and spatial analytics (Wolf, Malik). This is
complemented with expertise in data visualisation (Bach, Wolf), data provenance
(Cheney, Perera, Malik), and programming languages (Cheney, Perera, Petricek).
The following three tracks will run in parallel, building on a proof-of-concept
also developed within the TPS programme; see
\url{https://www.turing.ac.uk/research/research-projects/data-science-toolkit-explorable-data-visualisations}.

\subsection*{Track 1: Domain use cases}

\paragraph{[Wolf, Malik, Bach, Perera, 0.5 FTE RA]} We will develop a public
repository of substantive geospatial visualisation use cases that demonstrate
the explorability features of our approach to prospective users. This will
ensure that the work in track 1 stays focused on real needs of researchers in
the urban analytics and other geospatial domains.

\subsection*{Track 2: Infrastructure for explorable data visualisations}

\paragraph{[Perera, Petricek, Cheney, Malik, Bach, 0.5 FTE RA]} This will
involve user interface and library design, and new programming language
infrastructure. Driven by our domain use cases (track 1), we will further
develop our toolkit, building on a dynamic program analysis technique developed
by Perera and Cheney \cite{perera12a,ricciotti17}. We will combine the linking
feature with the ability to explore the pipeline of data and visual
transformations that yielded the final artefact, for example by tabbing through
the various intermediate artefacts. We call the resulting explorable
visualisations ``self-explaining'' because they can be explored in situ (say in
an online paper) to reveal how the visualisation was created.

\subsection*{Track 3: Wrattler integration}

\paragraph{[Perera, Petricek, 0.5 FTE RSE]} We will integrate our toolkit with
the Wrattler notebook \cite{petricek18}; this strand of work will also be
supported by the ``Evolving Wrattler into the Turing tools platform" work
package. This is central to our impact strategy and how we connect to other
Turing SPF projects and is discussed in detail later.

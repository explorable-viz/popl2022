\section*{Research challenges (300 words):}

The project is an unusual blend of programming languages, data visualisation and
HCI research, and faces technical, usability, and interoperability challenges.
We have tried to anticipate these with an eclectic but synergistic team.

The basic technical challenge of our project is implementing infrastructure for
``linking'', or multiple-coordinated views \cite{tobiasz09}, in an
application-independent way. We are taking a principled approach based on our
own prior work on dynamic dependency analysis and provenance; in contrast to
prior work on provenance in data visualisation \cite{callahan06}, our approach
is much more fine-grained and is able to associate specific bits of data or code
with parts of a visualisation in a precise way. While the theoretical technique
is proven, this approach has never been applied to data visualisation before.

A central usability challenge is visualising these complex relationships between
the various parts of a visualisation and the relevant data and/or visualisation
code. This is essentially a higher-order visualisation problem: visualising
information about the provenance of visualisations. We will draw on Bach's
expertise in temporal data visualisation \cite{bach16} and data-driven
storytelling \cite{bach18}, and Cheney, Malik and Perera's background in data
provenance. We may also build on recent work on ``literate'' visualisation
\cite{wood19}.

Finally, we will need to solve challenging interoperability problems in order to
coexist usefully with popular visualisation libraries such as Bokeh
\cite{jolly18}, visual specification languages like Vega \cite{satyanarayan17},
spatial analytics libraries such as GeoPandas and PySAL \cite{rey07}, and
notebooks like Wrattler and Jupyter. To use our analysis techniques with such
third-party frameworks we may need to propose new metadata formats or find
efficient ways to instrument existing libraries; Wolf's experience with
developing open source spatial analytics software and Petricek's Wrattler
expertise will be useful for exploring appropriate techniques.

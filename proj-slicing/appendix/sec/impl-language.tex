\section{Implementation language}

We now describe a more realistic implementation language with \emph{data types}
(\secref{impl-language:data-types}), which replace recursive types, binary
sums and binary products by a single construct for named recursive
sums-of-products; primitive operations
(\secref{impl-language:primitives}), \ttt{match}...\ttt{as}
(\secref{impl-language:match-as}), and mutually recursive functions
(\secref{impl-language:recursion}).

\subsection{Data types}
\label{sec:impl-language:data-types}

\figref{impl-language:syntax} includes all the syntax associated with data
types, including data type names $D$, constructors $c$, constructor expressions
$\exConstr{c}{\vec{e}}$ where $\vec{e}$ is a vector of arguments, constructor
(or \emph{sum}) eliminators $\Sigma$ of the form $\family{\kappa_c}{c}{\tilde{c}}$
where $\tilde{c}$ is a set of constructors. Every data type name $D$ has an
interpretation $\interpret{D}$ as a family of vectors of types
$\family{\vec{A}_c}{c}{\tilde{c}}$.

\subsection{Primitives}
\label{sec:impl-language:primitives}

\begin{figure}
\begin{syntaxfig}
\mbox{Eliminator}
&
\sigma, \tau
&
::=
&
\ldots
\\
&&&
\elimBoolTrue{\kappa}
\\
&&&
\elimBoolFalse{\kappa}
\\
&&&
\elimListSingleton{\branchNil{\kappa}}
\\
&&&
\elimListSingleton{\branchCons{\sigma}}
\\[2mm]
\mbox{Raw term}
&
r
&
::=
&
\ldots
\\
&&&
\exLambda{\sigma}
&
\text{anonymous function}
\end{syntaxfig}
\caption{Additional syntax}
\end{figure}

\begin{figure}
\begin{syntaxfig}
\mbox{Value}
&
u, v
&
::=
&
\exTrue \mid \exFalse
&
\text{Boolean}
\\
&&&
\exTrueSel \mid \exFalseSel
\\
&&&
\exInt{n}
\\
&&&
\exIntSel{n}
&
\text{integer}
\\
&&&
\exClosureRec{\rho}{f}{\sigma}
&
\text{closure}
\\
&&&
\exPair{u}{v}
&
\text{pair}
\\
&&&
\exPairSel{u}{v}
\\
&&&
\exNil
\\
&&&
\exNilSel
&
\text{nil}
\\
&&&
\exCons{u}{v}
&
\text{cons}
\\
&&&
\exConsSel{u}{v}
\\[2mm]
\mbox{Environment}
&
\rho
&
::=
&
\envEmpty
&
\text{empty}
\\
&&&
\envExtend{\rho}{x}{v}
&
\text{extend}
\end{syntaxfig}
\caption{Values and environments}
\end{figure}


\figref{impl-language:syntax} includes all the syntax associated with primitive
operations, including pairwise-disjoint ground types $\tyGround{C}$, constants
$\exConst{k}$ of ground type, unary primitives $\phi$ which are first-class
functions, and binary primitives $\primOp$ which are infix and not first-class.
It is straightforward to add Haskell-style sections which convert binary
operators $\primOp$ into first-class values.

Each constant $\exConst{k}$ has a unique ground type $\typeof{\exConst{k}}$.
Each unary primitive $\phi$ has a unique type $\typeof{\phi}$ of the form $C \to
A$ and interpretation $\interpret{\phi} \in \Val{C} \to \Val{A}$. Unary
primitives have no expression form; they are introduced by
$\exPrimDef{\exVar{x}}$ definitions, which bind $\exVar{x}$ to its
interpretation $\interpret{\exVar{x}}$ as a unary primitive, if it has one.
(This rather complex setup is to allow each first-class primitive to be
associated with a specific bit of \emph{syntax} for slicing purposes.) Each
binary operator $\primOp$ has a unique type $\typeof{\primOp}$ and
interpretation $\interpret{\primOp} \in \Val{C_1} \times \Val{C_2} \to \Val{A}$.

\begin{figure}
\flushleft \shadebox{$\Gamma \vdash e: A$}
\begin{smathpar}
\inferrule*[right={$x : A \in \Gamma$}]
{
   \strut
}
{
   \Gamma \vdash \exVar{x}: A
}
%
\and
%
\inferrule*[right={$\primOp: A \in \Gamma$}]
{
   \strut
}
{
   \Gamma \vdash \exOp{\primOp}: A
}
%
\and
%
\inferrule*
{
   \strut
}
{
   \Gamma \vdash n: \tyInt
}
%
\and
%
\inferrule*
{
   \strut
}
{
   \Gamma \vdash \exTrue: \tyBool
}
%
\and
%
\inferrule*
{
   \strut
}
{
   \Gamma \vdash \exFalse: \tyBool
}
%
\and
%
\inferrule*
{
   \Gamma \vdash u: A
   \\
   \Gamma \vdash v: B
}
{
   \Gamma \vdash \exPair{u}{v}: \tyProd{A}{B}
}
%
\and
%
\inferrule*[right={$\primOp: \tyFun{\tyInt}{\tyFun{\tyInt}{\tyInt}} \in \Gamma$}]
{
   \Gamma \vdash e: \tyInt
   \\
   \Gamma \vdash e': \tyInt
}
{
   \Gamma \vdash \exBinaryApp{e}{\primOp}{e'}: \tyInt
}
%
\and
%
\inferrule*
{
   \cxtExtend{\Gamma}{f}{\tyFun{A}{B}} \vdash \sigma: \tyFun{A}{B}
}
{
   \Gamma \vdash \exRec{f}{\sigma}: \tyFun{A}{B}
}
%
\and
%
\inferrule*
{
   \Gamma \vdash e: \tyFun{A}{B}
   \\
   \Gamma \vdash e': A
}
{
   \Gamma \vdash \exApp{e}{e'}: B
}
%
\and
%
\inferrule*
{
   \Gamma \vdash e: A
   \\
   \cxtExtend{\Gamma}{x}{A} \vdash e': B
}
{
   \Gamma \vdash \exLet{x}{e}{e'}: B
}
%
\and
%
\inferrule*
{
   \strut
}
{
   \Gamma \vdash \exNil: \tyList{\tyInt}
}
%
\and
%
\inferrule*
{
   \Gamma \vdash u: \tyInt
   \\
   \Gamma \vdash v: \tyList{\tyInt}
}
{
   \vdash \exCons{u}{v}: \tyList{\tyInt}
}
\end{smathpar}
\caption{Typing rules for terms}
\end{figure}

\begin{figure}
\flushleft \shadebox{$\elimType{\sigma}{B}{A}{\Gamma}$}
\begin{smathpar}
\inferrule*
{
   \contType{\kappa}{B}{\cxtExtend{\Gamma}{x}{A}}
}
{
   \elimType{\elimVar{x}{\kappa}}{B}{A}{\Gamma}
}
%
\and
%
\inferrule*[right={\textnormal{$\dom{\interpret{D}} = \dom{\Sigma}$}}]
{
   \elimType{\Sigma(c)}{A}{\interpret{D}(c)}{\Gamma}
   \\
   (\forall c \in \dom{\interpret{D}})
}
{
   \elimType{\elimConstr{\Sigma}}{A}{\tyData{D}}{\Gamma}
}
\end{smathpar}
\vspace{5pt}
\flushleft \shadebox{$\elimType{\kappa}{B}{\vec{A}}{\Gamma}$}
\begin{smathpar}
\inferrule*
{
   \contType{\kappa}{A}{\Gamma}
}
{
   \elimType{\kappa}{A}{\seqEmpty}{\Gamma}
}
%
\and
%
\inferrule*
{
   \elimType{\sigma}{\elimTypePartial{B}{\vec{A}}}{A}{\Gamma}
}
{
   \elimType{\sigma}{B}{A \concat \vec{A}}{\Gamma}
}
\end{smathpar}
\caption{Typing rules for eliminators}
\end{figure}

\begin{figure}
\flushleft \shadebox{$\Gamma \vdash \vec{d}: \Delta$}
\begin{smathpar}
   \inferrule*
   {
      \strut
   }
   {
      \Gamma \vdash \seqEmpty: \cxtEmpty
   }
   %
   \and
   %
   \inferrule*
   {
      \Gamma \vdash d: \Delta
      \\
      \Gamma \concat \Delta \vdash \vec{d}: \Delta'
   }
   {
      \Gamma \vdash d \concat \vec{d}: \Delta \concat \Delta'
   }
\end{smathpar}
\vspace{5pt}

\flushleft \shadebox{$\Gamma \vdash d: \Delta$}
\begin{smathpar}
   \inferrule*
   {
      \exprType{e}{\Gamma}{A}
   }
   {
      \Gamma \vdash \exLetDef{x}{e}: \cxtExtend{\cxtEmpty}{x}{A}
   }
   %
   \and
   %
   \inferrule*[right={
      \textnormal{$\Delta = \set{f \mapsto A_f \mid f \in \dom{\delta}}$}
   }]
   {
      \exprType{\exFun{\sigma_f}}{\Gamma \concat \Delta}{A_f}
      \quad
      (\forall f \mapsto \exFun{\sigma_f} \in \delta)
   }
   {
      \Gamma \vdash \exLetrecDef{\delta}: \Delta
   }
   %
   \and
   %
   \inferrule*
   {
      \interpret{x} \in \interpret{A}
   }
   {
      \Gamma \vdash \exPrimDef{x}: \cxtExtend{\cxtEmpty}{x}{A}
   }
\end{smathpar}
\vspace{5pt}

\flushleft \shadebox{$\valExplTypeNew{u}{A}$}
\begin{smathpar}
\inferrule*[right={$\interpret{\exConst{c}} \in C$}]
{
   \strut
}
{
   \valExplTypeNew{\exConst{k}}{\tyGround{C}}
}
%
\and
%
\inferrule*[right={\textnormal{$\interpret{\phi} \in \interpret{C} \to \interpret{A}$}}]
{
   \strut
}
{
   \valExplTypeNew{\phi}{\tyFun{C}{A}}
}
%
\and
%
\inferrule*
{
   \envType{\Gamma}{\rho}
   \\
   \Gamma \vdash \exLetrecDef{\delta}: \Delta
   \\
   \exprType{\exFun{\sigma}}{\Gamma \concat \Delta}{\tyFun{A}{B}}
}
{
   \valExplTypeNew{\exClosureNew{\rho}{\delta}{\exFun{\sigma}}}{\tyFun{A}{B}}
}
%
\and
%
\inferrule*[right={\textnormal{$c \in \dom{\interpret{D}}$}}]
{
   \valExplTypeNew{\vec{v}}{\interpret{D}(c)}
}
{
   \valExplTypeNew{\exConstr{c}{\vec{v}}}{\tyData{D}}
}
\end{smathpar}
\vspace{5pt}

\flushleft \shadebox{$\envType{\Gamma}{\rho}$}
\begin{smathpar}
\inferrule*
{
   \strut
}
{
   \envType{\cxtEmpty}{\envEmpty}
}
%
\and
%
\inferrule*
{
   \envType{\Gamma}{\rho}
   \\
   \valExplType{\Gamma'}{u}{A}
}
{
   \envType{\cxtExtend{\Gamma}{x}{A}}{\envExtend{\rho}{x}{u}}
}
\end{smathpar}
\caption{Typing rules for partial values, environments and recursive definitions}
\end{figure}

% \begin{figure}
\flushleft \shadebox{$\explType{\Gamma}{T}{A}$}
\begin{smathpar}
\inferrule*
{
   \envType{\Gamma}{\rho}
}
{
   \explType{\Gamma}{\sub{\trEmpty}{\rho}}{A}
}
%
\and
%
\inferrule*[right={$x : A \in \Gamma$}]
{
   \envType{\Gamma}{\rho}
}
{
   \explType{\Gamma}{\trVarTwo{x}{\rho}}{A}
}
%
\and
%
\inferrule*
{
   \explType{\Gamma}{T}{(\tyFun{A}{B})}
   \\
   \explType{\Gamma}{T'}{A}
   \\
   \elimType{\xi}{K}{A}{\Gamma'}
   \\
   \explType{\Gamma'}{U}{B}
}
{
   \explType
      {\Gamma}
      {\trApp{T}{T'}{\matchPlug{\xi}{U}}}
      {B}
}
\end{smathpar}
\caption{Typing rules for explanations}
\end{figure}


\begin{figure}[H]
\flushleft \shadebox{$p, \sigma \matchp \kappa$}
\begin{smathpar}
   \inferrule*[lab={\ruleName{$\matchp$-var}}]
   {
      \strut
   }
   {
      x, \elimVar{x}{\kappa}
      \matchp
      \kappa
   }
   %
   \and
   %
   \inferrule*[lab={\ruleName{$\matchp$-true}}]
   {
      \strut
   }
   {
      \exTrue, \elimBool{\kappa}{\kappa'}
      \matchp
      \kappa
   }
   %
   \and
   %
   \inferrule*[lab={\ruleName{$\matchp$-false}}]
   {
      \strut
   }
   {
      \exFalse, \elimBool{\kappa}{\kappa'}
      \matchp
      \kappa'
   }
   %
   \and
   %
   \inferrule*[lab={\ruleName{$\matchp$-pair}}]
   {
      p_1, \sigma \matchp, \tau
      \\
      p_2, \tau \matchp \kappa
   }
   {
      \exPair{p_1}{p_2}, \elimProd{\sigma}
      \matchp
      \kappa
   }
   %
   \and
   %
   \inferrule*[lab={\ruleName{$\matchp$-nil}}]
   {
      \strut
   }
   {
      \exNil, \elimList{\branchNil{\kappa}}{\branchCons{\sigma}}
      \matchp
      \kappa
   }
   %
   \and
   %
   \inferrule*[lab={\ruleName{$\matchp$-cons}}]
   {
      p_1, \sigma \matchp \tau
      \\
      p_2, \tau \matchp \kappa'
   }
   {
      \exCons{p_1}{p_2}, \elimList{\branchNil{\kappa}}{\branchCons{\sigma}}
      \matchp
      \kappa'
   }
\end{smathpar}
\caption{Pattern-matching (by pattern)}
\end{figure}

\begin{figure}
\flushleft \shadebox{$\rho, \matchPlug{\xi}{\kappa}, \alpha \unlookupR{} u, \sigma$}
\begin{smathpar}
   \inferrule*[lab={\ruleName{$\unlookupR{}$-var}}]
   {
   }
   {
      \envExtend{\envEmpty}{x}{u}, \matchVar{x}{\matchHole{\kappa}}, \alpha
      \unlookupR{}
      u, \trieVar{x}{\kappa}
   }
   %
   \and
   %
   \inferrule*[lab={\ruleName{$\unlookupR{}$-constr}}]
   {
      \rho, \matchPlug{\Psi}{\kappa}, \alpha \unlookupR{} \vec{u}, \kappa'
   }
   {
      \rho, \envExtend{\Sigma}{c}{\matchPlug{\Psi}{\kappa}}, \alpha
      \unlookupR{}
      \annot{\exConstr{c}{\vec{u}}}{\alpha}, \envExtend{\sub{\bot}{\Sigma}}{c}{\kappa'}
   }
\end{smathpar}
\\[2mm]
\flushleft \shadebox{$\rho, \matchPlug{\Psi}{\kappa}, \alpha \unlookupR{} \vec{u}, \kappa$}
\begin{smathpar}
   \inferrule*[lab={\ruleName{$\unlookupR{}$-args-nil}}]
   {
   }
   {
      \envEmpty, \matchHole{\kappa}, \alpha \unlookupR{} \seqEmpty, \kappa
   }
   %
   \and
   %
   \inferrule*[
      lab={\ruleName{$\unlookupR{}$-args-cons}}
   ]
   {
      \rho_2, \matchPlug{\Psi}{\kappa}, \alpha \unlookupR{} \vec{u}, \kappa'
      \\
      \rho_1, \matchPlug{\xi}{\kappa'}, \alpha \unlookupR{} u, \sigma
   }
   {
      \rho_1 \concat \rho_2, \matchProd{\xi}{\matchPlug{\Psi}{\kappa}}, \alpha
      \unlookupR{}
      u \concat \vec{u}, \trieProd{\sigma}
   }
\end{smathpar}
\caption{Reverse pattern-matching}
\label{fig:impl-language:unmatch}
\end{figure}

\begin{figure}[p]
\flushleft \shadebox{$\rho, e \eval{} T, u$}
\begin{smathpar}
   \inferrule*[left={\ruleName{$\eval{}$-annot}}]
   {
      \rho, r
      \eval{}
      T, \annot{v}{\alpha'}
   }
   {
      \rho,
      \annot{r}{\alpha}
      \eval{}
      T, \annot{v}{\alpha \wedge \alpha'}
   }
\end{smathpar}
\\
\flushleft \shadebox{$\rho, r \eval{} T, u$}
\begin{smathpar}
   \inferrule*[
      left={\ruleName{$\eval{}$-var}},
      right={\textnormal{$\exThunkVar{x}{u} \in \rho$}}
   ]
   {
      \strut
   }
   {
      \rho, \exVar{x} \eval{} \trVarTwo{x}{\rho}, u
   }
   %
   \and
   %
   \inferrule*[left={\ruleName{$\eval{}$-const}}]
   {
      \strut
   }
   {
      \rho,
      \exConst{k}
      \eval{}
      \sub{\trEmpty}{\rho}, \exConst{k}
   }
   %
   \and
   %
   \inferrule*[left={\ruleName{$\eval{}$-op}}]
   {
      \strut
   }
   {
      \rho,
      \phi
      \eval{}
      \sub{\trEmpty}{\rho}, \phi
   }
   %
   \and
   %
   \inferrule*[left={\ruleName{$\eval{}$-fun}}]
   {
      \strut
   }
   {
      \rho,
      \exFun{\sigma}
      \eval{}
      \sub{\trEmpty}{\rho}, \exClosureNew{\rho}{\seqEmpty}{\exFun{\sigma}}
   }
   %
   \and
   %
   \inferrule*[left={\ruleName{$\eval{}$-app}}]
   {
      \rho, e \eval{} T, \annot{\exClosureNew{\rho_1}{\delta}{\exFun{\sigma}}}{\alpha}
      \\
      \rho_1, \delta \closeDefs \rho_2
      \\
      \rho, e' \eval{} T', u
      \\
      u, \sigma \lookupR \rho_3, \matchPlug{\xi}{e^\twoPrime}, \alpha'
      \\
      \rho_1 \concat \rho_2 \concat \rho_3, e^\twoPrime
      \eval{}
      U, \annot{v}{\alpha^\twoPrime}
   }
   {
      \rho,
      \exApp{e}{e'}
      \eval{}
      \trApp{T}{T'}{\matchPlug{\xi}{U}},
      \annot{v}{\alpha \wedge \alpha' \wedge \alpha^\twoPrime}
   }
   %
   \and
   %
   \inferrule*[
      left={\ruleName{$\eval{}$-app-unary}},
      right={\textnormal{$k \in \dom{\interpret{\phi}}$}}
   ]
   {
      \rho, e \eval{} \explVal{T}{\annot{\phi}{\alpha}}
      \\
      \rho, e' \eval{} \explVal{T'}{\annot{k}{\alpha'}}
   }
   {
      \rho,
      \exApp{e}{e'}
      \eval{}
      \trUnaryApp{(\explVal{T}{\phi})}{(\explVal{T'}{\exConst{k}})},
      \annot{\interpret{\phi}(k)}{\alpha \wedge \alpha'}
   }
   %
   \and
   %
   \inferrule*[
      left={\ruleName{$\eval{}$-app-binary}},
      right={\textnormal{$(k, k') \in \dom{\interpret{\primOp}}$
      }}
   ]
   {
      \rho, e \eval{} \explVal{T}{\annot{k}{\alpha'}}
      \\
      \rho, e' \eval{} \explVal{T'}{\annot{k'}{\smash{\alpha^\twoPrime}}}
   }
   {
      \rho,
      \exPrimApp{e}{\annot{\primOp}{\alpha}}{e'}
      \eval{}
      \trPrimApp{(\explVal{T}{k})}{\primOp}{(\explVal{T'}{k'})},
      \annot{\interpret{\primOp}(k, k')}{\smash{\alpha \wedge \alpha' \wedge \alpha^\twoPrime}}
   }
   %
   \and
   %
   \inferrule*[left={\ruleName{$\eval{}$-constr}}]
   {
      \rho,
      \vec{e}
      \eval{}
      \vec{\explVal{T}{u}}
   }
   {
      \rho,
      \exConstr{d}{\vec{e}}
      \eval{}
      \explVal{\sub{\trEmpty}{\rho}}{\exConstr{d}{\vec{\explVal{T}{u}}}}
   }
   %
   \and
   %
   \inferrule*[left={\ruleName{$\eval{}$-match}}]
   {
      \rho, e \eval{} \explVal{T}{u}
      \\
      u, \sigma \lookupR \rho_1, \matchProd{\xi}{e'}, \alpha
      \\
      \rho \concat \rho_1, e' \eval{} \explVal{T'}{\annot{v}{\alpha'}}
   }
   {
      \rho,
      \exMatch{e}{\sigma}
      \eval{}
      \explVal{\trMatch{T}{\matchPlug{\xi}{T'}}}{\annot{v}{\alpha \wedge \alpha'}}
   }
   %
   \and
   %
   \inferrule*[left={\ruleName{$\eval{}$-let}}]
   {
      \rho, e \eval{} \explVal{T}{u}
      \\
      \envExtend{\rho}{x}{u}, e' \eval{} \explVal{U}{u'}
   }
   {
      \rho, \exLet{x}{e}{e'}
      \eval{}
      \explVal{\trLet{x}{\explVal{T}{u}}{U}}{u'}
   }
   %
   \and
   %
   \inferrule*[left={\ruleName{$\eval{}$-letrec}}]
   {
      \rho, \delta \closeDefs \rho_1
      \\
      \rho \concat \rho_1, e \eval{} \explVal{T}{u}
   }
   {
      \rho, \exLetrec{\delta}{e}
      \eval{}
      \explVal{\trLetrec{\delta}{T}}{u}
   }
   %
   \and
   %
   \inferrule*[left={\ruleName{$\eval{}$-defs}}]
   {
      \rho, \vec{d} \closeDefs \rho_1, \vec{z}
      \\
      \rho \concat \rho_1, e \eval{} \explVal{T}{u}
   }
   {
      \rho, \exDefs{\vec{d}}{e}
      \eval{}
      \explVal{\trDefs{\vec{z}}{T}}{u}
   }
\end{smathpar}
\vspace{5pt}

\flushleft \shadebox{$\rho, \vec{e} \eval{} \vec{\explVal{T}{u}}$}
\begin{smathpar}
   \inferrule*
   {
      \strut
   }
   {
      \rho,
      \exUnit
      \eval{}
      \seqEmpty
   }
   \and
   %
   \inferrule*
   {
      \rho, e \eval{} \explVal{T}{u}
      \\
      \rho, \vec{e} \eval{} \vec{\explVal{T}{u}}
   }
   {
      \rho,
      e \concat \vec{e}
      \eval{}
      (\explVal{T}{u}) \concat (\vec{\explVal{T}{u}})
   }
\end{smathpar}
\caption{Evaluation for terms and term sequences}
\end{figure}

\begin{figure}
\flushleft \shadebox{$T, u \uneval \rho, e$}
\begin{smathpar}
   \inferrule*[left={\ruleName{$\uneval$-var}}]
   {
      \strut
   }
   {
      {\trVarTwo{x}{\rho}}, \annot{v}{\alpha}
      \uneval
      \envExtend{\sub{\bot}{\rho}}{x}{\annot{v}{\alpha}},
      \annot{\exVar{x}}{\alpha}
   }
   %
   \and
   %
   \inferrule*[left={\ruleName{$\uneval$-const}}]
   {
      \strut
   }
   {
      \explVal{\sub{\trEmpty}{\rho}}{\annot{\exConst{k}}{\alpha}}
      \uneval
      \sub{\bot}{\rho},
      \annot{\exConst{k}}{\alpha}
   }
   %
   \and
   %
   \inferrule*[left={\ruleName{$\uneval$-fun}}]
   {
      \strut
   }
   {
      \sub{\trEmpty}{\rho}, \annot{\exClosureNew{\rho'}{\seqEmpty}{\exFun{\sigma}}}{\alpha}
      \uneval
      \rho',
      \annot{(\exFun{\sigma})}{\alpha}
   }
   %
   \and
   %
   \inferrule*[left={\ruleName{$\uneval$-app}}]
   {
      U, \annot{v}{\alpha} \uneval \rho_1' \concat \rho_2 \concat \rho_3, e^\twoPrime
      \\
      \rho_3, \matchPlug{\xi}{e^\twoPrime}, \alpha \unlookupR{} u, \sigma
      \\
      T', u \uneval \rho, e'
      \\
      \rho_2 \uncloseDefs \rho_1^\twoPrime, \delta'
      \\
      T, \annot{\exClosureNew{\rho_1' \vee \rho_1^\twoPrime}{\delta'}{\exFun{\sigma}}}{\alpha}
      \uneval
      \rho', e
   }
   {
      \trApp{T}{T'}{\matchPlug{\xi}{U}}, \annot{v}{\alpha}
      \uneval
      \rho \vee \rho',
      \annot{(\exApp{e}{e'})}{\alpha}
   }
   %
   \and
   %
   \inferrule*[
      left={\ruleName{$\uneval$-app-unary}}
   ]
   {
      T', \annot{c}{\alpha} \uneval \rho', e'
      \\
      T, \annot{\phi}{\alpha} \uneval \rho, e
   }
   {
      \trApp{(\explVal{T}{\phi})}{(\explVal{T'}{c})}{\sub{\trEmpty}{\envEmpty}},
      \annot{v}{\alpha}
      \uneval
      \rho \vee \rho',
      \annot{(\exApp{e}{e'})}{\alpha}
   }
   %
   \and
   %
   \inferrule*[
      left={\ruleName{$\uneval$-app-binary}}
   ]
   {
      T', \annot{c'}{\alpha} \uneval \rho', e'
      \\
      T, \annot{c}{\alpha} \uneval \rho, e
   }
   {
      \trPrimApp{(\explVal{T}{c})}{\primOp}{(\explVal{T'}{c'})},
      \annot{v}{\alpha}
      \uneval
      \rho \vee \rho',
      \annot{(\exPrimApp{e}{\annot{\primOp}{\alpha}}{e'})}{\alpha}
   }
   %
   \and
   %
   \inferrule*[left={\ruleName{$\uneval$-constr}}]
   {
      \vec{T}, \vec{u} \uneval \rho', \vec{e}
   }
   {
      \trConstr{c}{\vec{T}}, \annot{\exConstr{c}{\vec{u}}}{\alpha}
      \uneval
      \rho',
      \annot{\exConstr{c}{\vec{e}}}{\alpha}
   }
   %
   \and
   %
   \inferrule*[left={\ruleName{$\uneval$-match}}]
   {
      \explVal{T'}{\annot{v}{\alpha}} \uneval \rho' \concat \rho_1, e'
      \\
      \rho_1, \matchPlug{\xi}{e'}, \alpha \unlookupR{} u, \sigma
      \\
      \explVal{T}{u} \uneval \rho, e
   }
   {
      \explVal{\trMatch{T}{\matchPlug{\xi}{T'}}}{\annot{v}{\alpha}}
      \uneval
      \rho \vee \rho',
      \annot{(\exMatch{e}{\sigma})}{\alpha}
   }
   %
   \and
   %
   \inferrule*[left={\ruleName{$\uneval$-let}}]
   {
      \explVal{U}{\annot{v}{\alpha}} \uneval \envExtend{\rho}{x}{u'}, e'
      \\
      \explVal{T}{u'} \uneval \rho', e
   }
   {
      \explVal{\trLet{x}{\explVal{T}{u}}{U}}{\annot{v}{\alpha}}
      \uneval
      \rho \vee \rho', \annot{(\exLet{x}{e}{e'})}{\alpha}
   }
   %
   \and
   %
   \inferrule*[left={\ruleName{$\uneval$-letrec}}]
   {
      \explVal{T}{\annot{v}{\alpha}} \uneval \rho \concat \rho_1, e
      \\
      \rho_1 \uncloseDefs \rho', \delta'
   }
   {
      \explVal{\trLetrec{\delta}{T}}{\annot{v}{\alpha}}
      \uneval
      \rho \vee \rho', \annot{(\exLetrec{\delta'}{e})}{\alpha}
   }
\end{smathpar}
\\
\flushleft \shadebox{$\vec{T}, \vec{u} \uneval \rho, \vec{e}$}
\begin{smathpar}
   \inferrule*
   {
      \strut
   }
   {
      \annot{\seqEmpty}{\rho}, \seqEmpty
      \uneval
      \sub{\bot}{\rho},
      \exUnit
   }
   \and
   %
   \inferrule*
   {
      \vec{T}, \vec{u} \uneval \rho_1, \vec{e}
      \\
      T, u \uneval \rho_1, e
   }
   {
      T \concat \vec{T},
      u \concat \vec{u}
      \uneval
      \rho_1 \vee \rho_2,
      e \concat \vec{e}
   }
\end{smathpar}
\caption{Reverse evaluation for terms and term sequences}
\end{figure}


% \begin{figure}
\begin{syntaxfig}
\mbox{Match}
&
\xi
&
::=
&
\matchVar{x}{\matchHole{}}
&
\text{variable}
\\
&&&
\matchUnit{\matchHole{}}
&
\text{unit}
\\
&&&
\matchSumL{\xi}{\sigma}
&
\text{inject left}
\\
&&&
\matchSumR{\sigma}{\xi}
&
\text{inject right}
\\
&&&
\matchProd{\xi}{\xi'}
&
\text{product}
\\
&&&
\matchRoll{\xi}
&
\text{roll}
\end{syntaxfig}
\caption{Syntax of matches}
\end{figure}

% \begin{figure}
\flushleft \shadebox{$\matchType{\Gamma}{\xi}{K}{A}{\Gamma'}$}
\begin{smathpar}
\inferrule*[right={$\kappa \in K_{\cxtExtend{\Gamma'}{x}{A}}$}]
{
   \strut
}
{
   \matchType{\Gamma}{\matchedVar{x}{\kappa}}{K}{A}{\Gamma'}
}
%
\and
%
\inferrule*[right={$\kappa \in K_{\Gamma'} \wedge \interpret{c} \in C$}]
{
   \strut
}
{
   \matchType{\Gamma}{\matchedGround{c}{\kappa}}{K}{C}{\Gamma'}
}
%
\and
%
\inferrule*[right={$\kappa \in K_{\Gamma'}$}]
{
   \valType{\exClosure{\rho}{\exFun{\sigma}}}{\tyFun{A}{B}}
}
{
   \matchType{\Gamma}{\matchedFun{\exClosure{\rho}{\exFun{\sigma}}}{\kappa}}{K}{(\tyFun{A}{B})}{\Gamma'}
}
%
\and
%
\inferrule*[right={\textnormal{$\dom{\interpret{D}} = \dom{\Xi} = \set{d} \cup \tilde{d}$}}]
{
   \matchType{\Gamma}{\Xi(d)}{K}{\interpret{D}(d)}{\Gamma'}
   \\
   \elimType{\Xi(d')}{K}{\interpret{D}(d')}{\Gamma'}
   \\
   (\forall d' \in \tilde{d})
}
{
   \matchType{\Gamma}{\matchedConstr{\Xi}}{K}{\tyData{D}}{\Gamma'}
}
\end{smathpar}
\vspace{5pt}

\flushleft \shadebox{$\explMatchType{\Gamma}{\Psi}{K}{\vec{A}}{\Gamma'}$}
\begin{smathpar}
\inferrule*[right={$\kappa \in K_{\Gamma'}$}]
{
   \strut
}
{
   \explMatchType{\Gamma}{\matchedUnit{\kappa}}{K}{\seqEmpty}{\Gamma'}
}
%
\and
%
\inferrule*
{
   \explMatchType{\Gamma}{\explVal{T}{\xi}}{\matchTypePartial{K}{\vec{A}}}{A}{\Gamma'}
}
{
   \explMatchType{\Gamma}{\matchedProd{\explVal{T}{\xi}}}{K}{A \concat \vec{A}}{\Gamma'}
}
\end{smathpar}
\caption{Typing rules for matches and match products}
\end{figure}

% \begin{figure}
{\small
\begin{align*}
   \fmapTwo{f}{g}{\matchedInj{\sigma}}
   &=
   \matchedInj{\fmap{g}{\sigma}}
   \\
   \fmapTwo{f}{g}{(\matchedVar{x}{\kappa})}
   &=
   \matchedVar{x}{f\kappa}
   \\
   \fmapTwo{f}{g}{(\matchedUnit{\kappa})}
   &=
   \matchedUnit{f\kappa}
   \\
   \fmapTwo{f}{g}{\matchedSumL{\explVal{T}{\xi}}{\sigma}}
   &=
   \matchedSumL{\explVal{T}{\fmapTwo{f}{g}{\xi}}}{\fmap{g}{\sigma}}
   \\
   \fmapTwo{f}{g}{\matchedSumR{\sigma}{\explVal{T}{\xi}}}
   &=
   \matchedSumR{\fmap{g}{\sigma}}{\explVal{T}{\fmapTwo{f}{g}{\xi}}}
   \\
   \fmapTwo{f}{g}{\matchedProd{\explVal{T}{\xi}}}
   &=
   \matchedProd{\explVal{T}{\fmapTwo{(\fmapTwoPartial{f}{g})}{(\fmapTwoPartial{g}{g})}{\xi}}}
   \\
   \fmapTwo{f}{g}{(\matchedRoll{\explVal{T}{\xi}})}
   &=
   \matchedRoll{\explVal{T}{\fmapTwo{f}{g}{\xi}}}
   \\[2mm]
   \fmapTwo{f}{g}{(\explVal{T}{\xi})}
   &=
   \explVal{T}{\fmapTwo{f}{g}{\xi}}
\end{align*}}
\caption{Mapping families of functions $f_{\Gamma}, g_{\Gamma}: K_{\Gamma} \to K'_{\Gamma}$ to an (explained) match}
\end{figure}

% \begin{figure}
\begin{align*}
\exprTr{\exVar{x}}
&=
\explVal{\trVar{x}{}{\trBot}}{\exBot}
\\
\exprTr{\exConst{c}}
&=
\explVal{\trEmpty}{\exConst{c}}
\\
\exprTr{\exConst{\phi}}
&=
\explVal{\trEmpty}{\exConst{\phi}}
\\
\exprTr{\exFun{\sigma}}
&=
\explVal{\trEmpty}{\exClosure{\envBot}{\exFun{\exprTr{\sigma}}}}
\\
\exprTr{\exApp{e}{e'}}
&=
\explVal{\trApp{\exprTr{e}}{\exprTr{e'}}{\trBot}}{\exBot}
\\
\exprTr{\exPrimApp{e_1}{\primOp}{e_2}}
&=
\explVal{\trPrimApp{\exprTr{e_1}}{\primOp}{\exprTr{e_2}}}{\exBot}
\\
\exprTr{\exConstr{c}{\vec{e}}}
&=
\explVal{\trEmpty}{\exConstr{c}{\exprTr{\vec{e}}}}
\\
\exprTr{\exMatch{e}{\sigma}}
&=
\explVal{\trMatch{\exprTr{e}}{\exprTr{\sigma}}{}{\exBot}}{\exBot}
\\
\exprTr{\exLetrec{\delta}{e}}
&=
\explVal{\trLetrec{\exprTr{\delta}}{T}}{\exBot}
\text{ where }
\explVal{T}{v} = \exprTr{e}
\end{align*}
\caption{Embedding of expressions into explained values}
\end{figure}


% \begin{figure}
\flushleft \shadebox{$\rho, \indexed{e} \eval{\sigma} \explVal{\indexed{U}}{\indexed{u}}, \rho', \kappa$}
\begin{smathpar}
\inferrule*
{
   \strut
}
{
   \rho,
   \indexed{e}
   \eval{\trieVar{x}{\kappa}}
   \explVal{\indexed{e}}{\exBot},
   \envExtend{\envEmpty}{x}{\recEnvEntry{\rho}{\envEmpty}{\indexed{e}}},
   \kappa
}
%
\and
%
\inferrule*[right={$
   \exThunkVar{x}{\recEnvEntry{\rho'}{\delta}{\indexed{e}}} \in \rho \wedge
   \envType{\Gamma}{\rho' \concat \closeDefs{\delta}{\rho'}}
$}]
{
   \rho' \concat \closeDefs{\delta}{\rho'},
   \shift{\closeDefs{\delta}{\rho'}}{\indexed{e}}
   \eval{\tau}
   \explVal{\indexed{T}}{\indexed{v}},
   \rho^\twoPrime,
   \kappa
}
{
   \rho,
   \raw{\exVar{x}}{\idx{i}}
   \eval{\tau}
   \explVal{\raw{(\trVar{x}{\Gamma}{\indexed{T}})}{\idx{i}}}{\indexed{v}},
   \rho^\twoPrime,
   \kappa
}
%
\and
%
\inferrule*[right={$\interpret{\exConst{c}} \in \tyGround{C}$}]
{
   \strut
}
{
   \rho,
   \raw{\exConst{c}}{\idx{i}}
   \eval{\trieGround{C}{\kappa}}
   \explVal
      {\raw{\trEmpty}{\idx{i}}}
      {\raw{\exConst{c}}{\idx{i}}},
   \envEmpty,
   \kappa
}
%
\and
%
\inferrule*
{
   \strut
}
{
   \rho,
   \raw{\phi}{\idx{i}}
   \eval{\trieFun{\kappa}}
   \explVal
      {\raw{\trEmpty}{\idx{i}}}
      {\raw{\phi}{\idx{i}}},
   \envEmpty,
   \kappa
}
%
\and
%
\inferrule*
{
   \strut
}
{
   \rho,
   \raw{\exFun{\sigma}}{\idx{i}}
   \eval{\trieFun{\kappa}}
   \explVal
      {\raw{\trEmpty}{\idx{i}}}
      {\raw{\exClosure{\rho}{\exFun{\sigma}}}{\idx{i}}},
   \envEmpty,
   \kappa
}
%
\and
%
   \inferrule*[right={$\envType{\Gamma}{\rho_1 \concat \rho_2}$}]
   {
      \rho, \indexed{e} \eval{\trieFun{\metaunit}}
      \explVal{\indexed{T}}{\raw{\exClosure{\rho_1}{\exFun{\sigma}}}{\idx{j}}}, \envEmpty, \metaunit
      \\
      \rho, \indexed{e}' \eval{\sigma} \explVal{\indexed{T}'}{\indexed{u}}, \rho_2, \indexed{e}^\twoPrime
      \\
      \rho_1 \concat \rho_2,
      \shift{\rho_2}{\indexed{e}^\twoPrime}
      \eval{\tau}
      \explVal{\indexed{T}^\twoPrime}{\indexed{v}}, \rho', \kappa
   }
   {
      \rho,
      \raw{(\exApp{\indexed{e}}{\indexed{e}'})}{\idx{i}}
      \eval{\tau}
      \explVal
         {\raw
            {\smash{(\trApp{(\explVal{\indexed{T}}{\raw{\exClosure{\rho_1}{\exFun{\sigma}}}{\idx{j}}})}
                           {(\explVal{\indexed{T'}}{\indexed{u}})}
                           {\indexed{T^\twoPrime}})}}
            {\idx{i}}}
         {\indexed{v}},
      \rho',
      \kappa
   }
%
\and
%
\inferrule*[right={\textnormal{
   $\interpret{\phi} \in \UnaryOp{C}{A} \wedge \interpret{\phi}(c, \tau) = (v, \kappa)$
}}]
{
   \rho, \indexed{e} \eval{\trieFun{\metaunit}} \explVal{\indexed{T}}{\raw{\phi}{\idx{j}}}, \envEmpty, \metaunit
   \\
   \rho, \indexed{e}' \eval{\trieGround{C}{\metaunit}} \explVal{\indexed{T}'}{\raw{c}{\idx{k}}}, \envEmpty, \metaunit
}
{
   \rho,
   \raw{(\exApp{\indexed{e}}{\indexed{e}'})}{\idx{i}}
   \eval{\tau}
   \explVal
      {\raw{\smash{(\trApp{(\explVal{\indexed{T}}{\raw{\phi}{\idx{j}}})}
                          {(\explVal{\indexed{T}'}{\raw{c}{\idx{k}}})}
                          {\trBot})}}
           {\idx{i}}}
      {\raw{v}{\idx{i}}},
   \envEmpty,
   \kappa
}
%
\and
%
\inferrule*[right={\textnormal{
   $\interpret{\primOp} \in \BinaryOp{C}{C'}{A} \wedge \interpret{\primOp}(c, c', \tau) = (v, \kappa)$
}}]
{
   \rho, e \eval{\trieGround{C}{\metaunit}} \explVal{\indexed{T}}{\raw{c}{\idx{j}}}, \envEmpty, \metaunit
   \\
   \rho, e' \eval{\trieGround{C'}{\metaunit}} \explVal{\indexed{T}'}{\raw{\smash{c'}}{\idx{k}}}, \envEmpty, \metaunit
}
{
   \rho,
   \raw{(\exPrimApp{\indexed{e}}{\primOp}{\indexed{e}'})}{\idx{i}}
   \eval{\tau}
   \explVal
      {\raw{\smash{\trPrimApp{(\explVal{\indexed{T}}{\raw{c}{\idx{j}}})}
                             {\primOp}
                             {(\explVal{\indexed{T}'}{\raw{\smash{c'}}{\idx{k}}})}}}{\idx{i}}}
      {\raw{v}{\idx{i}}},
   \envEmpty,
   \kappa
}
%
\and
%
\inferrule*
{
   \rho,
   \vec{\indexed{e}}
   \eval{\Sigma(d)}
   \vec{\explVal{\indexed{T}}{\indexed{u}}},
   \rho',
   \kappa
}
{
   \rho,
   \raw{\exConstr{d}{\vec{\indexed{e}}}}{\idx{i}}
   \eval{\Sigma}
   \explVal
      {\raw{\trEmpty}{\idx{i}}}
      {\raw{\exConstr{d}{\vec{\explVal{\indexed{T}}{\indexed{u}}}}}{\idx{i}}},
   \rho',
   \kappa
}
%
\and
%
\inferrule*
{
   \rho, \indexed{e} \eval{\sigma} \explVal{\indexed{U}}{\indexed{u}}, \rho', \indexed{e}'
   \\
   \rho \concat \rho',
   \shift{\rho'}{\indexed{e}'}
   \eval{\tau}
   \explVal{\indexed{T}}{\indexed{v}},
   \rho^\twoPrime,
   \kappa
}
{
   \rho,
   \raw{(\exMatch{\indexed{e}}{\sigma})}{\idx{i}}
   \eval{\tau}
   \explVal
      {\raw{(\trMatch{\explVal{\indexed{U}}{u}}{\sigma}{\Gamma}{\indexed{T}})}{\idx{i}}}
      {\indexed{v}},
   \rho^\twoPrime,
   \kappa
}
%
\and
%
\inferrule*
{
   \rho \concat \closeDefs{\delta}{\rho},
   \shift{\closeDefs{\delta}{\rho}}{\indexed{e}}
   \eval{\tau}
   \explVal{\indexed{T}}{\indexed{v}},
   \rho',
   \kappa
}
{
   \rho,
   \raw{(\exLetrec{\delta}{\indexed{e}})}{\idx{i}}
   \eval{\tau}
   \explVal
      {\raw{(\trLetrec{\delta}{\indexed{T}})}{\idx{i}}}
      {\indexed{v}},
   \rho',
   \kappa
}
\end{smathpar}
\caption{Evaluation for persistent terms}
\end{figure}


\subsection{\ttt{match}...\ttt{as}}
\label{sec:impl-language:match-as}

Pattern-matching expressions, which are essentially generalised $\kw{let}$
forms, are easily added as syntactic sugar for a beta-redex. Specifically,
$\exMatch{e}{\sigma}$ desugars to $\exApp{\exFun{\sigma'}}{e'}$ where $\sigma'$
and $e'$ are the desugaring of $\sigma$ and $e$. Alternatively, the rules shown
are derivable and give the same static and dynamic semantics as the desugaring.
Similarly, $\exLet{x}{e}{e'}$ can also be expressed as a desugaring to
$\exMatch{e}{\elimVar{x}{e^\twoPrime}}$ where $e^\twoPrime$ is the desugaring of
$e'$.

\subsection{Mutual recursion}
\label{sec:impl-language:recursion}

\begin{figure}
\flushleft\shadebox{$\rho, \delta \closeDefs \rho'$}
\begin{salign}
   \rho, \delta
   &\closeDefs
   [f: \exClosure{\rho}{\delta}{\exLambda{\sigma}} \mid \exFun{f}{\sigma} \in \delta]
\end{salign}
\flushleft\shadebox{$\rho, \delta, \alpha \closeDefsFwd \rho'$}
\begin{salign}
   \rho, \delta, \alpha
   &\closeDefsFwd
   [f : \annot{\exClosure{\rho}{\delta}{\exLambda{\sigma}}}{\alpha} \mid \exFun{f}{\sigma} \in \delta]
\end{salign}
\flushleft\shadebox{$\rho \closeDefsBwd \rho', \delta, \alpha$}
\begin{salign}
   [f : \annot{\exClosure{\rho_f}{\delta_f}{\exLambda{\sigma_f}}}{\alpha_f} \mid f \in F]
   &\closeDefsFwd
   {\textstyle{\bigvee}\rho}, [
   \exFun{f}{\sigma_f} \mid f \in F] \join {\textstyle{\bigvee}\delta},
   {\textstyle{\bigvee}\alpha}
\end{salign}
\caption{Mutually recursive functions $\delta$ close to $\rho'$ in $\rho$}
\end{figure}


\begin{definition}
\label{def:closedefs}
   \figref{impl-language:closedefs} defines the deterministic relation $\closeDefs$.
\end{definition}

\begin{definition}
\label{def:unclosedefs}
   \figref{impl-language:closedefs} defines the deterministic relation $\uncloseDefs$.
\end{definition}

\begin{lemma}
\label{lem:gc-closedefs}

Suppose $\rho, \exLetrecDef{\delta} \closeDefs \rho', \exLetrecDef{\delta}$
where $\rho$, $\delta$ and $\rho'$ are unannotated, and write $\closeDefs_{\rho,
\delta}$ for $\closeDefs$ domain-restricted to $\Ann{\rho,
\exLetrecDef{\delta}}$ and $\uncloseDefs_{\rho, \delta}$ for $\uncloseDefs$
domain-restricted to $\Ann{\rho',\exLetrecDef{\delta}}$. Then $\closeDefs_{\rho,
\delta}$ and $\uncloseDefs_{\rho, \delta}$ form a Galois injection:

\[
   {\closeDefs_{\rho, \delta}} \adjoint {\uncloseDefs_{\rho, \delta}}:
   \Ann{\rho, \exLetrecDef{\delta}} \inj \Ann{\rho',\exLetrecDef{\delta}}
\]

\end{lemma}

\begin{abstract}
   We present new language-based data provenance techniques for linking visualisations and other structured outputs to data in a fine-grained way, allowing a user to interactively explore how data attributes map to visual or other output elements by selecting (focusing on) substructures of interest. This can help both programmers and end-users understand how data sources and complex outputs are related, which can be a challenge even for an expert. Our approach builds on bidirectional program slicing techiques based on Galois connections, which provide desirable round-tripping properties.

   Unlike prior work in this area, our approach allows selections to be negated (inverted). When combined with negation, the bidirectional analysis can also be used to link different outputs generated from the same input. This offers a principled language-based foundation for a popular interactive visualisation feature called \emph{brushing and linking} where selections in one chart automatically select corresponding elements in another related chart. Although such view coordination features are valuable comprehension aids, they tend be to hard-coded into specific applications or libraries, or require programmer effort.
\end{abstract}

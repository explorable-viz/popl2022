\begin{abstract}
   We propose new language-based data provenance techniques for creating visualisations and other structured outputs which are automatically linked to data in a fine-grained way, allowing a user to interactively explore how data attributes map to visual or other output elements by selecting (focusing on) a substructure of interest. This can help a user understand how data sources and complex outputs are related, which can be a challenge even for an expert. Our approach builds on bidirectional program slicing techiques based on Galois connections, which provide desirable round-tripping properties.

   Unlike prior work in this area, our bidirectional analysis allows a selection to be negated (inverted). When combined with negation, input-output linking can be used to link different outputs generated from the same input. This offers a principled language-based foundation for a popular interactive visualisation feature called \emph{brushing and linking} where selections in one chart automatically select corresponding elements in another related chart. Although such view coordination features are valuable comprehension aids, they tend be to hard-coded into specific applications or libraries, or require programmer effort.
\end{abstract}

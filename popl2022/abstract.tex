\begin{abstract}
   Charts and graphics often provide a link to the data sources they were derived from. But understanding how a data source and complex visualisation are related — how data attributes map to visual elements — is a challenge even for a knowledgeable reader. We propose new language-based data provenance techniques for creating visualisations which are automatically linked to data in a fine-grained way, allowing a reader to explore these relationships interactively. Our approach builds on bidirectional program slicing techiques based on Galois connections, which characterise desirable round-tripping properties of the data provenance.

   When multiple visualisations are used to present different aspects of a data set, the reader is also tasked with understanding how these views related. Visualisations sometimes provide a feature called \emph{brushing and linking} where selections in one chart automatically select corresponding elements in another related chart. Although such view coordination features are valuable comprehension aids, they tend be to hard-coded into specific applications or libraries, or require programmer effort. We show how our bidirectional framework, unlike earlier work, supports a notion of negation which can be used to implement a fully automatic brushing and linking mechanism.
\end{abstract}

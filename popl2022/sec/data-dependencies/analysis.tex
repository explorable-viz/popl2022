\subsection{Forward and backward Galois dependency}
\label{sec:core-language:fwd-bwd}

\todo{Introduce this section}. %The annotated syntactic forms introduced in \secref{data-dependencies} allow us to now describe how data dependencies can be tracked during evaluation by propagated selection information during forward and backward dependency analysis.
\todo{Given that forward analysis is notated with $\evalFwdF{T}$, should we introduce this section by describing how forward analysis builds on top of the idea of evaluation but on annotated syntactic forms?}

\subsubsection{Pattern-matching}

\begin{figure}
\flushleft \shadebox{$v, \sigma \match \rho, \matchPlugLow{\xi}{e}$}
\begin{smathpar}
   \inferrule*[lab={\ruleName{$\match$-var}}]
   {
      \strut
   }
   {
      v, \elimVar{x}{e} \match \envExtend{\envEmpty}{x}{v}, \matchVar{x}{e}
   }
   \\
   %
   \and
   %
   \inferrule*[lab={\ruleName{$\match$-true}}]
   {
      \strut
   }
   {
      \exTrue, \elimBool{e}{e'}
      \match
      \envEmpty, \matchTrueLow{e}{e'}
   }
   %
   \and
   %
   \inferrule*[lab={\ruleName{$\match$-false}}]
   {
      \strut
   }
   {
      \exFalse, \elimBool{e}{e'}
      \match
      \envEmpty, \matchFalseLow{e}{e'}
   }
   %
   \and
   %
   \inferrule*[lab={\ruleName{$\matchFwd$-pair}}]
   {
      \strut
   }
   {
      \exPair{v}{v'}, \elimPair{x}{y}{e}
      \matchFwd
      \envExtend{\envExtend{\envEmpty}{x}{v}}{y}{v'}, e, \top
   }
   %
   \and
   %
   \inferrule*[lab={\ruleName{$\matchFwd$-pair-del}}]
   {
      \strut
   }
   {
      \exPairDel{v}{v'}, \elimPair{x}{y}{e}
      \matchFwd
      \envExtend{\envExtend{\envEmpty}{x}{v}}{y}{v'}, e, \bot
   }
   %
   \and
   %
   \inferrule*[lab={\ruleName{$\match$-nil}}]
   {
      \strut
   }
   {
      \exNil, \elimList{\branchNil{e}}{\branchCons{x}{y}{e'}}
      \match
      \envEmpty, \matchNilLow{e}{x}{y}{e'}
   }
   %
   \and
   %
   \inferrule*[lab={\ruleName{$\match$-cons}}]
   {
      \strut
   }
   {
      \exCons{v}{v'}, \elimList{\branchNil{e}}{\branchCons{x}{y}{e'}}
      \match
      \envExtend{\envExtend{\envEmpty}{x}{v}}{y}{v'}, \matchConsLow{e}{x}{y}{e'}
   }
\end{smathpar}
\caption{Pattern-matching}
\end{figure}


\subsubsection{Evaluation}

The idea of forward dependency analysis is to propagate selections on the input resources in order to determine the selections on the output value. The function $\evalFwdF{T} :: \tyProd{\Sel{\rho, e}{A}}{\textcal{A}} \to \Sel{v}{A}$ describes how we take a selected term $e$ and environment $\rho$, along with the selection $\alpha$ of the current context, and consume them to construct the selection information of the output value $v$. When working in the forwards direction, we want to express that a value is only selected if all the terms involved in its construction are selected, hence we use the meet operation to combine selection states.

Evaluating a variable $x$ is equivalent to looking it up in the environment $\rho$, where $\envLookupFwdF{}$ is defined in \figref{core-language:slicing:eval-aux}. Anonymous functions $\exLambda{\sigma}$ simply evaluate to closures, which are unannotated. We begin to see how the $\meet$ operation is used when evaluating an integer $n_{\alpha'}$ with $\alpha$ as the current selection context - this returns $\annot{n}{\alpha \meet \alpha'}$, expressing that the value $n$ is selected if both the term $n$ and the term containing $n$ were selected. The same logic applies to empty lists $\annot{\exNil}{\alpha'}$.

The recursive behaviour of forward analysis is first seen in the rule for records, $\exRec{\vec{\bind{x}{e}}}_{\alpha}$. We begin by forward analysing each term $e_i$ in the record to a selected value $v_i$, and then construct a record $\annot{\vec{\bind{x}{v}}}{\alpha \meet \alpha'}$, which is only selected if the record itself was selected along with the current context $\alpha'$.

For the cons rule, $\exPair{(\rho,\annCons{e_1}{e_2}{\alpha'})}{\alpha}$, we forward analyse $e_1$ and $e_2$ to get annotated values $v_1$ and $v_2$, and then construct the value $\annCons{v_1}{v_2}{\alpha \meet \alpha'}$ where the cons constructor is only selected if both $\exCons{e_1}{e_2}$ and the outer term containing it are selected.

% Give the intuition that what we do with annotation propagation is specifying how we consume resources in the input in order to construct things in the output.

\paragraph{Primitive operations}

Each primitive operation $\phi: \tyInt^{i} \to \tyInt$ must for every $\vec{n}$ with $\length{\vec{n}} = i$ provide a Galois connection $(\primFwdBool{\phi}{\vec{n}}, \primBwdBool{\phi}{\vec{n}})$ between $\Bool^i$ and $\Bool$, which we lift to a Galois connection $(\primFwd{\phi}{\vec{n}}, \primBwd{\phi}{\vec{n}})$ between $\Below{\vec{n}}$ and $\Below{\phi(\vec{n})}$ by defining
\begin{definition}
\label{def:core-language:primop-gc}
\begin{salign}
   \primFwd{\phi}{\vec{n}}(\vec{\annInt{n}{\alpha}}) &= \annInt{m}{\beta}
   \text{ where }
   \primFwdBool{\phi}{\vec{n}}(\vec{\alpha}) = \beta
   \\
   \primBwd{\phi}{\vec{n}}(\annInt{m}{\beta}) &= \vec{\annInt{n}{\alpha}}
   \text{ where }
   \primBwdBool{\phi}{\vec{n}}(\beta) = \vec{\alpha}
\end{salign}
\end{definition}

\noindent where $\vec{\annInt{n}{\alpha}}$ denotes the zip of same-length sequences $\vec{n}$ and $\vec{\alpha}$ with the constructor for integer values. For any $\vec{n}$ with $\length{\vec{n}} \numlt i$, any such $\phi$ also gives rise to an isomorphism between $\Below{\vec{n}}$ and the lattice of partial applications $\Below{\exPrimOp{\phi}{\vec{n}}}$.

\begin{figure}
\flushleft\shadebox{$\rho, \delta \closeDefs \rho'$}
\begin{salign}
   \rho, \delta
   &\closeDefs
   [f: \exClosure{\rho}{\delta}{\sigma} \mid \exFun{f}{\sigma} \in \delta]
\end{salign}
\\[5mm]
\flushleft \shadebox{$\explVal{T}{\rho, e \eval v}$}
\begin{smathpar}
   \inferrule*[lab={\ruleName{$\eval$-lambda}}]
   {
      \strut
   }
   {
      \explVal
         {\trLambda{\sigma}}
         {\rho, \exLambda{\sigma} \eval \exClosure{\rho}{\seqEmpty}{\sigma}}
   }
   %
   \and
   %
   \inferrule*[lab={\ruleName{$\eval$-apply}}]
   {
      \explVal{T}{\rho, e \eval \exClosure{\rho_1}{\delta}{\sigma}}
      \\
      \rho_1, \delta \closeDefs \rho_2
      \\
      \explVal{U}{\rho, e' \eval v}
      \\
      \explVal{\xi}{v, \sigma \match \rho_3, e^\twoPrime}
      \\
      \explVal{T'}{\rho_1 \concat \rho_2 \concat \rho_3, e^\twoPrime \eval v'}
   }
   {
      \explVal{\trApp{T}{U}{\matchPlug{\xi}{T'}}}{\rho, \exApp{e}{e'} \eval v'}
   }
   %
   \and
   %
   \inferrule*[lab={\ruleName{$\eval$-let-struct}}]
   {
      \explVal{T}{\rho, e \eval v}
      \\
      \explVal{\xi}{v, \sigma \match \rho', e'}
      \\
      \explVal{U}{\rho \concat \rho', e' \eval v'}
   }
   {
      \explVal
         {\trLetStructured{\singleton{\xi}{T}}{U}}
         {\rho, \exLetStructured{\singleton{\sigma}{e}}{e'} \eval v'}
   }
\end{smathpar}
\caption{Additional evaluation rules}
\end{figure}

\begin{figure}[H]
\small
$\begin{array}{lll|llll}
\arrayrulecolor{lightgray}
&&&\textit{where}
\\
\rowcolor{verylightgray}
\envLookupGC{\rho,z} &::& \Below{\rho} \to \Below{v}
&\envLookup{\rho}{z}{v}
&&
\\
\envLookupFwdF{\rho \concat \bind{x}{v},x}(\rho' \concat \bind{x}{u},x) &=& u
\\
\envLookupFwdF{\rho \concat \bind{y}{v},x}(\rho' \concat \bind{y}{u},x)
&=&
\envLookupFwdF{\rho,x}(\rho',x)
&
x \neq y
\\[3mm]
\envLookupBwdF{\rho \concat \bind{x}{v},x}(u)
&=&
\sub{\hole}{\rho} \concat \bind{x}{u}
\\
\envLookupBwdF{\rho \concat \bind{y}{v},x}(u)
&=&
\envLookupBwdF{\rho}(u) \concat \bind{y}{\hole}
&
x \neq y
\\[3mm]
%
%
%
\rowcolor{verylightgray}
\closeDefsGC{\rho,h} &::& \Below{(\rho, h)} \to \Below{\rho'}
&\rho, h \closeDefsR \rho'
&&
\\
\closeDefsFwdF{\rho,h}(\rho',h')
&=&
\vec{\bind{x}{v}}
&
h' &=& \vec{\bind{x}{\sigma}}
\\
&&&
v_i &=& \exClosure{\rho'}{h'}{\sigma_i}
\\[3mm]
\closeDefsBwdF{\rho,h}(\vec{\bind{x}{v}})
&=&
\bigjoin\vec{\rho}', \vec{\bind{x}{\sigma}} \join {\bigjoin\vec{h}'}
&
v_i &=& \exClosure{\rho'_i}{h'_i}{\sigma_i}
\end{array}$
\caption{Auxiliary functions: forward and backward slicing}
\label{fig:slicing:eval-aux}
\end{figure}


\begin{definition}[Hole environment]
Define $\hole_{\vec{\bind{x}{v}}} = \vec{\bind{x}{\hole}}$
\end{definition}

\begin{lemma}[Least environment for $\rho$]
\label{lem:core-language:hole-env}If $\vdash \rho: \Gamma$ then $\hole_{\rho} \leq \rho$.
\end{lemma}

\begin{definition}[Forward and backward functions for environment lookup]
   Suppose $\envLookup{\rho}{x}{v}$. Then define $\envLookupFwdF{\rho,x}: \Below{\rho} \to \Below{v}$ and $\envLookupBwdF{\rho,x}: \Below{v} \to \Below{\rho}$ to be $\bind{x}{-}\envLookupR$ and $\envLookupBwdR{\rho}\bind{x}{-}$ restricted to $\Below{\rho}$ and $\Below{v}$ respectively.
\end{definition}

\begin{lemma}[Galois connection for environment]
\label{lem:core-language:env-get-put}Suppose $\envLookup{\rho}{x}{v}$.
\begin{enumerate}
   \item \label{lem:core-language:env-get-put:1} $\envLookupFwdF{\rho,x}(\envLookupBwdF{\rho,x}(v)) \geq v$.
   \item \label{lem:core-language:env-get-put:2} $\envLookupBwdF{\rho,x}(\envLookupFwdF{\rho,x}(\rho')) \leq \rho'$.
\end{enumerate}
\end{lemma}

\begin{definition}
   \label{def:core-language:closeDefs-bwd}
   Define the relation $\closeDefsBwdR$ as given in \figref{core-language:slicing:eval-aux}.
\end{definition}

\begin{definition}[Forward and backward functions for recursive bindings]
   Suppose $\rho, h \closeDefsR \rho'$. Define $\closeDefsFwdF{\rho,h}: \Below{(\rho, h)} \to \Below{\rho'}$ and $\closeDefsBwdF{\rho,h}: \Below{\rho'} \to \Below{(\rho, h)}$ to be $\closeDefsR$ and $\closeDefsBwdR$ restricted to $\Below{(\rho, h)}$ and $\Below{\rho'}$ respectively.
\end{definition}

\begin{theorem}[Galois connection for recursive bindings]
\label{thm:core-language:closeDefs:gc}
   Suppose $\rho, h \closeDefsR \rho'$.  Then $\closeDefsFwdF{\rho,h} \adjoint \closeDefsBwdF{\rho,h}$.
\end{theorem}

\begin{definition}[Forward and backward functions for evaluation]
   Suppose $T: \rho, e \evalR v$. Define $\evalFwdF{T}: \Below{(\rho, e, \TT)} \to \Below{v}$ and $\evalBwdF{T}: \Below{v} \to \Below{(\rho, e, \TT)}$ to be $\evalFwdR{T}$ and $\evalBwdR{T}$ restricted to $\Below{(\rho, e, \TT)}$ and $\Below{v}$ respectively.
\end{definition}

\begin{theorem}[Galois connection for evaluation]
\label{thm:core-language:eval:gc}
   Suppose $T: \rho, e \evalFwdS v$.  Then $\evalFwdF{T} \adjoint \evalBwdF{T}$.
\end{theorem}

\begin{lemma}[Determinism of pattern-matching]
   Suppose $v, \sigma \matchFwdR{w} \rho, \kappa$ and $v', \sigma' \matchFwdR{w} \rho', \kappa'$. If $(v, \sigma) \eq (v', \sigma)$ then $(\rho, \kappa) \eq (\rho', \kappa')$.
\end{lemma}

\begin{definition}[Forward and backward functions for pattern-matching]
   Suppose $w: v, \sigma \matchR \rho, \kappa$. Define $\matchFwdF{w}: \Below{(v,\sigma,\TT)} \to \Below{(\rho,\kappa)}$ and $\matchBwdF{w}: \Below{(\rho,\kappa)} \to \Below{(v,\sigma,\TT)}$ to be $\matchFwdR{w}$ and $\matchBwdR{w}$ domain-restricted to $\Below{(v,\sigma,\TT)}$ and $\Below{(\rho,\kappa)}$ respectively.
\end{definition}

\begin{theorem}[Galois connection for pattern-matching]
\label{thm:core-language:match:gc}
   Suppose $w: v, \sigma \matchFwdS \rho, \kappa$.  Then $\matchFwdF{w} \adjoint \matchBwdF{w}$.
\end{theorem}

\subsection{Galois dependency}
\label{sec:data-dependencies:analyses}

We now show how to interpret \todo{revisit} a selection on a term $\raw{e}$ or its environment $\raw{\rho}$ (collectively: an \emph{input selection}) as a selection on its output $\raw{v}$, and vice versa. In the forward direction, selections represent \emph{availability}, with the input selection at each step giving rise to an output selection indicating the data available to the downstream computation. In the backward direction, selections represent demands, with the output selection at each step giving rise to an input selection identifying the data needed from the upstream computation. Both analyses are with respect to a fixed computation $T :: \raw{\rho}, \raw{e} \evalR \raw{v}$, where $T$ is a trace. (Typically one would first evaluate a program to obtain its trace $T$, and then run multiple forward or backward analyses over $T$ with appropriate lattices.)

We call our analysis \emph{Galois dependency}, because the two directions form a Galois connection \todo{explain why this matters, perhaps pulling some of the motivation regarding GCs from 3.5 forward to here}, a pair of functions that are each other's closest approximate inverse: from above in one direction, from below in the other (\secref{data-dependencies:analysis:galois-connections}).  We present the forward direction first and explain the underlying principles.

\subsection{Forward data dependency}
\label{sec:data-dependencies:analyses:fwd}

The forward data dependency analysis ``replays'' evaluation, turing input availability into output availability, with $T$ guiding the analysis whenever holes in the input selection would mean the analysis would otherwise get stuck. (Forward) pattern-matching introduces the idea of a selection as the data available to a downstream computation, so we start with that.

\begin{figure}
{\small \flushleft \shadebox{$\matchFwd{v}{\sigma}{w}{\rho}{\kappa}{\alpha}$}%
\hfill \textbfit{$v$ and $\sigma$ forward-match along $w$ to $\rho$ and $\kappa$, with ambient availability $\alpha$}}
\begin{smathpar}
   \inferrule*[
      lab={\ruleName{$\matchFwdS$-hole-1}}
   ]
   {
      \hole \eq v
      \\
      \matchFwd{v}{\sigma}{w}{\rho}{\kappa}{\alpha}
   }
   {
      \matchFwd{\hole}{\sigma}{w}{\rho}{\kappa}{\alpha}
   }
   %
   \and
   %
   \inferrule*[
      lab={\ruleName{$\matchFwdS$-hole-2}}
   ]
   {
      \hole \eq \sigma
      \\
      \matchFwd{v}{\sigma}{w}{\rho}{\kappa}{\alpha}
   }
   {
      \matchFwd{v}{\hole}{w}{\rho}{\kappa}{\alpha}
   }
   %
   \and
   %
   \inferrule*[
      lab={\ruleName{$\matchFwdS$-var}}
   ]
   {
      \strut
   }
   {
      \matchFwd{v}{\elimVar{x}{\kappa}}{\matchVar{x}}{\bind{x}{v}}{\kappa}{\top}
   }
   %
   \and
   %
   \inferrule*[
      lab={\ruleName{$\matchFwdS$-true}}
   ]
   {
      \strut
   }
   {
      \matchFwd{\annTrue{\alpha}}
               {\elimBool{\kappa}{\kappa'}}
               {\matchTrue}
               {\seqEmpty}{\kappa}{\alpha}
   }
   %
   \and
   %
   \inferrule*[
      lab={\ruleName{$\matchFwdS$-false}}
   ]
   {
      \strut
   }
   {
      \matchFwd{\annFalse{\alpha}}
               {\elimBool{\kappa}{\kappa'}}
               {\matchFalse}
               {\seqEmpty}{\kappa'}{\alpha}
   }
   %
   \and
   %
   \inferrule*[
      lab={\ruleName{$\matchFwdS$-unit}}
   ]
   {
      \strut
   }
   {
      \matchFwd{\annot{\exRecEmpty}{\alpha}}
               {\elimRecEmpty{\kappa}}
               {\matchRecEmpty}
               {\seqEmpty}
               {\kappa}
               {\alpha}
   }
   %
   \and
   %
   \inferrule*[
      lab={\ruleName{$\matchFwdS$-record}}
   ]
   {
      \matchFwd{\annRec{\vec{\bind{x}{v}}}{\top}}
               {\elimRec{\vec{x}}{\sigma}}
               {\matchRec{\vec{\bind{x}{w}}}}
               {\rho}
               {\sigma'}
               {\beta}
      \\
      \matchFwd{u}{\sigma'}{w}{\rho'}{\kappa}{\beta'}
   }
   {
      \matchFwd{\annRec{\vec{\bind{x}{v}} \concat \bind{y}{u}}{\alpha}}
               {\elimRec{\vec{x} \concat y}{\sigma}}
               {\matchRec{\vec{\bind{x}{w}} \concat \bind{y}{w'}}}
               {\rho \concat \rho'}
               {\kappa}
               {\alpha \meet \beta \meet \beta'}
   }
   %
   \and
   %
   \inferrule*[
      lab={\ruleName{$\matchFwdS$-nil}}
   ]
   {
      \strut
   }
   {
      \matchFwd{\annNil{\alpha}}
               {\elimList{\kappa}{\sigma'}}
               {\matchNil}
               {\seqEmpty}{\kappa}{\alpha}
   }
   %
   \and
   %
   \inferrule*[
      lab={\ruleName{$\matchFwdS$-cons}}
   ]
   {
      \matchFwd{v}{\sigma}{w}{\rho}{\tau}{\beta}
      \\
      \matchFwd{v'}{\tau}{w'}{\rho'}{\kappa'}{\beta'}
   }
   {
      \matchFwd{\annCons{v}{v'}{\alpha}}
               {\elimList{\kappa}{\sigma}}
               {\matchCons{w}{w'}}
               {\rho \concat \rho'}{\kappa'}{\alpha \meet \beta \meet \beta'}
   }
\end{smathpar}

\vspace{2mm}
{\small \flushleft \shadebox{$\evalFwd{\rho}{e}{\alpha}{T}{v}$}%
\hfill \textbfit{$\rho$ and $e$, with ambient availability $\alpha$, forward-evaluate along $T$ to $v$}}
\begin{smathpar}
   \mprset{center}
   \inferrule*[
      lab={\ruleName{$\evalFwdS$-hole}}
   ]
   {
      \hole \eq e
      \\
      \evalFwd{\rho}{e}{\alpha}{T}{v}
   }
   {
      \evalFwd{\rho}{\hole}{\alpha}{T}{v}
   }
   %
   \and
   %
   \inferrule*[
      lab={\ruleName{$\evalFwdS$-var}}
   ]
   {
      \envLookup{\rho}{x}{v}
   }
   {
      \evalFwd{\rho}{\exVar{x}}{\alpha}{\trVar{x}{\rho}}{v}
   }
   %
   \and
   %
   \inferrule*[lab={\ruleName{$\evalFwdS$-lambda}}]
   {
      \strut
   }
   {
      \evalFwd{\rho}
              {\exLambda{\sigma}}
              {\alpha}
              {\trLambda{\sigma'}}
              {\annClosure{\rho}{\seqEmpty}{\alpha}{\sigma}}
   }
   %
   \and
   %
   \inferrule*[lab={\ruleName{$\evalFwdS$-int}}]
   {
      \strut
   }
   {
      \evalFwd{\rho}
              {\annInt{n}{\alpha'}}
              {\alpha}
              {\trInt{n}{\rho}}
              {\annInt{n}{\alpha \meet \alpha'}}
   }
   %
   \and
   %
   \inferrule*[lab={\ruleName{$\evalFwdS$-record}}]
   {
      \evalFwd{\rho}{e_i}{\alpha}{T_i}{v_i}
      \quad
      (\forall i \numleq \length{\vec{x}})
   }
   {
      \evalFwd{\rho}
              {\annRec{\vec{\bind{x}{e}}}{\alpha'}}
              {\alpha}
              {\trRec{\vec{\bind{x}{T}}}}
              {\annRec{\vec{\bind{x}{v}}}{\alpha \meet \alpha'}}
   }
   %
   \and
   %
   \inferrule*[
      lab={\ruleName{$\evalFwdS$-project}}
   ]
   {
      \evalFwdEq{\rho}{e}{\alpha}{T}{\annRec{\vec{\bind{x}{u}}}{\beta}}
      \\
      \envLookup{\vec{\bind{x}{u}}}{y}{v'}
   }
   {
      \evalFwd{\rho}
              {\exRecProj{e}{y}}
              {\alpha}
              {\trRecProj{T}{\vec{\bind{x}{v}}}{y}}
              {v'}
   }
   %
   \and
   %
   \inferrule*[lab={\ruleName{$\evalFwdS$-nil}}]
   {
      \strut
   }
   {
      \evalFwd{\rho}
              {\annNil{\alpha'}}
              {\alpha}
              {\trNil{\rho}}
              {\annNil{\alpha \meet \alpha'}}
   }
   %
   \and
   %
   \inferrule*[
      lab={\ruleName{$\evalFwdS$-cons}},
   ]
   {
      \evalFwd{\rho}{e}{\alpha}{T}{v}
      \\
      \evalFwd{\rho}{e'}{\alpha}{U}{v'}
   }
   {
      \evalFwd{\rho}
              {\annCons{e}{e'}{\alpha'}}
              {\alpha}
              {\trCons{T}{U}}
              {\annCons{v}{v'}{\alpha \meet \alpha'}}
   }
   %
   \and
   %
   \inferrule*[
      lab={\ruleName{$\evalFwdS$-apply-prim}},
      right={$\primFwdBool{\phi}{\vec{n}}(\vec{\beta}) = \alpha'$}
   ]
   {
      \evalFwdEq{\rho}{e_i}{\alpha}{U_i}{\annInt{{n_i}}{\beta_i}}
      \quad
      (\forall i \numleq \length{\vec{n}})
   }
   {
      \evalFwd{\rho}
              {\exAppPrim{\phi}{\vec{e}}}
              {\alpha}
              {\trAppPrimNew{\phi}{U}{n}}
              {\annInt{\exAppPrim{\phi}{\vec{n}}}{\alpha'}}
   }
   %
   \and
   %
   \inferrule*[
      lab={\ruleName{$\evalFwdS$-apply}},
      width={3.4in}
   ]
   {
      \evalFwdEq{\rho}{e}{\alpha}{T}{\annClosure{\rho_1}{h}{\beta}{\sigma}}
      \\
      \rho_1, h, \beta \closeDefsFwdR \rho_2
      \\
      \evalFwd{\rho}{e'}{\alpha}{U}{v}
      \\
      \matchFwd{v}{\sigma}{w}{\rho_3}{e^\twoPrime}{\beta'}
      \\
      \evalFwd{\rho_1 \concat \rho_2 \concat \rho_3}{e^\twoPrime}{\beta \meet \beta'}{T'}{v'}
   }
   {
      \evalFwd{\rho}{\exApp{e}{e'}}{\alpha}{\trApp{T}{U}{w}{T'}}{v'}
   }
   %
   \and
   %
   \inferrule*[lab={
      \ruleName{$\evalFwdS$-let-rec}}
   ]
   {
      \rho, h', \alpha \closeDefsFwdR \rho'
      \\
      \evalFwd{\rho \concat \rho'}{e}{\alpha}{T}{v}
   }
   {
      \evalFwd{\rho}{\exLetRecMutual{h'}{e}}{\alpha}{\trLetRecMutual{h}{T}}{v}
   }
\end{smathpar}
\vspace{2mm}
{\small \flushleft \shadebox{$\rho, h, \alpha \closeDefsFwdR \rho'$}%
\hfill \textbfit{$h$ forward-generates to $\rho'$ in $\rho$ and $\alpha$}}
\begin{smathpar}
   \inferrule*[
      lab={\ruleName{$\closeDefsFwdR$-rec-defs}}
   ]
   {
      v_i = \annClosure{\rho}{\vec{\bind{x}{\sigma}}}{\alpha}{\sigma_i}
      \quad
      (\forall i \in \length{\vec{x}})
   }
   {
      \rho, \vec{\bind{x}{\sigma}}, \alpha
      \closeDefsFwdR
      \vec{\bind{x}{v}}
   }
\end{smathpar}
\caption{Forward data dependency (Boolean cases for $\evalFwdR{T}$ omitted)}
\label{fig:data-dependencies:fwd}
\end{figure}


\subsubsection{Forward matching}
\label{sec:data-dependencies:analyses:fwd:pattern-match}

\figref{data-dependencies:fwd} defines a family of \emph{forward-matching} functions $\matchFwdF{w}$ of type $\Sel{\raw{v}, \raw{\sigma}}{A} \to \Sel{\raw{\rho}, \raw{\kappa}}{A} \times \Ann{A}$ for any $w :: \raw{v}, \raw{\sigma} \matchR \raw{\rho}, \raw{\kappa}$. (The definition is presented in a relational style for readability, but should be understood as a total function defined by structural recursion on $w$.) Forward-match replays the match witnessed by $w$, transferring the selections on the relevant parts of $v \in \Sel{\raw{v}}{A}$ to the output environment $\rho \in \Sel{\raw{\rho}}{A}$, and from the relevant part of $\sigma \in \Sel{\raw{\sigma}}{A}$ to the chosen continuation $\kappa \in \Sel{\raw{\kappa}}{A}$.

$\matchFwdF{w}$ also returns the \emph{meet} of the selection states associated with the part of $v$ consumed by $\sigma$, which we call the \emph{availability} of $v$ (in the context of $\sigma$), since it represents the extent to which the demand implied by $\sigma$ was met. A variable match consumes nothing of $v$ and so the availability is always $\top$, the unit for meet. A Boolean match consumes either $\annot{\exTrue}{\alpha}$ or $\annot{\exFalse}{\alpha}$, with availability $\alpha$; empty list and record matches are similar. When we match a cons, we return the meet of the $\alpha$ on the cons node itself with the availabilities $\beta$ and $\beta'$ computed for $v$ and $v'$. Non-empty records are similar, but to process the initial part of the record, we supply the neutral selection state $\top$ on the subrecord in order to use the definition recursively. (Subrecords are not first-class, but exist only as intermediate artefacts of the interpreter.)

One might imagine dispensing with the need for $w$ by simply defining $\matchFwdF{w}$ by case analysis on $v$ and $\sigma$. However, it is then unclear how to proceed in the event that $v$ is a hole. It would be legitimate to produce $\hole$ as the output continuation and $\bot$ as the output selection state, but for the output environment to be well-typed, it must provide variable bindings corresponding to those introduced in the baseline computation where $\raw{v}$ was matched against $\raw{\sigma}$. If $\matchFwdS$ is defined with respect to a known $w$, this can be achieved via an additional rule \ruleName{$\matchFwdS$-hole-1} that defines the behaviour at hole to be the same as the behaviour at any $\eq$-equivalent value in $\Sel{\raw{v}}{A}$. The \ruleName{$\matchFwdS$-hole-2} makes a similar provision for $\sigma$, which may also be a hole. Operationally, these rules can be interpreted as ``expanding'' enough of the holes in $v$ or $\sigma$ to make another rule of the definition match. There will be exactly one non-hole rule that matches, corresponding to the execution path originally taken, and although there may be multiple such expansions, the following property implies that the result of $\matchFwdF{w}$ will be unique up to $\eq$.

\begin{lemma}[Monotonicity of $\matchFwdF{w}$]
   Suppose $w :: \raw{v}, \raw{\sigma} \matchR \raw{\rho}, \raw{\kappa}$, with $v, \sigma \matchFwdR{w} \rho, \kappa, \alpha$ and $v', \sigma' \matchFwdR{w} \rho', \kappa', \alpha'$. If $(v, \sigma) \leq (v', \sigma)$ then $(\rho, \kappa, \alpha) \leq (\rho', \kappa', \alpha')$.
\end{lemma}

\subsubsection{Forward evaluation}
\label{sec:data-dependencies:forward-eval}

\figref{data-dependencies:fwd} also defines a family of \emph{forward-evaluation} functions $\evalFwdF{T}$ of type $\Sel{\raw{\rho}, \raw{e}}{A} \times \Ann{A} \to \Sel{\raw{v}}{A}$ for any $T :: \raw{\rho}, \raw{e} \evalR \raw{v}$. Like forward-match, forward-evaluation is presented in a relational style, but should be understood as a total function defined by structural recursion on $T$. Forward evaluation replays $T$, using the input selection $\rho, e \in \Sel{\raw{\rho}, \raw{e}}{A}$ to determine an output selection $v \in \Sel{\raw{v}}{A}$. The rules mirror those of the evaluation relation $\evalR$, although there is an additional selection state input $\alpha$ called the \emph{ambient availability}. We explain this with reference to the application rule, which is the only point where a new ambient availability is assigned.

\paragraph{Function application} The rule assumes the application $\exApp{e}{e'}$ already has an ambient availability $\alpha$; at the outermost level this will usually be $\top$. The rule passes $\alpha$ down when recursively forward-evaluating $e$ and $e'$, but computes a new selection state $\alpha' \meet \beta$ when transferring control to the function, combining ambient availability $\alpha'$ captured by the closure and $\beta$ obtained by forward-matching the argument $v$ with the eliminator $\sigma$ of the closure, representing the availability of those parts of $v$ demanded by the function. The ambient availability is used to upper-bound the availability of any selectable values constructed in the dynamic context of that function, establishing a dependency between the resources consumed by functions and resources they produce. The auxiliary function $\closeDefsFwdF{\rho,h}: \Sel{\raw{\rho},\raw{h}}{A} \times \Ann{A} \to \Sel{\raw{\rho}'}{A}$ for any $\smash{\raw{\rho}, \raw{h} \closeDefsR \raw{\rho'}}$ is given at the bottom of \figref{data-dependencies:fwd} and follows $\closeDefsR$, but captures the ambient availability into each closure.

\paragraph{Primitive application} Primitive operations are the other source of input-output dependencies beyond user-defined functions. Since a primitive operation is opaque, these dependencies cannot be derived from its execution, but must be specified by the primitive operation directly. More specifically, $\phi \in \tyInt^{i} \to \tyInt$ is required to provide a forward-dependency function $\primFwdBool{\phi}{\vec{n}}: \Ann{A}^i \to \Ann{A}$ for every $\vec{n} \in \tyInt^i$ which specifies how to turn an input selection $\vec{\alpha} \in \Ann{A}^i$ for $\vec{n}$ into an output selection $\alpha'$ on $\exAppPrim{\phi}{\vec{n}}$. There is one such function per possible input $\vec{n}$, since the dynamic dependencies for a primitive operation with an annihilator, such as multiplication, depend on the values passed to the operation. Primitives are free to implement forward-dependency however they want, except that \secref{data-dependencies:analyses:bwd:eval} will also require $\phi$ to provide a backward-dependency function for any input $\vec{n}$, and \secref{data-dependencies:analysis:galois-connections} will require these to be related in a certain way for the consistency of the whole system to be guaranteed.

\paragraph{Other rules} All other rules pass the ambient availability into any subcomputations unchanged. Variable lookup disregards the ambient $\alpha$, simply preserving the selection on the returned value. The lambda rule captures it in the closure, along with the environment; the letrec rule passes it on to $\closeDefsFwdR$ so it is captured by recursive closures as well. Record projection is more interesting, disregarding not only the ambient $\alpha$ but also the availability $\beta$ of the record itself. Containers are considered to be independent of the values they contain: here, $v_i$ has its own internal availability which is preserved by projection, but there is no implied dependency of the field on the record from which it was projected. Record construction also reflects this principle, preserving the field selections into the resulting record selection unchanged. But it also sets the availability of the record value to be the meet of $\alpha$ and $\alpha'$, reflecting the dependency of the container on the constructing expression and also on any resources consumed by the ambient function. The rules for nil, cons and integers are similar.

\paragraph{Hole case} The hole rule for $\evalFwdF{T}$ \todo{needed because $\hole$ expression cannot just stop with output $\hole$, but must continue, possibly extracting non-$\hole$ values from $\rho$} is similar to the rules for $\matchFwdF{w}$, except that no special treatment is needed for the holes that arise inside $\rho$, and environments themselves have no special $\hole$ form. Moreover, the elimination rules (application and record projection) must accommodate the case where the selection on the closure or record being eliminated is represented by $\hole$. In these rules $\evalFwdR{T}\eq$ denotes the relational composition of $\evalFwdR{T}$ and $\eq$. As with $\matchFwdF{w}$, these uses of $\eq$ can be understood operationally as using the information in $T$ to expand $\hole$ sufficiently for the rule to apply, with a result which is unique up to $\eq$, which is again implied by the following property.

\begin{lemma}[Monotonicity of $\evalFwdF{T}$]
   Suppose $T :: \raw{\rho}, \raw{e} \evalR \raw{v}$ with $\rho, e, \alpha \evalFwdR{T} v$ and $\rho', e', \alpha' \evalFwdR{T} v'$. If $(\rho, e, \alpha) \leq (\rho', e', \alpha')$ then $v \leq v'$.
\end{lemma}

\subsection{Backward data dependency}
\label{sec:data-dependencies:analyses:bwd}

Backward analysis ``rewinds'' evaluation, turning output demand into input demand, with $T$ guiding the analysis backward.

\subsubsection{Backward matching}
\label{sec:data-dependencies:analyses:bwd:pattern-match}

\figref{data-dependencies:bwd} defines a family of \emph{backward-match} functions $\matchBwdF{w}$ of type $\Sel{\raw{\rho}, \raw{\kappa}}{A} \times \Ann{A} \to \Sel{\raw{v}, \raw{\sigma}}{A}$ for any $w :: \raw{v}, \raw{\sigma} \matchR \raw{\rho}, \raw{\kappa}$. Backward-match rewinds the match witnessed by $w$, turning demand on the environment and continuation into demand on the value and eliminator that were originally matched. The additional input $\alpha$ represents the downstream demand placed on any resources that were constructed in the context of this match; $\matchBwdF{w}$ transfers this to the matched portion of $\raw{v}$, establishing a backwards link between resources produced and resources consumed in a given function context.

\begin{figure}
   {\small \flushleft \shadebox{$\evalBwd{v}{T}{\rho}{e}{\alpha}$}%
   \hfill \textbfit{$v$ backward-evaluates along $T$ to $\rho$ and $e$, with argument demand $\alpha$}}
   \begin{smathpar}
      \inferrule*[
         lab={\ruleName{$\evalBwdS$-hole}}
      ]
      {
         \hole \eq v
         \\
         \evalBwd{v}{T}{\rho}{e}{\alpha}
      }
      {
         \evalBwd{\hole}{T}{\rho}{e}{\alpha}
      }
      %
      \and
      %
      \inferrule*[
         lab={\ruleName{$\evalBwdS$-var}}
      ]
      {
         \envLookupBwd{\rho'}{\rho}{\bind{x}{v}}
      }
      {
         \evalBwd{v}{\trVar{x}{\rho}}{\rho'}{\exVar{x}}{\bot}
      }
      %
      \and
      %
      \inferrule*[
         lab={\ruleName{$\evalBwdS$-lambda}}
      ]
      {
         \strut
      }
      {
         \evalBwd{\annClosure{\rho}{\seqEmpty}{\alpha}{\sigma}}
                 {\trLambda{\sigma'}}
                 {\rho}
                 {\exLambda{\sigma}}
                 {\alpha}
      }
      %
      \and
      %
      \inferrule*[
         lab={\ruleName{$\evalBwdS$-int}}
      ]
      {
         \strut
      }
      {
         \evalBwd{\annInt{n}{\alpha}}
                 {\trInt{n}{\rho}}
                 {\hole_{\raw{\rho}}}
                 {\annInt{n}{\alpha}}
                 {\alpha}
      }
      %
      \and
      %
      \inferrule*[
         lab={\ruleName{$\evalBwdS$-true}}
      ]
      {
         \strut
      }
      {
         \evalBwd{\annTrue{\alpha}}
                 {\trTrue{\rho}}
                 {\hole_{\raw{\rho}}}
                 {\annTrue{\alpha}}
                 {\alpha}
      }
      %
      \and
      %
      \inferrule*[
         lab={\ruleName{$\evalBwdS$-false}}
      ]
      {
         \strut
      }
      {
         \evalBwd{\annFalse{\alpha}}
                 {\trFalse{\rho}}
                 {\hole_{\raw{\rho}}}
                 {\annFalse{\alpha}}
                 {\alpha}
      }
      %
      \and
      %
      \inferrule*[lab={\ruleName{$\evalBwdS$-record}}]
      {
         \evalBwd{v_i}{T_i}{\rho_i}{e_i}{\alpha_i'}
         \quad
         (\forall i \numleq \length{\vec{x}})
      }
      {
         \evalBwd{\annRec{\vec{\bind{x}{v}}}{\alpha}}
                 {\trRec{\vec{\bind{x}{T}}}}
                 {\bigjoin\vec{\rho}}
                 {\annRec{\vec{\bind{x}{e}}}{\alpha}}
                 {\alpha \join \bigjoin\vec{\alpha}'}
      }
      %
      \and
      %
      \inferrule*[lab={\ruleName{$\evalBwdS$-project}}]
      {
         \envLookupBwd{\vec{\bind{x}{u}}}{\vec{\bind{x}{v}}}{\bind{y}{v'}}
         \\
         \evalBwd{\annRec{\vec{\bind{x}{u}}}{\bot}}
                 {T}
                 {\rho}
                 {e}
                 {\alpha}
      }
      {
         \evalBwd{v'}
                 {\trRecProj{T}{\vec{\bind{x}{v}}}{y}}
                 {\rho}
                 {\exRecProj{e}{y}}
                 {\alpha}
      }
      %
      \and
      %
      \inferrule*[
         lab={\ruleName{$\evalBwdS$-nil}}
      ]
      {
         \strut
      }
      {
         \evalBwd{\annNil{\alpha}}
                 {\trNil{\rho}}
                 {\hole_{\raw{\rho}}}
                 {\annNil{\alpha}}
                 {\alpha}
      }
      %
      \and
      %
      \inferrule*[
         lab={\ruleName{$\evalBwdS$-cons}}
      ]
      {
         \evalBwd{v}{T}{\rho}{e}{\alpha}
         \\
         \evalBwd{v'}{U}{\rho'}{e'}{\alpha'}
      }
      {
         \evalBwd{\annCons{v}{v'}{\beta}}
                 {\trCons{T}{U}}
                 {\rho \join \rho'}
                 {\annCons{e}{e'}{\beta}}
                 {\beta \join \alpha \join \alpha'}
      }
      %
      \and
      %
      \inferrule*[
         lab={\ruleName{$\evalBwdS$-let-rec}}
      ]
      {
         \evalBwd{v}{T}{\rho \concat \rho_1}{e}{\alpha}
         \\
         \rho_1 \closeDefsBwdR \rho', h', \alpha'
      }
      {
         \evalBwd{v}{\trLetRecMutual{h}{T}}{\rho \join \rho'}{\exLetRecMutual{h'}{e}}{\alpha \join \alpha'}
      }
      %
      \and
      %
      \inferrule*[
         lab={\ruleName{$\evalBwdS$-apply-prim}},
         right={$\primBwdBool{\phi}{\vec{n}}(\alpha') = \vec{\alpha}$}
      ]
      {
         \evalBwd{\annInt{{n_i}}{\alpha_i}}{U_i}{\rho_i}{e_i}{\beta_i}
         \quad
         (\forall i \in \length{\vec{n}})
      }
      {
         \evalBwd{\annInt{m}{\alpha'}}
                 {\trAppPrimNew{\phi}{U}{n}}
                 {\bigjoin\vec{\rho}}
                 {\exAppPrim{\phi}{\vec{e}}}
                 {\bigjoin\vec{\beta}}
      }
      %
      \and
      %
      \inferrule*[
         lab={\ruleName{$\evalBwdS$-apply}},
         width={3.3in}
      ]
      {
         \evalBwd{v}{T'}{\rho_1 \concat \rho_2 \concat \rho_3}{e}{\beta}
         \\
         \matchBwd{\rho_3}{e}{\beta}{w}{v'}{\sigma}
         \\
         \evalBwd{v'}{U}{\rho}{e_2}{\alpha}
         \\
         \rho_2 \closeDefsBwdR \rho_1', h, \beta'
         \\
         \evalBwd{\annClosure{\rho_1 \join \rho_1'}{h}{\beta \join \beta'}{\sigma}}{T}{\rho'}{e_1}{\alpha'}
      }
      {
         \evalBwd{v}{\trApp{T}{U}{w}{T'}}{\rho \join \rho'}{\exApp{e_1}{e_2}}{\alpha \join \alpha'}
      }
   \end{smathpar}
   \vspace{1mm}

\begin{minipage}[t]{0.48\textwidth}%
   {\small\flushleft \shadebox{$\envLookupBwd{\rho'}{\rho}{\bind{x}{v}}$}%
   \hfill \textbfit{$\rho'$ contains $\bind{x}{v}$}}
   \begin{smathpar}
      \inferrule*[
         lab={\ruleName{$\envLookupBwdS$-head}}
      ]
      {
         \strut
      }
      {
         \envLookupBwd{(\hole_{\raw{\rho}} \concat \bind{x}{u})}
                      {\rho \concat \bind{x}{v}}
                      {\bind{x}{u}}
      }
      %
      \and
      %
      \inferrule*[
         lab={\ruleName{$\envLookupBwdS$-tail}},
      ]
      {
         \envLookupBwd{\rho'}{\rho}{\bind{x}{u}}
         \\
         x \neq y
      }
      {
         \envLookupBwd{(\rho' \concat \bind{y}{\hole})}
                      {\rho \concat \bind{y}{v}}
                      {\bind{x}{u}}
      }
   \end{smathpar}
\end{minipage}%
\hfill
\begin{minipage}[t]{0.47\textwidth}%
   {\small\flushleft\shadebox{$\rho \closeDefsBwdR \rho', h, \alpha$}%
   \hfill \textbfit{$\rho$ backward-generates to $\rho'$, $h$, $\alpha$}}
   \begin{smathpar}
      \inferrule*[
         lab={\ruleName{$\closeDefsBwdR$-rec-defs}}
      ]
      {
         v_i = \annClosure{\rho_i}{h_i}{\alpha_i}{\sigma_i}
         \quad
         (\forall i \in \length{\vec{x}})
      }
      {
         \vec{\bind{x}{v}}
         \closeDefsBwdR
         \bigjoin\vec{\rho}, \vec{\bind{x}{\sigma}} \join {\bigjoin\vec{h}}, \bigjoin{\vec{\alpha}}
      }
   \end{smathpar}
\end{minipage}

\caption{\vspace{-2mm}Backward evaluation}
\label{fig:data-dependencies:bwd}
\end{figure}


In the variable case, no proper part of $\raw{v}$ was matched, so $\alpha$ is disregarded. The rule need only ensure that the demand $v$ in the singleton environment $\bind{x}{v}$ is propagated backward. If a Boolean constant was matched, $\alpha$ becomes the demand on that constant, and $\kappa$, capturing the demand on the continuation, is used to construct the demand on the original eliminator, with $\hole$ used to represent the absence of demand on the non-taken branch. (Using $\hole$ for this means matches $w$ need only retain information about taken branches.) The nil case is similar.

For a cons match $\matchCons{w}{w'}$, we split the environment into $\rho$ and $\rho'$ (there is a unique well-typed decomposition) and then backward-match $w$ and $w'$ recursively to obtain $v$ and $v'$, representing the demand on the head and tail of the list. These are combined into the demand on the entire list, using $\alpha$ as the demand on the cons node. $\sigma$ represents the demand on the interim eliminator used to match the tail, and $\tau$ the demand on the eliminator used to match the head, which are then combined into a demand on the eliminator used to match the whole list, with $\hole$ again used to represent the absence of demand on the nil branch. Records are similar, except there is only a single branch, and the selection state $\beta$ computed for the initial part of the record is an artefact of processing records recursively, and is disregarded.

\begin{lemma}[Monotonicity of $\matchBwdF{w}$]
   Suppose $w :: \raw{v}, \raw{\sigma} \matchR \raw{\rho}, \raw{\kappa}$, with $\rho, \kappa, \alpha \matchBwdR{w} v, \sigma$ and $\rho', \kappa', \alpha' \matchBwdR{w} v', \sigma'$. If $(\rho, \kappa, \alpha) \leq (\rho', \kappa', \alpha')$ then $(v, \sigma) \leq (v', \sigma)$.
\end{lemma}

\subsubsection{Backward evaluation}
\label{sec:data-dependencies:analyses:bwd:eval}

\figref{data-dependencies:bwd} also defines a family of \emph{backward-evaluation} functions $\evalBwdF{T}$ of type $\Sel{\raw{v}}{A} \to \Sel{\raw{\rho}, \raw{e}}{A} \times \Ann{A}$ for any $T :: \raw{\rho}, \raw{e} \evalR \raw{v}$. Backward evaluation rewinds $T$, using the output selection $v \in \Sel{\raw{v}}{A}$ to determine an input selection $\rho, e \in \Sel{\raw{\rho}, \raw{e}}{A}$. The rules resemble those of the evaluation relation $\evalR$ with inputs and outputs flipped, and with an additional output $\alpha$ called the \emph{ambient demand}. The general pattern is that each backward rule takes the join of the demand attached to any partial values constructed at that step, and the ambient demand associated with any subcomputations, and passes it upwards as the new ambient demand. The output environment is constructed similarly, by joining the demand flowing back through the environment copies used to evaluate subcomputations. Demand is also attached to the source expression when it is the expression form responsible for the construction of a demanded value.

\paragraph{Function application} The application rule is where the ambient demand is used and the function context changes, so we start here. The rule essentially runs the forward evaluation rule in reverse, using the trace $T'$ to backward-evaluate the function body. The ambient demand $\beta$ associated with $T'$ is the join of the demand on any resources constructed directly by that function invocation, and is transferred to the matched part of the function argument by the backward-match function $\matchBwdF{w}$. The ambient demand passed upwards into the enclosing function context is $\alpha \join \alpha'$, representing the resources needed along $T$ and $U$. The family of auxiliary functions $\smash{\closeDefsBwdF{\rho, h}}$ of type $\smash{\Sel{\raw{\rho'}}{A} \to \Sel{\raw{\rho}, \raw{h}}{A}}$ for any $\raw{\rho}, \raw{h} \closeDefsR \raw{\rho'}$ defined at the bottom of \figref{data-dependencies:bwd} is used to turn $\rho_2$, capturing the demand flowing back through any recursive uses of the function and any others with which it was mutually defined, into information that can be merged back into the demand on the closure. The function $\closeDefsBwdF{\rho,h}$ is also used in the letrec rule, which otherwise follows the generic pattern describe above.

\paragraph{Primitive operations} Each primitive operation $\phi: \tyInt^{i} \to \tyInt$ must provide a backward-dependency  function $\primBwdBool{\phi}{\vec{n}}: \Ann{A} \to \Ann{A}^i$ for every $\vec{n} \in \tyInt^i$ which specifies how to turn the output selection $\alpha'$ on $\exAppPrim{\phi}{\vec{n}}$ into an input selection $\vec{a} \in \Ann{A}^i$ on $\vec{n}$. The rule for primitive application uses this information to pair each argument $n_i$ with its demand $\alpha_i$ and then backwards-evaluate the argument. The ambient demand passed upward is the join of those arising from these subcomputations, and is unrelated to the execution of the primitive itself, similar to a function application. Here $\bigjoin{\vec{\beta}}$ means the fold of $\join$ (with unit $\bot$) over the sequence of selection states $\seqRange{\beta_1}{\beta_{\length{\vec{x}}}'}$. Environment demands $\vec{\rho} = \seqRange{\rho_1}{\rho_{\length{\vec{n}}}}$ are joined (pointwise) in a similar fashion.

\paragraph{Other rules} In the variable case, no partial values were constructed during evaluation and there are no subcomputations, so the ambient demand is $\bot$, the unit for $\join$. The returned environment selection demands $v$ for the variable $x$ and $\hole$ for all other variables, using the family of \emph{backwards lookup} functions $\envLookupBwdF{-}{\rho}{x}{-}$ of type $\Sel{\raw{v}}{A} \to \Sel{\raw{\rho}}{A}$ for any $\envLookup{\raw{\rho}}{x}{\raw{v}}$ also defined in \figref{data-dependencies:bwd}. (The output of the function is on the left in the relational notation.) For atomic values such as integers and nil, the ambient demand is simply the demand $\alpha$ associated with the constructed value, which also attached to the corresponding expression, and the environment demand has $\hole$ for every variable in the original environment $\raw{\rho}$, written $\hole_{\raw{\rho}}$.

Closures are not directly selectable, so the ambient demand is $\bot$, and the rule simply unpacks the components, preserving any internal selections on $\rho$ and $\sigma$. Composite values such as records and cons cells follow the general pattern; thus for records, the ambient demands $\alpha'_i$ for the subcomputations are joined with the $\alpha$ on the record itself to produce the ambient demand passed upward. Record projection never demands the record constructor itself, but simply promotes the field demand into a record demand, using using $\envLookupBwdR{\vec{\bind{x}{\raw{v}}}}$ to demand any other fields with $\hole$.

\paragraph{Hole rule} The hole rule, as elsewhere, ensures that the function is defined when $v$ is $\hole$, and it is easy to show that $\evalBwdF{T}$  preserves $\leq$, and thus $\eq$.

\begin{lemma}[Monotonicity of $\evalBwdF{T}$]
   Suppose $T :: \raw{\rho}, \raw{e} \evalR \raw{v}$ with $v \evalBwdR{T} \rho, e, \alpha $ and $v' \evalBwdR{T} \rho', e', \alpha' $. If $v \leq v'$ then $(\rho, e, \alpha) \leq (\rho', e', \alpha')$.
\end{lemma}

\subsection{Round-tripping properties of $\evalFwdF{T}$ and $\evalBwdF{T}$}
\label{sec:data-dependencies:analysis:galois-connections}

The analyses above can be used to answer the kind of question we motivated in \secref{introduction:data-linking}. In particular $\evalBwdF{T}$ can be used to answer questions like: ``what data is needed to compute this bar in a bar chart?'' (In fact we used our implementation to generate \figref{introduction:data-linking}.) To rely on the answer, we need the analyses to satisfy certain correctness properties. Intuitively, we expect $\evalBwdF{T}$ to be ``sound'': to pick out all (and hopefully only) the input data needed for the output selection. Similarly, we expect the forward analysis $\evalFwdF{T}$ to be ``complete'': to pick out only (and hopefully all) output data that can be computed from the selected input.

We can understand these requirements as \emph{round-tripping} properties that relate $\evalFwdF{T}$ to $\evalBwdF{T}$: we expect $\evalFwdF{T}(\evalBwdF{T}(v))$ to produce a value selection $v' \geq v$, and $\evalBwdF{T}(\evalFwdF{T}(\rho,e))$ to produce an input selection $(\rho',e') \leq (\rho,e)$. (We will present a visual example of this in \secref{de-morgan:example}.) Pairs of (monotonic) functions $f: X \to Y$ and $g: Y \to X$ that are related in this way are called \emph{Galois connections}. Galois connections generalise isomorphisms: $f$ and $g$ are not quite mutual inverses, but are the nearest to an inverse each can get to the other.

\begin{definition}[Galois connection]
   Suppose $X$ and $Y$ are sets equipped with partial orders $\numleq_X$ and $\numleq_Y$. Then monotonic functions $f: X \to Y$ and $g: Y \to X$ form a \emph{Galois connection} $(g, f): Y \to X$ iff $f(g(y)) \numgeq_Y y$ and $g(f(x)) \numleq_X x$.
\end{definition}

\noindent Galois connections are also adjoint functors between poset categories, with right and left adjoints $f$ and $g$  usually called the \emph{upper} and \emph{lower} adjoints, because $f$ approximates an inverse of $g$ from above, and $g$ an inverse of $f$ from below. Galois connections compose component-wise, so it is useful to think of them as morphisms, by convention with domain and codomain given by the lower adjoint. If $f: X \to Y$ is a Galois connection, we will write $\lowerAdj{f}$ and $\upperAdj{f}$ for the lower and upper adjoints respectively; an important property we will return to is that $\lowerAdj{f}$ preserves joins and $\upperAdj{f}$ preserves meets. We now show that $\evalBwdF{T}$ and $\evalFwdF{T}$ form a Galois connection $\evalGC{T}$ for any $\Ann{A}$ (\thmref{core-language:eval:gc}) by first establishing that the relevant auxiliary functions also form Galois connections.

\begin{theorem}[Galois connection for pattern-matching]
   \label{thm:core-language:match:gc}
      Suppose $w :: \raw{v}, \raw{\sigma} \matchR \raw{\rho}, \raw{\kappa}$.  Then $(\matchBwdF{w}, \matchFwdF{w}): \Sel{\raw{\rho},\raw{\kappa}}{A} \to \Sel{\raw{v}, \raw{\sigma}}{A}$ is a Galois connection.
\end{theorem}

\begin{proof}
\ifappendices See \appref{proofs:match:gc}. \else \ProofInSupplementaryMaterial \fi
\end{proof}

\begin{lemma}[Galois connection for environment lookup]
\label{lem:core-language:env-get-put}Suppose $\envLookup{\raw{\rho}}{x}{\raw{v}}$. Then $(\envLookupBwdF{-}{\raw{\rho}}{x}{},\envLookupFwdF{}{\raw{\rho}}{x}{-}): \Sel{\raw{v}}{A} \to \Sel{\raw{\rho}}{A}$ is a Galois connection. \todo{better notation here}
\end{lemma}

\begin{proof}
   \ifappendices See \appref{proofs:lookup:gc}. \else \ProofInSupplementaryMaterial \fi
   \end{proof}

   \begin{theorem}[Galois connection for recursive bindings]
\label{thm:core-language:closeDefs:gc}
   Suppose $\raw{\rho}, \raw{h} \closeDefsR \raw{\rho}'$. Then $({\closeDefsBwdF{\raw{\rho},\raw{h}}, \closeDefsFwdF{\raw{\rho},\raw{h}}}): \Sel{\raw{\rho}'}{A} \to \Sel{\raw{\rho},\raw{h}}{A}$ is a Galois connection.
\end{theorem}

\begin{proof}
   \ifappendices See \appref{proofs:closeDefs:gc}. \else \ProofInSupplementaryMaterial \fi
   \end{proof}

We assume (rather than prove) that the forward and backwards dependency functions $\primGCBool{\phi}{\vec{n}} = (\primFwdBool{\phi}{\vec{n}}, \primBwdBool{\phi}{\vec{n}})$ provided for every primitive operation $\phi: \tyInt^i \to \tyInt$ form a Galois connection of type $\Ann{A} \to \Ann{A}^i$. Under this assumption the following holds.

\begin{theorem}[Galois connection for evaluation]
\label{thm:core-language:eval:gc}
   Suppose $T :: \raw{\rho}, \raw{e} \evalR \raw{v}$.  Then $(\evalBwdF{T}, \evalFwdF{T}): \Sel{\raw{\rho}, \raw{e}}{A} \to \Sel{\raw{v}}{A}$ is a Galois connection.
\end{theorem}

\begin{proof}
   \ifappendices See \appref{proofs:eval:gc}. \else \ProofInSupplementaryMaterial \fi
\end{proof}

Establishing that $(\evalBwdF{T}, \evalFwdF{T})$ is an adjoint pair might seem rather weak as a correctess property: it merely ensures that the two analyses are related in a sensible way, not that they actually capture any useful information. This is a familiar problem from other approximate analyses like type systems and model checking, where properties like soundness or completeness are essential but do not by themselves guarantee utility. One could certainly define versions of $\evalBwdF{T}$ and $\evalFwdF{T}$ that are too coarse grained to be useful yet still satisfy \thmref{core-language:eval:gc}. However Galois connections do at least require that every tightening or tweak to the forward analysis is paired with a corresponding adjustment to the backward analysis, and vice-versa. In \secref{conclusion} we consider how other ideas from provenance and program slicing might be adapted to provide additional correctness criteria.

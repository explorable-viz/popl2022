\section{A bidirectional dynamic dependency analysis}
\label{sec:data-dependencies}

We now extend the core language described in \secref{core-language} with a bidirectional mechanism for tracking data dependencies. The first thing to establish is a way of selecting parts of the program of interest. Consider the \kw{data} field of the bar chart computed in \figref{introduction:data-linking}, which contains a list of records; the record corresponding to China is $\exRec{\bind{\kw{x}}{\kw{"China"}} \concat \bind{\kw{y}}{\kw{295.3}}}$. In the example, we selected the value of the field $\kw{y}$ in order to see the corresponding selection on the input.

Our approach to representing selections is shown in \figref{core-syntax-selection}. The basic idea is to parameterise the type $\Set{Val}$ of values by a (bounded) lattice $\Ann{A}$ of \emph{selection states} $\alpha$. We add selection states to Booleans, integers, records and lists; for our present purposes, we are only interested in dependencies between first-order data, so closures are not (directly) selectable, although they have selectable parts. We parameterise the type $\Set{Term}$ of terms in the same way, but only add selection states to the term constructors corresponding to selectable values, allowing us to trace data dependencies back to expressions that appear in the source code. We return to this in \secref{surface-language}.

\begin{figure}[H]
{\small
\begingroup
\renewcommand*{\arraystretch}{1}
\begin{minipage}[t]{0.5\textwidth}
   \begin{tabularx}{\textwidth}{rL{2.5cm}L{3cm}}
   &\textbfit{Selectable term}&
   \\
   $e \in \TermF{\Ann{A}} ::=$
   &
   \ldots
   \\
   &
   $\hole$
   &
   hole
   \\
   &
   $\annTrue{\alpha} \mid \annFalse{\alpha}$
   &
   Boolean
   \\
   &
   $\annInt{n}{\alpha}$
   &
   integer
   \\
   &
   $\annRec{\vec{\bind{x}{e}}}{\alpha}$
   &
   record
   \\
   &
   $\annNil{\alpha} \mid \annCons{e}{e'}{\alpha}$
   &
   list
   \\[2mm]
   $\alpha, \beta \in \Ann{A}$
   &&
   \text{selection state}
   \end{tabularx}
\end{minipage}%
\begin{minipage}[t]{0.5\textwidth}
   \begin{tabularx}{\textwidth}{rL{2.5cm}L{3cm}}
   &\textbfit{Selectable value}&
   \\
   $u, v \in \ValF{\Ann{A}}::=$
   &
   $\hole$
   &
   hole
   \\
   &
   $\annTrue{\alpha} \mid \annFalse{\alpha}$
   &
   Boolean
   \\
   &
   $\annInt{n}{\alpha}$
   &
   integer
   \\
   &
   $\annRec{\vec{\bind{x}{v}}}{\alpha}$
   &
   record
   \\
   &
   $\annNil{\alpha} \mid \annCons{u}{v}{\alpha}$
   &
   list
   \\
   &
   $\annClosure{\rho}{h}{\alpha}{\sigma}$
   &
   closure
   \\
   \\[2mm]
   \end{tabularx}
\end{minipage}
\endgroup
}
\caption{Selection states, selectable terms and selectable values}
\label{fig:core-syntax-selection}
\end{figure}


The fact that $\Ann{A}$ is a bounded lattice means it must have top and bottom elements $\top$ and $\bot$ representing fully selected/unselected, plus meet and join operations $\meet$ and $\join$ for combining selection information, with $\top$ and $\bot$ as their respective units. The Boolean lattice $\Bool \eqdef \Lattice{\Bool}{\TT}{\FF}{\wedge}{\vee}$ can represent the selection above as $\annRec{\bind{\kw{x}}{\annStr{\kw{"China"}}{\FF}} \concat \bind{\kw{y}}{\annInt{\kw{295.3}}{\TT}}}{\FF}$, indicating that the number $\exInt{\kw{295.3}}$ is selected, but neither the string $\exStr{\kw{"China"}}$, nor the record itself is selected. More interesting lattices such vectors of Booleans can be used to represent multiple selections simultaneously.

\begin{figure}
   \flushleft \shadebox{$v \leq v'$}
\begin{smathpar}
   \inferrule*[
   ]
   {
      \strut
   }
   {
      \hole \leq v
   }
   %
   \and
   %
   \inferrule*[
   ]
   {
      \alpha \leq \alpha'
   }
   {
      \annInt{n}{\alpha} \leq \annInt{n}{\alpha'}
   }
   %
   \and
   %
   \inferrule*[
   ]
   {
      \strut
   }
   {
      \annInt{n}{\bot} \leq \hole
   }
   %
   \and
   %
   \inferrule*[
   ]
   {
      \alpha \leq \alpha'
   }
   {
      \annTrue{\alpha} \leq \annTrue{\alpha'}
   }
   %
   \and
   %
   \inferrule*[
   ]
   {
      \strut
   }
   {
      \annTrue{\bot} \leq \hole
   }
   %
   \and
   %
   \inferrule*[
   ]
   {
      \alpha \leq \alpha'
   }
   {
      \annFalse{\alpha} \leq \annFalse{\alpha'}
   }
   %
   \and
   %
   \inferrule*[
   ]
   {
      \strut
   }
   {
      \annFalse{\bot} \leq \hole
   }
   %
   \and
   %
   \inferrule*[
   ]
   {
      \alpha \leq \alpha'
      \quad
      v_i \leq u_i
      \quad
      (\forall i \in \length{\vec{x}})
   }
   {
      \annRec{\vec{\bind{x}{v}}}{\alpha} \leq \annRec{\vec{\bind{x}{u}}}{\alpha'}
   }
   %
   \and
   %
   \inferrule*[
   ]
   {
      v_i \leq \hole
      \quad
      (\forall i \in \length{\vec{x}})
   }
   {
      \annRec{\vec{\bind{x}{v}}}{\bot} \leq \hole
   }
   %
   \and
   %
   \inferrule*[
   ]
   {
      \alpha \leq \alpha'
   }
   {
      \annNil{\alpha} \leq \annNil{\alpha'}
   }
   %
   \and
   %
   \inferrule*[
   ]
   {
      \strut
   }
   {
      \annNil{\bot} \leq \hole
   }
   %
   \and
   %
   \inferrule*[
   ]
   {
      (\alpha, v, v') \leq (\alpha', v, v')
   }
   {
      \annCons{v}{v'}{\alpha} \leq \annCons{u}{u'}{\alpha'}
   }
   %
   \and
   %
   \inferrule*[
   ]
   {
      (v, v') \leq (\hole, \hole)
   }
   {
      \annCons{v}{v'}{\bot} \leq \hole
   }
   %
   \and
   %
   \inferrule*[
   ]
   {
      (\rho, h, \alpha, \sigma) \leq (\rho', h', \alpha', \sigma')
   }
   {
      \annClosure{\rho}{h}{\alpha}{\sigma} \leq \annClosure{\rho'}{h'}{\alpha'}{\sigma'}
   }
   %
   \and
   %
   \inferrule*[
   ]
   {
      (\rho, h, \sigma) \leq (\sub{\hole}{\rho}, \hole, \hole)
   }
   {
      \annClosure{\rho}{h}{\bot}{\sigma} \leq \hole
   }
\end{smathpar}

   \caption{Partial order on values}
   \label{fig:data-dependencies:leq}
\end{figure}

\subsection{Lattices of selections}
\label{sec:data-dependencies:lattices-of-selections}

Our approach to representing selections is shown in \figref{core-syntax-selection}. The basic idea is to parameterise the type $\Set{Val}$ of values by a (bounded) lattice $\Ann{A}$ of \emph{selection states} $\alpha$. We add selection states to Booleans, integers, records and lists; while it would present no complications to equip closures with selection states too, for present purposes we are only interested in dependencies between first-order data, so closures are not (directly) selectable. Closures do however have selectable parts, and moreover capture the current \emph{argument availability}, explained in \secref{data-dependencies:forward-eval} below, which is also a selection state $\alpha$. We parameterise the type $\Set{Term}$ of terms similarly, allowing us to trace data dependencies back to expressions that appear in the source code, but only add selection states to the term constructors corresponding to selectable values. We return to this in \secref{surface-language}.

\begin{figure}[H]
{\small
\begingroup
\renewcommand*{\arraystretch}{1}
\begin{minipage}[t]{0.5\textwidth}
   \begin{tabularx}{\textwidth}{rL{2.5cm}L{3cm}}
   &\textbfit{Selectable term}&
   \\
   $e \in \TermF{\Ann{A}} ::=$
   &
   \ldots
   \\
   &
   $\hole$
   &
   hole
   \\
   &
   $\annTrue{\alpha} \mid \annFalse{\alpha}$
   &
   Boolean
   \\
   &
   $\annInt{n}{\alpha}$
   &
   integer
   \\
   &
   $\annRec{\vec{\bind{x}{e}}}{\alpha}$
   &
   record
   \\
   &
   $\annNil{\alpha} \mid \annCons{e}{e'}{\alpha}$
   &
   list
   \\[2mm]
   $\alpha, \beta \in \Ann{A}$
   &&
   \text{selection state}
   \end{tabularx}
\end{minipage}%
\begin{minipage}[t]{0.5\textwidth}
   \begin{tabularx}{\textwidth}{rL{2.5cm}L{3cm}}
   &\textbfit{Selectable value}&
   \\
   $u, v \in \ValF{\Ann{A}}::=$
   &
   $\hole$
   &
   hole
   \\
   &
   $\annTrue{\alpha} \mid \annFalse{\alpha}$
   &
   Boolean
   \\
   &
   $\annInt{n}{\alpha}$
   &
   integer
   \\
   &
   $\annRec{\vec{\bind{x}{v}}}{\alpha}$
   &
   record
   \\
   &
   $\annNil{\alpha} \mid \annCons{u}{v}{\alpha}$
   &
   list
   \\
   &
   $\annClosure{\rho}{h}{\alpha}{\sigma}$
   &
   closure
   \\
   \\[2mm]
   \end{tabularx}
\end{minipage}
\endgroup
}
\caption{Selection states, selectable terms and selectable values}
\label{fig:core-syntax-selection}
\end{figure}


The top and bottom elements $\top$ and $\bot$ of $\Ann{A}$ represent fully selected and fully unselected; the meet and join operations $\meet$ and $\join$, which have $\top$ and $\bot$ as their respective units, are used to combine selection information. In \figref{introduction:data-linking}, the \kw{data} field of \kw{BarChart} expects a list of records with fields \kw{x} and \kw{y}, mapping strings representing categorical data to floats determining the height of the corresponding bar; the record computed for China is $\exRec{\bind{\kw{x}}{\kw{"China"}} \concat \bind{\kw{y}}{\kw{295.3}}}$. The two-point lattice $\Bool \eqdef \Lattice{\set{\TT,\FF}}{\TT}{\FF}{\wedge}{\vee}$ can be used to represent the selection of the field $\kw{y}$ within this record as $\annRec{\bind{\kw{x}}{\annStr{\kw{"China"}}{\FF}} \concat \bind{\kw{y}}{\annInt{\kw{295.3}}{\TT}}}{\FF}$, indicating that the number $\exInt{\kw{295.3}}$ is selected, but that neither the string $\exStr{\kw{"China"}}$, nor the record itself, is selected. Because lattices are closed under component-wise products, we sometimes write $(\alpha, \beta) \leq (\alpha', \beta')$ to mean that $\alpha \leq \alpha'$ and $\beta \leq \beta'$. This also suggests more interesting lattices of selections, such as vectors of Booleans to represent multiple selections simultaneously, which might be visualised using different colours (as in \figref{introduction:data-linking}).

\subsubsection{Selections of a value}
\label{sec:data-dependencies:selections}

The analyses which follow will be defined with respect to a fixed computation, and so we will often need to talk about the selections of a given value. To make this notion precise, consider that the raw (selection-free) syntax described in \secref{core-language} can be recovered from a term selection via an erasure operation $\erase{\param}: \ValF{\Ann{A}} \to \ValF{\Unit}$ which forgets the selection information, where $\Unit$ is the trivial one-point lattice. We refer to $\erase{v}$ as the \emph{shape} of $v$. Allowing $\raw{u}, \raw{v}$ from now on to range over raw values, and reserving $u, v$ for value selections, we can then define:

\begin{definition}[Selections of $\raw{v}$]
   Define $\Sel{\raw{v}}{A}$ to be the set of all values $v \in \ValF{\Ann{A}}$ with shape $\raw{v}$, i.e.
   ~that erase to $\raw{v}$.
\end{definition}

Since its elements have a fixed shape, the pointwise comparison of any $v, v' \in \Sel{\raw{v}}{A}$ using the partial order $\leq$ of $\Ann{A}$ is well defined, as is the pointwise application (zip) of a binary operation~\cite{gibbons17}. It should therefore be clear that if $\Ann{A}$ is a lattice, then $\Sel{\raw{v}}{A}$ is also a lattice, with $\top_{\raw{v}}$, $\bot_{\raw{v}}$, $\meet_{\raw{v}}$, and $\join_{\raw{v}}$ defined pointwise. For example, if $u$ and $u'$ have the same shape and $v$ and $v'$ have the same shape, the join of the lists $(\annCons{u}{v}{\alpha})$ and $(\annCons{u'}{v'}{\alpha'})$ is defined and equal to $\annCons{(u \join u')}{(v \join v')}{\alpha \join \alpha'}$. Similarly, the top element of $\Sel{\raw{v}}{A}$ is the selection of $\raw{v}$ which has $\top$ at every selection position. (We omit the $\raw{v}$ indices from these lattice operations if it is clear which lattice is being referred to.) The notion of the ``selections'' of $\raw{v}$ extends to the other syntactic forms.

\begin{figure}
   \flushleft \shadebox{$v \leq v'$}
\begin{smathpar}
   \inferrule*[
   ]
   {
      \strut
   }
   {
      \hole \leq v
   }
   %
   \and
   %
   \inferrule*[
   ]
   {
      \alpha \leq \alpha'
   }
   {
      \annInt{n}{\alpha} \leq \annInt{n}{\alpha'}
   }
   %
   \and
   %
   \inferrule*[
   ]
   {
      \strut
   }
   {
      \annInt{n}{\bot} \leq \hole
   }
   %
   \and
   %
   \inferrule*[
   ]
   {
      \alpha \leq \alpha'
   }
   {
      \annTrue{\alpha} \leq \annTrue{\alpha'}
   }
   %
   \and
   %
   \inferrule*[
   ]
   {
      \strut
   }
   {
      \annTrue{\bot} \leq \hole
   }
   %
   \and
   %
   \inferrule*[
   ]
   {
      \alpha \leq \alpha'
   }
   {
      \annFalse{\alpha} \leq \annFalse{\alpha'}
   }
   %
   \and
   %
   \inferrule*[
   ]
   {
      \strut
   }
   {
      \annFalse{\bot} \leq \hole
   }
   %
   \and
   %
   \inferrule*[
   ]
   {
      \alpha \leq \alpha'
      \quad
      v_i \leq u_i
      \quad
      (\forall i \in \length{\vec{x}})
   }
   {
      \annRec{\vec{\bind{x}{v}}}{\alpha} \leq \annRec{\vec{\bind{x}{u}}}{\alpha'}
   }
   %
   \and
   %
   \inferrule*[
   ]
   {
      v_i \leq \hole
      \quad
      (\forall i \in \length{\vec{x}})
   }
   {
      \annRec{\vec{\bind{x}{v}}}{\bot} \leq \hole
   }
   %
   \and
   %
   \inferrule*[
   ]
   {
      \alpha \leq \alpha'
   }
   {
      \annNil{\alpha} \leq \annNil{\alpha'}
   }
   %
   \and
   %
   \inferrule*[
   ]
   {
      \strut
   }
   {
      \annNil{\bot} \leq \hole
   }
   %
   \and
   %
   \inferrule*[
   ]
   {
      (\alpha, v, v') \leq (\alpha', v, v')
   }
   {
      \annCons{v}{v'}{\alpha} \leq \annCons{u}{u'}{\alpha'}
   }
   %
   \and
   %
   \inferrule*[
   ]
   {
      (v, v') \leq (\hole, \hole)
   }
   {
      \annCons{v}{v'}{\bot} \leq \hole
   }
   %
   \and
   %
   \inferrule*[
   ]
   {
      (\rho, h, \alpha, \sigma) \leq (\rho', h', \alpha', \sigma')
   }
   {
      \annClosure{\rho}{h}{\alpha}{\sigma} \leq \annClosure{\rho'}{h'}{\alpha'}{\sigma'}
   }
   %
   \and
   %
   \inferrule*[
   ]
   {
      (\rho, h, \sigma) \leq (\sub{\hole}{\rho}, \hole, \hole)
   }
   {
      \annClosure{\rho}{h}{\bot}{\sigma} \leq \hole
   }
\end{smathpar}

   \caption{Preorder on value selections}
   \label{fig:data-dependencies:leq}
\end{figure}

\subsubsection{Environment selections and hole equivalence}

The notion of the ``selections'' of $\raw{v}$ also extends (pointwise) to environments, so that $\Sel{\raw{\rho}}{A}$ means the set of environment selections $\rho'$ of shape $\raw{\rho}$, where the variables in $\rho'$ are bound to selections of the corresponding variables in $\raw{\rho}$. One challenge arises from the pointwise use of $\join$ to combine environment selections. Since environments contain other environments recursively, via closures, a naive implementation of environment join is a very expensive operation. One solution is to employ an efficient representation of values which are fully unselected, which is often the case during the backward analysis.

We therefore augment the set of value selections $\ValF{\Ann{A}}$ with a distinguished element \emph{hole}, written $\hole$, which is an alternative notation for $\bot_{\raw{v}}$ for any $\raw{v}$, i.e.~the selection of shape $\raw{v}$ which has $\bot$ at every selection position, and generalise this idea to terms and eliminators. The equivalence of $\hole$ to any such bottom element is established explicitly by the preorder order defined (for values) in \figref{data-dependencies:leq}: the first rule always allows $\hole$ on the left-hand side of $\leq$, and other rules allow $\hole$ on the right-hand side of $\leq$ as long as all the selections that appear on the left-hand side are $\bot$. (The rules for terms $e$ and eliminators $\sigma$ are analogous and are omitted.) If we write $\eq$ for the equivalence relation induced by $\leq$ on values selections, which we call \emph{hole-equivalence}, it should be clear that $\hole \join v \eq v$ and $\hole \meet v \eq \hole$. This means the join of two selections $v, v'$ of $\raw{v}$ can be implemented efficiently, whenever one selection is $\hole$, by simply discarding $\hole$ and returning the other selection without further processing.

\begin{definition}[Hole equivalence]
   Define $\eq$ as the intersection of $\leq$ and $\geq$.
\end{definition}

Because $\hole$ is equivalent to $\bot_{\raw{v}}$ for any $\raw{v}$, all such bottom elements are hole-equivalent. For example, the value selection $\annCons{\hole}{\hole}{\top}$ is hole-equivalent to $\annCons{5_{\bot}}{\hole}{\top}$, but also to $\annCons{6_{\bot}}{{\exNil_{\bot}}}{\top}$, and so the last two selections, even though they have different shapes, are hole-equivalent by transitivity. In practice we only use the hole ordering to compare selections with the same shape.

\subsection{Forward data dependency}
\label{sec:data-dependencies:analyses:fwd}

We now define the core bidirectional data dependency analyses for a fixed computation $T :: \raw{\rho}, \raw{e} \evalR \raw{v}$, where $T$ is a trace. In practice one would obtain $T$ by first evaluating $\raw{e}$ in $\raw{\rho}$, and then run multiple forward or backward analyses over $T$ with appropriate lattices. We start with the forward dependency function $\evalFwdF{T}$ which ``replays'' evaluation, turning input availability into output availability, with $T$ guiding the analysis whenever holes in the input selection would mean the analysis would otherwise get stuck. $\evalFwdF{T}$ uses the auxiliary function $\matchFwdF{w}$ for forward-analysing a pattern-match; we explain this first, as it introduces the key idea of a selection as identifying the data available to a downstream computation.

\begin{figure}
{\small \flushleft \shadebox{$\matchFwd{v}{\sigma}{w}{\rho}{\kappa}{\alpha}$}%
\hfill \textbfit{$v$ and $\sigma$ forward-match along $w$ to $\rho$ and $\kappa$, with ambient availability $\alpha$}}
\begin{smathpar}
   \inferrule*[
      lab={\ruleName{$\matchFwdS$-hole-1}}
   ]
   {
      \hole \eq v
      \\
      \matchFwd{v}{\sigma}{w}{\rho}{\kappa}{\alpha}
   }
   {
      \matchFwd{\hole}{\sigma}{w}{\rho}{\kappa}{\alpha}
   }
   %
   \and
   %
   \inferrule*[
      lab={\ruleName{$\matchFwdS$-hole-2}}
   ]
   {
      \hole \eq \sigma
      \\
      \matchFwd{v}{\sigma}{w}{\rho}{\kappa}{\alpha}
   }
   {
      \matchFwd{v}{\hole}{w}{\rho}{\kappa}{\alpha}
   }
   %
   \and
   %
   \inferrule*[
      lab={\ruleName{$\matchFwdS$-var}}
   ]
   {
      \strut
   }
   {
      \matchFwd{v}{\elimVar{x}{\kappa}}{\matchVar{x}}{\bind{x}{v}}{\kappa}{\top}
   }
   %
   \and
   %
   \inferrule*[
      lab={\ruleName{$\matchFwdS$-true}}
   ]
   {
      \strut
   }
   {
      \matchFwd{\annTrue{\alpha}}
               {\elimBool{\kappa}{\kappa'}}
               {\matchTrue}
               {\seqEmpty}{\kappa}{\alpha}
   }
   %
   \and
   %
   \inferrule*[
      lab={\ruleName{$\matchFwdS$-false}}
   ]
   {
      \strut
   }
   {
      \matchFwd{\annFalse{\alpha}}
               {\elimBool{\kappa}{\kappa'}}
               {\matchFalse}
               {\seqEmpty}{\kappa'}{\alpha}
   }
   %
   \and
   %
   \inferrule*[
      lab={\ruleName{$\matchFwdS$-unit}}
   ]
   {
      \strut
   }
   {
      \matchFwd{\annot{\exRecEmpty}{\alpha}}
               {\elimRecEmpty{\kappa}}
               {\matchRecEmpty}
               {\seqEmpty}
               {\kappa}
               {\alpha}
   }
   %
   \and
   %
   \inferrule*[
      lab={\ruleName{$\matchFwdS$-record}}
   ]
   {
      \matchFwd{\annRec{\vec{\bind{x}{v}}}{\top}}
               {\elimRec{\vec{x}}{\sigma}}
               {\matchRec{\vec{\bind{x}{w}}}}
               {\rho}
               {\sigma'}
               {\beta}
      \\
      \matchFwd{u}{\sigma'}{w}{\rho'}{\kappa}{\beta'}
   }
   {
      \matchFwd{\annRec{\vec{\bind{x}{v}} \concat \bind{y}{u}}{\alpha}}
               {\elimRec{\vec{x} \concat y}{\sigma}}
               {\matchRec{\vec{\bind{x}{w}} \concat \bind{y}{w'}}}
               {\rho \concat \rho'}
               {\kappa}
               {\alpha \meet \beta \meet \beta'}
   }
   %
   \and
   %
   \inferrule*[
      lab={\ruleName{$\matchFwdS$-nil}}
   ]
   {
      \strut
   }
   {
      \matchFwd{\annNil{\alpha}}
               {\elimList{\kappa}{\sigma'}}
               {\matchNil}
               {\seqEmpty}{\kappa}{\alpha}
   }
   %
   \and
   %
   \inferrule*[
      lab={\ruleName{$\matchFwdS$-cons}}
   ]
   {
      \matchFwd{v}{\sigma}{w}{\rho}{\tau}{\beta}
      \\
      \matchFwd{v'}{\tau}{w'}{\rho'}{\kappa'}{\beta'}
   }
   {
      \matchFwd{\annCons{v}{v'}{\alpha}}
               {\elimList{\kappa}{\sigma}}
               {\matchCons{w}{w'}}
               {\rho \concat \rho'}{\kappa'}{\alpha \meet \beta \meet \beta'}
   }
\end{smathpar}

\vspace{2mm}
{\small \flushleft \shadebox{$\evalFwd{\rho}{e}{\alpha}{T}{v}$}%
\hfill \textbfit{$\rho$ and $e$, with ambient availability $\alpha$, forward-evaluate along $T$ to $v$}}
\begin{smathpar}
   \mprset{center}
   \inferrule*[
      lab={\ruleName{$\evalFwdS$-hole}}
   ]
   {
      \hole \eq e
      \\
      \evalFwd{\rho}{e}{\alpha}{T}{v}
   }
   {
      \evalFwd{\rho}{\hole}{\alpha}{T}{v}
   }
   %
   \and
   %
   \inferrule*[
      lab={\ruleName{$\evalFwdS$-var}}
   ]
   {
      \envLookup{\rho}{x}{v}
   }
   {
      \evalFwd{\rho}{\exVar{x}}{\alpha}{\trVar{x}{\rho}}{v}
   }
   %
   \and
   %
   \inferrule*[lab={\ruleName{$\evalFwdS$-lambda}}]
   {
      \strut
   }
   {
      \evalFwd{\rho}
              {\exLambda{\sigma}}
              {\alpha}
              {\trLambda{\sigma'}}
              {\annClosure{\rho}{\seqEmpty}{\alpha}{\sigma}}
   }
   %
   \and
   %
   \inferrule*[lab={\ruleName{$\evalFwdS$-int}}]
   {
      \strut
   }
   {
      \evalFwd{\rho}
              {\annInt{n}{\alpha'}}
              {\alpha}
              {\trInt{n}{\rho}}
              {\annInt{n}{\alpha \meet \alpha'}}
   }
   %
   \and
   %
   \inferrule*[lab={\ruleName{$\evalFwdS$-record}}]
   {
      \evalFwd{\rho}{e_i}{\alpha}{T_i}{v_i}
      \quad
      (\forall i \numleq \length{\vec{x}})
   }
   {
      \evalFwd{\rho}
              {\annRec{\vec{\bind{x}{e}}}{\alpha'}}
              {\alpha}
              {\trRec{\vec{\bind{x}{T}}}}
              {\annRec{\vec{\bind{x}{v}}}{\alpha \meet \alpha'}}
   }
   %
   \and
   %
   \inferrule*[
      lab={\ruleName{$\evalFwdS$-project}}
   ]
   {
      \evalFwdEq{\rho}{e}{\alpha}{T}{\annRec{\vec{\bind{x}{u}}}{\beta}}
      \\
      \envLookup{\vec{\bind{x}{u}}}{y}{v'}
   }
   {
      \evalFwd{\rho}
              {\exRecProj{e}{y}}
              {\alpha}
              {\trRecProj{T}{\vec{\bind{x}{v}}}{y}}
              {v'}
   }
   %
   \and
   %
   \inferrule*[lab={\ruleName{$\evalFwdS$-nil}}]
   {
      \strut
   }
   {
      \evalFwd{\rho}
              {\annNil{\alpha'}}
              {\alpha}
              {\trNil{\rho}}
              {\annNil{\alpha \meet \alpha'}}
   }
   %
   \and
   %
   \inferrule*[
      lab={\ruleName{$\evalFwdS$-cons}},
   ]
   {
      \evalFwd{\rho}{e}{\alpha}{T}{v}
      \\
      \evalFwd{\rho}{e'}{\alpha}{U}{v'}
   }
   {
      \evalFwd{\rho}
              {\annCons{e}{e'}{\alpha'}}
              {\alpha}
              {\trCons{T}{U}}
              {\annCons{v}{v'}{\alpha \meet \alpha'}}
   }
   %
   \and
   %
   \inferrule*[
      lab={\ruleName{$\evalFwdS$-apply-prim}},
      right={$\primFwdBool{\phi}{\vec{n}}(\vec{\beta}) = \alpha'$}
   ]
   {
      \evalFwdEq{\rho}{e_i}{\alpha}{U_i}{\annInt{{n_i}}{\beta_i}}
      \quad
      (\forall i \numleq \length{\vec{n}})
   }
   {
      \evalFwd{\rho}
              {\exAppPrim{\phi}{\vec{e}}}
              {\alpha}
              {\trAppPrimNew{\phi}{U}{n}}
              {\annInt{\exAppPrim{\phi}{\vec{n}}}{\alpha'}}
   }
   %
   \and
   %
   \inferrule*[
      lab={\ruleName{$\evalFwdS$-apply}},
      width={3.4in}
   ]
   {
      \evalFwdEq{\rho}{e}{\alpha}{T}{\annClosure{\rho_1}{h}{\beta}{\sigma}}
      \\
      \rho_1, h, \beta \closeDefsFwdR \rho_2
      \\
      \evalFwd{\rho}{e'}{\alpha}{U}{v}
      \\
      \matchFwd{v}{\sigma}{w}{\rho_3}{e^\twoPrime}{\beta'}
      \\
      \evalFwd{\rho_1 \concat \rho_2 \concat \rho_3}{e^\twoPrime}{\beta \meet \beta'}{T'}{v'}
   }
   {
      \evalFwd{\rho}{\exApp{e}{e'}}{\alpha}{\trApp{T}{U}{w}{T'}}{v'}
   }
   %
   \and
   %
   \inferrule*[lab={
      \ruleName{$\evalFwdS$-let-rec}}
   ]
   {
      \rho, h', \alpha \closeDefsFwdR \rho'
      \\
      \evalFwd{\rho \concat \rho'}{e}{\alpha}{T}{v}
   }
   {
      \evalFwd{\rho}{\exLetRecMutual{h'}{e}}{\alpha}{\trLetRecMutual{h}{T}}{v}
   }
\end{smathpar}
\vspace{2mm}
{\small \flushleft \shadebox{$\rho, h, \alpha \closeDefsFwdR \rho'$}%
\hfill \textbfit{$h$ forward-generates to $\rho'$ in $\rho$ and $\alpha$}}
\begin{smathpar}
   \inferrule*[
      lab={\ruleName{$\closeDefsFwdR$-rec-defs}}
   ]
   {
      v_i = \annClosure{\rho}{\vec{\bind{x}{\sigma}}}{\alpha}{\sigma_i}
      \quad
      (\forall i \in \length{\vec{x}})
   }
   {
      \rho, \vec{\bind{x}{\sigma}}, \alpha
      \closeDefsFwdR
      \vec{\bind{x}{v}}
   }
\end{smathpar}
\caption{Forward data dependency (Boolean cases for $\evalFwdR{T}$ omitted)}
\label{fig:data-dependencies:fwd}
\end{figure}


\subsubsection{Forward match}
\label{sec:data-dependencies:analyses:fwd:pattern-match}

\figref{data-dependencies:fwd} defines a family of \emph{forward-match} functions $\matchFwdF{w}$ of type $\Sel{\raw{v}, \raw{\sigma}}{A} \to (\Sel{\raw{\rho}, \raw{\kappa}}{A}) \times \Ann{A}$ for any $w :: \raw{v}, \raw{\sigma} \matchR \raw{\rho}, \raw{\kappa}$. (The definition is presented in a relational style for readability, but should be understood as a total function defined by structural recursion on $w$, which appears in grey to emphasise the connection to \figref{core-language:semantics}.) Forward-match replays the match witnessed by $w$, transferring the selections on the relevant parts of $v \in \Sel{\raw{v}}{A}$ to the output environment $\rho \in \Sel{\raw{\rho}}{A}$, and from the relevant part of $\sigma \in \Sel{\raw{\sigma}}{A}$ to the chosen continuation $\kappa \in \Sel{\raw{\kappa}}{A}$.

$\matchFwdF{w}$ also returns the \emph{meet} of the selection states associated with the part of $v$ consumed by $\sigma$. We call this the \emph{availability} of $v$ (in the context of $\sigma$), since it represents the extent to which the demand implied by $\sigma$ was met. A variable match consumes nothing of $v$ and so the availability of $v$ in this context is simply $\top$, the unit for $\meet$. A Boolean match consumes either $\annot{\exTrue}{\alpha}$ or $\annot{\exFalse}{\alpha}$, with availability $\alpha$; empty list and empty record matches are similar. When we match a cons, we return the meet of the $\alpha$ on the cons node itself with the availabilities $\beta$ and $\beta'$ computed for $v$ and $v'$. Non-empty records are similar, but to process the initial part of the record, we supply the neutral selection state $\top$ on the subrecord in order to use the definition recursively. (Note, however, that subrecords are not first-class; they exist only as intermediate artefacts of the interpreter.)

One might imagine dispensing with the need for $w$ by simply defining $\matchFwdS$ by case analysis on $v$ and $\sigma$. However, it is then unclear how to proceed in the event that either $v$ or $\sigma$ is a hole. In particular, it is not clear how to obtain the $\raw{\rho}$ associated with the original pattern-match, in order to produce an environment selection $\rho' \in \Sel{\raw{\rho}}{A}$. If $\matchFwdS$ is defined with respect to a known $w$, this can be achieved via additional rules \ruleName{$\matchFwdS$-hole-$v$} and \ruleName{$\matchFwdS$-hole-$\sigma$} that define the behaviour at hole to be the same as the behaviour at any $\eq$-equivalent value in $\Sel{\raw{v}}{A}$ or $\Sel{\raw{\sigma}}{A}$. Operationally, these rules can be interpreted as ``expanding'' the holes in $v$ or $\sigma$, in a shape-preserving way, until another rule of the definition applies. There will be exactly one non-hole rule that applies, corresponding to the execution path originally taken, and although there may be multiple such expansions, the following property implies that the result of $\matchFwdF{w}$ will be unique up to $\eq$.

\begin{lemma}[Monotonicity of $\matchFwdF{w}$]
   Suppose $w :: \raw{v}, \raw{\sigma} \matchR \raw{\rho}, \raw{\kappa}$, with $v, \sigma \matchFwdR{w} \rho, \kappa, \alpha$ and $v', \sigma' \matchFwdR{w} \rho', \kappa', \alpha'$. If $(v, \sigma) \leq (v', \sigma)$ then $(\rho, \kappa, \alpha) \leq (\rho', \kappa', \alpha')$.
\end{lemma}

\subsubsection{Forward evaluation}
\label{sec:data-dependencies:forward-eval}

\figref{data-dependencies:fwd} also defines a family of \emph{forward-evaluation} functions $\evalFwdF{T}$ of type $(\Sel{\raw{\rho}, \raw{e}}{A}) \times \Ann{A} \to \Sel{\raw{v}}{A}$ for any $T :: \raw{\rho}, \raw{e} \evalR \raw{v}$. Like forward-match, forward-evaluation is presented in a relational style, but should be read as a total function defined by structural recursion on $T$. Forward evaluation replays $T$, using the input selection $(\rho, e) \in \Sel{\raw{\rho}, \raw{e}}{A}$ to determine an output selection $v \in \Sel{\raw{v}}{A}$. The rules mirror those of the evaluation relation $\evalR$, although there is an additional selection state input $\alpha$ called the \emph{ambient availability}. We explain this with reference to the application rule, which is the only point where the ambient availability changes.

\Paragraph{Function application} The rule assumes the application $\exApp{e}{e'}$ already has an ambient availability $\alpha$; at the outermost level this will usually be $\top$. The rule passes $\alpha$ down when recursively forward-evaluating $e$ and $e'$, but computes a new selection state $\beta \meet \beta'$ when transferring control to the function, combining ambient availability $\beta$ captured by the closure and $\beta'$ obtained by forward-matching the argument $v$ with the eliminator $\sigma$ of the closure, representing the availability of those parts of $v$ demanded by the function. The ambient availability is used to upper-bound the availability of any value selections constructed in the dynamic context of that function, establishing a dependency between the resources consumed by functions and resources they produce. The auxiliary function $\closeDefsFwdF{\rho,h}: (\Sel{\raw{\rho},\raw{h}}{A}) \times \Ann{A} \to \Sel{\raw{\rho}'}{A}$ for any $\smash{\raw{\rho}, \raw{h} \closeDefsR \raw{\rho'}}$ is given at the bottom of \figref{data-dependencies:fwd} and follows $\closeDefsR$, but captures the ambient availability into each closure.

\Paragraph{Primitive application} Primitive operations are the other source of input-output dependencies beyond user-defined functions. Since a primitive operation is opaque, these dependencies cannot be derived from its execution, but must be specified by the primitive operation directly. More specifically, $\phi \in \tyInt^{i} \to \tyInt$ is required to provide a forward-dependency function $\primFwdBool{\phi}{\vec{n}}: \Ann{A}^i \to \Ann{A}$ for every $\vec{n} \in \tyInt^i$ which specifies how to turn an input selection $\vec{\alpha} \in \Ann{A}^i$ for $\vec{n}$ into an output selection $\alpha'$ on $\exAppPrim{\phi}{\vec{n}}$. There is one such function per possible input $\vec{n}$ so that the dynamic dependencies for that specific call can depend on the values passed to the operation. For example, in our implementation, the dependency function for multiplication establishes that (for non-zero $n$) both $n * 0$ and $0 * n$ depend only on $0$. However, primitives are free to implement forward-dependency however they choose, with the caveat that \secref{data-dependencies:analyses:bwd:eval} will also require $\phi$ to provide a backward-dependency function for any input $\vec{n}$, and \secref{data-dependencies:galois-connections} will require these to be related in a certain way for the consistency of the whole system to be guaranteed.

\Paragraph{Other rules} All other rules pass the ambient availability into any subcomputations unchanged. Variable lookup disregards the ambient $\alpha$, simply preserving the selection on the returned value. The lambda rule captures it in the closure, along with the environment; the letrec rule passes it on to $\closeDefsFwdR$ so it is captured by recursive closures as well. Record projection is more interesting, disregarding not only the ambient $\alpha$ but also the availability $\beta$ of the record itself. Containers are considered to be independent of the values they contain: here, $v_i$ has its own internal availability which is preserved by projection, but there is no implied dependency of the field on the record from which it was projected. Record construction also reflects this principle, preserving the field selections into the resulting record selection unchanged. But it also sets the availability of the record value to be $\alpha \meet \alpha'$, reflecting the dependency of the container on both the constructing expression and the ambient availability. The rules for nil, cons and integers are similar.

\Paragraph{Hole case} Environments have no special $\hole$ form. However, a hole rule is needed to allow forward evaluation to continue in the event that $e$ is $\hole$; this is essential because subsequent steps may extract non-$\hole$ values from $\rho$ and result in non-$\hole$ outputs. The rule is similar to the rules for $\matchFwdF{w}$ and again can be understood operationally as using the information in $T$ to expand $\hole$ sufficiently for another rule to apply, with a result which is unique up to $\eq$. In addition, application and record projection must accommodate the case where the selection on the closure or record being eliminated is represented by $\hole$. In these rules $\evalFwdR{T}\eq$ is used to denote the relational composition of $\evalFwdR{T}$ and $\eq$.

\begin{lemma}[Monotonicity of $\evalFwdF{T}$]
   Suppose $T :: \raw{\rho}, \raw{e} \evalR \raw{v}$ with $\rho, e, \alpha \evalFwdR{T} v$ and $\rho', e', \alpha' \evalFwdR{T} v'$. If $(\rho, e, \alpha) \leq (\rho', e', \alpha')$ then $v \leq v'$.
\end{lemma}

\subsection{Backward data dependency}
\label{sec:data-dependencies:analyses:bwd}

The backward dependency function $\evalBwdF{T}$ ``rewinds'' evaluation, turning output demand into input demand, with $T$ guiding the analysis backward. We start with the auxiliary function $\matchBwdF{w}$ which is used for backward-analysing a pattern-match.

\subsubsection{Backward match}
\label{sec:data-dependencies:analyses:bwd:pattern-match}

\figref{data-dependencies:bwd} defines a family of \emph{backward-match} functions $\matchBwdF{w}$ of type $(\Sel{\raw{\rho}, \raw{\kappa}}{A}) \times \Ann{A} \to \Sel{\raw{v}, \raw{\sigma}}{A}$ for any $w :: \raw{v}, \raw{\sigma} \matchR \raw{\rho}, \raw{\kappa}$. Backward-match rewinds the match witnessed by $w$, turning demand on the environment and continuation into demand on the value and eliminator that were originally matched. The additional input $\alpha$ represents the downstream demand placed on any resources that were constructed in the context of this match; $\matchBwdF{w}$ transfers this to the matched portion of $\raw{v}$, establishing a backwards link between resources produced and resources consumed in a given function context.

\begin{figure}
   {\small \flushleft \shadebox{$\evalBwd{v}{T}{\rho}{e}{\alpha}$}%
   \hfill \textbfit{$v$ backward-evaluates along $T$ to $\rho$ and $e$, with argument demand $\alpha$}}
   \begin{smathpar}
      \inferrule*[
         lab={\ruleName{$\evalBwdS$-hole}}
      ]
      {
         \hole \eq v
         \\
         \evalBwd{v}{T}{\rho}{e}{\alpha}
      }
      {
         \evalBwd{\hole}{T}{\rho}{e}{\alpha}
      }
      %
      \and
      %
      \inferrule*[
         lab={\ruleName{$\evalBwdS$-var}}
      ]
      {
         \envLookupBwd{\rho'}{\rho}{\bind{x}{v}}
      }
      {
         \evalBwd{v}{\trVar{x}{\rho}}{\rho'}{\exVar{x}}{\bot}
      }
      %
      \and
      %
      \inferrule*[
         lab={\ruleName{$\evalBwdS$-lambda}}
      ]
      {
         \strut
      }
      {
         \evalBwd{\annClosure{\rho}{\seqEmpty}{\alpha}{\sigma}}
                 {\trLambda{\sigma'}}
                 {\rho}
                 {\exLambda{\sigma}}
                 {\alpha}
      }
      %
      \and
      %
      \inferrule*[
         lab={\ruleName{$\evalBwdS$-int}}
      ]
      {
         \strut
      }
      {
         \evalBwd{\annInt{n}{\alpha}}
                 {\trInt{n}{\rho}}
                 {\hole_{\raw{\rho}}}
                 {\annInt{n}{\alpha}}
                 {\alpha}
      }
      %
      \and
      %
      \inferrule*[
         lab={\ruleName{$\evalBwdS$-true}}
      ]
      {
         \strut
      }
      {
         \evalBwd{\annTrue{\alpha}}
                 {\trTrue{\rho}}
                 {\hole_{\raw{\rho}}}
                 {\annTrue{\alpha}}
                 {\alpha}
      }
      %
      \and
      %
      \inferrule*[
         lab={\ruleName{$\evalBwdS$-false}}
      ]
      {
         \strut
      }
      {
         \evalBwd{\annFalse{\alpha}}
                 {\trFalse{\rho}}
                 {\hole_{\raw{\rho}}}
                 {\annFalse{\alpha}}
                 {\alpha}
      }
      %
      \and
      %
      \inferrule*[lab={\ruleName{$\evalBwdS$-record}}]
      {
         \evalBwd{v_i}{T_i}{\rho_i}{e_i}{\alpha_i'}
         \quad
         (\forall i \numleq \length{\vec{x}})
      }
      {
         \evalBwd{\annRec{\vec{\bind{x}{v}}}{\alpha}}
                 {\trRec{\vec{\bind{x}{T}}}}
                 {\bigjoin\vec{\rho}}
                 {\annRec{\vec{\bind{x}{e}}}{\alpha}}
                 {\alpha \join \bigjoin\vec{\alpha}'}
      }
      %
      \and
      %
      \inferrule*[lab={\ruleName{$\evalBwdS$-project}}]
      {
         \envLookupBwd{\vec{\bind{x}{u}}}{\vec{\bind{x}{v}}}{\bind{y}{v'}}
         \\
         \evalBwd{\annRec{\vec{\bind{x}{u}}}{\bot}}
                 {T}
                 {\rho}
                 {e}
                 {\alpha}
      }
      {
         \evalBwd{v'}
                 {\trRecProj{T}{\vec{\bind{x}{v}}}{y}}
                 {\rho}
                 {\exRecProj{e}{y}}
                 {\alpha}
      }
      %
      \and
      %
      \inferrule*[
         lab={\ruleName{$\evalBwdS$-nil}}
      ]
      {
         \strut
      }
      {
         \evalBwd{\annNil{\alpha}}
                 {\trNil{\rho}}
                 {\hole_{\raw{\rho}}}
                 {\annNil{\alpha}}
                 {\alpha}
      }
      %
      \and
      %
      \inferrule*[
         lab={\ruleName{$\evalBwdS$-cons}}
      ]
      {
         \evalBwd{v}{T}{\rho}{e}{\alpha}
         \\
         \evalBwd{v'}{U}{\rho'}{e'}{\alpha'}
      }
      {
         \evalBwd{\annCons{v}{v'}{\beta}}
                 {\trCons{T}{U}}
                 {\rho \join \rho'}
                 {\annCons{e}{e'}{\beta}}
                 {\beta \join \alpha \join \alpha'}
      }
      %
      \and
      %
      \inferrule*[
         lab={\ruleName{$\evalBwdS$-let-rec}}
      ]
      {
         \evalBwd{v}{T}{\rho \concat \rho_1}{e}{\alpha}
         \\
         \rho_1 \closeDefsBwdR \rho', h', \alpha'
      }
      {
         \evalBwd{v}{\trLetRecMutual{h}{T}}{\rho \join \rho'}{\exLetRecMutual{h'}{e}}{\alpha \join \alpha'}
      }
      %
      \and
      %
      \inferrule*[
         lab={\ruleName{$\evalBwdS$-apply-prim}},
         right={$\primBwdBool{\phi}{\vec{n}}(\alpha') = \vec{\alpha}$}
      ]
      {
         \evalBwd{\annInt{{n_i}}{\alpha_i}}{U_i}{\rho_i}{e_i}{\beta_i}
         \quad
         (\forall i \in \length{\vec{n}})
      }
      {
         \evalBwd{\annInt{m}{\alpha'}}
                 {\trAppPrimNew{\phi}{U}{n}}
                 {\bigjoin\vec{\rho}}
                 {\exAppPrim{\phi}{\vec{e}}}
                 {\bigjoin\vec{\beta}}
      }
      %
      \and
      %
      \inferrule*[
         lab={\ruleName{$\evalBwdS$-apply}},
         width={3.3in}
      ]
      {
         \evalBwd{v}{T'}{\rho_1 \concat \rho_2 \concat \rho_3}{e}{\beta}
         \\
         \matchBwd{\rho_3}{e}{\beta}{w}{v'}{\sigma}
         \\
         \evalBwd{v'}{U}{\rho}{e_2}{\alpha}
         \\
         \rho_2 \closeDefsBwdR \rho_1', h, \beta'
         \\
         \evalBwd{\annClosure{\rho_1 \join \rho_1'}{h}{\beta \join \beta'}{\sigma}}{T}{\rho'}{e_1}{\alpha'}
      }
      {
         \evalBwd{v}{\trApp{T}{U}{w}{T'}}{\rho \join \rho'}{\exApp{e_1}{e_2}}{\alpha \join \alpha'}
      }
   \end{smathpar}
   \vspace{1mm}

\begin{minipage}[t]{0.48\textwidth}%
   {\small\flushleft \shadebox{$\envLookupBwd{\rho'}{\rho}{\bind{x}{v}}$}%
   \hfill \textbfit{$\rho'$ contains $\bind{x}{v}$}}
   \begin{smathpar}
      \inferrule*[
         lab={\ruleName{$\envLookupBwdS$-head}}
      ]
      {
         \strut
      }
      {
         \envLookupBwd{(\hole_{\raw{\rho}} \concat \bind{x}{u})}
                      {\rho \concat \bind{x}{v}}
                      {\bind{x}{u}}
      }
      %
      \and
      %
      \inferrule*[
         lab={\ruleName{$\envLookupBwdS$-tail}},
      ]
      {
         \envLookupBwd{\rho'}{\rho}{\bind{x}{u}}
         \\
         x \neq y
      }
      {
         \envLookupBwd{(\rho' \concat \bind{y}{\hole})}
                      {\rho \concat \bind{y}{v}}
                      {\bind{x}{u}}
      }
   \end{smathpar}
\end{minipage}%
\hfill
\begin{minipage}[t]{0.47\textwidth}%
   {\small\flushleft\shadebox{$\rho \closeDefsBwdR \rho', h, \alpha$}%
   \hfill \textbfit{$\rho$ backward-generates to $\rho'$, $h$, $\alpha$}}
   \begin{smathpar}
      \inferrule*[
         lab={\ruleName{$\closeDefsBwdR$-rec-defs}}
      ]
      {
         v_i = \annClosure{\rho_i}{h_i}{\alpha_i}{\sigma_i}
         \quad
         (\forall i \in \length{\vec{x}})
      }
      {
         \vec{\bind{x}{v}}
         \closeDefsBwdR
         \bigjoin\vec{\rho}, \vec{\bind{x}{\sigma}} \join {\bigjoin\vec{h}}, \bigjoin{\vec{\alpha}}
      }
   \end{smathpar}
\end{minipage}

\caption{\vspace{-2mm}Backward evaluation}
\label{fig:data-dependencies:bwd}
\end{figure}


In the variable case, no proper part of $\raw{v}$ was matched, so $\alpha$ is disregarded. The rule need only ensure that the demand $v$ in the singleton environment $\bind{x}{v}$ is propagated backward. If a Boolean constant was matched, $\alpha$ becomes the demand on that constant, and $\kappa$, capturing the demand on the continuation, is used to construct the demand on the original eliminator, with $\hole$ used to represent the absence of demand on the non-taken branch. (Using $\hole$ for this means matches $w$ need only retain information about taken branches.) The nil case is similar.

For a cons match $\matchCons{w}{w'}$, we split the environment into $\rho$ and $\rho'$ (there is a unique well-typed decomposition) and then backward-match $w$ and $w'$ recursively to obtain $v$ and $v'$, representing the demand on the head and tail of the list. These are combined into the demand on the entire list, using $\alpha$ as the demand on the cons node itself. $\sigma$ represents the demand on the interim eliminator used to match the tail, and $\tau$ the demand on the eliminator used to match the head, which are then combined into a demand on the eliminator used to match the whole list, with $\hole$ again used to represent the absence of demand on the nil branch. Records are similar, except there is only a single branch. The selection state $\beta$ computed for the initial part of the record is an artefact of processing records recursively, and is disregarded.

\begin{lemma}[Monotonicity of $\matchBwdF{w}$]
   Suppose $w :: \raw{v}, \raw{\sigma} \matchR \raw{\rho}, \raw{\kappa}$, with $\rho, \kappa, \alpha \matchBwdR{w} v, \sigma$ and $\rho', \kappa', \alpha' \matchBwdR{w} v', \sigma'$. If $(\rho, \kappa, \alpha) \leq (\rho', \kappa', \alpha')$ then $(v, \sigma) \leq (v', \sigma)$.
\end{lemma}

\subsubsection{Backward evaluation}
\label{sec:data-dependencies:analyses:bwd:eval}

\figref{data-dependencies:bwd} also defines a family of \emph{backward-evaluation} functions $\evalBwdF{T}$ of type $\Sel{\raw{v}}{A} \to (\Sel{\raw{\rho}, \raw{e}}{A}) \times \Ann{A}$ for any $T :: \raw{\rho}, \raw{e} \evalR \raw{v}$. Backward evaluation rewinds $T$, using the output selection $v \in \Sel{\raw{v}}{A}$ to determine an input selection $(\rho, e) \in \Sel{\raw{\rho}, \raw{e}}{A}$. The rules resemble those of the evaluation relation $\evalR$ with inputs and outputs flipped, and with an additional output $\alpha$ called the \emph{ambient demand}. The general pattern is that each backward rule takes the join of the demand attached to any partial values constructed at that step, and the ambient demand associated with any subcomputations, and passes it upwards as the new ambient demand. The output environment is constructed similarly, by joining the demand flowing back through the environment copies used to evaluate subcomputations. Demand is also attached to the source expression when it is the expression form responsible for the construction of a demanded value.

\Paragraph{Function application} The application rule is where the ambient demand is used and the function context changes, so we start here. The rule essentially runs the forward evaluation rule in reverse, using the trace $T'$ to backward-evaluate the function body. The ambient demand $\beta$ associated with $T'$ is the join of the demand on any resources constructed directly by that function invocation, and is transferred to the matched part of the function argument by the backward-match function $\matchBwdF{w}$. The ambient demand passed upwards into the enclosing function context is $\alpha \join \alpha'$, representing the resources needed along $T$ and $U$. The auxiliary function $\smash{\closeDefsBwdF{\rho, h}}: \smash{\Sel{\raw{\rho'}}{A} \to (\Sel{\raw{\rho}, \raw{h}}{A})} \times \Ann{A}$ for any $\raw{\rho}, \raw{h} \closeDefsR \raw{\rho'}$ defined at the bottom of \figref{data-dependencies:bwd} is used to turn $\rho_2$, capturing the demand flowing back through any recursive uses of the function and any others with which it was mutually defined, into information that can be merged back into the demand on the closure. The function $\closeDefsBwdF{\rho,h}$ is also used in the letrec rule, which otherwise follows the generic pattern described above.

\Paragraph{Primitive application} Each primitive operation $\phi: \tyInt^{i} \to \tyInt$ must provide a backward-dependency  function $\primBwdBool{\phi}{\vec{n}}: \Ann{A} \to \Ann{A}^i$ for every $\vec{n} \in \tyInt^i$ which specifies how to turn the output selection $\alpha'$ on $\exAppPrim{\phi}{\vec{n}}$ into an input selection $\vec{a} \in \Ann{A}^i$ on $\vec{n}$. The rule for primitive application uses this information to pair each argument $n_i$ with its demand $\alpha_i$ and then backwards-evaluate the argument. The ambient demand passed upward is the join of those arising from these subcomputations, and is unrelated to the execution of the primitive itself, similar to a function application. Here $\bigjoin{\vec{\beta}}$ means the fold of $\join$ (with unit $\bot$) over the sequence of selection states $\seqRange{\beta_1}{\beta_{\length{\vec{x}}}'}$. Environment demands $\vec{\rho} = \seqRange{\rho_1}{\rho_{\length{\vec{n}}}}$ are joined (pointwise) in a similar fashion.

\Paragraph{Other rules} In the variable case, no partial values were constructed during evaluation and there are no subcomputations, so the ambient demand is $\bot$, the unit for $\join$. The returned environment selection demands $v$ for the variable $x$ and $\hole$ for all other variables, using the family of \emph{backwards lookup} functions $\envLookupBwdF{-}{\rho}{x}{-}$ of type $\Sel{\raw{v}}{A} \to \Sel{\raw{\rho}}{A}$ for any $\envLookup{\raw{\rho}}{x}{\raw{v}}$ also defined in \figref{data-dependencies:bwd}. (The output of the function is on the left in the relational notation.) For atomic values such as integers and nil, the ambient demand is simply the demand $\alpha$ associated with the constructed value, which is also attached to the corresponding expression, and the environment demand has $\hole$ for every variable in the original environment $\raw{\rho}$, written $\hole_{\raw{\rho}}$.

For closures, the ambient demand is unpacked along with the other components, preserving any internal selections on $\rho$ and $\sigma$. Composite values such as records and cons cells follow the general pattern; thus for records, the ambient demands $\alpha'_i$ for the subcomputations are joined with the $\alpha$ on the record itself to produce the ambient demand passed upward. Record projection never demands the record constructor itself, but simply promotes the field demand into a record demand, using $\envLookupBwdR{\vec{\bind{x}{\raw{v}}}}$ to demand fields other than $y$ with $\hole$.

\Paragraph{Hole rule} The hole rule, as elsewhere, ensures that the function is defined when $v$ is $\hole$, and it is easy to show that $\evalBwdF{T}$  preserves $\leq$, and thus $\eq$.

\begin{lemma}[Monotonicity of $\evalBwdF{T}$]
   Suppose $T :: \raw{\rho}, \raw{e} \evalR \raw{v}$ with $v \evalBwdR{T} \rho, e, \alpha $ and $v' \evalBwdR{T} \rho', e', \alpha' $. If $v \leq v'$ then $(\rho, e, \alpha) \leq (\rho', e', \alpha')$.
\end{lemma}

\newpage
\subsection{Round-tripping properties of $\evalFwdF{T}$ and $\evalBwdF{T}$}
\label{sec:data-dependencies:galois-connections}

We now establish more formally the round-tripping properties, alluded at the beginning of the section, that relate $\evalFwdF{T}$ to $\evalBwdF{T}$. For the analyses to be coherent, we expect $\evalFwdF{T}(\evalBwdF{T}(v))$ to produce a value selection $v' \geq v$, and $\evalBwdF{T}(\evalFwdF{T}(\rho,e))$ to produce an input selection $(\rho',e') \leq (\rho,e)$. Pairs of (monotonic) functions $f: X \to Y$ and $g: Y \to X$ that are related in this way are called \emph{Galois connections}. Galois connections generalise isomorphisms: $f$ and $g$ are not quite mutual inverses, but are the nearest to an inverse each can get to the other. We will present a visual example of some of these round-tripping properties in \secref{de-morgan:example}; here we establish the relevant theorems.

\begin{definition}[Galois connection]
   Suppose $X$ and $Y$ are sets equipped with partial orders $\numleq_X$ and $\numleq_Y$. Then monotonic functions $f: X \to Y$ and $g: Y \to X$ form a \emph{Galois connection} $(f, g): X \to Y$ iff $g(f(x)) \numgeq_X x$ and $f(g(y)) \numleq_Y y$.
\end{definition}

\noindent Galois connections are also adjoint functors between poset categories, with left and right adjoints $f$ and $g$  usually called the \emph{lower} and \emph{upper} adjoints, because $f$ approximates an inverse of $g$ from below, and $g$ an inverse of $f$ from above. Galois connections compose component-wise, so it is useful to think of them as having a type $X \to Y$, with the direction (by convention) given by the lower adjoint. If $\gamma: X \to Y$ is a Galois connection, we will write $\lowerAdj{\gamma}$ and $\upperAdj{\gamma}$ for the lower and upper adjoints respectively; an important property we will return to is that $\lowerAdj{\gamma}$ preserves joins and $\upperAdj{\gamma}$ preserves meets. We now show that, for any $\Ann{A}$, $\evalBwdF{T}$ and $\evalFwdF{T}$ form a Galois connection (\thmref{core-language:eval:gc}), by first establishing that the relevant auxiliary functions also form Galois connections.

\begin{theorem}[Galois connection for pattern-matching]
   \label{thm:core-language:match:gc}
   Suppose $w :: \raw{v}, \raw{\sigma} \matchR \raw{\rho}, \raw{\kappa}$.  Then $(\matchBwdF{w}, \matchFwdF{w}): (\Sel{\raw{\rho},\raw{\kappa}}{A}) \times \Ann{A} \to \Sel{\raw{v}, \raw{\sigma}}{A}$ is a Galois connection.
\end{theorem}

\begin{proof}
\ifappendices See \appref{proofs:match:gc}. \else \ProofInSupplementaryMaterial \fi
\end{proof}

\begin{lemma}[Galois connection for environment lookup]
   \label{lem:core-language:env-get-put}
   Suppose $\envLookup{\raw{\rho}}{x}{\raw{v}}$. Then $(\envLookupBwdF{-}{\raw{\rho}}{x}{},\envLookupFwdF{}{\raw{\rho}}{x}{-}): \Sel{\raw{v}}{A} \to \Sel{\raw{\rho}}{A}$ is a Galois connection.
\end{lemma}

\begin{proof}
   \ifappendices See \appref{proofs:lookup:gc}. \else \ProofInSupplementaryMaterial \fi
   \end{proof}

   \begin{theorem}[Galois connection for recursive bindings]
\label{thm:core-language:closeDefs:gc}
   Suppose $\raw{\rho}, \raw{h} \closeDefsR \raw{\rho}'$. Then $({\closeDefsBwdF{\raw{\rho},\raw{h}}, \closeDefsFwdF{\raw{\rho}, \raw{h}}}): \Sel{\raw{\rho}'}{A} \to (\Sel{\raw{\rho},\raw{h}}{A}) \times \Ann{A}$ is a Galois connection.
\end{theorem}

\begin{proof}
   \ifappendices See \appref{proofs:closeDefs:gc}. \else \ProofInSupplementaryMaterial \fi
   \end{proof}

We assume (rather than prove) that the backward and forward dependency functions $\primBwdBool{\phi}{\vec{n}}$ and $\primFwdBool{\phi}{\vec{n}}$ provided for every primitive operation $\phi: \tyInt^i \to \tyInt$ and every $\vec{n}$ of length $i$ form a Galois connection of type $\Ann{A} \to \Ann{A}^i$. Under this assumption the following holds.

\begin{theorem}[Galois connection for evaluation]
\label{thm:core-language:eval:gc}
   Suppose $T :: \raw{\rho}, \raw{e} \evalR \raw{v}$.  Then $(\evalBwdF{T}, \evalFwdF{T}): \Sel{\raw{v}}{A} \to (\Sel{\raw{\rho}, \raw{e}}{A}) \times \Ann{A}$ is a Galois connection.
\end{theorem}

\begin{proof}
   \ifappendices See \appref{proofs:eval:gc}. \else \ProofInSupplementaryMaterial \fi
\end{proof}

Establishing that $(\evalBwdF{T}, \evalFwdF{T})$ is an adjoint pair might seem rather weak as a correctess property: it merely ensures that the two analyses are related in a sensible way, not that they actually capture any useful information. This is a familiar problem from other approximate analyses like type systems and model checking, where properties like soundness or completeness are essential but do not by themselves guarantee utility. One could certainly define versions of $\evalBwdF{T}$ and $\evalFwdF{T}$ that are too coarse grained to be useful, yet still satisfy \thmref{core-language:eval:gc}. However Galois connections do at least require that every tightening or tweak to the forward analysis is paired with a corresponding adjustment to the backward analysis, and vice-versa. In \secref{conclusion} we consider how other ideas from provenance and program slicing might be adapted to provide additional correctness criteria.


\section{A bidirectional dynamic dependency analysis}
\label{sec:data-dependencies}

We now show how to extend the core language described in \secref{core-language} with a bidirectional mechanism for tracking data dependencies. The first thing to establish is a way of selecting parts of the program of interest. For example, consider representing the bar chart in \figref{introduction:data-linking} as the record $\exRec{\bind{\kw{China}}{295.3}, \bind{\kw{USA}}{196.7}, \bind{\kw{Germany}}{91.8}}$ and wanting to select only the value of the field $\kw{China}$.

We represent selections using an abstract type $\textcal{A}$ of annotations and then parameterise the data type of values $\Set{Val}$ by $\textcal{A}$ so that individual values can be annotated by their selection state, as defined in \figref{core-syntax-selection}. Because we are primarily interested in tracking data dependencies between values, we only annotate values which are first-order data forms and terms which construct them. The annotation-augmented forms for values include: Booleans $\annTrue{\alpha}$ and $\annFalse{\alpha}$, integers $\annInt{n}{\alpha}$, records $\annRec{\vec{\bind{x}{v}}}{\alpha}$, lists $\annNil{\alpha}$ and $\annCons{u}{v}{\alpha}$, and vectors $\annVecVal{\vec{v}}{\annInt{j}{\alpha}}{\smash{\alpha'}}$. The corresponding annotated forms for terms are defined similarly.

% We also introduce holes $\hole$ as a special values that express annotating every position within a value with the bottom element of a lattice.

The selection type $\textcal{A}$ is chosen to remain abstract as there are various forms of selection information we may want to express in our language, however we do require $\textcal{A}$ to be a lattice. In other words, selections must at least have top $\top$ and bottom $\bot$ elements representing fully selected and fully unselected respectively, and a meet $\meet$ and join $\join$ operation for combining selection information using intersection and union where $\top$ and $\bot$ are their left and right units.

The most obvious notion of selection is achieved by using the Boolean lattice, $\textcal{2}$, where the top $\top$ and bottom $\bot$ elements are assigned true $\TT$ and false $\FF$, and the meet $\meet$ and $\join$ operations are conjunction $\land$ and disjunction $\lor$. Using this, we can represent the selection of the value of the field $\kw{China}$ in the record given earlier as $\annot{\exRec{\bind{\kw{China}}{\annot{295.3}{\TT}}, \bind{\kw{USA}}{\annot{196.7}{\FF}}, \bind{\kw{Germany}}{\annot{91.8}{\FF}}}}{\FF}$. However, more sophisticated lattices can also be used such as a vector of Booleans to represent multiple selections simultaneously.

\begin{figure}
\begin{syntaxfig}
\mbox{Eliminator}
&
\sigma, \tau
&
::=
&
\ldots
\\
&&&
\elimBoolTrue{\kappa}
\\
&&&
\elimBoolFalse{\kappa}
\\
&&&
\elimListSingleton{\branchNil{\kappa}}
\\
&&&
\elimListSingleton{\branchCons{\sigma}}
\\[2mm]
\mbox{Raw term}
&
r
&
::=
&
\ldots
\\
&&&
\exLambda{\sigma}
&
\text{anonymous function}
\end{syntaxfig}
\caption{Additional syntax}
\end{figure}

\begin{figure}[H]
\flushleft \shadebox{$h \leq h'$}
\begin{smathpar}
   \inferrule*[
   ]
   {
      \sigma_i \leq \tau_i
      \quad
      (\forall i \numleq \length{\vec{x}})
   }
   {
      \vec{\bind{x}{\sigma}}
      \leq
      \vec{\bind{x}{\tau}}
   }
   %
   \and
   %
   \inferrule*[
   ]
   {
      \sigma_i \leq \hole
      \quad
      (\forall i \numleq \length{x})
   }
   {
      \vec{\bind{x}{\sigma}} \leq \hole
   }
\end{smathpar}

\vspace{5pt}
\flushleft \shadebox{$v \leq v'$}
\begin{smathpar}
   \inferrule*[
   ]
   {
      \strut
   }
   {
      \hole \leq v
   }
   %
   \and
   %
   \inferrule*[
      right={$\alpha \leq \alpha'$}
   ]
   {
      \strut
   }
   {
      \annInt{n}{\alpha} \leq \annInt{n}{\alpha'}
   }
   %
   \and
   %
   \inferrule*[
   ]
   {
      \strut
   }
   {
      \annInt{n}{\FF} \leq \hole
   }
   %
   \and
   %
   \inferrule*[
      right={$\alpha \leq \alpha'$}
   ]
   {
      \strut
   }
   {
      \annTrue{\alpha} \leq \annTrue{\alpha'}
   }
   %
   \and
   %
   \inferrule*[
   ]
   {
      \strut
   }
   {
      \annTrue{\FF} \leq \hole
   }
   %
   \and
   %
   \inferrule*[
      right={$\alpha \leq \alpha'$}
   ]
   {
      \strut
   }
   {
      \annFalse{\alpha} \leq \annFalse{\alpha'}
   }
   %
   \and
   %
   \inferrule*[
   ]
   {
      \strut
   }
   {
      \annFalse{\FF} \leq \hole
   }
   %
   \and
   %
   \inferrule*[
      right={$\alpha \leq \alpha'$}
   ]
   {
      v_i \leq u_i
      \quad
      (\forall i \in \length{\vec{x}})
   }
   {
      \annRec{\vec{\bind{x}{v}}}{\alpha} \leq \annRec{\vec{\bind{x}{u}}}{\alpha'}
   }
   %
   \and
   %
   \inferrule*[
   ]
   {
      v \leq \hole
      \\
      v' \leq \hole
   }
   {
      \annPair{v}{v'}{\FF} \leq \hole
   }
   %
   \and
   %
   \inferrule*[
      right={$\alpha \leq \alpha'$}
   ]
   {
      \strut
   }
   {
      \annNil{\alpha} \leq \annNil{\alpha'}
   }
   %
   \and
   %
   \inferrule*[
   ]
   {
      \strut
   }
   {
      \annNil{\FF} \leq \hole
   }
   %
   \and
   %
   \inferrule*[
      right={$\alpha \leq \alpha'$}
   ]
   {
      v \leq u
      \\
      v' \leq u'
   }
   {
      \annCons{v}{v'}{\alpha} \leq \annCons{u}{u'}{\alpha'}
   }
   %
   \and
   %
   \inferrule*[
   ]
   {
      v \leq \hole
      \\
      v' \leq \hole
   }
   {
      \annCons{v}{v'}{\FF} \leq \hole
   }
   %
   \and
   %
   \inferrule*[
      right={$(\alpha,\beta) \leq (\alpha',\beta')$}
   ]
   {
      v_{i} \leq u_{i}
      \quad
      (\forall i \numleq j)
   }
   {
      \annVecVal{\vec{v}}{\annInt{j}{\beta}}{\alpha}
      \leq
      \annVecVal{\vec{u}}{\annInt{j}{\beta'}}{\alpha'}
   }
   %
   \and
   %
   \inferrule*[
   ]
   {
      v_{i} \leq \hole
      \quad
      (\forall i \numleq j)
   }
   {
      \annVecVal{\vec{v}}{\annInt{j}{\FF}}{\FF} \leq \hole
   }
   %
   \and
   %
   \inferrule*[
   ]
   {
      v_i \leq u_i
      \quad
      (\forall i \numleq j)
   }
   {
      \exPrimOp{\phi}{\seqRange{v_1}{v_j}} \leq \exPrimOp{\phi}{\seqRange{u_1}{u_j}}
   }
   %
   \and
   %
   \inferrule*[
   ]
   {
      v_i \leq \hole
      \quad
      (\forall i \numleq \length{\vec{v}})
   }
   {
      \exPrimOp{\phi}{\vec{v}} \leq \hole
   }
   %
   \and
   %
   \inferrule*[
   ]
   {
      \rho \leq \rho'
      \\
      h \leq h'
      \\
      \sigma \leq \sigma'
   }
   {
      \exClosure{\rho}{h}{\sigma} \leq \exClosure{\rho'}{h'}{\sigma'}
   }
   %
   \and
   %
   \inferrule*[
   ]
   {
      \rho \leq \sub{\hole}{\rho}
      \\
      h \leq \hole
      \\
      \sigma \leq \hole
   }
   {
      \exClosure{\rho}{h}{\sigma} \leq \hole
   }
\end{smathpar}

\vspace{5pt}
\flushleft \shadebox{$\sigma \leq \sigma'$}
\begin{smathpar}
   \inferrule*[
   ]
   {
      \strut
   }
   {
      \hole \leq \sigma
   }
   %
   \and
   %
   \inferrule*[
   ]
   {
      \kappa \leq \kappa'
   }
   {
      \elimVar{x}{\kappa} \leq \elimVar{x}{\kappa'}
   }
   %
   \and
   %
   \inferrule*[
   ]
   {
      \kappa \leq \hole
   }
   {
      \elimVar{x}{\kappa} \leq \hole
   }
   %
   \and
   %
   \inferrule*[
   ]
   {
      \kappa_1 \leq \kappa_1'
      \\
      \kappa_1 \leq \kappa_1'
   }
   {
      \elimBool{\kappa_1}{\kappa_2} \leq \elimBool{\kappa_1'}{\kappa_2'}
   }
   %
   \and
   %
   \inferrule*[
   ]
   {
      \kappa_1 \leq \hole
      \\
      \kappa_1 \leq \hole
   }
   {
      \elimBool{\kappa_1}{\kappa_2} \leq \hole
   }
   %
   \and
   %
   \inferrule*[
   ]
   {
      \sigma \leq \sigma'
   }
   {
      \elimProd{\sigma} \leq \elimProd{\sigma'}
   }
   %
   \and
   %
   \inferrule*[
   ]
   {
      \sigma \leq \hole
   }
   {
      \elimProd{\sigma} \leq \hole
   }
   %
   \and
   %
   \inferrule*[
   ]
   {
      \kappa \leq \kappa'
      \\
      \sigma \leq \sigma'
   }
   {
      \elimList{\kappa}{\sigma} \leq \elimList{\kappa'}{\sigma'}
   }
   %
   \and
   %
   \inferrule*[
   ]
   {
      \kappa \leq \hole
      \\
      \sigma \leq \hole
   }
   {
      \elimList{\kappa}{\sigma} \leq \hole
   }
\end{smathpar}
\caption{Partial orders on values and eliminators}
\end{figure}

\begin{figure}[H]
\flushleft \shadebox{$e \leq e'$}
\begin{smathpar}
   \inferrule*[
   ]
   {
      \strut
   }
   {
      \hole \leq e
   }
   %
   \and
   %
   \inferrule*[
      right={$\alpha \leq \alpha'$}
   ]
   {
      \strut
   }
   {
      \annInt{n}{\alpha} \leq \annInt{n}{\alpha'}
   }
   %
   \and
   %
   \inferrule*[
   ]
   {
      \strut
   }
   {
      \annInt{n}{\FF} \leq \hole
   }
   %
   \and
   %
   \inferrule*[
      right={$\alpha \leq \alpha'$}
   ]
   {
      \strut
   }
   {
      \annTrue{\alpha} \leq \annTrue{\alpha'}
   }
   %
   \and
   %
   \inferrule*[
   ]
   {
      \strut
   }
   {
      \annTrue{\FF} \leq \hole
   }
   %
   \and
   %
   \inferrule*[
      right={$\alpha \leq \alpha'$}
   ]
   {
      \strut
   }
   {
      \annFalse{\alpha} \leq \annFalse{\alpha'}
   }
   %
   \and
   %
   \inferrule*[
   ]
   {
      \strut
   }
   {
      \annFalse{\FF} \leq \hole
   }
   %
   \and
   %
   \inferrule*[
      right={$\alpha \leq \alpha'$}
   ]
   {
      e_i \leq e'_i
      \quad
      (\forall i \numleq \length{\vec{x}})
   }
   {
      \annRec{\vec{\bind{x}{e}}}{\alpha} \leq \annRec{\vec{\bind{x}{e}}'}{\alpha'}
   }
   %
   \and
   %
   \inferrule*[
   ]
   {
      e_i \leq \hole
      \quad
      (\forall i \numleq \length{\vec{x}})
   }
   {
      \annRec{\vec{\bind{x}{e}}}{\FF} \leq \hole
   }
   %
   \and
   %
   \inferrule*[
      right={$\alpha \leq \alpha'$}
   ]
   {
      \strut
   }
   {
      \annNil{\alpha} \leq \annNil{\alpha'}
   }
   %
   \and
   %
   \inferrule*[
   ]
   {
      \strut
   }
   {
      \annNil{\FF} \leq \hole
   }
   %
   \and
   %
   \inferrule*[
      right={$\alpha \leq \alpha'$}
   ]
   {
      e_1 \leq e_1'
      \\
      e_2 \leq e_2'
   }
   {
      \annCons{e_1}{e_2}{\alpha} \leq \annCons{e_1'}{e_2'}{\alpha'}
   }
   %
   \and
   %
   \inferrule*[
   ]
   {
      e_1 \leq \hole
      \\
      e_2 \leq \hole
   }
   {
      \annCons{e_1}{e_2}{\FF} \leq \hole
   }
   %
   \and
   %
   \inferrule*[
      right={$\alpha \leq \alpha'$}
   ]
   {
      e_1 \leq e_1'
      \\
      e_2 \leq e_2'
   }
   {
      \annVec{e_1}{x}{e_2}{\alpha} \leq \annVec{e_1'}{x}{e_2'}{\alpha'}
   }
   %
   \and
   %
   \inferrule*[
   ]
   {
      e_1 \leq \hole
      \\
      e_2 \leq \hole
   }
   {
      \annVec{e_1}{x}{e_2}{\FF} \leq \hole
   }
   %
   \and
   %
   \inferrule*[
   ]
   {
      \strut
   }
   {
      \exVar{x} \leq \exVar{x}
   }
   %
   \and
   %
   \inferrule*[
   ]
   {
      \strut
   }
   {
      \exVar{x} \leq \hole
   }
   %
   \and
   %
   \inferrule*[
   ]
   {
      \sigma \leq \sigma'
   }
   {
      \exLambda{\sigma} \leq \exLambda{\sigma}
   }
   %
   \and
   %
   \inferrule*[
   ]
   {
      \sigma \leq \hole
   }
   {
      \exLambda{\sigma} \leq \hole
   }
   %
   \and
   %
   \inferrule*[
   ]
   {
      e_1 \leq e_1'
      \\
      e_2 \leq e_2'
   }
   {
      \exApp{e_1}{e_2} \leq \exApp{e_1}{e_2}
   }
   %
   \and
   %
   \inferrule*[
   ]
   {
      e_1 \leq \hole
      \\
      e_2 \leq \hole
   }
   {
      \exApp{e_1}{e_2} \leq \hole
   }
   %
   \and
   %
   \inferrule*[
   ]
   {
      e_1 \leq e_1'
      \\
      e_2 \leq e_2'
   }
   {
      \exVecLookup{e_1}{e_2} \leq \exVecLookup{e_1}{e_2}
   }
   %
   \and
   %
   \inferrule*[
   ]
   {
      e_1 \leq \hole
      \\
      e_2 \leq \hole
   }
   {
      \exVecLookup{e_1}{e_2} \leq \hole
   }
   %
   \and
   %
   \inferrule*[
   ]
   {
      e \leq e'
   }
   {
      \exVecLen{e} \leq \exVecLen{e'}
   }
   %
   \and
   %
   \inferrule*[
   ]
   {
      e \leq \hole
   }
   {
      \exVecLen{e} \leq \hole
   }
   %
   \and
   %
   \inferrule*[
   ]
   {
      e_1 \leq \hole
      \\
      e_2 \leq \hole
   }
   {
      \exLet{x}{e_1}{e_2} \leq \hole
   }
   %
   \and
   %
   \inferrule*[
   ]
   {
      e \leq e'
      \\
      h \leq h'
   }
   {
      \exLetRecMutual{h}{e} \leq \exLetRecMutual{h'}{e'}
   }
   %
   \and
   %
   \inferrule*[
   ]
   {
      e \leq \hole
      \\
      h \leq \hole
   }
   {
      \exLetRecMutual{h}{e} \leq \hole
   }
\end{smathpar}
\caption{Partial order on terms}
\end{figure}


% \begin{figure}[H]
\flushleft\shadebox{$h \join h'$}
\begin{salign}
   (\seqRange{x_1: \sigma_1}{x_n: \sigma_n})
   \join
   (\seqRange{x_1: \sigma'_1}{x_n: \sigma'_n})
   &=
   \seqRange{x_1: (\sigma_1 \join \sigma'_1)}{x_n: (\sigma_n \join \sigma'_n)}
\end{salign}

\vspace{5pt}
\flushleft\shadebox{$v \join v'$}
\begin{salign}
   \hole \join v &= v
   \\
   v \join \hole &= v
   \\
   \annTrue{\alpha} \join \annTrue{\alpha'} &= \annTrue{\alpha \join \alpha'}
   \\
   \annFalse{\alpha} \join \annFalse{\alpha'} &= \annFalse{\alpha \join \alpha'}
   \\
   \annInt{n}{\alpha} \join \annInt{n}{\alpha'} &= \annInt{n}{\alpha \join \alpha'}
   \\
   \annPair{v_1}{v_2}{\alpha} \join \annPair{v_1'}{v_2'}{\alpha'} &=
   \annPair{v_1 \join v_1'}{v_2 \join v_2'}{\alpha \join \alpha'}
   \\
   \annNil{\alpha} \join \annNil{\alpha} &= \annNil{\alpha \join \alpha'}
   \\
   (\annCons{u}{v}{\alpha}) \join (\annCons{u'}{v'}{\alpha'})
   &=
   \annCons{(u \join u')}{(v \join v')}{\alpha \join \alpha'}
   \\
   \annMatrix{v_{i,j}}{i}{j}{\exPair{i'}{j'}}{\alpha}
   \join
   \annMatrix{v'_{i,j}}{i}{j}{\exPair{i'}{j'}}{\alpha'}
   &=
   \annMatrix{v_{i,j} \join v'_{i,j}}{i}{j}{\exPair{i'}{j'}}{\alpha \join \alpha'}
   \\
   \exPrim{\phi} \join \exPrim{\phi} &= \exPrim{\phi}
   \\
   \exClosureRec{\Gamma}{\rho}{h}{\sigma} \join \exClosureRec{\Gamma}{\rho'}{h'}{\sigma'}
   &=
   \exClosureRec{\Gamma}{\rho \join \rho'}{h \join h'}{\sigma \join \sigma'}
\end{salign}
\caption{Join of compatible values}
\end{figure}

% \begin{figure}[H]
\flushleft\shadebox{$\sigma \join \sigma'$}
\begin{salign}
   \hole \join \sigma &= \sigma
   \\
   \sigma \join \hole &= \sigma
   \\
   (\elimVar{x}{\kappa}) \join (\elimVar{x}{\kappa'})
   &=
   \elimVar{x}{\kappa \join \kappa'}
   \\
   \elimBool{\kappa_1}{\kappa_2} \join \elimBool{\kappa_1'}{\kappa_2'}
   &=
   \elimBool{\kappa_1 \join \kappa_1'}{\kappa_2 \join \kappa_2'}
   \\
   \elimRec{\vec{x}}{\kappa} \join \elimRec{\vec{x}}{\kappa'}
   &=
   \elimRec{\vec{x}}{\kappa \join \kappa'}
   \\
   \elimList{\kappa}{\sigma} \join \elimList{\kappa'}{\sigma'}
   &=
   \elimList{\kappa \join \kappa'}{\sigma \join \sigma'}
\end{salign}
\caption{Join of compatible eliminators}
\label{fig:join-elim}
\end{figure}


\noindent

\subsection{Lattices of selections}
\label{sec:data-dependencies:lattices-of-selections}

Our approach to representing selections is shown in \figref{core-syntax-selection}. The basic idea is to parameterise the type $\Set{Val}$ of values by a (bounded) lattice $\Ann{A}$ of \emph{selection states} $\alpha$. We add selection states to Booleans, integers, records and lists; while it would present no complications to equip closures with a selection state too, for present purposes we are only interested in dependencies between first-order data, so closures are not (directly) selectable, although they do have selectable parts. Moreover closures must capture the \emph{ambient availability} (explained in \secref{data-dependencies:forward-eval} below), which is also a selection state $\alpha$. We parameterise the type $\Set{Term}$ of terms in the same way, but only add selection states to the term constructors corresponding to selectable values, allowing us to trace data dependencies back to expressions that appear in the source code. We return to this in \secref{surface-language}.

\begin{figure}
\begin{syntaxfig}
\mbox{Eliminator}
&
\sigma, \tau
&
::=
&
\ldots
\\
&&&
\elimBoolTrue{\kappa}
\\
&&&
\elimBoolFalse{\kappa}
\\
&&&
\elimListSingleton{\branchNil{\kappa}}
\\
&&&
\elimListSingleton{\branchCons{\sigma}}
\\[2mm]
\mbox{Raw term}
&
r
&
::=
&
\ldots
\\
&&&
\exLambda{\sigma}
&
\text{anonymous function}
\end{syntaxfig}
\caption{Additional syntax}
\end{figure}


\vspace{-3mm}
The fact that $\Ann{A}$ is a bounded lattice means it has top and bottom elements $\top$ and $\bot$ representing fully selected/unselected, plus meet and join operations $\meet$ and $\join$ for combining selection information, with $\top$ and $\bot$ as their respective units. In \figref{introduction:data-linking}, the \kw{data} field of \kw{BarChart} expects a list of records with fields \kw{x} and \kw{y}, mapping strings representing categorical data to floats determining the height of the corresponding bar; the record computed for China is $\exRec{\bind{\kw{x}}{\kw{"China"}} \concat \bind{\kw{y}}{\kw{295.3}}}$. The two-point lattice $\Bool \eqdef \Lattice{\set{\TT,\FF}}{\TT}{\FF}{\wedge}{\vee}$ can be used to represent the selection of the field $\kw{y}$ within this record as $\annRec{\bind{\kw{x}}{\annStr{\kw{"China"}}{\FF}} \concat \bind{\kw{y}}{\annInt{\kw{295.3}}{\TT}}}{\FF}$, indicating that the number $\exInt{\kw{295.3}}$ is selected, but that neither the string $\exStr{\kw{"China"}}$, nor the record itself, is selected.

Because lattices are closed under component-wise products, we often write $(\alpha, \beta) \leq (\alpha', \beta')$ to mean that $\alpha \leq \alpha'$ and $\beta \leq \beta'$. This also suggests more interesting lattices of selections, such as vectors of Booleans to represent multiple selections simultaneously, which might be visualised using different colours (as in \figref{introduction:data-linking}).

\begin{figure}
   \flushleft \shadebox{$v \leq v'$}
\begin{smathpar}
   \inferrule*[
   ]
   {
      \strut
   }
   {
      \hole \leq v
   }
   %
   \and
   %
   \inferrule*[
   ]
   {
      \alpha \leq \alpha'
   }
   {
      \annInt{n}{\alpha} \leq \annInt{n}{\alpha'}
   }
   %
   \and
   %
   \inferrule*[
   ]
   {
      \strut
   }
   {
      \annInt{n}{\bot} \leq \hole
   }
   %
   \and
   %
   \inferrule*[
   ]
   {
      \alpha \leq \alpha'
   }
   {
      \annTrue{\alpha} \leq \annTrue{\alpha'}
   }
   %
   \and
   %
   \inferrule*[
   ]
   {
      \strut
   }
   {
      \annTrue{\bot} \leq \hole
   }
   %
   \and
   %
   \inferrule*[
   ]
   {
      \alpha \leq \alpha'
   }
   {
      \annFalse{\alpha} \leq \annFalse{\alpha'}
   }
   %
   \and
   %
   \inferrule*[
   ]
   {
      \strut
   }
   {
      \annFalse{\bot} \leq \hole
   }
   %
   \and
   %
   \inferrule*[
   ]
   {
      \alpha \leq \alpha'
      \quad
      v_i \leq u_i
      \quad
      (\forall i \in \length{\vec{x}})
   }
   {
      \annRec{\vec{\bind{x}{v}}}{\alpha} \leq \annRec{\vec{\bind{x}{u}}}{\alpha'}
   }
   %
   \and
   %
   \inferrule*[
   ]
   {
      v_i \leq \hole
      \quad
      (\forall i \in \length{\vec{x}})
   }
   {
      \annRec{\vec{\bind{x}{v}}}{\bot} \leq \hole
   }
   %
   \and
   %
   \inferrule*[
   ]
   {
      \alpha \leq \alpha'
   }
   {
      \annNil{\alpha} \leq \annNil{\alpha'}
   }
   %
   \and
   %
   \inferrule*[
   ]
   {
      \strut
   }
   {
      \annNil{\bot} \leq \hole
   }
   %
   \and
   %
   \inferrule*[
   ]
   {
      (\alpha, v, v') \leq (\alpha', v, v')
   }
   {
      \annCons{v}{v'}{\alpha} \leq \annCons{u}{u'}{\alpha'}
   }
   %
   \and
   %
   \inferrule*[
   ]
   {
      (v, v') \leq (\hole, \hole)
   }
   {
      \annCons{v}{v'}{\bot} \leq \hole
   }
   %
   \and
   %
   \inferrule*[
   ]
   {
      (\rho, h, \alpha, \sigma) \leq (\rho', h', \alpha', \sigma')
   }
   {
      \annClosure{\rho}{h}{\alpha}{\sigma} \leq \annClosure{\rho'}{h'}{\alpha'}{\sigma'}
   }
   %
   \and
   %
   \inferrule*[
   ]
   {
      (\rho, h, \sigma) \leq (\sub{\hole}{\rho}, \hole, \hole)
   }
   {
      \annClosure{\rho}{h}{\bot}{\sigma} \leq \hole
   }
\end{smathpar}

   \caption{Preorder on values}
   \label{fig:data-dependencies:leq}
\end{figure}

\subsubsection{Selections of a value}
\label{sec:data-dependencies:selections}

The analyses which follow will be defined with respect to a fixed computation, and so we will often need to talk about the selections of a given value. To make this notion precise, consider that the raw (unselectable) syntax described in \secref{core-language} can be recovered from a selectable term via an erasure operation $\erase{\param}: \ValF{\Ann{A}} \to \ValF{\Unit}$ which forgets the selection information, where $\Unit$ is the trivial one-point lattice. We refer to $\erase{v}$ as the \emph{shape} of $v$. Allowing $\raw{u}, \raw{v}$ from now on to range over raw values, and reserving $u, v$ for selectable values, we can then define:

\begin{definition}[Selections of $\raw{v}$]
   Define $\Sel{\raw{v}}{A}$ to be the set of all values $v \in \ValF{\Ann{A}}$ with shape $\raw{v}$, i.e.
   ~that erase to $\raw{v}$.
\end{definition}

Since its elements have a fixed shape, the pointwise comparison of any $v, v'$ in $\Sel{\raw{v}}{A}$ using the partial order $\leq$ of $\Ann{A}$ is well defined, as is the pointwise application (zip) of a binary operation~\cite{gibbons17}. It should therefore be clear that if $\Ann{A}$ is a lattice, then $\Sel{\raw{v}}{A}$ is also a lattice, with $\top_{\raw{v}}$, $\bot_{\raw{v}}$, $\meet_{\raw{v}}$, and $\join_{\raw{v}}$ defined pointwise. For example, if $u$ and $u'$ have the same shape and $v$ and $v'$ have the same shape, the join of the lists $(\annCons{u}{v}{\alpha})$ and $(\annCons{u'}{v'}{\alpha'})$ is defined and equal to $\annCons{(u \join u')}{(v \join v')}{\alpha \join \alpha'}$. Similarly, the top element of $\Sel{\raw{v}}{A}$ is the selection of $\raw{v}$ which has $\top$ at every selection position. (We omit the $\raw{v}$ indices from the lattice operations if it is clear which lattice is being referred to.) The notion of the ``selections'' of $\raw{v}$ extends to the other syntactic forms.

\subsubsection{Environment selections and hole equivalence}

The notion of a ``selection'' of $\raw{v}$ also extends pointwise to environments, so that $\Sel{\raw{\rho}}{A}$ means the set of selectable environments $\rho'$ of shape $\raw{\rho}$, where the variables in $\rho'$ are bound to selections of the corresponding variables in $\raw{\rho}$. One challenge arises from the pointwise use of $\join$ to combine environment selections. Since environments contain other environments recursively through closures, environment join is a very expensive operation if treated naively. One implementation option is to employ an efficient representation of values which are fully unselected, which is the case for the majority of bindings in an environment.

We therefore enrich the set of selectable values $\ValF{\Ann{A}}$ with a distinguished element \emph{hole}, written $\hole$, which is an alternative notation for $\bot_{\raw{v}}$ for any $\raw{v}$, i.e.~the selectable value of shape $\raw{v}$ which has $\bot$ at every selection position. The equivalence of $\hole$ to any such bottom element is established explicitly by the preorder order defined in \figref{data-dependencies:leq}: the first rule always allows $\hole$ on the left-hand side of $\leq$, and other rules allow $\hole$ on the right-hand side of $\leq$ as long as all the selections that appear on the left-hand side are $\bot$. (The rules for recursive definitions $h$, eliminators $\sigma$ and terms $e$ are analogous and are omitted.) If we write $\eq$ for the equivalence relation induced by $\leq$ on selectable values, which we call \emph{hole-equivalence}, it should be clear that $\hole \join v \eq v$ and $\hole \meet v \eq \hole$. This means the join of two selections $v, v'$ of $\raw{v}$ can be implemented efficiently, whenever one selection is $\hole$, by simply discarding $\hole$ and returning the other selection without further processing.

\begin{definition}[Hole equivalence]
   Define $\eq$ as the intersection of $\leq$ and $\geq$.
\end{definition}

Because $\hole$ is equivalent to $\bot_{\raw{v}}$ for any $\raw{v}$, all such bottom elements are hole-equivalent. For example, the selectable value $\annCons{\hole}{\hole}{\top}$ is hole-equivalent to $\annCons{5_{\bot}}{\hole}{\top}$, but also to $\annCons{6_{\bot}}{{\exNil_{\bot}}}{\top}$, and so the last two selections, even though they have different shapes, are hole-equivalent by transitivity. In practice we only use the hole ordering to compare selections with the same shape.

\subsection{Forward and backward Galois dependency}
\label{sec:core-language:fwd-bwd}

Now that we have a way of expressing selections on values and program fragments (\secref{data-dependencies}), we want a means of analysing a given computation and specifying how selection information on its input can determine the selections on its output, and vice versa. This will allow us to achieve the crux of our goal as set out in \secref{introduction:motivation} to bidirectionally link structured outputs to structured inputs in order to track data dependencies.

One can think of selections on the input as the resources that are available to the program, and unselected inputs as unavailable resources. The intuition behind \textit{forward dependency analysis} is then to express that an output is only selected if all the resources required to construct it are available, i.e. they are all selected, hence we use the $\meet$ operation to combine selection information.

Conversely we can think of selections on the output as specifications for resources we want in the output, and unselected outputs as resources of uninterest. \textit{Backward dependency analysis} then expresses that an input is only selected if it was a resource required to construct the selected output.

Both forwards and backwards analyses are performed with respect to a provided fixed computation $\explVal{T}{\rho, e \evalR v}$ as previously defined in \figref{core-language:semantics}. Forward analysis can be thought of as ``replaying'' evaluation and propagating annotations on inputs to annotations on outputs; the trace $\textcolor{gray}{T}$ is necessary to guide forward analysis through the original computation because the presence of holes in the program results in loss of information about the program structure. Backward analysis can be thought of ``rewinding'' evaluation and propagating annotations on outputs to annotations on inputs; the necessity for traces $\textcolor{gray}{T}$ are more obvious as we have no other information about the original program structure. Operationally, one would first evaluate a program to obtain its trace $\textcolor{gray}{T}$ and then run a forward or background analysis over $\textcolor{gray}{T}$ with any lattice desired.

\newpage
\subsubsection{Pattern-matching}

It is necessary to first describe how the notion of pattern-matching translates to forward and backward dependency analysis, as this is how annotated input values are consumed to determine whether we have all the resources required to construct an output. This is defined in \figref{match:fwd-bwd}.

The function $\matchFwdF{w} :: \Sel{v, \boldsymbol{\sigma}}{A} \to \tyProd{\Sel{\boldsymbol{\rho}, \boldsymbol{\kappa}}{A}}{\textcal{A}}$ for forward analysis pattern-matching can be understood in two parts.
%First, it expresses that we match a value $v$ against an eliminator $\sigma$, both of which are annotated, to produce an annotated environment $\rho$ and continuation $\kappa$.
First it re-expresses the relation $\explVal{w}{v, \sigma \matchR \rho, \kappa}$ (given previously in \figref{core-language:semantics}) except that $v$, $\sigma$, $\rho$, and $\kappa$ are now annotated. Second, we make use of the input trace $\textcolor{gray}{w}$ which records the eliminator's pattern we originally matched against. This informs us to take the conjunction of all values in $v$ which correspond to $\textcolor{gray}{w}$ -- the resulting selection $\alpha$ determines if we have all the input resources necessary for consumption when pattern-matching.

Matching against a variable eliminator $\elimVar{x}{\kappa}$ always returns the selection $\top$, as a variable eliminator can accept any value. For Boolean eliminators, the output selection $\alpha$ is simply the selection on the Boolean value we match with i.e. $\annot{\exTrue}{\alpha}$ or $\annot{\exFalse}{\alpha}$ -- intuitively, if this value is selected then it constitutes all the resources required for the match. This is also the case for $\annot{\exNil}{\alpha}$.

For the non-empty list, $\annListComp{v_1}{v_2}{\alpha}$, we must first determine whether $v_1$ and $v_2$ are available as resources by performing forward match analysis on $v_1$ and $v_2$ as the head and tail pattern against the cons case eliminator $\sigma'$ in $\elimList{\kappa}{\sigma'}$; this returns selections $\beta$ and $\beta'$. We then take the meet of these with the selection $\alpha$ of the cons constructor to yield $\alpha \meet \beta \meet \beta'$. In other words, the output from matching against a non-empty list is only available if the head, tail, and cons constructor are available to match with. A similar story holds for forward-matching against records, where the output is only available if both the record constructor and all of its components are available.

Now working in the backwards direction, the function $\matchBwdF{w} :: \tyProd{\Sel{\boldsymbol{\rho}, \boldsymbol{\kappa}}{A}}{\textcal{A}} \to \Sel{v, \boldsymbol{\sigma}}{A}$ can also be understood in two parts. First, we use the trace $\textcolor{gray}{w}$ to reconstruct the shape of the eliminator $\sigma$. The environment $\rho$ and continuation $\kappa$ resulting from the forward pattern-match are then used to determine the original continuation for the branch corresponding to $\textcolor{gray}{w}$; all other branches are considered unused and hence have hole as their continuations. Second, we take the output selection $\alpha$ which tells us if all input resources were available when forward matching, and use this to annotate the original input value $v$.

The rule for variables, $\textcolor{gray}{x}$, discards its output selection $\alpha$ as no selections were originally consumed when forward pattern-matching against a variable eliminator. For traces $\textcolor{gray}{w}$ representing either $\textcolor{gray}{\exTrue}$ or $\textcolor{gray}{\exFalse}$, we take the output selection $\alpha$ and use this to annotate the Boolean value corresponding to $\textcolor{gray}{w}$. The same logic applies to $\textcolor{gray}{\matchNil}$. For non-empty lists, $\textcolor{gray}{\matchCons{w_1}{w_2}}$, we backward match in reverse order against the tail pattern $w_2$ then the head pattern $w_1$ to acquire annotated values $v_2$ and $v_1$. Using this, we reconstruct the value $\annCons{v_1}{v_2}{\alpha}$ where the cons constructor is annotated by the output selection $\alpha$. Similarly for records, $\textcolor{gray}{\matchRec{\vec{\bind{x}{w}}} \concat \bind{y}{w'}}$, we recursively backward match against all of the components of record and annotate the record constructor with the output selection $\alpha$.
% what pattern matching is doing is taking the conjunction of all selections on the values in v that correspond to the trace w

% The intuition behind mapping selections on the input to selections on the output is as a resource consumption, and evaluation consumes values and produces values, and the two kinds of consumptions (ways you can consume values) are primitive operations and pattern matching. Whenever we try to construct an output, we need to check that the resources are available. As primitive operations induce demand on their arguments in that in order to produce an output, this means checking that the selection on the required arguments are there. Structured values like lists are consumed by pattern-matching - whether we are able to construct an output depends on whether any data that was consumed in that local context was available.

% Every function operates as a context that consumes some input through its eliminator and all the selections on the consumed input determine whether any values constructed in that context can be constructed.

\begin{figure}[H]
\flushleft \shadebox{$p, \sigma \matchp \kappa$}
\begin{smathpar}
   \inferrule*[lab={\ruleName{$\matchp$-var}}]
   {
      \strut
   }
   {
      x, \elimVar{x}{\kappa}
      \matchp
      \kappa
   }
   %
   \and
   %
   \inferrule*[lab={\ruleName{$\matchp$-true}}]
   {
      \strut
   }
   {
      \exTrue, \elimBool{\kappa}{\kappa'}
      \matchp
      \kappa
   }
   %
   \and
   %
   \inferrule*[lab={\ruleName{$\matchp$-false}}]
   {
      \strut
   }
   {
      \exFalse, \elimBool{\kappa}{\kappa'}
      \matchp
      \kappa'
   }
   %
   \and
   %
   \inferrule*[lab={\ruleName{$\matchp$-pair}}]
   {
      p_1, \sigma \matchp, \tau
      \\
      p_2, \tau \matchp \kappa
   }
   {
      \exPair{p_1}{p_2}, \elimProd{\sigma}
      \matchp
      \kappa
   }
   %
   \and
   %
   \inferrule*[lab={\ruleName{$\matchp$-nil}}]
   {
      \strut
   }
   {
      \exNil, \elimList{\branchNil{\kappa}}{\branchCons{\sigma}}
      \matchp
      \kappa
   }
   %
   \and
   %
   \inferrule*[lab={\ruleName{$\matchp$-cons}}]
   {
      p_1, \sigma \matchp \tau
      \\
      p_2, \tau \matchp \kappa'
   }
   {
      \exCons{p_1}{p_2}, \elimList{\branchNil{\kappa}}{\branchCons{\sigma}}
      \matchp
      \kappa'
   }
\end{smathpar}
\caption{Pattern-matching (by pattern)}
\end{figure}


\subsubsection{Evaluation}

The idea of forward dependency analysis is to propagate selections on the input resources in order to determine the selections on the output value. The function $\evalFwdF{T} :: \tyProd{\Sel{\boldsymbol{\rho},\, e}{A}}{\textcal{A}} \to \Sel{v}{A}$ describes how we take a selected term $e$ and environment $\rho$, along with the selection $\alpha$ of the current context, and consume them to construct the selection information of the output value $v$. When working in the forwards direction, we want to express that a value is only selected if all the terms involved in its construction are selected, hence we use the meet operation to combine selection states.

Evaluating a variable $x$ is equivalent to looking it up in the environment $\rho$, where $\envLookupFwdF{}$ is defined in \figref{core-language:slicing:eval-aux}. Anonymous functions $\exLambda{\sigma}$ simply evaluate to closures, which are unannotated. We begin to see how the $\meet$ operation is used when evaluating an integer $n_{\alpha'}$ with $\alpha$ as the current selection context - this returns $\annot{n}{\alpha \meet \alpha'}$, expressing that the value $n$ is selected if both the term $n$ and the term containing $n$ were selected. The same logic applies to empty lists $\annot{\exNil}{\alpha'}$.

The recursive behaviour of forward analysis is first seen in the rule for records, $\exRec{\vec{\bind{x}{e}}}_{\alpha}$. We begin by forward analysing each term $e_i$ in the record to a selected value $v_i$, and then construct a record $\annot{\vec{\bind{x}{v}}}{\alpha \meet \alpha'}$, which is only selected if the record itself was selected along with the current context $\alpha'$.

For the cons rule, $\exPair{(\rho,\annCons{e_1}{e_2}{\alpha'})}{\alpha}$, we forward analyse $e_1$ and $e_2$ to get annotated values $v_1$ and $v_2$, and then construct the value $\annCons{v_1}{v_2}{\alpha \meet \alpha'}$ where the cons constructor is only selected if both $\exCons{e_1}{e_2}$ and the outer term containing it are selected.

% Give the intuition that what we do with annotation propagation is specifying how we consume resources in the input in order to construct things in the output.

\paragraph{Primitive operations}

Each primitive operation $\phi: \tyInt^{i} \to \tyInt$ must for every $\vec{n}$ with $\length{\vec{n}} = i$ provide a Galois connection $(\primFwdBool{\phi}{\vec{n}}, \primBwdBool{\phi}{\vec{n}})$ between $\Bool^i$ and $\Bool$, which we lift to a Galois connection $(\primFwd{\phi}{\vec{n}}, \primBwd{\phi}{\vec{n}})$ between $\Below{\vec{n}}$ and $\Below{\phi(\vec{n})}$ by defining
\begin{definition}
\label{def:core-language:primop-gc}
\begin{salign}
   \primFwd{\phi}{\vec{n}}(\vec{\annInt{n}{\alpha}}) &= \annInt{m}{\beta}
   \text{ where }
   \primFwdBool{\phi}{\vec{n}}(\vec{\alpha}) = \beta
   \\
   \primBwd{\phi}{\vec{n}}(\annInt{m}{\beta}) &= \vec{\annInt{n}{\alpha}}
   \text{ where }
   \primBwdBool{\phi}{\vec{n}}(\beta) = \vec{\alpha}
\end{salign}
\end{definition}

\noindent where $\vec{\annInt{n}{\alpha}}$ denotes the zip of same-length sequences $\vec{n}$ and $\vec{\alpha}$ with the constructor for integer values. For any $\vec{n}$ with $\length{\vec{n}} \numlt i$, any such $\phi$ also gives rise to an isomorphism between $\Below{\vec{n}}$ and the lattice of partial applications $\Below{\exPrimOp{\phi}{\vec{n}}}$.

\begin{figure}
{\small \flushleft \shadebox{$\evalFwd{\rho}{e}{\alpha}{T}{v}$}%
\hfill \textbfit{$\rho$ and $e$, with argument availability $\alpha$, forward-analyse along $T$ to $v$}}
\begin{smathpar}
   \mprset{center}
   \inferrule*[
      lab={\ruleName{$\evalFwdS$-eq}}
   ]
   {
      e \eq e'
      \\
      \evalFwd{\rho}{e'}{\alpha}{T}{v}
   }
   {
      \evalFwd{\rho}{e}{\alpha}{T}{v}
   }
   %
   \and
   %
   \inferrule*[
      lab={\ruleName{$\evalFwdS$-var}},
      right={$\envLookup{\rho}{x}{v}$}
   ]
   {
      \strut
   }
   {
      \evalFwd{\rho}{\exVar{x}}{\alpha}{\trVar{x}{\rho}}{v}
   }
   %
   \and
   %
   \inferrule*[lab={\ruleName{$\evalFwdS$-lambda}}]
   {
      \strut
   }
   {
      \evalFwd{\rho}
              {\exLambda{\sigma}}
              {\alpha}
              {\trLambda{\sigma'}}
              {\exClosure{\rho}{\seqEmpty}{\sigma}}
   }
   %
   \and
   %
   \inferrule*[lab={\ruleName{$\evalFwdS$-int}}]
   {
      \strut
   }
   {
      \evalFwd{\rho}
              {\annInt{n}{\alpha}}
              {\alpha'}
              {\trInt{n}{\rho}}
              {\annInt{n}{\alpha \meet \alpha'}}
   }
   %
   \and
   %
   \inferrule*[lab={\ruleName{$\evalR$-record}}]
   {
      \evalFwd{\rho}{e_i}{\alpha'}{T_i}{v_i}
      \quad
      (\forall i \numleq \length{\vec{x}})
   }
   {
      \evalFwd{\rho}
              {\annRec{\vec{\bind{x}{e}}}{\alpha}}
              {\alpha'}
              {\trRec{\vec{\bind{x}{T}}}}
              {\annRec{\vec{\bind{x}{v}}}{\alpha \meet \alpha'}}
   }
   %
   \and
   %
   \inferrule*[
      lab={\ruleName{$\evalR$-project}},
      right={$i \numleq \length{\vec{x}}$}
   ]
   {
      \evalFwd{\rho}{e}{\alpha}{T}{\exRec{\vec{\bind{x}{v}}}}
   }
   {
      \evalFwd{\rho}
              {\exRecProj{e}{x}}
              {\alpha}
              {\trRecProj{T}{\vec{x}}{x}}
              {v_i}
   }
   %
   \and
   %
   \inferrule*[lab={\ruleName{$\evalFwdS$-nil}}]
   {
      \strut
   }
   {
      \evalFwd{\rho}
              {\annNil{\alpha}}
              {\alpha'}
              {\trNil{\rho}}
              {\annNil{\alpha \meet \alpha'}}
   }
   %
   \and
   %
   \inferrule*[
      lab={\ruleName{$\evalFwdS$-cons}},
   ]
   {
      \evalFwd{\rho}{e}{\alpha'}{T}{v}
      \\
      \evalFwd{\rho}{e'}{\alpha'}{U}{v'}
   }
   {
      \evalFwd{\rho}
              {\annCons{e}{e'}{\alpha}}
              {\alpha'}
              {\trCons{T}{U}}
              {\annCons{v}{v'}{\alpha \meet \alpha'}}
   }
   %
   \and
   %
   \inferrule*[
      lab={\ruleName{$\evalFwdS$-apply-prim}}
   ]
   {
      \evalFwdEq{\rho}{e}{\alpha}{T}{\exPrimOp{\phi}{\vec{v}}}
      \\
      \evalFwd{\rho}{e'}{\alpha}{U}{u}
   }
   {
      \evalFwd{\rho}
              {\exApp{e}{e'}}
              {\alpha}
              {\trAppPrim{T}{\phi}{\vec{n}}{U}{\exInt{m}}}
              {\primFwd{\phi}{\vec{n} \concat m}(\vec{v} \concat u)}
   }
   %
   \and
   %
   \inferrule*[
      lab={\ruleName{$\evalFwdS$-apply}},
      width={3in}
   ]
   {
      \evalFwdEq{\rho}{e}{\alpha}{T}{\exClosure{\rho_1}{h}{\sigma}}
      \\
      \rho_1, h \closeDefsR \rho_2
      \\
      \evalFwd{\rho}{e'}{\alpha}{U}{v}
      \\
      \matchFwd{v}{\sigma}{w}{\rho_3}{e^\twoPrime}{\beta}
      \\
      \evalFwd{\rho_1 \concat \rho_2 \concat \rho_3}{e^\twoPrime}{\beta}{T'}{v'}
   }
   {
      \evalFwd{\rho}{\exApp{e}{e'}}{\alpha}{\trApp{T}{U}{w}{T'}}{v'}
   }
   %
   \and
   %
   \inferrule*[lab={
      \ruleName{$\evalFwdS$-let-rec}}
   ]
   {
      \rho, h' \closeDefsR \rho'
      \\
      \evalFwd{\rho \concat \rho'}{e}{\alpha}{T}{v}
   }
   {
      \evalFwd{\rho}{\exLetRecMutual{h'}{e}}{\alpha}{\trLetRecMutual{h}{T}}{v}
   }
\end{smathpar}
\vspace{1mm}

{\small \flushleft \shadebox{$\matchFwd{v}{\sigma}{w}{\rho}{\kappa}{\alpha}$}%
\hfill \textbfit{$v$ and $\sigma$ forward-analyse along $w$ to $\rho$ and $\kappa$, with argument availability $\alpha$}}
\begin{smathpar}
   \inferrule*[
      lab={\ruleName{$\matchFwdS$-eq}}
   ]
   {
      (v,\sigma) \eq (v',\sigma')
      \\
      \matchFwd{v'}{\sigma'}{w}{\rho}{\kappa}{\alpha}
   }
   {
      \matchFwd{v}{\sigma}{w}{\rho}{\kappa}{\alpha}
   }
   %
   \and
   %
   \inferrule*[
      lab={\ruleName{$\matchFwdS$-var}}
   ]
   {
      \strut
   }
   {
      \matchFwd{v}{\elimVar{x}{\kappa}}{\matchVar{x}}{\bind{x}{v}}{\kappa}{\TT}
   }
   %
   \and
   %
   \inferrule*[
      lab={\ruleName{$\matchFwdS$-true}}
   ]
   {
      \strut
   }
   {
      \matchFwd{\annTrue{\alpha}}
               {\elimBool{\kappa}{\kappa'}}
               {\matchTrue}
               {\seqEmpty}{\kappa}{\alpha}
   }
   %
   \and
   %
   \inferrule*[
      lab={\ruleName{$\matchFwdS$-false}}
   ]
   {
      \strut
   }
   {
      \matchFwd{\annFalse{\alpha}}
               {\elimBool{\kappa}{\kappa'}}
               {\matchFalse}
               {\seqEmpty}{\kappa'}{\alpha}
   }
   %
   \and
   %
   \inferrule*[
      lab={\ruleName{$\matchFwdS$-unit}}
   ]
   {
      \strut
   }
   {
      \matchFwd{\annot{\exRecEmpty}{\alpha}}
               {\elimRecEmpty{\kappa}}
               {\matchRecEmpty}
               {\seqEmpty}
               {\kappa}
               {\alpha}
   }
   %
   \and
   %
   \inferrule*[
      lab={\ruleName{$\matchFwdS$-record}}
   ]
   {
      \matchFwd{\annRec{\vec{\bind{x}{v}}}{\FF}}
               {\elimRec{\vec{x}}{\sigma}}
               {\matchRec{\vec{\bind{x}{w}}}}
               {\rho}
               {\sigma'}
               {\beta}
      \\
      \matchFwd{u}{\sigma'}{w}{\rho'}{\kappa}{\beta'}
   }
   {
      \matchFwd{\annRec{\vec{\bind{x}{v}} \concat \bind{y}{u}}{\alpha}}
               {\elimRec{\vec{x} \concat y}{\sigma}}
               {\matchRec{\vec{\bind{x}{w}} \concat \bind{y}{w'}}}
               {\rho \concat \rho'}
               {\kappa}
               {\alpha \meet \beta \meet \beta'}
   }
   %
   \and
   %
   \inferrule*[
      lab={\ruleName{$\matchFwdS$-nil}}
   ]
   {
      \strut
   }
   {
      \matchFwd{\annNil{\alpha}}
               {\elimList{\kappa}{\sigma'}}
               {\matchNil}
               {\seqEmpty}{\kappa}{\alpha}
   }
   %
   \and
   %
   \inferrule*[
      lab={\ruleName{$\matchFwdS$-cons}}
   ]
   {
      \matchFwd{v}{\sigma}{w}{\rho}{\tau}{\beta}
      \\
      \matchFwd{v'}{\tau}{w'}{\rho'}{\kappa}{\beta'}
   }
   {
      \matchFwd{\annCons{v}{v'}{\alpha}}
               {\elimList{\kappa}{\sigma}}
               {\matchCons{w}{w'}}
               {\rho \concat \rho'}{\kappa}{\alpha \meet \beta \meet \beta'}
   }
\end{smathpar}
\caption{Forward data dependency (Boolean cases for $\evalFwdR{T}$ omitted)}
\label{fig:data-dependencies:fwd}
\end{figure}

\begin{figure}
{\small \flushleft \shadebox{$\evalBwd{v}{T}{\rho}{e}{\alpha}$}%
\hfill \textbfit{$v$ backward-analyses along $T$ to $\rho$ and $e$, with argument demand $\alpha$}}
\begin{smathpar}
   \inferrule*[
      lab={\ruleName{$\evalBwdS$-eq}}
   ]
   {
      v \eq v'
      \\
      \evalBwd{v'}{T}{\rho}{e}{\alpha}
   }
   {
      \evalBwd{v}{T}{\rho}{e}{\alpha}
   }
   %
   \and
   %
   \inferrule*[
      lab={\ruleName{$\evalBwdS$-var}},
      right={$\envLookupBwd{\rho'}{\rho}{\bind{x}{v}}$}
   ]
   {
      \strut
   }
   {
      \evalBwd{v}{\trVar{x}{\rho}}{\rho'}{\exVar{x}}{\FF}
   }
   %
   \and
   %
   \inferrule*[
      lab={\ruleName{$\evalBwdS$-lambda}}
   ]
   {
      \strut
   }
   {
      \evalBwd{\exClosure{\rho}{\seqEmpty}{\sigma}}
              {\trLambda{\sigma'}}
              {\rho}
              {\exLambda{\sigma}}
              {\FF}
   }
   %
   \and
   %
   \inferrule*[
      lab={\ruleName{$\evalBwdS$-int}}
   ]
   {
      \strut
   }
   {
      \evalBwd{\annInt{n}{\alpha}}
              {\trInt{n}{\rho}}
              {\hole_{\rho}}
              {\annInt{n}{\alpha}}
              {\alpha}
   }
   %
   \and
   %
   \inferrule*[lab={\ruleName{$\evalBwdS$-record}}]
   {
      \evalBwd{v_i}{T_i}{\rho_i}{e_i}{\alpha_i'}
      \quad
      (\forall i \numleq \length{\vec{x}})
   }
   {
      \evalBwd{\annRec{\vec{\bind{x}{v}}}{\alpha}}
              {\trRec{\vec{\bind{x}{T}}}}
              {\bigjoin\vec{\rho}}
              {\annRec{\vec{\bind{x}{e}}}{\alpha}}
              {\alpha \join \bigjoin\vec{\alpha}'}
   }
   %
   \and
   %
   \inferrule*[lab={\ruleName{$\evalBwdS$-project}}]
   {
      \evalBwd{\annRec{\mapUpdate{\vec{\bind{x}{\hole}}}{x_i}{v}}{\FF}}
              {T}
              {\rho}
              {e}
              {\alpha}
   }
   {
      \evalBwd{v}
              {\trRecProj{T}{\vec{x}}{x_i}}
              {\rho}
              {\exRecProj{e}{x_i}}
              {\alpha}
   }
   %
   \and
   %
   \inferrule*[
      lab={\ruleName{$\evalBwdS$-nil}}
   ]
   {
      \strut
   }
   {
      \evalBwd{\annNil{\alpha}}
              {\trNil{\rho}}
              {\hole_{\rho}}
              {\annNil{\alpha}}
              {\alpha}
   }
   %
   \and
   %
   \inferrule*[
      lab={\ruleName{$\evalBwdS$-cons}}
   ]
   {
      \evalBwd{v}{T}{\rho}{e}{\alpha}
      \\
      \evalBwd{v'}{U}{\rho'}{e'}{\alpha'}
   }
   {
      \evalBwd{\annCons{v}{v'}{\beta}}
              {\trCons{T}{U}}
              {\rho \join \rho'}
              {\annCons{e}{e'}{\beta}}
              {\beta \join \alpha \join \alpha'}
   }
   %
   \and
   %
   \inferrule*[
      lab={\ruleName{$\evalBwdS$-apply-prim}},
      right={$\primBwd{\phi}{\vec{n} \concat m}(v) = \vec{u} \concat u'$}
   ]
   {
      \evalBwd{\exPrimOp{\phi}{\vec{u}}}{T}{\rho}{e_1}{\alpha}
      \\
      \evalBwd{u'}{U}{\rho'}{e_2}{\alpha'}
   }
   {
      \evalBwd{v}
              {\trAppPrim{T}{\phi}{\vec{n}}{U}{\exInt{m}}}
              {\rho \join \rho'}
              {\exApp{e_1}{e_2}}
              {\alpha \join \alpha'}
   }
   %
   \and
   %
   \inferrule*[
      lab={\ruleName{$\evalBwdS$-apply}},
      width={3.25in}
   ]
   {
      \evalBwd{v}{T'}{\rho_1 \concat \rho_2 \concat \rho_3}{e}{\beta}
      \\
      \matchBwd{\rho_3}{e}{\beta}{w}{v'}{\sigma}
      \\
      \evalBwd{v'}{U}{\rho}{e_2}{\alpha}
      \\
      \rho_2 \closeDefsBwdR \rho_1', h
      \\
      \evalBwd{\exClosure{\rho_1 \join \rho_1'}{h}{\sigma}}{T}{\rho'}{e_1}{\alpha'}
   }
   {
      \evalBwd{v}{\trApp{T}{U}{w}{T'}}{\rho \join \rho'}{\exApp{e_1}{e_2}}{\alpha \join \alpha'}
   }
   %
   \and
   %
   \inferrule*[
      lab={\ruleName{$\evalBwdS$-let-rec}}
   ]
   {
      \evalBwd{v}{T}{\rho \concat \rho_1}{e}{\alpha}
      \\
      \rho_1 \closeDefsBwdR \rho', h'
   }
   {
      \evalBwd{v}{\trLetRecMutual{h}{T}}{\rho \join \rho'}{\exLetRecMutual{h'}{e}}{\alpha}
   }
\end{smathpar}
\vspace{1mm}

{\small \flushleft \shadebox{$\matchBwd{\rho}{\kappa}{\alpha}{w}{v}{\sigma}$}%
\hfill \textbfit{$\rho$ and $\kappa$, with argument demand $\alpha$, backward-analyse along $w$ to $v$ and $\sigma$}}
\begin{smathpar}
   \inferrule*[lab={\ruleName{$\matchBwdS$-var}}]
   {
      \strut
   }
   {
      \matchBwd{\bind{x}{v}}{\kappa}{\alpha}{\matchVar{x}}{v}{\elimVar{x}{\kappa}}
   }
   %
   \and
   %
   \inferrule*[lab={\ruleName{$\matchBwdS$-unit}}]
   {
      \strut
   }
   {
      \matchBwd{\seqEmpty}
               {\kappa}
               {\alpha}
               {\matchRecEmpty}
               {\annot{\exRecEmpty}{\alpha}}
               {\elimRecEmpty{\kappa}}
   }
   %
   \and
   %
   \inferrule*[lab={\ruleName{$\matchBwdS$-nil}}]
   {
      \strut
   }
   {
      \matchBwd{\seqEmpty}{\kappa}{\alpha}{\matchNil}{\annNil{\alpha}}{\elimList{\kappa}{\hole}}
   }
   %
   \and
   %
   \inferrule*[lab={\ruleName{$\matchBwdS$-record}}]
   {
      \matchBwd{\rho'}{\kappa}{\alpha}{w'}{u}{\sigma}
      \\
      \matchBwd{\rho}
               {\sigma}
               {\alpha}
               {\matchRec{\vec{\bind{x}{w}}}}
               {\annRec{\vec{\bind{x}{v}}}{\beta}}
               {\tau}
   }
   {
      \matchBwd{\rho \concat \rho'}
               {\kappa}
               {\alpha}
               {\matchRec{\vec{\bind{x}{w}} \concat \bind{y}{w'}}}
               {\annRec{\vec{\bind{x}{v}} \concat \bind{y}{u}}{\alpha}}
               {\elimRec{\vec{x} \concat y}{\tau}}
   }
   %
   \and
   %
   \inferrule*[lab={\ruleName{$\matchBwdS$-cons}}]
   {
      \matchBwd{\rho'}{\kappa}{\alpha}{w'}{v'}{\sigma}
      \\
      \matchBwd{\rho}{\sigma}{\alpha}{w}{v}{\tau}
   }
   {
      \matchBwd{\rho \concat \rho'}
               {\kappa}
               {\alpha}
               {\matchCons{w}{w'}}
               {\annCons{v}{v'}{\alpha}}
               {\elimList{\hole}{\tau}}
   }
   %
   \and
   %
   \inferrule*[lab={\ruleName{$\matchBwdS$-true}}]
   {
      \strut
   }
   {
      \matchBwd{\seqEmpty}{\kappa}{\alpha}{\matchTrue}{\annTrue{\alpha}}{\elimBool{\kappa}{\hole}}
   }
   %
   \and
   %
   \inferrule*[lab={\ruleName{$\matchBwdS$-false}}]
   {
      \strut
   }
   {
      \matchBwd{\seqEmpty}{\kappa}{\alpha}{\matchFalse}{\annFalse{\alpha}}{\elimBool{\hole}{\kappa}}
   }
\end{smathpar}
\vspace{1mm}

{\small\flushleft \shadebox{$\envLookupBwd{\rho'}{\rho}{\bind{x}{v}}$}%
\hfill \textbfit{$\bind{x}{v}$ backward-analyses along $\rho$ to $\rho'$}}
\begin{smathpar}
   \inferrule*[
      lab={\ruleName{$\envLookupBwdS$-head}}
   ]
   {
      \strut
   }
   {
      \envLookupBwd{(\hole_{\rho} \concat \bind{x}{u})}
                   {\rho \concat \bind{x}{v}}
                   {\bind{x}{u}}
   }
   %
   \and
   %
   \inferrule*[
      lab={\ruleName{$\envLookupBwdS$-tail}},
      right={$x \neq y$}
   ]
   {
      \envLookupBwd{\rho'}{\rho}{\bind{x}{u}}
   }
   {
      \envLookupBwd{(\rho' \concat \bind{y}{\hole})}
                   {\rho \concat \bind{y}{v}}
                   {\bind{x}{u}}
   }
\end{smathpar}
\vspace{1mm}

{\small\flushleft\shadebox{$\rho \closeDefsBwdR \rho', h$}%
\hfill \textbfit{$\rho$ backward-analyses to $\rho'$ and $h$}}
\begin{salign}
   \vec{\bind{x}{v}}
   &\closeDefsBwdR
   (\bigjoin\vec{\rho}, \vec{\bind{x}{\sigma}} \join {\bigjoin\vec{h}})
   &
   \text{ where }
   v_i = \exClosure{\rho_i}{h_i}{\sigma_i}
\end{salign}

\caption{Backward data dependency (Boolean cases for $\evalBwdR{T}$ omitted)}
\label{fig:data-dependencies:bwd}
\end{figure}

\begin{figure}[H]
\small
$\begin{array}{lllll}
&&&
\textit{where}
\\
\primAppF{\phi}(\vec{v})
&=&
\exPrimOp{\phi}{\vec{v}}
&
\arity{\phi} \numgt \length{\vec{v}}
\\
\primAppF{\phi}(\vec{v})
&=&
\primFwd{\phi}(\vec{v})
&
\arity{\phi} = \length{\vec{n}}
\end{array}$
\caption{Auxiliary functions: forward and backward slicing}
\end{figure}


\begin{definition}[Hole environment]
Define $\hole_{\vec{\bind{x}{v}}} = \vec{\bind{x}{\hole}}$
\end{definition}

\begin{lemma}[Least environment for $\rho$]
\label{lem:core-language:hole-env}If $\vdash \rho: \Gamma$ then $\hole_{\rho} \leq \rho$.
\end{lemma}

\begin{definition}[Forward and backward functions for environment lookup]
   Suppose $\envLookup{\rho}{x}{v}$. Then define $\envLookupFwdF{\rho,x}: \Below{\rho} \to \Below{v}$ and $\envLookupBwdF{\rho,x}: \Below{v} \to \Below{\rho}$ to be $\bind{x}{-}\envLookupR$ and $\envLookupBwdR{\rho}\bind{x}{-}$ restricted to $\Below{\rho}$ and $\Below{v}$ respectively.
\end{definition}

\begin{lemma}[Galois connection for environment]
\label{lem:core-language:env-get-put}Suppose $\envLookup{\rho}{x}{v}$.
\begin{enumerate}
   \item \label{lem:core-language:env-get-put:1} $\envLookupFwdF{\rho,x}(\envLookupBwdF{\rho,x}(v)) \geq v$.
   \item \label{lem:core-language:env-get-put:2} $\envLookupBwdF{\rho,x}(\envLookupFwdF{\rho,x}(\rho')) \leq \rho'$.
\end{enumerate}
\end{lemma}

\begin{definition}
   \label{def:core-language:closeDefs-bwd}
   Define the relation $\closeDefsBwdR$ as given in \figref{core-language:slicing:eval-aux}.
\end{definition}

\begin{definition}[Forward and backward functions for recursive bindings]
   Suppose $\rho, h \closeDefsR \rho'$. Define $\closeDefsFwdF{\rho,h}: \Below{(\rho, h)} \to \Below{\rho'}$ and $\closeDefsBwdF{\rho,h}: \Below{\rho'} \to \Below{(\rho, h)}$ to be $\closeDefsR$ and $\closeDefsBwdR$ restricted to $\Below{(\rho, h)}$ and $\Below{\rho'}$ respectively.
\end{definition}

\begin{theorem}[Galois connection for recursive bindings]
\label{thm:core-language:closeDefs:gc}
   Suppose $\rho, h \closeDefsR \rho'$.  Then $\closeDefsFwdF{\rho,h} \adjoint \closeDefsBwdF{\rho,h}$.
\end{theorem}

\begin{definition}[Forward and backward functions for evaluation]
   Suppose $T: \rho, e \evalR v$. Define $\evalFwdF{T}: \Below{(\rho, e, \TT)} \to \Below{v}$ and $\evalBwdF{T}: \Below{v} \to \Below{(\rho, e, \TT)}$ to be $\evalFwdR{T}$ and $\evalBwdR{T}$ restricted to $\Below{(\rho, e, \TT)}$ and $\Below{v}$ respectively.
\end{definition}

\begin{theorem}[Galois connection for evaluation]
\label{thm:core-language:eval:gc}
   Suppose $T: \rho, e \evalFwdS v$.  Then $\evalFwdF{T} \adjoint \evalBwdF{T}$.
\end{theorem}

\begin{lemma}[Determinism of pattern-matching]
   Suppose $v, \sigma \matchFwdR{w} \rho, \kappa$ and $v', \sigma' \matchFwdR{w} \rho', \kappa'$. If $(v, \sigma) \eq (v', \sigma)$ then $(\rho, \kappa) \eq (\rho', \kappa')$.
\end{lemma}

\begin{definition}[Forward and backward functions for pattern-matching]
   Suppose $w: v, \sigma \matchR \rho, \kappa$. Define $\matchFwdF{w}: \Below{(v,\sigma,\TT)} \to \Below{(\rho,\kappa)}$ and $\matchBwdF{w}: \Below{(\rho,\kappa)} \to \Below{(v,\sigma,\TT)}$ to be $\matchFwdR{w}$ and $\matchBwdR{w}$ domain-restricted to $\Below{(v,\sigma,\TT)}$ and $\Below{(\rho,\kappa)}$ respectively.
\end{definition}

\begin{theorem}[Galois connection for pattern-matching]
\label{thm:core-language:match:gc}
   Suppose $w: v, \sigma \matchFwdS \rho, \kappa$.  Then $\matchFwdF{w} \adjoint \matchBwdF{w}$.
\end{theorem}


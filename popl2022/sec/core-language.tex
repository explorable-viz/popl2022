\section{A core language for fine-grained data dependencies}
\label{sec:core-language}

\subsection{Syntax and typing}
\begin{figure}[H]
\begin{syntaxfig}
\mbox{Type}
&
A, B
&
::=
&
\tyBool
&
\text{Booleans}
\\
&&&
\tyInt
&
\text{integers}
\\
&&&
\tyProd{A}{B}
&
\text{pairs}
\\
&&&
\tyList{A}
&
\text{lists}
\\
&&&
\tyVec{A}
&
\text{vectors}
\\
&&&
\tyFun{A}{B}
&
\text{functions}
\\[2mm]
\mbox{Identifier}
&
x, y
\\[2mm]
\mbox{Typing context}
&
\Gamma, \Delta
&
::=
&
\seq{\bind{z}{A}}
\end{syntaxfig}
\caption{Typing contexts and types}
\end{figure}


The language consists of primitive types such as booleans, integers, and functions, but additionally lists, vectors, and records as data structures in order to better demonstrate more interesting relationships which can occur between the program structure and output. The notation $\seq{\bind{x}{A}}$ found in the typing context and records represents a sequence of bindings from variables $x$ to types $A$ which can be unzipped to yield $\seq{x}$ and $\seq{A}$.

\begin{figure}
{\small
\begingroup
\renewcommand*{\arraystretch}{1}
\begin{minipage}[t]{0.5\textwidth}
\begin{tabularx}{\textwidth}{rL{2cm}L{3cm}}
%\rowcolor{verylightgray}
&\textbfit{Type}&
\\
$A, B ::=$
&
$\tyBool$
&
Booleans
\\
&
$\tyInt$
&
integers
\\
&
$\tyRec{\vec{\bind{x}{A}}}$
&
records
\\
&
$\tyList{A}$
&
lists
\\
&
$\tyFun{A}{B}$
&
functions
\\[2mm]
$\Gamma, \Delta ::=$
&
$\seq{\bind{x}{A}}$
&
typing context
\\[2mm]
&\textbfit{Term}&
\\
$e ::=$
&
$\exTrue \mid \exFalse$
&
Boolean
\\
&
$\exInt{n}$
&
integer
\\
&
$\exVar{x}$
&
variable
\\
&
$\exAppPrim{\phi}{\vec{e}}$
&
primitive application
\\
&
$\exApp{e}{e'}$
&
application
\\
&
$\exNil \mid \exCons{u}{v}$
&
list
\\
&
$\exRec{\vec{\bind{x}{e}}}$
&
record
\\
&
$\exRecProj{e}{x}$
&
record projection
\\
&
$\exLambda{\sigma}$
&
anonymous function
\\
&
$\exLetRecMutual{h}{e}$
&
recursive let
\\[2mm]
$h ::=$
&
$\seq{\bind{x}{\sigma}}$
&
recursive functions
\end{tabularx}
\end{minipage}%
\begin{minipage}[t]{0.5\textwidth}
\begin{tabularx}{\textwidth}{rL{2.6cm}L{2.9cm}}
&\textbfit{Continuation type}&
\\
$K ::=$
&
$A$
&
term
\\
&
$\elimTy{A}{K}$
&
eliminator
\\[2mm]
&\textbfit{Continuation}&
\\
$\kappa ::=$
&
$e$
&
term
\\
&
$\sigma$
&
eliminator
\\[2mm]
&\textbfit{Eliminator}&
\\
$\sigma, \tau ::=$
&
$\elimVar{x}{\kappa}$
&
variable
\\
&
$\elimBool{\kappa}{\kappa'}$
&
Boolean
\\
&
$\elimRec{\vec{x}}{\kappa}$
&
record
\\
&
$\elimList{\kappa}{\sigma}$
&
list
\\[2mm]
&\textbfit{Value}&
\\
$u, v ::=$
&
$\exTrue \mid \exFalse$
&
Boolean
\\
&
$\exInt{n}$
&
integer
\\
&
$\exNil \mid \exCons{u}{v}$
&
list
\\
&
$\exRec{\vec{\bind{x}{v}}}$
&
record
\\
&
$\exClosure{\rho}{h}{\sigma}$
&
closure
\\[2mm]
$\rho ::=$
&
$\seq{\bind{x}{v}}$
&
environment
\\[2mm]
%$x, y$
%&
%&
%identifier
%\\
%$i, j$
%&&
%positive integer
%\\[2mm]
\end{tabularx}
\end{minipage}
\endgroup
}
\caption{Syntax of core language}
\label{fig:core-language:syntax}
\end{figure}


The language includes basic terms $e$: booleans $\exTrue$ and $\exFalse$, integers $n$, empty $\exNil$ and non-empty lists $\exCons{e}{e'}$, anonymous functions $\exLambda{\sigma}$, and application $\exApp{e}{e'}$. Recursive let-bindings $\exLetRecMutual{h}{e}$ are defined through declaring a sequence of mutually recursive functions $h = \seq{\bind{x}{\sigma}}$ and body $e$. In addition, we provide vectors $\exVec{e}{x}{e'}$ which are constructed in a list-comprehension style, records $\exRec{\vec{\bind{x}{v}}}$ as a sequence of bindings between field names and values, and various operations over these. Holes $\hole$ can be used to substitute for terms which have been considered of no interest.

Eliminators $\sigma$ express pattern matching against terms $e$ to result in program continuations $\kappa$ which can be either terms or further eliminators. Rather than one elimination-form per type, we provide a single pattern-matching elimination form called an eliminator \cite{hinze00}. We provide eliminators for matching against variables, booleans, records, and lists.An common alternative to this is to use case-expressions, however, real-world functional languages have rich pattern-matching features such as piecewise definitions, view patterns, pattern synonyms, etc with non-trivial exhaustiveness-checking rules. Instead, we want an efficient, simple core pattern-matching feature that is total by construction and can serve as an elaborate target for more advanced features -- in section \ref{sec:surface-language} we show how to desugar piecewise definitions that satisfy certain well-formedness conditions into eliminators. 

\begin{figure}
\begin{syntaxfig}
\mbox{Value}
&
v
&
::=
&
\bot
&
\text{absent}
\\
&&&
\exTrue \mid \exFalse
&
\text{Boolean}
\\
&&&
\exInt{n}
&
\text{integer}
\\
&&&
\exClosureRec{\rho}{f}{\sigma}
&
\text{closure}
\\
&&&
\exPair{v}{v'}
&
\text{pair}
\\
&&&
\exPairDel{v}{v'}
\\
&&&
\exNil
&
\text{nil}
\\
&&&
\exCons{v}{v'}
&
\text{cons}
\\
&&&
\exConsDel{v}{v'}
\\[2mm]
\mbox{Environment}
&
\rho
&
::=
&
\envEmpty
&
\text{empty}
\\
&&&
\envExtend{\rho}{x_{\alpha}}{v}
&
\text{extend}
\end{syntaxfig}
\caption{Values and environments}
\end{figure}


The environment $\rho$ of a program consists of bindings between variables and values. The values $v$ which terms can evaluate to include: holes $\hole$, integers $n$, records $\exRec{\vec{\bind{x}{v}}}$, empty $\exNil$ and non-empty lists $\exCons{u}{v}$ of values, and vectors $\exVecVal{\vec{v}}{j}$ of values $\vec{v}$ with fixed-length $j$. Additionally, we have partially applied primitives $\exPrimOp{\phi}{\vec{v}}$ where $\phi$ is a primitive operation applied to an incomplete sequence of values $\vec{v}$, and function closures $\exClosure{\rho}{h}{\sigma}$ consisting of an environment $\rho$, a sequence of mutually recursive functions $h$, and eliminator $\sigma$. 

\begin{figure}[H]
{\small
\begingroup
\renewcommand*{\arraystretch}{1}
\begin{minipage}[t]{0.5\textwidth}
\begin{tabularx}{\textwidth}{rL{2cm}L{3cm}}
&\textbfit{Trace}&
\\
$T, U ::=$
&
$\trVar{x}{\rho}$
&
variable
\\
&
$\trTrue{\rho} \mid \trFalse{\rho}$
&
Boolean
\\
&
$\trInt{n}{\rho}$
&
integer
\\
&
$\trRec{\vec{\bind{x}{T}}}$
&
record
\\
&
$\trRecProj{T}{\vec{\bind{x}{v}}}{y}$
&
record projection
\\
&
$\trNil{\rho} \mid \trCons{T}{U}$
&
list
\\
&
$\trLambda{\sigma}$
&
anonymous function
\\
&
$\trApp{T}{U}{w}{T'}$
&
application
\end{tabularx}
\end{minipage}%
\begin{minipage}[t]{0.5\textwidth}
\begin{tabularx}{\textwidth}{rL{2.5cm}L{3cm}}
\\
&
$\trAppPrimNew{\phi}{U}{\exInt{n}}$
&
primitive application
\\
&
$\trLetRecMutual{h}{T}$
&
recursive let
\\[2mm]
&\textbfit{Match}&
\\
$w ::=$
&
$\matchVar{x}$
&
variable
\\
&
$\matchTrue \mid \matchFalse$
&
Boolean
\\
&
$\matchRec{\vec{\bind{x}{w}}}$
&
record
\\
&
$\matchNil \mid \matchCons{w}{w'}$
&
list
\\[4mm]
\end{tabularx}
\end{minipage}
\endgroup
}
\caption{Syntax of traces and matches}
\label{fig:core-language:syntax-trace}
\end{figure}


Traces $T$ closely mirror the terms of our language, and are a means of recording the details of program execution and providing an explanation of how a result was computed. These are computed during forward slicing, and allows slicing to then proceed backwards afterwards. Matches $w$ are specifically used to record which pattern was matched against in an eliminator and hence which branch was taken during program execution. 

\begin{figure}[H]
\caption{Typing rules for environments and values}
\label{core-typing-value}
\end{figure}


The typing rule for environments $\vdash \rho: \Gamma$ states that if all the values $v_i$ of the environment are well typed under the empty typing context, then the environment itself is well-typed. Similarly, the typing rule for mutually recursive functions states that if all functions $\sigma_i$ are well-typed under the typing context $\Gamma \concat \Delta$, then the environment of mutually recursive functions is well-typed under context $\Gamma$ with the type $\Delta$.

The typing rules for values $\vdash v: A$ are self-explanatory for integers, booleans, records, lists, and vectors. Given a primitive operation $\phi$ which takes $i+j$ integer arguments to output an integer, when partially applied to a sequence of $i$ integers this produces a function which takes $j$ more integers.

\begin{figure}[H]
\flushleft \shadebox{$\Gamma \vdash e: A$}
\begin{smathpar}
\inferrule*[right={$x : A \in \Gamma$}]
{
   \strut
}
{
   \Gamma \vdash \exVar{x}: A
}
%
\and
%
\inferrule*[right={$\primOp: A \in \Gamma$}]
{
   \strut
}
{
   \Gamma \vdash \exOp{\primOp}: A
}
%
\and
%
\inferrule*
{
   \strut
}
{
   \Gamma \vdash n: \tyInt
}
%
\and
%
\inferrule*
{
   \strut
}
{
   \Gamma \vdash \exTrue: \tyBool
}
%
\and
%
\inferrule*
{
   \strut
}
{
   \Gamma \vdash \exFalse: \tyBool
}
%
\and
%
\inferrule*
{
   \Gamma \vdash u: A
   \\
   \Gamma \vdash v: B
}
{
   \Gamma \vdash \exPair{u}{v}: \tyProd{A}{B}
}
%
\and
%
\inferrule*[right={$\primOp: \tyFun{\tyInt}{\tyFun{\tyInt}{\tyInt}} \in \Gamma$}]
{
   \Gamma \vdash e: \tyInt
   \\
   \Gamma \vdash e': \tyInt
}
{
   \Gamma \vdash \exBinaryApp{e}{\primOp}{e'}: \tyInt
}
%
\and
%
\inferrule*
{
   \Gamma \vdash \sigma : \tyFun{A}{B}
}
{
   \Gamma \vdash \exLambda{\sigma} : \tyFun{A}{B}
}
%
\and
%
\inferrule*
{
   \Gamma \vdash e: \tyFun{A}{B}
   \\
   \Gamma \vdash e': A
}
{
   \Gamma \vdash \exApp{e}{e'}: B
}
%
\and
%
\inferrule*
{
   \strut
}
{
   \Gamma \vdash \exNil: \tyList{A}
}
%
\and
%
\inferrule*
{
   \Gamma \vdash u: A
   \\
   \Gamma \vdash v: \tyList{A}
}
{
   \Gamma \vdash (\exCons{u}{v}): \tyList{A}
}
%
\and
%
\inferrule*
{
   \Gamma \vdash e': \tyProd{\tyInt}{\tyInt}
   \\
   \Gamma \concat \bind{x}{\tyInt} \concat \bind{y}{\tyInt} \vdash e: A
}
{
   \Gamma \vdash \exMatrix{e}{x}{y}{e'}: \tyMatrix{A}
}
%
\and
%
\inferrule*
{
   \Gamma \vdash e: \tyMatrix{A}
   \\
   \Gamma \vdash e': \tyProd{\tyInt}{\tyInt}
}
{
   \Gamma \vdash \exMatrixAccess{e}{e'}: A
}
%
\and
%
\inferrule*
{
   \Gamma \vdash e: \tyMatrix{A}
}
{
   \Gamma \vdash \exMatrixDims{e}: \tyProd{\tyInt}{\tyInt}
}
%
\and
%
\inferrule*
{
   \Gamma \vdash h: \Delta
   \\
   \Gamma \concat \Delta \vdash e: A
}
{
   \Gamma \vdash \exLetRecMutual{h}{e}: A
}
\end{smathpar}

\vspace{5pt}
\flushleft \shadebox{$\Gamma \vdash \sigma: \tyFun{A}{B}$}
\begin{smathpar}
\inferrule*
{
   \Gamma \concat \bind{x}{A} \vdash \kappa: B
}
{
   \Gamma \vdash (\elimVar{x}{\kappa}): \tyFun{A}{B}
}
%
\and
%
\inferrule*
{
   \Gamma \vdash \kappa: A
   \\
   \Gamma \vdash \kappa': A
}
{
   \Gamma \vdash \elimBool{\kappa}{\kappa'}: \tyFun{\tyBool}{A}
}
%
\and
%
\inferrule*
{
   \Gamma \vdash \sigma: \tyFun{A}{\tyFun{A'}{B}}
}
{
   \Gamma \vdash \elimProd{\sigma}: \tyFun{\tyProd{A}{A'}}{B}
}
%
\and
%
\inferrule*
{
   \Gamma \vdash \kappa: B
   \\
   \Gamma \vdash \sigma: \tyFun{A}{\tyFun{\tyList{A}}{B}}
}
{
   \Gamma \vdash \elimList{\kappa}{\sigma}: \tyFun{\tyList{A}}{B}
}
\end{smathpar}
\caption{Typing rules for terms and eliminators}
\end{figure}


The judgement for eliminator typing rules, $\Gamma \vdash \sigma: \tyFun{A}{B}$, decomposes how the well-typedness of an eliminator decomposes into the types of its patterns and branches. 

\subsection{Evaluation}
{\small \flushleft \shadebox{$\explVal{w}{v, \sigma \matchR \rho, \kappa}$}%
\hfill \textbfit{$w$ witnesses that $\sigma$ matches $v$ and yields $\rho$ and $\kappa$}}
\begin{smathpar}
   \inferrule*[lab={\ruleName{$\matchR$-true}}]
   {
      \strut
   }
   {
      \explVal{\matchTrue}{\exTrue, \elimBool{\kappa}{\kappa'} \matchR \seqEmpty, \kappa}
   }
   %
   \and
   %
   \inferrule*[lab={\ruleName{$\matchR$-false}}]
   {
      \strut
   }
   {
      \explVal{\matchFalse}{\exFalse, \elimBool{\kappa}{\kappa'} \matchR \seqEmpty, \kappa'}
   }
   %
   \and
   %
   \inferrule*[lab={\ruleName{$\matchR$-var}}]
   {
      \strut
   }
   {
      \explVal{\matchVar{x}}{v, \elimVar{x}{\kappa} \matchR \bind{x}{v}, \kappa}
   }
   %
   \and
   %
   \inferrule*[lab={\ruleName{$\matchR$-nil}}]
   {
      \strut
   }
   {
      \explVal
         {\matchNil}
         {\exNil, \elimList{\kappa}{\sigma} \matchR \seqEmpty, \kappa}
   }
   %
   \and
   %
   \inferrule*[
      lab={\ruleName{$\matchR$-unit}}
   ]
   {
      \strut
   }
   {
      \explVal{\exRecEmpty}
              {\seqEmpty, \elimRecEmpty{\kappa} \matchR \seqEmpty, \kappa}
   }
   %
   \and
   %
   \inferrule*[lab={\ruleName{$\matchR$-cons}}]
   {
      \explVal{w}{v, \sigma \matchR \rho, \tau}
      \\
      \explVal{w'}{v', \tau \matchR \rho', \kappa'}
   }
   {
      \explVal
         {(\matchCons{w}{w'})}
         {\exCons{v}{v'}, \elimList{\kappa}{\sigma}
         \matchR
         \rho \concat \rho', \kappa'}
   }
   %
   \and
   %
   \inferrule*[
      lab={\ruleName{$\matchR$-record}}
   ]
   {
      \explVal{\matchRec{\vec{\bind{x}{w}}}}{\vec{\bind{x}{v}}, \sigma \matchR \rho, \sigma'}
      \\
      \explVal{w'}{u, \sigma' \matchR \rho', \kappa}
   }
   {
      \explVal{\exRec{\vec{\bind{x}{w}} \concat \bind{y}{w'}}}
              {\exRec{\vec{\bind{x}{v}} \concat \bind{y}{u}},
               \elimRec{\vec{x} \concat y}{\sigma}
               \matchR
               \rho \concat \rho',
               \kappa}
   }
\end{smathpar}

{\small \flushleft \shadebox{$\explVal{T}{\rho, e \evalR v}$}%
\hfill \textbfit{$T$ witnesses that $e$ evaluates to $v$ in $\rho$}}
\begin{smathpar}
   \inferrule*[
      lab={\ruleName{$\evalR$-var}}
   ]
   {
      \envLookup{\rho}{x}{v}
   }
   {
      \explVal{\trVar{x}{\rho}}{\rho, x \evalR v}
   }
   %
   \and
   %
   \inferrule*[lab={\ruleName{$\evalR$-lambda}}]
   {
      \strut
   }
   {
      \explVal
         {\trLambda{\sigma}}
         {\rho, \exLambda{\sigma} \evalR \exClosure{\rho}{\seqEmpty}{\sigma}}
   }
   %
   \and
   %
   \inferrule*[
      lab={\ruleName{$\evalR$-true}}
   ]
   {
      \strut
   }
   {
      \explVal{\trTrue{\rho}}{\rho, \exTrue \evalR \exTrue}
   }
   %
   \and
   %
   \inferrule*[
      lab={\ruleName{$\evalR$-false}}
   ]
   {
      \strut
   }
   {
      \explVal{\trFalse{\rho}}{\rho, \exFalse \evalR \exFalse}
   }
   %
   \and
   %
   \inferrule*[
      lab={\ruleName{$\evalR$-int}}
   ]
   {
      \strut
   }
   {
      \explVal{\trInt{n}{\rho}}{\rho, \exInt{n} \evalR \exInt{n}}
   }
   %
   \and
   %
   \inferrule*[lab={\ruleName{$\evalR$-record}}]
   {
      \explVal{T_i}{\rho, e_i \evalR v_i}
      \quad
      (\forall i \numleq \length{\vec{x}})
   }
   {
      \explVal{\trRec{\vec{\bind{x}{T}}}}
              {\rho, \exRec{\vec{\bind{x}{e}}} \evalR \exRec{\vec{\bind{x}{v}}}}
   }
   %
   \and
   %
   \inferrule*[
      lab={\ruleName{$\evalR$-project}}
   ]
   {
      \explVal{T}{\rho, e \evalR \exRec{\vec{\bind{x}{v}}}}
      \\
      \envLookup{\vec{\bind{x}{v}}}{y}{v'}
   }
   {
      \explVal{\trRecProj{T}{\vec{\bind{x}{v}}}{y}}
              {\rho, \exRecProj{e}{y} \evalR v'}
   }
   %
   \and
   %
   \inferrule*[
      lab={\ruleName{$\evalR$-nil}}
   ]
   {
      \strut
   }
   {
      \explVal{\trNil{\rho}}{\rho, \exNil \evalR \exNil}
   }
   %
   \and
   %
   \inferrule*[lab={\ruleName{$\evalR$-cons}}]
   {
      \explVal{T}{\rho, e \evalR v}
      \\
      \explVal{U}{\rho, e' \evalR v'}
   }
   {
      \explVal{\trCons{T}{U}}{\rho, \exCons{e}{e'} \evalR \exCons{v}{v'}}
   }
   %
   \and
   %
   \inferrule*[
      lab={\ruleName{$\evalR$-apply-prim}},
      right={$\phi \in \Int^j \to \Int$}
   ]
   {
      \explVal{U}{\rho, e_i \evalR \exInt{n}_i}
      \quad
      (\forall i \numleq j)
   }
   {
      \explVal{\trAppPrimNew{\phi}{U}{n}}
              {\rho, \exAppPrim{\phi}{\vec{e}} \evalR \exPrimOp{\phi}{\vec{n}}}
   }
   %
   \and
   %
   \inferrule*[
      lab={\ruleName{$\evalR$-let-rec}}
   ]
   {
      \rho, h \closeDefsR \rho'
      \\
      \explVal{T}{\rho \concat \rho', e \evalR v}
   }
   {
      \explVal{\trLetRecMutual{h}{T}}{\rho, \exLetRecMutual{h}{e} \evalR v}
   }
   %
   \and
   %
   \inferrule*[
      lab={\ruleName{$\evalR$-apply}},
      width=3in,
   ]
   {
      \explVal{T}{\rho, e \evalR \exClosure{\rho_1}{h}{\sigma'}}
      \\
      \rho_1, h \closeDefsR \rho_2
      \\
      \explVal{U}{\rho, e' \evalR v}
      \\
      \explVal{w}{v, \sigma' \matchR \rho_3, e^\twoPrime}
      \\
      \explVal{T'}{\rho_1 \concat \rho_2 \concat \rho_3, e^\twoPrime \evalR v'}
   }
   {
      \explVal{\trApp{T}{U}{w}{T'}}{\rho, \exApp{e}{e'} \evalR v'}
   }
\end{smathpar}

\begin{minipage}[t]{0.52\textwidth}%
\vspace{1mm}%
{\small \flushleft \shadebox{$\envLookup{\rho}{x}{v}$}%
\hfill \textbfit{$\bind{x}{v}$ is contained by $\rho$}}
\begin{smathpar}
   \inferrule*[
      lab={\ruleName{$\envLookupS$-head}}
   ]
   {
      \strut
   }
   {
      \envLookup{(\rho \concat \bind{x}{v})}{x}{v}
   }
   %
   \and
   %
   \inferrule*[
      lab={\ruleName{$\envLookupS$-tail}},
      right={$x \neq y$}
   ]
   {
      \envLookup{\rho}{x}{v}
   }
   {
      \envLookup{(\rho \concat \bind{y}{u})}{x}{v}
   }
\end{smathpar}
\end{minipage}%
\hfill
\begin{minipage}[t]{0.36\textwidth}%
\vspace{1mm}%
{\small \flushleft \shadebox{$\rho, h \closeDefsR \rho'$}%
\hfill \textbfit{$h$ generates $\rho'$ in $\rho$}}
\begin{smathpar}
   \inferrule*[
      lab={\ruleName{$\closeDefsR$-rec-defs}}
   ]
   {
      v_i = \exClosure{\rho}{\vec{\bind{x}{\sigma}}}{\sigma_i}
      \quad
      (\forall i \in \length{\vec{x}})
   }
   {
      \rho, \vec{\bind{x}{\sigma}}
      \closeDefsR
      \vec{\bind{x}{v}}
   }
\end{smathpar}
%\begin{salign}
%   \rho, \vec{\bind{x}{\sigma}}
%   &\closeDefsR
%   \vec{\bind{x}{v}}
%   & \text{where $v_i = \exClosure{\rho}{\vec{\bind{x}{\sigma}}}{\sigma_i}$}
%\end{salign}
\end{minipage}


\begin{definition}
   \label{def:core-language:closeDefs}
   Define the relation $\closeDefsR$ as given in \figref{core-language:eval-aux}.
\end{definition}

\subsection{Forward and backward Galois dependency}
\begin{figure}[H]
\flushleft \shadebox{$h \leq h'$}
\begin{smathpar}
   \inferrule*[
   ]
   {
      \sigma_i \leq \sigma'_i
      \quad
      (\forall i \numleq n)
   }
   {
      \seqRange{\bind{x_1}{\sigma_1}}{\bind{x_n}{\sigma_n}}
      \leq
      \seqRange{\bind{x_1}{\sigma'_1}}{\bind{x_n}{\sigma'_n}}
   }
   %
   \and
   %
   \inferrule*[
   ]
   {
      \sigma_i \leq \hole
      \quad
      (\forall i \numleq n)
   }
   {
      \seqRange{\bind{x_1}{\sigma_1}}{\bind{x_n}{\sigma_n}} \leq \hole
   }
\end{smathpar}

\vspace{5pt}
\flushleft \shadebox{$v \leq v'$}
\begin{smathpar}
   \inferrule*[
   ]
   {
      \strut
   }
   {
      \hole \leq v
   }
   %
   \and
   %
   \inferrule*[
      right={$\alpha \leq \alpha'$}
   ]
   {
      \strut
   }
   {
      \annInt{n}{\alpha} \leq \annInt{n}{\alpha'}
   }
   %
   \and
   %
   \inferrule*[
   ]
   {
      \strut
   }
   {
      \annInt{n}{\FF} \leq \hole
   }
   %
   \and
   %
   \inferrule*[
      right={$\alpha \leq \alpha'$}
   ]
   {
      \strut
   }
   {
      \annTrue{\alpha} \leq \annTrue{\alpha'}
   }
   %
   \and
   %
   \inferrule*[
   ]
   {
      \strut
   }
   {
      \annTrue{\FF} \leq \hole
   }
   %
   \and
   %
   \inferrule*[
      right={$\alpha \leq \alpha'$}
   ]
   {
      \strut
   }
   {
      \annFalse{\alpha} \leq \annFalse{\alpha'}
   }
   %
   \and
   %
   \inferrule*[
   ]
   {
      \strut
   }
   {
      \annFalse{\FF} \leq \hole
   }
   %
   \and
   %
   \inferrule*[
      right={$\alpha \leq \alpha'$}
   ]
   {
      v \leq u
      \\
      v' \leq u'
   }
   {
      \annPair{v}{v'}{\alpha} \leq \annPair{u}{u'}{\alpha'}
   }
   %
   \and
   %
   \inferrule*[
   ]
   {
      v \leq \hole
      \\
      v' \leq \hole
   }
   {
      \annPair{v}{v'}{\FF} \leq \hole
   }
   %
   \and
   %
   \inferrule*[
      right={$\alpha \leq \alpha'$}
   ]
   {
      \strut
   }
   {
      \annNil{\alpha} \leq \annNil{\alpha'}
   }
   %
   \and
   %
   \inferrule*[
   ]
   {
      \strut
   }
   {
      \annNil{\FF} \leq \hole
   }
   %
   \and
   %
   \inferrule*[
      right={$\alpha \leq \alpha'$}
   ]
   {
      v \leq u
      \\
      v' \leq u'
   }
   {
      \annCons{v}{v'}{\alpha} \leq \annCons{u}{u'}{\alpha'}
   }
   %
   \and
   %
   \inferrule*[
   ]
   {
      v \leq \hole
      \\
      v' \leq \hole
   }
   {
      \annCons{v}{v'}{\FF} \leq \hole
   }
   %
   \and
   %
   \inferrule*[
      right={$(\alpha,\beta,\gamma) \leq (\alpha',\beta',\gamma')$}
   ]
   {
      v_{i,j} \leq u_{i,j}
      \quad
      (\forall (i,j) \numleq (\ihat,\jhat))
   }
   {
      \annMatrix{v_{i,j}}{i}{j}{(\annInt{\ihat}{\beta},\annInt{\jhat}{\gamma})}{\alpha}
      \leq
      \annMatrix{u_{i,j}}{i}{j}{(\annInt{\ihat}{\beta'},\annInt{\jhat}{\gamma'})}{\alpha'}
   }
   %
   \and
   %
   \inferrule*[
   ]
   {
      v_{i,j} \leq \hole
      \quad
      (\forall (i,j) \leq (\ihat,\jhat))
   }
   {
      \annMatrix{v_{i,j}}{i}{j}{(\annInt{\ihat}{\FF},\annInt{\jhat}{\FF})}{\FF} \leq \hole
   }
   %
   \and
   %
   \inferrule*[
   ]
   {
      \strut
   }
   {
      \exPrim{\phi} \leq \exPrim{\phi}
   }
   %
   \and
   %
   \inferrule*[
   ]
   {
      \strut
   }
   {
      \exPrim{\phi} \leq \hole
   }
   %
   \and
   %
   \inferrule*[
   ]
   {
      \rho \leq \rho'
      \\
      h \leq h'
      \\
      \sigma \leq \sigma'
   }
   {
      \exClosureRec{\Gamma}{\rho}{h}{\sigma} \leq \exClosureRec{\Gamma}{\rho'}{h'}{\sigma'}
   }
   %
   \and
   %
   \inferrule*[
   ]
   {
      \rho \leq \hole_{\Gamma}
      \\
      h \leq \hole
      \\
      \sigma \leq \hole
   }
   {
      \exClosureRec{\Gamma}{\rho}{h}{\sigma} \leq \hole
   }
\end{smathpar}

\vspace{5pt}
\flushleft \shadebox{$\sigma \leq \sigma'$}
\begin{smathpar}
   \inferrule*[
   ]
   {
      \strut
   }
   {
      \hole \leq \sigma
   }
   %
   \and
   %
   \inferrule*[
   ]
   {
      \kappa \leq \kappa'
   }
   {
      \elimVar{x}{\kappa} \leq \elimVar{x}{\kappa'}
   }
   %
   \and
   %
   \inferrule*[
   ]
   {
      \kappa \leq \hole
   }
   {
      \elimVar{x}{\kappa} \leq \hole
   }
   %
   \and
   %
   \inferrule*[
   ]
   {
      \kappa_1 \leq \kappa_1'
      \\
      \kappa_1 \leq \kappa_1'
   }
   {
      \elimBool{\kappa_1}{\kappa_2} \leq \elimBool{\kappa_1'}{\kappa_2'}
   }
   %
   \and
   %
   \inferrule*[
   ]
   {
      \kappa_1 \leq \hole
      \\
      \kappa_1 \leq \hole
   }
   {
      \elimBool{\kappa_1}{\kappa_2} \leq \hole
   }
   %
   \and
   %
   \inferrule*[
   ]
   {
      \sigma \leq \sigma'
   }
   {
      \elimProd{\sigma} \leq \elimProd{\sigma'}
   }
   %
   \and
   %
   \inferrule*[
   ]
   {
      \sigma \leq \hole
   }
   {
      \elimProd{\sigma} \leq \hole
   }
   %
   \and
   %
   \inferrule*[
   ]
   {
      \kappa \leq \kappa'
      \\
      \sigma \leq \sigma'
   }
   {
      \elimList{\kappa}{\sigma} \leq \elimList{\kappa'}{\sigma'}
   }
   %
   \and
   %
   \inferrule*[
   ]
   {
      \kappa \leq \hole
      \\
      \sigma \leq \hole
   }
   {
      \elimList{\kappa}{\sigma} \leq \hole
   }
\end{smathpar}
\caption{Partial orders on values and eliminators}
\end{figure}

\begin{figure}[H]
\flushleft \shadebox{$e \leq e'$}
\begin{smathpar}
   \inferrule*[
   ]
   {
      \strut
   }
   {
      \hole \leq e
   }
   %
   \and
   %
   \inferrule*[
      right={$\alpha \leq \alpha'$}
   ]
   {
      \strut
   }
   {
      \annInt{n}{\alpha} \leq \annInt{n}{\alpha'}
   }
   %
   \and
   %
   \inferrule*[
   ]
   {
      \strut
   }
   {
      \annInt{n}{\bot} \leq \hole
   }
   %
   \and
   %
   \inferrule*[
      right={$\alpha \leq \alpha'$}
   ]
   {
      \strut
   }
   {
      \annTrue{\alpha} \leq \annTrue{\alpha'}
   }
   %
   \and
   %
   \inferrule*[
   ]
   {
      \strut
   }
   {
      \annTrue{\bot} \leq \hole
   }
   %
   \and
   %
   \inferrule*[
      right={$\alpha \leq \alpha'$}
   ]
   {
      \strut
   }
   {
      \annFalse{\alpha} \leq \annFalse{\alpha'}
   }
   %
   \and
   %
   \inferrule*[
   ]
   {
      \strut
   }
   {
      \annFalse{\bot} \leq \hole
   }
   %
   \and
   %
   \inferrule*[
      right={$\alpha \leq \alpha'$}
   ]
   {
      e_i \leq e'_i
      \quad
      (\forall i \numleq \length{\vec{x}})
   }
   {
      \annRec{\vec{\bind{x}{e}}}{\alpha} \leq \annRec{\vec{\bind{x}{e}}'}{\alpha'}
   }
   %
   \and
   %
   \inferrule*[
   ]
   {
      e_i \leq \hole
      \quad
      (\forall i \numleq \length{\vec{x}})
   }
   {
      \annRec{\vec{\bind{x}{e}}}{\bot} \leq \hole
   }
   %
   \and
   %
   \inferrule*
   {
      e \leq e'
   }
   {
      \exRecProj{e}{x} \leq \exRecProj{e'}{x}
   }
   %
   \and
   %
   \inferrule*
   {
      e \leq \hole
   }
   {
      \exRecProj{e}{x} \leq \hole
   }
   %
   \and
   %
   \inferrule*[
      right={$\alpha \leq \alpha'$}
   ]
   {
      \strut
   }
   {
      \annNil{\alpha} \leq \annNil{\alpha'}
   }
   %
   \and
   %
   \inferrule*[
   ]
   {
      \strut
   }
   {
      \annNil{\bot} \leq \hole
   }
   %
   \and
   %
   \inferrule*[
      right={$\alpha \leq \alpha'$}
   ]
   {
      e_1 \leq e_1'
      \\
      e_2 \leq e_2'
   }
   {
      \annCons{e_1}{e_2}{\alpha} \leq \annCons{e_1'}{e_2'}{\alpha'}
   }
   %
   \and
   %
   \inferrule*[
   ]
   {
      e_1 \leq \hole
      \\
      e_2 \leq \hole
   }
   {
      \annCons{e_1}{e_2}{\bot} \leq \hole
   }
   %
   \and
   %
   \inferrule*[
      right={$\alpha \leq \alpha'$}
   ]
   {
      e_1 \leq e_1'
      \\
      e_2 \leq e_2'
   }
   {
      \annVec{e_1}{x}{e_2}{\alpha} \leq \annVec{e_1'}{x}{e_2'}{\alpha'}
   }
   %
   \and
   %
   \inferrule*[
   ]
   {
      e_1 \leq \hole
      \\
      e_2 \leq \hole
   }
   {
      \annVec{e_1}{x}{e_2}{\bot} \leq \hole
   }
   %
   \and
   %
   \inferrule*[
   ]
   {
      \strut
   }
   {
      \exVar{x} \leq \exVar{x}
   }
   %
   \and
   %
   \inferrule*[
   ]
   {
      \strut
   }
   {
      \exVar{x} \leq \hole
   }
   %
   \and
   %
   \inferrule*[
   ]
   {
      \sigma \leq \sigma'
   }
   {
      \exLambda{\sigma} \leq \exLambda{\sigma}
   }
   %
   \and
   %
   \inferrule*[
   ]
   {
      \sigma \leq \hole
   }
   {
      \exLambda{\sigma} \leq \hole
   }
   %
   \and
   %
   \inferrule*[
   ]
   {
      e_1 \leq e_1'
      \\
      e_2 \leq e_2'
   }
   {
      \exApp{e_1}{e_2} \leq \exApp{e_1}{e_2}
   }
   %
   \and
   %
   \inferrule*[
   ]
   {
      e_1 \leq \hole
      \\
      e_2 \leq \hole
   }
   {
      \exApp{e_1}{e_2} \leq \hole
   }
   %
   \and
   %
   \inferrule*[
   ]
   {
      e_1 \leq e_1'
      \\
      e_2 \leq e_2'
   }
   {
      \exVecLookup{e_1}{e_2} \leq \exVecLookup{e_1}{e_2}
   }
   %
   \and
   %
   \inferrule*[
   ]
   {
      e_1 \leq \hole
      \\
      e_2 \leq \hole
   }
   {
      \exVecLookup{e_1}{e_2} \leq \hole
   }
   %
   \and
   %
   \inferrule*[
   ]
   {
      e \leq e'
   }
   {
      \exVecLen{e} \leq \exVecLen{e'}
   }
   %
   \and
   %
   \inferrule*[
   ]
   {
      e \leq \hole
   }
   {
      \exVecLen{e} \leq \hole
   }
   %
   \and
   %
   \inferrule*[
   ]
   {
      e_1 \leq \hole
      \\
      e_2 \leq \hole
   }
   {
      \exLet{x}{e_1}{e_2} \leq \hole
   }
   %
   \and
   %
   \inferrule*[
   ]
   {
      e \leq e'
      \\
      h \leq h'
   }
   {
      \exLetRecMutual{h}{e} \leq \exLetRecMutual{h'}{e'}
   }
   %
   \and
   %
   \inferrule*[
   ]
   {
      e \leq \hole
      \\
      h \leq \hole
   }
   {
      \exLetRecMutual{h}{e} \leq \hole
   }
\end{smathpar}
\caption{Partial order on terms}
\label{fig:leq-term}
\end{figure}


\begin{definition}[Order on selection state]
   Define $\leq$ to be the total order satisfying $\FF \leq \alpha$ and $\TT \leq \TT$.
\end{definition}

\begin{definition}[Hole equivalence]
   Define $\eq$ as the intersection of $\leq$ and $\geq$.
\end{definition}

\begin{figure}[H]
\flushleft\shadebox{$h \join h'$}
\begin{salign}
   \vec{\bind{x}{\sigma}} \join \vec{\bind{x}{\tau}}
   =
   \vec{\bind{x}{\sigma \join \tau}}
\end{salign}

\vspace{5pt}
\flushleft\shadebox{$v \join v'$}
\begin{salign}
   \hole \join v &= v
   \\
   v \join \hole &= v
   \\
   \annTrue{\alpha} \join \annTrue{\alpha'} &= \annTrue{\alpha \join \alpha'}
   \\
   \annFalse{\alpha} \join \annFalse{\alpha'} &= \annFalse{\alpha \join \alpha'}
   \\
   \annInt{n}{\alpha} \join \annInt{n}{\alpha'} &= \annInt{n}{\alpha \join \alpha'}
   \\
   \annRec{\vec{\bind{x}{u}}}{\alpha} \join \annRec{\vec{\bind{x}{v}}}{\alpha'} &=
   \annRec{\vec{\bind{x}{u \join v}}}{\alpha \join \alpha'}
   \\
   \annNil{\alpha} \join \annNil{\alpha} &= \annNil{\alpha \join \alpha'}
   \\
   (\annCons{u}{v}{\alpha}) \join (\annCons{u'}{v'}{\alpha'})
   &=
   \annCons{(u \join u')}{(v \join v')}{\alpha \join \alpha'}
   \\
   \annVecVal{\vec{v}}{\annInt{j}{\beta}}{\alpha}
   \join
   \annVecVal{\vec{u}}{\annInt{j}{\beta'}}{\alpha'}
   &=
   \annVecVal{\vec{v \join u}}
             {\annInt{j}{\beta \join \beta'}}{\alpha \join \alpha'}
   \\
   \exPrimOp{\phi}{\vec{v}} \join
   \exPrimOp{\phi}{\vec{u}}
   &=
   \exPrimOp{\phi}{\vec{v \join u}}
   \\
   \exClosure{\rho}{h}{\sigma} \join \exClosure{\rho'}{h'}{\sigma'}
   &=
   \exClosure{\rho \join \rho'}{h \join h'}{\sigma \join \sigma'}
\end{salign}
\caption{Join of compatible values}
\label{fig:join-value}
\end{figure}

\begin{figure}[H]
\flushleft\shadebox{$\sigma \join \sigma'$}
\begin{salign}
   \hole \join \sigma &= \sigma
   \\
   \sigma \join \hole &= \sigma
   \\
   (\elimVar{x}{\kappa}) \join (\elimVar{x}{\kappa'})
   &=
   \elimVar{x}{\kappa \join \kappa'}
   \\
   \elimBool{\kappa_1}{\kappa_2} \join \elimBool{\kappa_1'}{\kappa_2'}
   &=
   \elimBool{\kappa_1 \join \kappa_1'}{\kappa_2 \join \kappa_2'}
   \\
   \elimRec{\vec{x}}{\kappa} \join \elimRec{\vec{x}}{\kappa'}
   &=
   \elimRec{\vec{x}}{\kappa \join \kappa'}
   \\
   \elimList{\kappa}{\sigma} \join \elimList{\kappa'}{\sigma'}
   &=
   \elimList{\kappa \join \kappa'}{\sigma \join \sigma'}
\end{salign}
\caption{Join of compatible eliminators}
\label{fig:join-elim}
\end{figure}


\subsubsection{Pattern-matching}

\begin{figure}
{\small \flushleft \shadebox{$\matchFwd{v}{\sigma}{w}{\rho}{\kappa}{\alpha}$}%
\hfill \textbfit{$v$ and $\sigma$ forward-match along $w$ to $\rho$ and $\kappa$, with argument availability $\alpha$}}
\begin{smathpar}
   \inferrule*[
      lab={\ruleName{$\matchFwdS$-hole-$v$}}
   ]
   {
      \hole \eq v
      \\
      \matchFwd{v}{\sigma}{w}{\rho}{\kappa}{\alpha}
   }
   {
      \matchFwd{\hole}{\sigma}{w}{\rho}{\kappa}{\alpha}
   }
   %
   \and
   %
   \inferrule*[
      lab={\ruleName{$\matchFwdS$-hole-$\sigma$}}
   ]
   {
      \hole \eq \sigma
      \\
      \matchFwd{v}{\sigma}{w}{\rho}{\kappa}{\alpha}
   }
   {
      \matchFwd{v}{\hole}{w}{\rho}{\kappa}{\alpha}
   }
   %
   \and
   %
   \inferrule*[
      lab={\ruleName{$\matchFwdS$-var}}
   ]
   {
      \strut
   }
   {
      \matchFwd{v}{\elimVar{x}{\kappa}}{\matchVar{x}}{\bind{x}{v}}{\kappa}{\top}
   }
   %
   \and
   %
   \inferrule*[
      lab={\ruleName{$\matchFwdS$-true}}
   ]
   {
      \strut
   }
   {
      \matchFwd{\annTrue{\alpha}}
               {\elimBool{\kappa}{\kappa'}}
               {\matchTrue}
               {\seqEmpty}{\kappa}{\alpha}
   }
   %
   \and
   %
   \inferrule*[
      lab={\ruleName{$\matchFwdS$-false}}
   ]
   {
      \strut
   }
   {
      \matchFwd{\annFalse{\alpha}}
               {\elimBool{\kappa}{\kappa'}}
               {\matchFalse}
               {\seqEmpty}{\kappa'}{\alpha}
   }
   %
   \and
   %
   \inferrule*[
      lab={\ruleName{$\matchFwdS$-unit}}
   ]
   {
      \strut
   }
   {
      \matchFwd{\annot{\exRecEmpty}{\alpha}}
               {\elimRecEmpty{\kappa}}
               {\matchRecEmpty}
               {\seqEmpty}
               {\kappa}
               {\alpha}
   }
   %
   \and
   %
   \inferrule*[
      lab={\ruleName{$\matchFwdS$-record}}
   ]
   {
      \matchFwd{\annRec{\vec{\bind{x}{v}}}{\top}}
               {\elimRec{\vec{x}}{\sigma}}
               {\matchRec{\vec{\bind{x}{w}}}}
               {\rho}
               {\sigma'}
               {\beta}
      \\
      \matchFwd{u}{\sigma'}{w}{\rho'}{\kappa}{\beta'}
   }
   {
      \matchFwd{\annRec{\vec{\bind{x}{v}} \concat \bind{y}{u}}{\alpha}}
               {\elimRec{\vec{x} \concat y}{\sigma}}
               {\matchRec{\vec{\bind{x}{w}} \concat \bind{y}{w'}}}
               {\rho \concat \rho'}
               {\kappa}
               {\alpha \meet \beta \meet \beta'}
   }
   %
   \and
   %
   \inferrule*[
      lab={\ruleName{$\matchFwdS$-nil}}
   ]
   {
      \strut
   }
   {
      \matchFwd{\annNil{\alpha}}
               {\elimList{\kappa}{\sigma'}}
               {\matchNil}
               {\seqEmpty}{\kappa}{\alpha}
   }
   %
   \and
   %
   \inferrule*[
      lab={\ruleName{$\matchFwdS$-cons}}
   ]
   {
      \matchFwd{v}{\sigma}{w}{\rho}{\tau}{\beta}
      \\
      \matchFwd{v'}{\tau}{w'}{\rho'}{\kappa'}{\beta'}
   }
   {
      \matchFwd{\annCons{v}{v'}{\alpha}}
               {\elimList{\kappa}{\sigma}}
               {\matchCons{w}{w'}}
               {\rho \concat \rho'}{\kappa'}{\alpha \meet \beta \meet \beta'}
   }
\end{smathpar}
\caption{Forward match}
\label{fig:data-dependencies:match-fwd}
\end{figure}

\begin{figure}
   {\small \flushleft \shadebox{$\matchBwd{\rho}{\kappa}{\alpha}{w}{v}{\sigma}$}%
   \hfill \textbfit{$\rho$ and $\kappa$, with argument demand $\alpha$, backward-match along $w$ to $v$ and $\sigma$}}
   \begin{smathpar}
      \inferrule*[lab={\ruleName{$\matchBwdS$-true}}]
      {
         \strut
      }
      {
         \matchBwd{\seqEmpty}{\kappa}{\alpha}{\matchTrue}{\annTrue{\alpha}}{\elimBool{\kappa}{\hole}}
      }
      %
      \and
      %
      \inferrule*[lab={\ruleName{$\matchBwdS$-false}}]
      {
         \strut
      }
      {
         \matchBwd{\seqEmpty}{\kappa}{\alpha}{\matchFalse}{\annFalse{\alpha}}{\elimBool{\hole}{\kappa}}
      }
      \inferrule*[lab={\ruleName{$\matchBwdS$-var}}]
      {
         \strut
      }
      {
         \matchBwd{\bind{x}{v}}{\kappa}{\alpha}{\matchVar{x}}{v}{\elimVar{x}{\kappa}}
      }
      %
      \and
      %
      \inferrule*[lab={\ruleName{$\matchBwdS$-unit}}]
      {
         \strut
      }
      {
         \matchBwd{\seqEmpty}
                  {\kappa}
                  {\alpha}
                  {\matchRecEmpty}
                  {\annot{\exRecEmpty}{\alpha}}
                  {\elimRecEmpty{\kappa}}
      }
      %
      \and
      %
      \inferrule*[lab={\ruleName{$\matchBwdS$-nil}}]
      {
         \strut
      }
      {
         \matchBwd{\seqEmpty}{\kappa}{\alpha}{\matchNil}{\annNil{\alpha}}{\elimList{\kappa}{\hole}}
      }
      %
      \and
      %
      \inferrule*[lab={\ruleName{$\matchBwdS$-record}}]
      {
         \matchBwd{\rho'}{\kappa}{\alpha}{w'}{u}{\sigma}
         \\
         \matchBwd{\rho}
                  {\sigma}
                  {\alpha}
                  {\matchRec{\vec{\bind{x}{w}}}}
                  {\annRec{\vec{\bind{x}{v}}}{\beta}}
                  {\tau}
      }
      {
         \matchBwd{\rho \concat \rho'}
                  {\kappa}
                  {\alpha}
                  {\matchRec{\vec{\bind{x}{w}} \concat \bind{y}{w'}}}
                  {\annRec{\vec{\bind{x}{v}} \concat \bind{y}{u}}{\alpha}}
                  {\elimRec{\vec{x} \concat y}{\tau}}
      }
      %
      \and
      %
      \inferrule*[lab={\ruleName{$\matchBwdS$-cons}}]
      {
         \matchBwd{\rho'}{\kappa}{\alpha}{w'}{v'}{\sigma}
         \\
         \matchBwd{\rho}{\sigma}{\alpha}{w}{v}{\tau}
      }
      {
         \matchBwd{\rho \concat \rho'}
                  {\kappa}
                  {\alpha}
                  {\matchCons{w}{w'}}
                  {\annCons{v}{v'}{\alpha}}
                  {\elimList{\hole}{\tau}}
      }
   \end{smathpar}
\caption{Backward match}
\label{fig:data-dependencies:match-bwd}
\end{figure}


\begin{lemma}[Determinism of pattern-matching]
   Suppose $v, \sigma \matchFwdR{w} \rho, \kappa$ and $v', \sigma' \matchFwdR{w} \rho', \kappa'$. If $(v, \sigma) \eq (v', \sigma)$ then $(\rho, \kappa) \eq (\rho', \kappa')$.
\end{lemma}

\begin{definition}[Forward and backward functions for pattern-matching]
   Suppose $w: v, \sigma \matchR \rho, \kappa$. Define $\matchFwdF{w}: \Below{(v,\sigma,\TT)} \to \Below{(\rho,\kappa)}$ and $\matchBwdF{w}: \Below{(\rho,\kappa)} \to \Below{(v,\sigma,\TT)}$ to be $\matchFwdR{w}$ and $\matchBwdR{w}$ domain-restricted to $\Below{(v,\sigma,\TT)}$ and $\Below{(\rho,\kappa)}$ respectively.
\end{definition}

\begin{theorem}[Galois connection for pattern-matching]
\label{thm:core-language:match:gc}
   Suppose $w: v, \sigma \matchFwdS \rho, \kappa$.  Then $\matchFwdF{w} \adjoint \matchBwdF{w}$.
\end{theorem}

\subsubsection{Evaluation}

\paragraph{Primitive operations}

Each primitive operation $\phi: \tyInt^{i} \to \tyInt$ must for every $\vec{n}$ with $\length{\vec{n}} = i$ provide a Galois connection $(\primFwdBool{\phi}{\vec{n}}, \primBwdBool{\phi}{\vec{n}})$ between $\Bool^i$ and $\Bool$, which we lift to a Galois connection $(\primFwd{\phi}{\vec{n}}, \primBwd{\phi}{\vec{n}})$ between $\Below{\vec{n}}$ and $\Below{\phi(\vec{n})}$ by defining
\begin{definition}
\label{def:core-language:primop-gc}
\begin{salign}
   \primFwd{\phi}{\vec{n}}(\vec{\annInt{n}{\alpha}}) &= \annInt{m}{\beta}
   \text{ where }
   \primFwdBool{\phi}{\vec{n}}(\vec{\alpha}) = \beta
   \\
   \primBwd{\phi}{\vec{n}}(\annInt{m}{\beta}) &= \vec{\annInt{n}{\alpha}}
   \text{ where }
   \primBwdBool{\phi}{\vec{n}}(\beta) = \vec{\alpha}
\end{salign}
\end{definition}

\noindent where $\vec{\annInt{n}{\alpha}}$ denotes the zip of same-length sequences $\vec{n}$ and $\vec{\alpha}$ with the constructor for integer values. For any $\vec{n}$ with $\length{\vec{n}} \numlt i$, any such $\phi$ also gives rise to an isomorphism between $\Below{\vec{n}}$ and the lattice of partial applications $\Below{\exPrimOp{\phi}{\vec{n}}}$.

\begin{figure}
{\small \flushleft \shadebox{$\matchFwd{v}{\sigma}{w}{\rho}{\kappa}{\alpha}$}%
\hfill \textbfit{$v$ and $\sigma$ forward-match along $w$ to $\rho$ and $\kappa$, with ambient availability $\alpha$}}
\begin{smathpar}
   \inferrule*[
      lab={\ruleName{$\matchFwdS$-hole-1}}
   ]
   {
      \hole \eq v
      \\
      \matchFwd{v}{\sigma}{w}{\rho}{\kappa}{\alpha}
   }
   {
      \matchFwd{\hole}{\sigma}{w}{\rho}{\kappa}{\alpha}
   }
   %
   \and
   %
   \inferrule*[
      lab={\ruleName{$\matchFwdS$-hole-2}}
   ]
   {
      \hole \eq \sigma
      \\
      \matchFwd{v}{\sigma}{w}{\rho}{\kappa}{\alpha}
   }
   {
      \matchFwd{v}{\hole}{w}{\rho}{\kappa}{\alpha}
   }
   %
   \and
   %
   \inferrule*[
      lab={\ruleName{$\matchFwdS$-var}}
   ]
   {
      \strut
   }
   {
      \matchFwd{v}{\elimVar{x}{\kappa}}{\matchVar{x}}{\bind{x}{v}}{\kappa}{\top}
   }
   %
   \and
   %
   \inferrule*[
      lab={\ruleName{$\matchFwdS$-true}}
   ]
   {
      \strut
   }
   {
      \matchFwd{\annTrue{\alpha}}
               {\elimBool{\kappa}{\kappa'}}
               {\matchTrue}
               {\seqEmpty}{\kappa}{\alpha}
   }
   %
   \and
   %
   \inferrule*[
      lab={\ruleName{$\matchFwdS$-false}}
   ]
   {
      \strut
   }
   {
      \matchFwd{\annFalse{\alpha}}
               {\elimBool{\kappa}{\kappa'}}
               {\matchFalse}
               {\seqEmpty}{\kappa'}{\alpha}
   }
   %
   \and
   %
   \inferrule*[
      lab={\ruleName{$\matchFwdS$-unit}}
   ]
   {
      \strut
   }
   {
      \matchFwd{\annot{\exRecEmpty}{\alpha}}
               {\elimRecEmpty{\kappa}}
               {\matchRecEmpty}
               {\seqEmpty}
               {\kappa}
               {\alpha}
   }
   %
   \and
   %
   \inferrule*[
      lab={\ruleName{$\matchFwdS$-record}}
   ]
   {
      \matchFwd{\annRec{\vec{\bind{x}{v}}}{\top}}
               {\elimRec{\vec{x}}{\sigma}}
               {\matchRec{\vec{\bind{x}{w}}}}
               {\rho}
               {\sigma'}
               {\beta}
      \\
      \matchFwd{u}{\sigma'}{w}{\rho'}{\kappa}{\beta'}
   }
   {
      \matchFwd{\annRec{\vec{\bind{x}{v}} \concat \bind{y}{u}}{\alpha}}
               {\elimRec{\vec{x} \concat y}{\sigma}}
               {\matchRec{\vec{\bind{x}{w}} \concat \bind{y}{w'}}}
               {\rho \concat \rho'}
               {\kappa}
               {\alpha \meet \beta \meet \beta'}
   }
   %
   \and
   %
   \inferrule*[
      lab={\ruleName{$\matchFwdS$-nil}}
   ]
   {
      \strut
   }
   {
      \matchFwd{\annNil{\alpha}}
               {\elimList{\kappa}{\sigma'}}
               {\matchNil}
               {\seqEmpty}{\kappa}{\alpha}
   }
   %
   \and
   %
   \inferrule*[
      lab={\ruleName{$\matchFwdS$-cons}}
   ]
   {
      \matchFwd{v}{\sigma}{w}{\rho}{\tau}{\beta}
      \\
      \matchFwd{v'}{\tau}{w'}{\rho'}{\kappa'}{\beta'}
   }
   {
      \matchFwd{\annCons{v}{v'}{\alpha}}
               {\elimList{\kappa}{\sigma}}
               {\matchCons{w}{w'}}
               {\rho \concat \rho'}{\kappa'}{\alpha \meet \beta \meet \beta'}
   }
\end{smathpar}

\vspace{2mm}
{\small \flushleft \shadebox{$\evalFwd{\rho}{e}{\alpha}{T}{v}$}%
\hfill \textbfit{$\rho$ and $e$, with ambient availability $\alpha$, forward-evaluate along $T$ to $v$}}
\begin{smathpar}
   \mprset{center}
   \inferrule*[
      lab={\ruleName{$\evalFwdS$-hole}}
   ]
   {
      \hole \eq e
      \\
      \evalFwd{\rho}{e}{\alpha}{T}{v}
   }
   {
      \evalFwd{\rho}{\hole}{\alpha}{T}{v}
   }
   %
   \and
   %
   \inferrule*[
      lab={\ruleName{$\evalFwdS$-var}}
   ]
   {
      \envLookup{\rho}{x}{v}
   }
   {
      \evalFwd{\rho}{\exVar{x}}{\alpha}{\trVar{x}{\rho}}{v}
   }
   %
   \and
   %
   \inferrule*[lab={\ruleName{$\evalFwdS$-lambda}}]
   {
      \strut
   }
   {
      \evalFwd{\rho}
              {\exLambda{\sigma}}
              {\alpha}
              {\trLambda{\sigma'}}
              {\annClosure{\rho}{\seqEmpty}{\alpha}{\sigma}}
   }
   %
   \and
   %
   \inferrule*[lab={\ruleName{$\evalFwdS$-int}}]
   {
      \strut
   }
   {
      \evalFwd{\rho}
              {\annInt{n}{\alpha'}}
              {\alpha}
              {\trInt{n}{\rho}}
              {\annInt{n}{\alpha \meet \alpha'}}
   }
   %
   \and
   %
   \inferrule*[lab={\ruleName{$\evalFwdS$-record}}]
   {
      \evalFwd{\rho}{e_i}{\alpha}{T_i}{v_i}
      \quad
      (\forall i \numleq \length{\vec{x}})
   }
   {
      \evalFwd{\rho}
              {\annRec{\vec{\bind{x}{e}}}{\alpha'}}
              {\alpha}
              {\trRec{\vec{\bind{x}{T}}}}
              {\annRec{\vec{\bind{x}{v}}}{\alpha \meet \alpha'}}
   }
   %
   \and
   %
   \inferrule*[
      lab={\ruleName{$\evalFwdS$-project}}
   ]
   {
      \evalFwdEq{\rho}{e}{\alpha}{T}{\annRec{\vec{\bind{x}{u}}}{\beta}}
      \\
      \envLookup{\vec{\bind{x}{u}}}{y}{v'}
   }
   {
      \evalFwd{\rho}
              {\exRecProj{e}{y}}
              {\alpha}
              {\trRecProj{T}{\vec{\bind{x}{v}}}{y}}
              {v'}
   }
   %
   \and
   %
   \inferrule*[lab={\ruleName{$\evalFwdS$-nil}}]
   {
      \strut
   }
   {
      \evalFwd{\rho}
              {\annNil{\alpha'}}
              {\alpha}
              {\trNil{\rho}}
              {\annNil{\alpha \meet \alpha'}}
   }
   %
   \and
   %
   \inferrule*[
      lab={\ruleName{$\evalFwdS$-cons}},
   ]
   {
      \evalFwd{\rho}{e}{\alpha}{T}{v}
      \\
      \evalFwd{\rho}{e'}{\alpha}{U}{v'}
   }
   {
      \evalFwd{\rho}
              {\annCons{e}{e'}{\alpha'}}
              {\alpha}
              {\trCons{T}{U}}
              {\annCons{v}{v'}{\alpha \meet \alpha'}}
   }
   %
   \and
   %
   \inferrule*[
      lab={\ruleName{$\evalFwdS$-apply-prim}},
      right={$\primFwdBool{\phi}{\vec{n}}(\vec{\beta}) = \alpha'$}
   ]
   {
      \evalFwdEq{\rho}{e_i}{\alpha}{U_i}{\annInt{{n_i}}{\beta_i}}
      \quad
      (\forall i \numleq \length{\vec{n}})
   }
   {
      \evalFwd{\rho}
              {\exAppPrim{\phi}{\vec{e}}}
              {\alpha}
              {\trAppPrimNew{\phi}{U}{n}}
              {\annInt{\exAppPrim{\phi}{\vec{n}}}{\alpha'}}
   }
   %
   \and
   %
   \inferrule*[
      lab={\ruleName{$\evalFwdS$-apply}},
      width={3.4in}
   ]
   {
      \evalFwdEq{\rho}{e}{\alpha}{T}{\annClosure{\rho_1}{h}{\beta}{\sigma}}
      \\
      \rho_1, h, \beta \closeDefsFwdR \rho_2
      \\
      \evalFwd{\rho}{e'}{\alpha}{U}{v}
      \\
      \matchFwd{v}{\sigma}{w}{\rho_3}{e^\twoPrime}{\beta'}
      \\
      \evalFwd{\rho_1 \concat \rho_2 \concat \rho_3}{e^\twoPrime}{\beta \meet \beta'}{T'}{v'}
   }
   {
      \evalFwd{\rho}{\exApp{e}{e'}}{\alpha}{\trApp{T}{U}{w}{T'}}{v'}
   }
   %
   \and
   %
   \inferrule*[lab={
      \ruleName{$\evalFwdS$-let-rec}}
   ]
   {
      \rho, h', \alpha \closeDefsFwdR \rho'
      \\
      \evalFwd{\rho \concat \rho'}{e}{\alpha}{T}{v}
   }
   {
      \evalFwd{\rho}{\exLetRecMutual{h'}{e}}{\alpha}{\trLetRecMutual{h}{T}}{v}
   }
\end{smathpar}
\vspace{2mm}
{\small \flushleft \shadebox{$\rho, h, \alpha \closeDefsFwdR \rho'$}%
\hfill \textbfit{$h$ forward-generates to $\rho'$ in $\rho$ and $\alpha$}}
\begin{smathpar}
   \inferrule*[
      lab={\ruleName{$\closeDefsFwdR$-rec-defs}}
   ]
   {
      v_i = \annClosure{\rho}{\vec{\bind{x}{\sigma}}}{\alpha}{\sigma_i}
      \quad
      (\forall i \in \length{\vec{x}})
   }
   {
      \rho, \vec{\bind{x}{\sigma}}, \alpha
      \closeDefsFwdR
      \vec{\bind{x}{v}}
   }
\end{smathpar}
\caption{Forward data dependency (Boolean cases for $\evalFwdR{T}$ omitted)}
\label{fig:data-dependencies:fwd}
\end{figure}

\begin{figure}
   {\small \flushleft \shadebox{$\evalBwd{v}{T}{\rho}{e}{\alpha}$}%
   \hfill \textbfit{$v$ backward-evaluates along $T$ to $\rho$ and $e$, with argument demand $\alpha$}}
   \begin{smathpar}
      \inferrule*[
         lab={\ruleName{$\evalBwdS$-hole}}
      ]
      {
         \hole \eq v
         \\
         \evalBwd{v}{T}{\rho}{e}{\alpha}
      }
      {
         \evalBwd{\hole}{T}{\rho}{e}{\alpha}
      }
      %
      \and
      %
      \inferrule*[
         lab={\ruleName{$\evalBwdS$-var}}
      ]
      {
         \envLookupBwd{\rho'}{\rho}{\bind{x}{v}}
      }
      {
         \evalBwd{v}{\trVar{x}{\rho}}{\rho'}{\exVar{x}}{\bot}
      }
      %
      \and
      %
      \inferrule*[
         lab={\ruleName{$\evalBwdS$-lambda}}
      ]
      {
         \strut
      }
      {
         \evalBwd{\annClosure{\rho}{\seqEmpty}{\alpha}{\sigma}}
                 {\trLambda{\sigma'}}
                 {\rho}
                 {\exLambda{\sigma}}
                 {\alpha}
      }
      %
      \and
      %
      \inferrule*[
         lab={\ruleName{$\evalBwdS$-int}}
      ]
      {
         \strut
      }
      {
         \evalBwd{\annInt{n}{\alpha}}
                 {\trInt{n}{\rho}}
                 {\hole_{\raw{\rho}}}
                 {\annInt{n}{\alpha}}
                 {\alpha}
      }
      %
      \and
      %
      \inferrule*[
         lab={\ruleName{$\evalBwdS$-true}}
      ]
      {
         \strut
      }
      {
         \evalBwd{\annTrue{\alpha}}
                 {\trTrue{\rho}}
                 {\hole_{\raw{\rho}}}
                 {\annTrue{\alpha}}
                 {\alpha}
      }
      %
      \and
      %
      \inferrule*[
         lab={\ruleName{$\evalBwdS$-false}}
      ]
      {
         \strut
      }
      {
         \evalBwd{\annFalse{\alpha}}
                 {\trFalse{\rho}}
                 {\hole_{\raw{\rho}}}
                 {\annFalse{\alpha}}
                 {\alpha}
      }
      %
      \and
      %
      \inferrule*[lab={\ruleName{$\evalBwdS$-record}}]
      {
         \evalBwd{v_i}{T_i}{\rho_i}{e_i}{\alpha_i'}
         \quad
         (\forall i \numleq \length{\vec{x}})
      }
      {
         \evalBwd{\annRec{\vec{\bind{x}{v}}}{\alpha}}
                 {\trRec{\vec{\bind{x}{T}}}}
                 {\bigjoin\vec{\rho}}
                 {\annRec{\vec{\bind{x}{e}}}{\alpha}}
                 {\alpha \join \bigjoin\vec{\alpha}'}
      }
      %
      \and
      %
      \inferrule*[lab={\ruleName{$\evalBwdS$-project}}]
      {
         \envLookupBwd{\vec{\bind{x}{u}}}{\vec{\bind{x}{v}}}{\bind{y}{v'}}
         \\
         \evalBwd{\annRec{\vec{\bind{x}{u}}}{\bot}}
                 {T}
                 {\rho}
                 {e}
                 {\alpha}
      }
      {
         \evalBwd{v'}
                 {\trRecProj{T}{\vec{\bind{x}{v}}}{y}}
                 {\rho}
                 {\exRecProj{e}{y}}
                 {\alpha}
      }
      %
      \and
      %
      \inferrule*[
         lab={\ruleName{$\evalBwdS$-nil}}
      ]
      {
         \strut
      }
      {
         \evalBwd{\annNil{\alpha}}
                 {\trNil{\rho}}
                 {\hole_{\raw{\rho}}}
                 {\annNil{\alpha}}
                 {\alpha}
      }
      %
      \and
      %
      \inferrule*[
         lab={\ruleName{$\evalBwdS$-cons}}
      ]
      {
         \evalBwd{v}{T}{\rho}{e}{\alpha}
         \\
         \evalBwd{v'}{U}{\rho'}{e'}{\alpha'}
      }
      {
         \evalBwd{\annCons{v}{v'}{\beta}}
                 {\trCons{T}{U}}
                 {\rho \join \rho'}
                 {\annCons{e}{e'}{\beta}}
                 {\beta \join \alpha \join \alpha'}
      }
      %
      \and
      %
      \inferrule*[
         lab={\ruleName{$\evalBwdS$-let-rec}}
      ]
      {
         \evalBwd{v}{T}{\rho \concat \rho_1}{e}{\alpha}
         \\
         \rho_1 \closeDefsBwdR \rho', h', \alpha'
      }
      {
         \evalBwd{v}{\trLetRecMutual{h}{T}}{\rho \join \rho'}{\exLetRecMutual{h'}{e}}{\alpha \join \alpha'}
      }
      %
      \and
      %
      \inferrule*[
         lab={\ruleName{$\evalBwdS$-apply-prim}},
         right={$\primBwdBool{\phi}{\vec{n}}(\alpha') = \vec{\alpha}$}
      ]
      {
         \evalBwd{\annInt{{n_i}}{\alpha_i}}{U_i}{\rho_i}{e_i}{\beta_i}
         \quad
         (\forall i \in \length{\vec{n}})
      }
      {
         \evalBwd{\annInt{m}{\alpha'}}
                 {\trAppPrimNew{\phi}{U}{n}}
                 {\bigjoin\vec{\rho}}
                 {\exAppPrim{\phi}{\vec{e}}}
                 {\bigjoin\vec{\beta}}
      }
      %
      \and
      %
      \inferrule*[
         lab={\ruleName{$\evalBwdS$-apply}},
         width={3.3in}
      ]
      {
         \evalBwd{v}{T'}{\rho_1 \concat \rho_2 \concat \rho_3}{e}{\beta}
         \\
         \matchBwd{\rho_3}{e}{\beta}{w}{v'}{\sigma}
         \\
         \evalBwd{v'}{U}{\rho}{e_2}{\alpha}
         \\
         \rho_2 \closeDefsBwdR \rho_1', h, \beta'
         \\
         \evalBwd{\annClosure{\rho_1 \join \rho_1'}{h}{\beta \join \beta'}{\sigma}}{T}{\rho'}{e_1}{\alpha'}
      }
      {
         \evalBwd{v}{\trApp{T}{U}{w}{T'}}{\rho \join \rho'}{\exApp{e_1}{e_2}}{\alpha \join \alpha'}
      }
   \end{smathpar}
   \vspace{1mm}

\begin{minipage}[t]{0.48\textwidth}%
   {\small\flushleft \shadebox{$\envLookupBwd{\rho'}{\rho}{\bind{x}{v}}$}%
   \hfill \textbfit{$\rho'$ contains $\bind{x}{v}$}}
   \begin{smathpar}
      \inferrule*[
         lab={\ruleName{$\envLookupBwdS$-head}}
      ]
      {
         \strut
      }
      {
         \envLookupBwd{(\hole_{\raw{\rho}} \concat \bind{x}{u})}
                      {\rho \concat \bind{x}{v}}
                      {\bind{x}{u}}
      }
      %
      \and
      %
      \inferrule*[
         lab={\ruleName{$\envLookupBwdS$-tail}},
      ]
      {
         \envLookupBwd{\rho'}{\rho}{\bind{x}{u}}
         \\
         x \neq y
      }
      {
         \envLookupBwd{(\rho' \concat \bind{y}{\hole})}
                      {\rho \concat \bind{y}{v}}
                      {\bind{x}{u}}
      }
   \end{smathpar}
\end{minipage}%
\hfill
\begin{minipage}[t]{0.47\textwidth}%
   {\small\flushleft\shadebox{$\rho \closeDefsBwdR \rho', h, \alpha$}%
   \hfill \textbfit{$\rho$ backward-generates to $\rho'$, $h$, $\alpha$}}
   \begin{smathpar}
      \inferrule*[
         lab={\ruleName{$\closeDefsBwdR$-rec-defs}}
      ]
      {
         v_i = \annClosure{\rho_i}{h_i}{\alpha_i}{\sigma_i}
         \quad
         (\forall i \in \length{\vec{x}})
      }
      {
         \vec{\bind{x}{v}}
         \closeDefsBwdR
         \bigjoin\vec{\rho}, \vec{\bind{x}{\sigma}} \join {\bigjoin\vec{h}}, \bigjoin{\vec{\alpha}}
      }
   \end{smathpar}
\end{minipage}

\caption{\vspace{-2mm}Backward evaluation}
\label{fig:data-dependencies:bwd}
\end{figure}

\begin{figure}[H]
\small
$\begin{array}{lll|llll}
\arrayrulecolor{lightgray}
&&&\textit{where}
\\
\rowcolor{verylightgray}
\envLookupGC{\rho,z} &::& \Below{\rho} \to \Below{v}
&\envLookup{\rho}{z}{v}
&&
\\
\envLookupFwdF{\rho \concat \bind{x}{v},x}(\rho' \concat \bind{x}{u},x) &=& u
\\
\envLookupFwdF{\rho \concat \bind{y}{v},x}(\rho' \concat \bind{y}{u},x)
&=&
\envLookupFwdF{\rho,x}(\rho',x)
&
x \neq y
\\[3mm]
\envLookupBwdF{\rho \concat \bind{x}{v},x}(u)
&=&
\sub{\hole}{\rho} \concat \bind{x}{u}
\\
\envLookupBwdF{\rho \concat \bind{y}{v},x}(u)
&=&
\envLookupBwdF{\rho}(u) \concat \bind{y}{\hole}
&
x \neq y
\\[3mm]
%
%
%
\rowcolor{verylightgray}
\closeDefsGC{\rho,h} &::& \Below{(\rho, h)} \to \Below{\rho'}
&\rho, h \closeDefsR \rho'
&&
\\
\closeDefsFwdF{\rho,h}(\rho',h')
&=&
\vec{\bind{x}{v}}
&
h' &=& \vec{\bind{x}{\sigma}}
\\
&&&
v_i &=& \exClosure{\rho'}{h'}{\sigma_i}
\\[3mm]
\closeDefsBwdF{\rho,h}(\vec{\bind{x}{v}})
&=&
\bigjoin\vec{\rho}', \vec{\bind{x}{\sigma}} \join {\bigjoin\vec{h}'}
&
v_i &=& \exClosure{\rho'_i}{h'_i}{\sigma_i}
\end{array}$
\caption{Auxiliary functions: forward and backward slicing}
\label{fig:slicing:eval-aux}
\end{figure}


\begin{definition}[Hole environment]
Define $\hole_{\vec{\bind{x}{v}}} = \vec{\bind{x}{\hole}}$
\end{definition}

\begin{lemma}[Least environment for $\rho$]
\label{lem:core-language:hole-env}If $\vdash \rho: \Gamma$ then $\hole_{\rho} \leq \rho$.
\end{lemma}

\begin{definition}[Forward and backward functions for environment lookup]
   Suppose $\envLookup{\rho}{x}{v}$. Then define $\envLookupFwdF{\rho,x}: \Below{\rho} \to \Below{v}$ and $\envLookupBwdF{\rho,x}: \Below{v} \to \Below{\rho}$ to be $\bind{x}{-}\envLookupR$ and $\envLookupBwdR{\rho}\bind{x}{-}$ restricted to $\Below{\rho}$ and $\Below{v}$ respectively.
\end{definition}

\begin{lemma}[Galois connection for environment]
\label{lem:core-language:env-get-put}Suppose $\envLookup{\rho}{x}{v}$.
\begin{enumerate}
   \item \label{lem:core-language:env-get-put:1} $\envLookupFwdF{\rho,x}(\envLookupBwdF{\rho,x}(v)) \geq v$.
   \item \label{lem:core-language:env-get-put:2} $\envLookupBwdF{\rho,x}(\envLookupFwdF{\rho,x}(\rho')) \leq \rho'$.
\end{enumerate}
\end{lemma}

\begin{definition}
   \label{def:core-language:closeDefs-bwd}
   Define the relation $\closeDefsBwdR$ as given in \figref{core-language:slicing:eval-aux}.
\end{definition}

\begin{definition}[Forward and backward functions for recursive bindings]
   Suppose $\rho, h \closeDefsR \rho'$. Define $\closeDefsFwdF{\rho,h}: \Below{(\rho, h)} \to \Below{\rho'}$ and $\closeDefsBwdF{\rho,h}: \Below{\rho'} \to \Below{(\rho, h)}$ to be $\closeDefsR$ and $\closeDefsBwdR$ restricted to $\Below{(\rho, h)}$ and $\Below{\rho'}$ respectively.
\end{definition}

\begin{theorem}[Galois connection for recursive bindings]
\label{thm:core-language:closeDefs:gc}
   Suppose $\rho, h \closeDefsR \rho'$.  Then $\closeDefsFwdF{\rho,h} \adjoint \closeDefsBwdF{\rho,h}$.
\end{theorem}

\begin{definition}[Forward and backward functions for evaluation]
   Suppose $T: \rho, e \evalR v$. Define $\evalFwdF{T}: \Below{(\rho, e, \TT)} \to \Below{v}$ and $\evalBwdF{T}: \Below{v} \to \Below{(\rho, e, \TT)}$ to be $\evalFwdR{T}$ and $\evalBwdR{T}$ restricted to $\Below{(\rho, e, \TT)}$ and $\Below{v}$ respectively.
\end{definition}

\begin{theorem}[Galois connection for evaluation]
\label{thm:core-language:eval:gc}
   Suppose $T: \rho, e \evalFwdS v$.  Then $\evalFwdF{T} \adjoint \evalBwdF{T}$.
\end{theorem}

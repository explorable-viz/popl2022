\newpage
\section{Extending to a realistic surface language}
\label{sec:surface-language}

Up until now we have primarily considered how to track dependencies between input and output values, however, our approach also allows us to inspect selections on terms themselves to determine which fragments of our program are involved in the construction of a value. The ability to do this is not restricted to programs written using the core calculus given in \secref{core-language} -- one may want to observe how data dependencies interact with more sophisticated language constructs, such as list comprehensions and piece-wise definitions.

To achieve this, it is unnecessary to augment our core language to accommodate for every new syntactic form and their forward and backward dependency semantics -- we can instead create a surface language on top of our core language. The translation between our surface and core language then gives rise to another Galois connection. In the \textit{forward} direction, we desugar surface terms in their equivalent core representation whilst preserving their selection information. Conversely, in the \textit{backward} direction we reconstruct the original surface term from the core term whilst translating the selection information backward. The Galois connection for desugaring then composes with the Galois connection for dependency analysis in \secref{data-dependencies:analyses}, allowing our approach for data linking to easily extend to tracking dependencies between outputs and programs written in the surface language. In this section, we demonstrate how this can be achieved for a small program (\figref{surface-language:example-1}) written in our example surface language.

\begin{figure}
\begin{syntaxfig}
\mbox{Eliminator}
&
\sigma, \tau
&
::=
&
\ldots
\\
&&&
\elimBoolTrue{\kappa}
\\
&&&
\elimBoolFalse{\kappa}
\\
&&&
\elimListSingleton{\branchNil{\kappa}}
\\
&&&
\elimListSingleton{\branchCons{\sigma}}
\\[2mm]
\mbox{Raw term}
&
r
&
::=
&
\ldots
\\
&&&
\exLambda{\sigma}
&
\text{anonymous function}
\end{syntaxfig}
\caption{Additional syntax}
\end{figure}


\subsection{Surface language syntax}

Consider the program selection in \figref{surface-language:example-1}; the program uses list comprehensions to iterate through a list of records containing fields $\texttt{country}$ and $\texttt{output}$, and constructs a list of tuples containing countries and their total energy output. Assuming the output of this program were $\texttt{[("China", 304), ("USA", 125),}$ $\texttt{("Germany", 606)]}$, one could be interested in how selecting the string $\texttt{"China"}$ places a demand on list comprehensions and other constants that occur in the surface program as resources.

\begin{figure}[H]
\small
\begin{lstlisting}
let totalFor c rows = sum [ row.output | row $\leftarrow$ rows, row.country == c ]$_{\alpha_1}$
in  [ (c , totalFor c data)$_{\beta_1}$ | c $\leftarrow$ [$_{\alpha_3}$"China"$_{\beta_2}$ ,$_{\alpha_4}$ "USA"$_{\beta_3}$ ,$_{\alpha_5}$ "Germany"$_{\beta_4}$ ]$_{\alpha_6}$ ]$_{\alpha_2}$
\end{lstlisting}
\caption{List comprehension example: Surface language}
\label{fig:surface-language:example-1}
\end{figure}

\noindent
The full surface language syntax used to write this program (and others in the paper) is presented in \figref{surface-language:syntax} for completeness, however, here we focus primarily on list comprehensions and list notation. Surface terms are denoted by $s, t$; in particular, lists can now be written in the style $\kw{[} s \comma \ldots \comma s' \kw{]}$ using the term $\annList{s}{r}{\alpha}$, where $r$ either recursively defines the rest of the list, $\annListNext{s}{r}{\alpha}$, or is the end of the list, $\annListEnd{\alpha}$. List comprehensions, $\annListComp{s}{\vec{q}}{\alpha}$, then consist of a body $s$ which is constructed in reference to the environment introduced by a sequence of ``qualifiers'' $\vec{q}$. There are three variants of qualifiers: \textit{guards}, $\qualGuard{s}$, which enforce a Boolean condition $s$ to be true, \textit{generators}, $\qualGenerator{p}{s}$, which iterate through each of the elements in a list $s$ and binds them to a pattern $p$, and \textit{declarations}, $\qualDeclaration{p}{s}$, which bind an arbitrary term $s$ to $p$. We will only consider the first two forms for the purposes of \figref{surface-language:example-1}. It can be noted that the terms for lists and list comprehensions have selections because they correspond to selectable values i.e. $\annNil{\alpha}$ and $\annCons{u}{v}{\alpha}$.

\subsection{Forward desugaring}

Desugaring in the forwards direction is given by the function $\desugarFwdR$ of type $\Below{s} \to \Below{e}$ in \figref{desugar-fwd-short}. This specifies how to desugar a surface language term $s$ into its equivalent core form $e$, whilst correctly retaining the selection information made on $s$; the latter is done by taking the meet of all selections on resources in $s$ we consume, and propagating this onto the selections of all syntax generated by its desugaring.

{\flushleft \shadebox{$s \desugarFwdR e$}
\hfill \textbfit{$s$ forward-desugars to $e$}}
\begin{smathpar}
   \inferrule*[
      lab={\ruleName{$\desugarFwdR$-nil}}
   ]
   {
      \strut
   }
   {
      \annNil{\alpha} \desugarFwdR \annNil{\alpha}
   }
   %
   \and
   %
   \inferrule*[
      lab={\ruleName{$\desugarFwdR$-cons}}
   ]
   {
      s \desugarFwdR e
      \\
      s' \desugarFwdR e'
   }
   {
      \annCons{s}{s'}{\alpha} \desugarFwdR \annCons{e}{e'}{\alpha}
   }
   %
   \and
   %
   \inferrule*[
      lab={\ruleName{$\desugarFwdR$-non-empty-list}}
   ]
   {
      s \desugarFwdR e
      \\
      r \desugarFwdR e'
   }
   {
      \annList{s}{r}{\alpha} \desugarFwdR \annCons{e}{e'}{\alpha}
   }
   %
   \and
   %
   \inferrule*[lab={\ruleName{$\desugarFwdR$-list-comp-done}}]
   {
      s \desugarFwdR e
   }
   {
      \annListComp{s}{\seqEmpty}{\alpha}
      \desugarFwdR
      \annCons{e}{\annNil{\alpha}}{\alpha}
   }
   \and
   %
   \inferrule*[
      lab={\ruleName{$\desugarFwdR$-list-comp-gen}}
   ]
   {
      \annListComp{s}{\vec{q}}{\alpha} \desugarFwdR e
      \\
      p, e \clauseFwdR \sigma
      \\
      \totaliseFwd{\sigma}{\alpha}{p}{\sigma'}
      \\
      s' \desugarFwdR e'
   }
   {
      \annListComp{s}{\qualGenerator{p}{s'} \concat \vec{q}}{\alpha}
      \desugarFwdR
      \exApp{\exApp{\funConcatMap}{\exLambda{\sigma'}}}{e'}
   }
   %
   \and
   %
   \inferrule*[
      lab={\ruleName{$\desugarFwdR$-list-comp-guard}}
   ]
   {
      \annListComp{s}{\vec{q}}{\alpha} \desugarFwdR e
      \\
      s' \desugarFwdR e'
   }
   {
      \annListComp{s}{\qualGuard{s'} \concat \vec{q}}{\alpha}
      \desugarFwdR
      \exApp{\exLambda{\elimBool{e}{\annNil{\alpha}}}}{e'}
   }
   %
   \and
   %
   \inferrule*[
      lab={\ruleName{$\desugarFwdR$-list-comp-decl}}
   ]
   {
      \annListComp{s}{\vec{q}}{\alpha} \desugarFwdR e
      \\
      p, e \clauseFwdR \sigma
      \\
      s' \desugarFwdR e
   }
   {
      \annListComp{s}{\qualDeclaration{p}{s'} \concat \vec{q}}{\alpha}
      \desugarFwdR
      \exApp{\exLambda{\sigma}}{e}
   }
   \end{smathpar}

\vspace{5pt}
{\flushleft \shadebox{$r \desugarFwdR e$}
\hfill \textbfit{$r$ forward-desugars to $e$}}
\begin{smathpar}
   \inferrule*[
      lab={\ruleName{$\desugarFwdR$-list-rest-end}}
   ]
   {
      \strut
   }
   {
      \annListEnd{\alpha} \desugarFwdR \annNil{\alpha}
   }
   %
   \and
   %
   \inferrule*[
      lab={\ruleName{$\desugarFwdR$-list-rest-cons}}
   ]
   {
      s \desugarFwdR e
      \\
      r \desugarFwdR e'
   }
   {
      (\annListNext{s}{r}{\alpha}) \desugarFwdR \annCons{e}{e'}{\alpha}
   }
\end{smathpar}


\noindent
The rules for nil $\annNil{\alpha}$ and other constants simply preserve their selected form. For cons $\annCons{s}{s'}{\alpha}$ and pairs $\annPair{s}{s'}{\alpha}$, we recursively desugar $s$ and $s'$ but preserve their selected constructors. Non-empty lists $\annList{s}{r}{\alpha}$ are similar, but we also translate the sugared list constructors $\kw{[}_\alpha$, $\kw{,}_{\alpha}$, and $\kw{]}_{\alpha}$, to their core representations $\kw{(:)}_{\alpha}$ or $\annNil{\alpha}$. List comprehensions, $\annListComp{s}{\vec{q}}{\alpha}$, are slightly more involved; each of the rules accounts for a different possible qualifier at the head of the sequence $\vec{q}$.

When the first qualifier in a list comprehension is a generator $\qualGenerator{p}{s'}$, we want to elaborate $\annListComp{s}{\qualGenerator{p}{s'} \concat \vec{q}}{\alpha}$ into a function which maps over each element in the generator list $s'$. If the element matches the pattern $p$, the ``rest'' of the list comprehension should be returned; otherwise an empty list $\annNil{\alpha}$ with the selection of the original list comprehension is returned. The end result is a list of lists which then needs to be flattened. We achieve this by first desugaring the rest of the list comprehension $\annListComp{s}{\vec{q}}{\alpha}$ to obtain $e$, and desugaring the generator list $s'$ to obtain $e'$. The operation $p, e \clauseFwdR \sigma$ then constructs a singleton eliminator $\sigma$ where matching against $p$ returns $e$. To account for elements in the generator list which may not match against $p$, we must ensure that our eliminator is total; hence the \textit{totalise} operation $\totaliseFwd{\sigma}{\alpha}{p}{\sigma'}$ takes $\sigma$ and a selection $\alpha$ and recursively inserts a synthesised branch with body $\annNil{\alpha}$ for each missing case in $\sigma$. This returns us a \textit{total} eliminator $\sigma'$. The final desugared form is given by the application $\funConcatMap \; \exLambda{\sigma'} e'$, where $\funConcatMap$ first maps $\exLambda{\sigma'}$ over the generator list $e'$ to produce a list of lists, and then flattens it.

When the first qualifier in a list comprehension is a guard, $\qualGuard{s'}$, the process of desugaring is similar. We elaborate $\annListComp{s}{\qualGuard{s'} \concat \vec{q}}{\alpha}$, by again desugaring the rest of the list comprehension $\annListComp{s}{\qualGuard{s'} \concat \vec{q}}{\alpha}$ to $e$, and the Boolean condition $s'$ to $e'$. The idea is then to construct a Boolean eliminator and apply it to $e'$: if the condition $e'$ holds, the eliminator should return the rest of the list comprehension $e$, otherwise it returns an empty list $\annNil{\alpha}$ with the selection $\alpha$ of the original list comprehension.

If the sequence of qualifiers in the list comprehension is empty, i.e. $\annListComp{s}{\seqEmpty}{\alpha}$, then we simply desugar its body $s$ into $e$ and cons it to an empty list. This results in $\annCons{e}{\annNil{\alpha}}{\alpha}$ where the original selection $\alpha$ is propagated onto both the cons constructor and empty list.

\begin{figure}[H]
\small
\begin{lstlisting}
let totalFor c rows =
      sum (concatMap $\exLambda{\{\elimVar{\kw{row}}{  \exLambda{\elimBool{ \kw{(} \annCons{\kw{row.output}}{\annNil{\alpha_1}}{\alpha_1} \kw{)} }{\annNil{\alpha_1}}} \; \kw{(row.country == c)} }\}}$ rows)
in  concatMap $\exLambda{\{\elimVar{\kw{c}}{ \kw{(} \annCons{\kw{(c, totalFor c data)}_{\beta_1}}{\annNil{\alpha_2}}{\alpha_2} \kw{)}  }\}}$ ("China"$_{\beta_2}$:$_{\alpha_3}$ "USA"$_{\beta_3}$:$_{\alpha_4}$ "Germany"$_{\beta_4}$:$_{\alpha_5}$[]$_{\alpha_6}$)
\end{lstlisting}
\caption{List comprehension example: Surface language forward desugared}
\label{fig:surface-language:example-4}
\end{figure}

\noindent
Desugaring the surface language program in \figref{surface-language:example-1} with these rules yields the core program in \figref{surface-language:example-4}, where the selections $\alpha_i$ on the original list comprehensions and sugared list notation have been propagated and the selections $\beta_i$ on constants and the pair constructor have been preserved.

\subsection{Backward desugaring}

In order to desugar backwards, we need something that plays the role of a trace (as in the backward evaluation analysis in \secref{data-dependencies:analyses:bwd}). For desugaring, it is sufficient to use the original program and allow the trace $t$ to range over raw surface terms $s$. \figref{desugar-bwd-short} then denotes a family of \textit{backward desugaring} functions $\desugarBwdR{t}$ of type $\Below{e} \to \Below{s}$, where $\desugarBwdR{t}$ is defined by recursion over $t$. These specify how core terms $e \in \Below{e}$ backward desugar over a trace $t$ to recover their surface representation $s \in \Below{s}$, where $t$ corresponds to the unselected form of $s$. During this, the join operation is used to combine selection information; this expresses that if any of the generated terms that $s$ desugared into are needed as resources, then $s$ itself is also needed.

{\flushleft \shadebox{$e \desugarBwdR{t} s$}
\hfill \textbfit{$e$ backward-desugars along $t$ to $s$}}
\begin{smathpar}
   \inferrule*[
      lab={\ruleName{$\desugarBwdR{}$-nil}}
   ]
   {
      \strut
   }
   {
      \annNil{\alpha} \desugarBwdR{\exNil} \annNil{\alpha}
   }
   %
   \and
   %
   \inferrule*[
      lab={\ruleName{$\desugarBwdR{}$-cons}}
   ]
   {
      e \desugarBwdR{t} s
      \\
      e' \desugarBwdR{t'} s'
   }
   {
      \annCons{e}{e'}{\alpha}
      \desugarBwdR{\exCons{t}{t'}}
      \annCons{s}{s'}{\alpha}
   }
   %
   \and
   %
   \inferrule*[
      lab={\ruleName{$\desugarBwdR{}$-non-empty-list}}
   ]
   {
      e \desugarBwdR{t} s
      \\
      e' \desugarBwdR{r} r'
   }
   {
      \annCons{e}{e'}{\alpha}
      \desugarBwdR{\exList{t}{r}}
      \annList{s}{r'}{\alpha}
   }
   %
   \and
   %
   \inferrule*[lab={\ruleName{$\desugarBwdR{}$-list-comp-done}}]
   {
      e \desugarBwdR{t} s
   }
   {
      \annCons{e}{\annNil{\alpha}}{\alpha'}
      \desugarBwdR{\exListComp{t}{\seqEmpty}}
      \annListComp{s}{\seqEmpty}{\alpha \join \alpha'}
   }
   \inferrule*[
      lab={\ruleName{$\desugarBwdR{}$-list-comp-gen}},
   ]
   {
      e \desugarBwdR{t} s
      \\
      \totaliseBwd{\sigma}{p}{\sigma'}{\beta}
      \\
      \sigma'
      \clauseBwdR{p}
      e'
      \\
      e'
      \desugarBwdR{\exListComp{t'}{\vec{q}}}
      \annListComp{s'}{\vec{q}'}{\beta'}
   }
   {
      \exApp{\exApp{\funConcatMap}{\exLambda{\sigma}}}{e}
      \desugarBwdR{\exListComp{t'}{\qualGenerator{p}{t} \concat \vec{q}}}
      \annListComp{s'}{\qualGenerator{p}{s} \concat \vec{q}'}{\beta \join \beta'}
   }
   %
   \and
   %
   \inferrule*[
      lab={\ruleName{$\desugarBwdR{}$-list-comp-guard}},
   ]
   {
      e' \desugarBwdR{t'} s'
      \\
      e \desugarBwdR{\exListComp{t}{\vec{q}}} \annListComp{s}{\vec{q}'}{\beta}
   }
   {
      \exApp{\exLambda{\elimBool{e}{\annNil{\alpha}}}}{e'}
      \desugarBwdR{\exListComp{t}{\qualGuard{t'} \concat \vec{q}}}
      \annListComp{s}{\qualGuard{s'} \concat \vec{q}'}{\alpha \join \beta}
   }
\end{smathpar}

\vspace{3mm}
{\flushleft \shadebox{$e \desugarBwdR{r} r'$}
\hfill \textbfit{$e$ backward-desugars along $r$ to $r'$}}
\begin{smathpar}
   \inferrule*[
      lab={\ruleName{$\desugarBwdR{}$-list-rest-end}}
   ]
   {
      \strut
   }
   {
      \annot{\exNil}{\alpha}
      \desugarBwdR{\exListEnd}
      \annot{\exListEnd}{\alpha}
   }
   %
   \and
   %
   \inferrule*[
      lab={\ruleName{$\desugarBwdR{}$-list-rest-cons}}
   ]
   {
      e \desugarBwdR{t} s
      \\
      e' \desugarBwdR{r} r'
   }
   {
      \annCons{e}{e'}{\alpha}
      \desugarBwdR{(\exListNext{t}{r})}
      (\annListNext{s}{r'}{\alpha})
   }
\end{smathpar}


The rule for nil $\annNil{\alpha}$ and other constants simply preserve the selection state. For non-empty lists, $\annCons{e}{e'}{\alpha}$, we recursively backward desugar the head and tail of the list and use its trace, $\exCons{t}{t'}$ or $\exList{t}{r}$, to determine the original list constructor used in its surface representation.

If the trace is of the form $\exListComp{t'}{\qualGenerator{p}{t} \concat \vec{q}}$, then the core term we backward desugar is necessarily of the form $\exApp{\exApp{\funConcatMap}{\exLambda{\sigma}}}{e}$. First, we recover the surface representation $s$ of the generator list $e$. The operation $\totaliseBwd{\sigma}{p}{\sigma'}{\beta}$ then transforms the total eliminator $\sigma$ into a singleton eliminator $\sigma'$ containing the branch for pattern $p$; it also returns $\beta$ which represents the demand placed on the branches of $\sigma$ which were synthesized by \textit{totalise}. Afterwards, $\sigma' \clauseBwdR{p} e'$ extracts the body $e'$ from the singleton eliminator $\sigma'$; this then desugars into the ``rest'' of the list comprehension $\annListComp{s'}{\vec{q}'}{\beta'}$, which can be used to reconstruct the original surface term $\annListComp{s'}{\qualGenerator{p}{s} \concat \vec{q}'}{\beta \join \beta'}$, combining in the demand $\beta$ from the synthesized branches with the demand $\beta'$ on the list comprehension.

When the trace has form $\exListComp{t}{\qualGuard{t'} \concat \vec{q}}$, the core term is necessarily $\exApp{\exLambda{\elimBool{e}{\annNil{\alpha}}}}{e'}$. The surface Boolean $s'$ is first recovered from $e'$ and represents the guard's condition. The rest of the list comprehension, $\annListComp{s}{\vec{q}'}{\beta}$, is then recovered from backward desugaring the body $e$ of the Boolean eliminator's $\exTrue$ branch; using this, we recover the final surface term $\annListComp{s}{\qualGuard{s'} \concat \vec{q}'}{\alpha \join \beta}$, combining the demand placed on each branch of the Boolean eliminator.

Lastly, a trace $\exListComp{t}{\seqEmpty}$ representing a list comprehension with an empty qualifier sequence necessarily has a corresponding core term $\annCons{e}{\annNil{\alpha}}{\alpha'}$. By backward desugaring $e$, we obtain the body of the original list comprehension, $s$. We use this to return $\annListComp{s}{\seqEmpty}{\alpha \join \alpha'}$, merging the demand placed on the cons and nil constructors in its core representation.

\begin{definition}[Disjoint join of partial continuations]
   Define $\disjjoin$ to be the smallest partial symmetric function satisfying the equations in \figref{surface-language:disjoin-join-elim}.
\end{definition}

% \begin{definition}[Forward and backward functions for desugaring]
%      Suppose $s \desugarFwdR e$. Define $\desugarFwdF{s}: \Below{s} \to \Below{e}$ and $\desugarBwdF{s}: \Below{e} \to \Below{s}$ to be $\desugarFwdR$ and $\desugarBwdR{s}$ domain-restricted to $\Below{s}$ and $\Below{e}$ respectively.
% \end{definition}

\begin{theorem}[Galois connection for desugaring]
  \label{thm:surface-language:desugar:gc}
     Suppose $s \desugarFwdR e$. Then $\desugarFwdF{s} \adjoint \desugarBwdF{s}$.
\end{theorem}
\newpage
\section{Extending to a realistic surface language}
\label{sec:surface-language}

Up until now we have primarily considered how to track dependencies between input and output values, however, our approach also allows us to inspect selections on terms themselves to determine which fragments of our program are involved in the construction of a value. The ability to do this is not restricted to programs written using the core calculus given in \secref{core-language} -- one may want to observe how data dependencies interact with more sophisticated language constructs, such as list comprehensions and piece-wise definitions.

To achieve this, it is unnecessary to augment our core language to accommodate for every new syntactic form and their forward and backward dependency semantics -- we can instead create a surface language on top of our core language. The translation between our surface and core language then gives rise to another Galois connection. In the \textit{forward} direction, we desugar annotated surface terms in their equivalent core representation whilst preserving their selection information. Conversely, in the \textit{backward} direction we reconstruct the original surface term from the core term whilst translating the selection information backward. The Galois connection for desugaring then composes with the Galois connection for dependency analysis in \secref{data-dependencies:analyses}, allowing our approach for data linking to easily extend to tracking dependencies between outputs and programs written in the surface language. In this section, we demonstrate how this can be achieved for a small program (\figref{surface-language:example-1}) written in our example surface language.

\begin{figure}
\begin{syntaxfig}
\mbox{Eliminator}
&
\sigma, \tau
&
::=
&
\ldots
\\
&&&
\elimBoolTrue{\kappa}
\\
&&&
\elimBoolFalse{\kappa}
\\
&&&
\elimListSingleton{\branchNil{\kappa}}
\\
&&&
\elimListSingleton{\branchCons{\sigma}}
\\[2mm]
\mbox{Raw term}
&
r
&
::=
&
\ldots
\\
&&&
\exLambda{\sigma}
&
\text{anonymous function}
\end{syntaxfig}
\caption{Additional syntax}
\end{figure}


\subsection{Surface language syntax}

Consider the annotated program in \figref{surface-language:example-1}; it uses list comprehensions to iterate through a list of records containing fields $\texttt{country}$ and $\texttt{output}$, and constructs a list of tuples containing countries and their total energy output. Assuming the output of this program were $\texttt{[("China", 304), ("USA", 125),}$ $\texttt{("Germany", 606)]}$, one could be interested in how selecting the string $\texttt{"China"}$ places a demand on list comprehensions as a resource.

\begin{figure}[H]
\small
\begin{lstlisting}
let totalFor c rows = sum [ row.output | row $\leftarrow$ rows, row.country == c ]$_{\alpha_1}$
in  [ (c ,$_{\beta_1}$ totalFor c data) | c $\leftarrow$ ["China"$_{\beta_2}$ , "USA"$_{\beta_3}$ , "Germany"$_{\beta_4}$ ] ]$_{\alpha_2}$
\end{lstlisting}
\caption{List comprehension example: Surface language}
\label{fig:surface-language:example-1}
\end{figure}

\noindent
The syntax necessary to construct \figref{surface-language:example-1} can be found in \figref{surface-language:syntax}, where we also provide the full surface language syntax for completeness. Surface terms are denoted by $s, t$; in particular, lists can now be written in the style $\kw{[} s \comma \ldots \comma s' \kw{]}$ using the term $\annList{s}{r}{\alpha}$, where $r$ either recursively defines the rest of the list, $\annListNext{s}{r}{\alpha}$, or is the end of the list, $\annListEnd{\alpha}$. List comprehensions, $\annListComp{s}{\vec{q}}{\alpha}$, then consist of a body $s$ which is constructed in reference to the environment introduced by a sequence of ``qualifiers'' $\vec{q}$.  There are two variants of qualifiers we use in \figref{surface-language:example-1}: \textit{generators}, $\qualGenerator{p}{s}$, which iterate through each of the elements in a list $s$ and binds them to the pattern $p$, and \textit{guards}, $\qualGuard{s}$, which enforce a Boolean condition $s$ to be true. It can be noted that the terms for lists and list comprehensions are annotated because they correspond to selectable values i.e. $\annNil{\alpha}$ and $\annCons{u}{v}{\alpha}$.

\subsection{Forward desugaring}

Desugaring in the forwards direction is given by the relation $s \desugarFwdR e$ in \figref{desugar-fwd-short} which specifies how to desugar a surface language term $s$ into its equivalent core language form $e$, whilst correctly retaining the selection information made on $s$.

{\flushleft \shadebox{$s \desugarFwdR e$}
\hfill \textbfit{$s$ forward-desugars to $e$}}
\begin{smathpar}
   \inferrule*[
      lab={\ruleName{$\desugarFwdR$-nil}}
   ]
   {
      \strut
   }
   {
      \annNil{\alpha} \desugarFwdR \annNil{\alpha}
   }
   %
   \and
   %
   \inferrule*[
      lab={\ruleName{$\desugarFwdR$-cons}}
   ]
   {
      s \desugarFwdR e
      \\
      s' \desugarFwdR e'
   }
   {
      \annCons{s}{s'}{\alpha} \desugarFwdR \annCons{e}{e'}{\alpha}
   }
   %
   \and
   %
   \inferrule*[
      lab={\ruleName{$\desugarFwdR$-non-empty-list}}
   ]
   {
      s \desugarFwdR e
      \\
      r \desugarFwdR e'
   }
   {
      \annList{s}{r}{\alpha} \desugarFwdR \annCons{e}{e'}{\alpha}
   }
   %
   \and
   %
   \inferrule*[lab={\ruleName{$\desugarFwdR$-list-comp-done}}]
   {
      s \desugarFwdR e
   }
   {
      \annListComp{s}{\seqEmpty}{\alpha}
      \desugarFwdR
      \annCons{e}{\annNil{\alpha}}{\alpha}
   }
   \and
   %
   \inferrule*[
      lab={\ruleName{$\desugarFwdR$-list-comp-gen}}
   ]
   {
      \annListComp{s}{\vec{q}}{\alpha} \desugarFwdR e
      \\
      p, e \clauseFwdR \sigma
      \\
      \totaliseFwd{\sigma}{\alpha}{p}{\sigma'}
      \\
      s' \desugarFwdR e'
   }
   {
      \annListComp{s}{\qualGenerator{p}{s'} \concat \vec{q}}{\alpha}
      \desugarFwdR
      \exApp{\exApp{\funConcatMap}{\exLambda{\sigma'}}}{e'}
   }
   %
   \and
   %
   \inferrule*[
      lab={\ruleName{$\desugarFwdR$-list-comp-guard}}
   ]
   {
      \annListComp{s}{\vec{q}}{\alpha} \desugarFwdR e
      \\
      s' \desugarFwdR e'
   }
   {
      \annListComp{s}{\qualGuard{s'} \concat \vec{q}}{\alpha}
      \desugarFwdR
      \exApp{\exLambda{\elimBool{e}{\annNil{\alpha}}}}{e'}
   }
   %
   \and
   %
   \inferrule*[
      lab={\ruleName{$\desugarFwdR$-list-comp-decl}}
   ]
   {
      \annListComp{s}{\vec{q}}{\alpha} \desugarFwdR e
      \\
      p, e \clauseFwdR \sigma
      \\
      s' \desugarFwdR e
   }
   {
      \annListComp{s}{\qualDeclaration{p}{s'} \concat \vec{q}}{\alpha}
      \desugarFwdR
      \exApp{\exLambda{\sigma}}{e}
   }
   \end{smathpar}

\vspace{5pt}
{\flushleft \shadebox{$r \desugarFwdR e$}
\hfill \textbfit{$r$ forward-desugars to $e$}}
\begin{smathpar}
   \inferrule*[
      lab={\ruleName{$\desugarFwdR$-list-rest-end}}
   ]
   {
      \strut
   }
   {
      \annListEnd{\alpha} \desugarFwdR \annNil{\alpha}
   }
   %
   \and
   %
   \inferrule*[
      lab={\ruleName{$\desugarFwdR$-list-rest-cons}}
   ]
   {
      s \desugarFwdR e
      \\
      r \desugarFwdR e'
   }
   {
      (\annListNext{s}{r}{\alpha}) \desugarFwdR \annCons{e}{e'}{\alpha}
   }
\end{smathpar}


\noindent
The rules for nil $\annNil{\alpha}$, cons $\annCons{s}{s'}{\alpha}$, and non-empty lists $\annList{s}{r}{\alpha}$, are straightforward: they recursively desugar all elements of the list and translate the sugared list constructors $\kw{[}_\alpha$, $\kw{,}_{\alpha}$, and $\kw{]}_{\alpha}$, to their core representations $\kw{(:)}_{\alpha}$ or $\annNil{\alpha}$. Desugaring list comprehensions, $\annListComp{s}{\vec{q}}{\alpha}$, is slightly more involved; each of these rules accounts for a different possible qualifier at the head of the sequence $\vec{q}$.

The rule for $\annListComp{s}{\qualGenerator{p}{s'} \concat \vec{q}}{\alpha}$ describes the desugaring of a list comprehension with a generator as its first qualifier. On encountering a generator, we want to map over each element in the list $s'$: if the element matches the pattern $p$ we return the rest of the (desugared) list comprehension $\annListComp{s}{\vec{q}}{\alpha}$, otherwise we return an empty list $\annNil{\alpha}$ annotated by the original selection on the list comprehension. We hence use the relation $\vec{p}, \kappa \clauseFwdR \sigma$ with pattern $p$ as $\vec{p}$ and the desugared form of $\annListComp{s}{\vec{q}}{\alpha}$ as its continuation $\kappa$, to construct an singleton eliminator $\sigma$. Afterwards, we ``totalise'' $\sigma$ using the relation $\totaliseFwd{\kappa}{\alpha}{\vec{\pi}}{\kappa'}$; this operation takes our partial eliminator $\sigma$ as a continuation $\kappa$ and an annotation $\alpha$ and recursively inserts an branch with body $\annNil{\alpha}$ for each missing pattern in $\sigma$. This returns us a \textit{total} eliminator $\sigma'$. The final step is to desugar the generator list $s'$ into $e'$ and map $\sigma'$ over it; the result of this is a list of lists which we then flatten into a single list. The process of mapping and flattening is achieved using $\funConcatMap$.

List comprehensions of the shape $\annListComp{s}{\qualGuard{s'} \concat \vec{q}}{\alpha}$ correspond to when their first qualifier is a guard, $\qualGuard{s'}$. Let the Boolean condition $s'$ desugar into $e'$. The idea then here is to construct a Boolean eliminator and apply it to $e'$: if the condition $e'$ holds then we return the desugared form of the rest of the list comprehension $\annListComp{s}{\vec{q}}{\alpha}$, otherwise we return an empty list $\annNil{\alpha}$ annotated by the selection $\alpha$ of the original list comprehension.

The final list comprehension rule for $\annListComp{s}{\seqEmpty}{\alpha}$ handles when the sequence of qualifiers in the list comprehension are empty. In this situation, we simply desugar the body $s$ into $e$ and cons it to an empty list; this results in $\annCons{e}{\annNil{\alpha}}{\alpha}$ where we propagate the original annotation $\alpha$ onto both the cons constructor and empty list.

\begin{figure}[H]
\small
\begin{lstlisting}
let totalFor c rows =
      sum (concatMap $\exLambda{\{\elimVar{\kw{row}}{  \exLambda{\elimBool{ \kw{(} \annCons{\kw{row.output}}{\annNil{\alpha_1}}{\alpha_1} \kw{)} }{\annNil{\alpha_1}}} \; \kw{(row.country == c)} }\}}$ rows)
in  concatMap $\exLambda{\{\elimVar{\kw{c}}{ \kw{(} \annCons{\kw{(c ,}_{\beta_1} \kw{totalFor c data)}}{\annNil{\alpha_2}}{\alpha_2} \kw{)}  }\}}$ ["China"$_{\beta_2}$, "USA"$_{\beta_3}$, "Germany"$_{\beta_4}$]
\end{lstlisting}
\caption{List comprehension example: Surface language forwards desugared}
\label{fig:surface-language:example-4}
\end{figure}

\noindent
Desugaring the surface language program in \figref{surface-language:example-1} with these rules yields the core program in \figref{surface-language:example-4}. We can see that the selections $\alpha_1$ and $\alpha_2$ on the original list comprehensions have been propagated onto all the constructs they desugar to which correspond to values. The other selections $\beta_i$, on the pair constructor and strings in \figref{surface-language:example-1}, are annotations on terms which are already in our core language; hence they are preserved in their original form in \figref{surface-language:example-4}.

\subsection{Backward desugaring}

In the reverse direction, backward desugaring is given as the relation $e \desugarBwdR{t} s$ in \figref{desugar-bwd-short}. This says that the core term $e$ can be desugared backwards over a trace $t$ to its original surface representation $s$, where $t$ corresponds to the unannotated form of $s$ acquired from forward desugaring.

{\flushleft \shadebox{$e \desugarBwdR{t} s$}
\hfill \textbfit{$e$ backward-desugars along $t$ to $s$}}
\begin{smathpar}
   \inferrule*[
      lab={\ruleName{$\desugarBwdR{}$-nil}}
   ]
   {
      \strut
   }
   {
      \annNil{\alpha} \desugarBwdR{\exNil} \annNil{\alpha}
   }
   %
   \and
   %
   \inferrule*[
      lab={\ruleName{$\desugarBwdR{}$-cons}}
   ]
   {
      e \desugarBwdR{t} s
      \\
      e' \desugarBwdR{t'} s'
   }
   {
      \annCons{e}{e'}{\alpha}
      \desugarBwdR{\exCons{t}{t'}}
      \annCons{s}{s'}{\alpha}
   }
   %
   \and
   %
   \inferrule*[
      lab={\ruleName{$\desugarBwdR{}$-non-empty-list}}
   ]
   {
      e \desugarBwdR{t} s
      \\
      e' \desugarBwdR{r} r'
   }
   {
      \annCons{e}{e'}{\alpha}
      \desugarBwdR{\exList{t}{r}}
      \annList{s}{r'}{\alpha}
   }
   %
   \and
   %
   \inferrule*[lab={\ruleName{$\desugarBwdR{}$-list-comp-done}}]
   {
      e \desugarBwdR{t} s
   }
   {
      \annCons{e}{\annNil{\alpha}}{\alpha'}
      \desugarBwdR{\exListComp{t}{\seqEmpty}}
      \annListComp{s}{\seqEmpty}{\alpha \join \alpha'}
   }
   \inferrule*[
      lab={\ruleName{$\desugarBwdR{}$-list-comp-gen}},
   ]
   {
      e \desugarBwdR{t} s
      \\
      \totaliseBwd{\sigma}{p}{\sigma'}{\beta}
      \\
      \sigma'
      \clauseBwdR{p}
      e'
      \\
      e'
      \desugarBwdR{\exListComp{t'}{\vec{q}}}
      \annListComp{s'}{\vec{q}'}{\beta'}
   }
   {
      \exApp{\exApp{\funConcatMap}{\exLambda{\sigma}}}{e}
      \desugarBwdR{\exListComp{t'}{\qualGenerator{p}{t} \concat \vec{q}}}
      \annListComp{s'}{\qualGenerator{p}{s} \concat \vec{q}'}{\beta \join \beta'}
   }
   %
   \and
   %
   \inferrule*[
      lab={\ruleName{$\desugarBwdR{}$-list-comp-guard}},
   ]
   {
      e' \desugarBwdR{t'} s'
      \\
      e \desugarBwdR{\exListComp{t}{\vec{q}}} \annListComp{s}{\vec{q}'}{\beta}
   }
   {
      \exApp{\exLambda{\elimBool{e}{\annNil{\alpha}}}}{e'}
      \desugarBwdR{\exListComp{t}{\qualGuard{t'} \concat \vec{q}}}
      \annListComp{s}{\qualGuard{s'} \concat \vec{q}'}{\alpha \join \beta}
   }
\end{smathpar}

\vspace{3mm}
{\flushleft \shadebox{$e \desugarBwdR{r} r'$}
\hfill \textbfit{$e$ backward-desugars along $r$ to $r'$}}
\begin{smathpar}
   \inferrule*[
      lab={\ruleName{$\desugarBwdR{}$-list-rest-end}}
   ]
   {
      \strut
   }
   {
      \annot{\exNil}{\alpha}
      \desugarBwdR{\exListEnd}
      \annot{\exListEnd}{\alpha}
   }
   %
   \and
   %
   \inferrule*[
      lab={\ruleName{$\desugarBwdR{}$-list-rest-cons}}
   ]
   {
      e \desugarBwdR{t} s
      \\
      e' \desugarBwdR{r} r'
   }
   {
      \annCons{e}{e'}{\alpha}
      \desugarBwdR{(\exListNext{t}{r})}
      (\annListNext{s}{r'}{\alpha})
   }
\end{smathpar}


The rule for nil $\annNil{\alpha}$ simply returns the term unchanged. For non-empty lists, $\annCons{e}{e'}{\alpha}$, we recursively backward desugar the head and tail of the list and then use its trace ($\exCons{t}{t'}$ or $\exList{t}{r}$) to determine the list constructor used in the original surface term. The rules for list comprehensions use the join operation to combine selection information; this expresses that if any of the terms that list comprehensions desugar into are needed as resources, then the original list comprehension is also needed.

For list comprehensions with a generator as their first qualifier, we backward desugar the term $\exApp{\exApp{\funConcatMap}{\exLambda{\sigma}}}{e}$ with respect to trace $\exListComp{t'}{\qualGenerator{p}{t} \concat \vec{q}}$. First, we recover the surface representation $s$ of the list generator $e$. The operation $\totaliseBwd{\sigma}{p}{\sigma'}{\beta}$ then transforms our total eliminator $\sigma$ into a singleton eliminator $\sigma'$ containing the branch for pattern $p$; it also returns the annotation $\beta$ found on the other branches of $\sigma$. Afterwards, $\sigma' \clauseBwdR{p} e'$ extracts the body $e'$ from the singleton eliminator $\sigma'$, which then desugars into the ``rest'' of the list comprehension $\annListComp{s'}{\vec{q}'}{\beta'}$ without the generator qualifier. By inserting the generator $\qualGenerator{p}{s}$ into this, we recover the original surface term $\annListComp{s'}{\qualGenerator{p}{s} \concat \vec{q}'}{\beta \join \beta'}$, where we annotate it by the join of all intermediate selections produced.

A similar process applies for list comprehensions with guards, where we backward desugar the core term $\exApp{\exLambda{\elimBool{e}{\annNil{\alpha}}}}{e'}$ over the trace $\exListComp{t}{\qualGuard{t'} \concat \vec{q}}$. The surface Boolean $s'$ is recovered from $e'$ and represents the guard's condition. The rest of the list comprehension without the guard, $\annListComp{s}{\vec{q}'}{\beta}$, is then recovered from backward desugaring the body $e$ of the Boolean eliminator's $\exTrue$ branch. Inserting the guard $\qualGuard{s'}$ into this list comprehension yields the final surface term $\annListComp{s}{\qualGuard{s'} \concat \vec{q}'}{\alpha \join \beta}$, which is annotated by the join of the selections found on each branch of the Boolean eliminator it originally desugared to.

Lastly, we can reconstruct a list comprehension with an empty qualifier sequence by backward desugaring the singleton list $\annCons{e}{\annNil{\alpha}}{\alpha'}$ with respect to the trace $\exListComp{t}{\seqEmpty}$. We first recursively backward desugar $e$ to obtain $s$, the body of the original list comprehension. We use this to return $\annListComp{s}{\seqEmpty}{\alpha \join \alpha'}$, annotated by the join of the selections on the cons and nil constructors in its core representation.

As an example, consider wanting to backwards desugar the following the selected core program given in \figref{surface-language:example-4}. The result can be seen in \figref{surface-language:example-5}; we can see that where the selections on each list comprehension is the join of the selections on the components it desugared to. (For continuity reasons from \figref{surface-language:example-4}, selections $\alpha_1 \join \alpha_1 \join \alpha_1$ and $\alpha_2 \join \alpha_2$ are trivially joins over the same annotations, however, the same reasoning applies for arbitrary selections).

\begin{figure}[H]
\small
\begin{lstlisting}
let totalFor c rows = sum [ row.output | row $\leftarrow$ rows, row.country == c ]$_{\alpha_1 \join \alpha_1 \join \alpha_1}$
in  [ (c ,$_{\beta_1}$ totalFor c data) | c $\leftarrow$ ["China"$_{\beta_2}$ , "USA"$_{\beta_3}$ , "Germany"$_{\beta_4}$ ] ]$_{\alpha_2 \join \alpha_2}$
\end{lstlisting}
\caption{List comprehension example: Core language backwards desugared}
\label{fig:surface-language:example-5}
\end{figure}


\begin{definition}[Disjoint join of partial continuations]
   Define $\disjjoin$ to be the smallest partial symmetric function satisfying the equations in \figref{surface-language:disjoin-join-elim}.
\end{definition}

\begin{definition}[Forward and backward functions for desugaring]
     Suppose $s \desugarFwdR e$. Define $\desugarFwdF{s}: \Below{s} \to \Below{e}$ and $\desugarBwdF{s}: \Below{e} \to \Below{s}$ to be $\desugarFwdR$ and $\desugarBwdR{s}$ domain-restricted to $\Below{s}$ and $\Below{e}$ respectively.
\end{definition}

\begin{theorem}[Galois connection for desugaring]
  \label{thm:surface-language:desugar:gc}
     Suppose $s \desugarFwdR e$. Then $\desugarFwdF{s} \adjoint \desugarBwdF{s}$.
\end{theorem}

% \begin{enumerate}
%    \item Refer back to previous section (e.g. explain that using the core-language directly can be inconvenient for tracking more interesting data dependencies).
%    \item Build on top of previous section by motivating the necessity for a surface language which desugars into the core language by giving an example of a program which uses desired language features (such as list comprehensions (w/ generators), clauses, totalise, ).
%    \item Use this example to introduce the syntactic constructs required in the surface language (\figref{surface-language:syntax}), and mention which syntactic forms require annotations. If necessary, explain the corresponding type rules (\figref{surface-language:typing-term}).
%    \item Afterwards, we explain that the implementation needs to capture not only the typical desugaring process of general languages, but acts as an extra bidirectional stage on top of the bidirectional analysis of the core language.
%    \begin{enumerate}
%       \item In the forwards direction, it must specify how annotations on surface language expressions can be correspondingly positioned on the core language expressions that they desugar to.
%       \item In the backwards direction, it must use the original surface-language expression as a trace t in order to reconstruct the original surface-level program. It must also specify how annotations on the core language propagate backwards to form annotations on the surface language.
%    \end{enumerate}
%    \item Explain the forward desugaring of our example in chronological order, making reference to and elaborating on the necessary forward desugaring rules when necessary for terms (\figref{desugar-fwd}) and clauses (\figref{clause-fwd}), and also totalise (\figref{totalise-fwd}).
%    \item Explain how we backward desugar from the core-language version of our example over the forward-trace form in chronological order to reconstruct the original surface-level program. Make reference to and elaborate on the necessary backward rules - for terms (\figref{desugar-bwd}), clauses (\figref{clause-bwd}), and totalise (\figref{totalise-bwd}).
% \end{enumerate}

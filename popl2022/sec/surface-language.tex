\newpage
\section{Extending to a realistic surface language}
\label{sec:surface-language}

\begin{enumerate}
   \item Refer back to previous section (e.g. explain that using the core-language directly can be inconvenient for tracking more interesting data dependencies).
   \item Build on top of previous section by motivating the necessity for a surface language which desugars into the core language by giving an example of a program which uses desired language features (such as list comprehensions, clauses).
   \item Use this example to introduce the syntactic constructs required in the surface language, and mention which syntactic forms require annotations.
   \item Afterwards, we explain that the implementation needs to capture not only the typical desugaring process of general languages, but acts as an extra bidirectional stage before the core-language is evaluated.
   \begin{enumerate}
      \item In the forwards direction, it must specify how annotations on surface language expressions can be correctly/correspondingly positioned on the core language expressions that they desugar to. It must also record a trace of the surface language expressions which we desugar from.
      \item In the backwards direction, it must use the forward-desugaring traces to reconstruct the original surface-level program. It must also specify how annotations on the core language propagate backwards to form annotations on the surface language.
   \end{enumerate}
\end{enumerate}

\begin{figure}
{\small
\begingroup
\renewcommand*{\arraystretch}{1}
\begin{minipage}[t]{0.5\textwidth}
\begin{tabularx}{\textwidth}{rL{2.3cm}L{3cm}}
%\rowcolor{verylightgray}
&\textbfit{Identifier}&
\\
$x, y ::=$
&
$\ldots$
&
\\
&
$\primOp$
&
operator name
\\[2mm]
&\textbfit{Surface term}&
\\
$s, t ::=$
&
$\ldots$
&
\\
&
$\exOp{\primOp}$
&
first-class operator
\\
&
$\exBinaryApp{s}{\primOp}{s'}$
&
binary application
\\
&
$\exLetRecMutual{\vec{g}}{s}$
&
recursive functions
\\
&
$\exIfThenElse{s}{s}{s}$
&
if
\\
&
$\exMatch{s}{\vec{\clauseUncurried{p}{s}}}$
&
match
\\
&
$\exLet{p}{s}{s}$
&
structured let
\\
&
$\annList{s}{r}{\alpha}$
&
non-empty list
\\
&
$\exListEnum{s}{s}$
&
list enum
\\
&
$\annListComp{s}{\vec{q}}{\alpha}$
&
list comprehension
\\[2mm]
&\textbfit{List rest term}&
\\
$r ::=$
&
$\annListEnd{\alpha}$
&
end
\\
&
$\annListNext{s}{r}{\alpha}$
&
cons
\\
\\
\end{tabularx}
\end{minipage}%
\begin{minipage}[t]{0.5\textwidth}
\begin{tabularx}{\textwidth}{rL{2.8cm}L{2.9cm}}
&\textbfit{Recursive function}&
\\
$g ::=$
&
$\bind{x}{\vec{c}}$
&
\\[2mm]
&\textbfit{Clause}&
\\
$c ::=$
&
$\clause{\vec{p}}{s}$
&
\\[2mm]
&\textbfit{Pattern}&
\\
$p ::=$
&
$\pattVar{x}$
&
variable
\\
&
$\pattRec{\vec{\bind{x}{p}}}$
&
record
\\
&
$\pattNil$
&
nil
\\
&
$\pattCons{p}{p}$
&
cons
\\
&
$\pattList{p}{o}$
&
non-empty list
\\[2mm]
&\textbfit{List rest pattern}&
\\
$o ::=$
&
$\pattListEnd$
&
end
\\
&
$\pattListNext{p}{o}$
&
cons
\\[2mm]
&\textbfit{Qualifier}&
\\
$q ::=$
&
$\qualGuard{s}$
&
guard
\\
&
$\qualDeclaration{p}{s}$
&
declaration
\\
&
$\qualGenerator{p}{s}$
&
generator
%$x, y$
%&
%&
%identifier
%\\
%$i, j$
%&&
%positive integer
%\\[2mm]
\end{tabularx}
\end{minipage}
\endgroup
}
\vspace{-2mm}
\caption{(Selectable) syntax for surface language}
\vspace{-2mm}
\label{fig:surface-language:syntax}
\end{figure}


\begin{definition}[Disjoint join of partial continuations]
   Define $\disjjoin$ to be the smallest partial symmetric function satisfying the equations in \figref{surface-language:disjoin-join-elim}.
\end{definition}

\begin{definition}[Forward and backward functions for desugaring]
     Suppose $s \desugarFwdR e$. Define $\desugarFwdF{s}: \Below{s} \to \Below{e}$ and $\desugarBwdF{s}: \Below{e} \to \Below{s}$ to be $\desugarFwdR$ and $\desugarBwdR{s}$ domain-restricted to $\Below{s}$ and $\Below{e}$ respectively.
\end{definition}

\begin{theorem}[Galois connection for desugaring]
  \label{thm:surface-language:desugar:gc}
     Suppose $s \desugarFwdR e$. Then $\desugarFwdF{s} \adjoint \desugarBwdF{s}$.
\end{theorem}

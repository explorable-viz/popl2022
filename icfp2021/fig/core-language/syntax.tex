\begin{figure}[H]
\begin{syntaxfig}
\mbox{Continuation}
&
\kappa
&
::=
&
e
&
\text{term}
\\
&&&
\sigma
&
\text{eliminator}
\\[2mm]
\mbox{Eliminator}
&
\sigma, \tau
&
::=
&
\elimVar{x}{\kappa}
&
\text{variable}
\\
&&&
\elimBool{\kappa}{\kappa'}
&
\text{Boolean}
\\
&&&
\elimProd{\sigma}
&
\text{pair}
\\
&&&
\elimList{\branchNil{\kappa}}{\branchCons{\sigma}}
&
\text{list}
\\[2mm]
\mbox{Selection state}
&
\alpha, \beta, \gamma
&
::=
&
\TT
&
\text{selected}
\\
&&&
\FF
&
\text{unselected}
\\[2mm]
\mbox{Term}
&
e
&
::=
&
\hole
&
\text{hole}
\\
&&&
\annTrue{\alpha} \mid \annFalse{\alpha}
&
\text{Boolean}
\\
&&&
\annInt{n}{\alpha}
&
\text{integer}
\\
&&&
\annPair{e}{e'}{\alpha}
&
\text{pair}
\\
&&&
\annNil{\alpha}
&
\text{nil}
\\
&&&
\annCons{e}{e'}{\alpha}
&
\text{cons}
\\
&&&
\annMatrix{e'}{x}{y}{e}{\alpha}
&
\text{matrix}
\\
&&&
x
&
\text{variable}
\\
&&&
\exOp{\primOp}
&
\text{first-class operator}
\\
&&&
\exApp{e}{e'}
&
\text{application}
\\
&&&
\exBinaryApp{e}{\primOp}{e'}
&
\text{binary application}
\\
&&&
\exMatrixAccess{e}{e}
&
\text{matrix lookup}
\\
&&&
\exMatrixDims{e}
&
\text{matrix dimensions}
\\
&&&
\exLet{x}{e}{e'}
&
\text{let}
\\
&&&
\exLetRec{f}{\sigma}{e}
&
\text{recursive function}
\end{syntaxfig}
\caption{Syntax of terms}
\end{figure}

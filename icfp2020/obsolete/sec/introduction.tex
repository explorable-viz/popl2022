\section{Introduction}

The notion of how much output we need from a computation $X$ -- what we
will call the \emph{demand} active on $X$ -- is inherently time-varying
and compositional. For example a user may want more or less from $X$
over time, which they might express by expanding or collapsing UI
widgets. More generally, there may be another computation $Y$ consuming
the output of $X$, and the demand active on $Y$ it itself varying over
time. Intuitively at any rate, because the demand on $Y$ is varying, it
seems reasonable that the demand that $Y$ places on $X$ should vary in
the same direction.

Put like this, the observation that demand is time-varying and
compositional is banal and unsurprising. Building applications that
respond easily to varying demand, on the other hand, requires some
effort. One might first turn to lazy functional languages as a suitable
implementation choice for such demand-sensitive applications. However,
in such languages the notion of demand, at least at the top level, is
fixed. While sub-computations are demanded to varying degrees --
depending on the extent to which they are consumed in order to produce
the result -- the result itself is usually forced to a depth of a single
constructor. In other words the ``terminal'' terms (normal forms) have
the following fixed structure:
\[v ::= \exUnit \mid \exPair{e_1}{e_2} \mid \exInl{e} \mid \exInr{e} \mid \exLambda{x}{e} \]
where $e$ is a (closed) unevaluated expression. If a user subsequently
wants more output, they must explicitly pick one of the unevaluated
sub-expressions and evaluate it. If it is a client system that wants
additional output, then this possibility must have been explicitly
catered for by the programmer at the interface between the two systems.

Our goal here is a runtime scheme for automatically and efficiently
propagating changes in demand into an active computation. This only
really makes sense if we no longer regard execution as a ``single-shot''
batch-mode activity that yields a \emph{value}, but rather as an ongoing
dialogue between producer and consumer, where the consumer requests more
output and the producer responds by computing more output. What we
propose here is therefore related to forms of incremental computation,
such as self-adjusting computation \cite{acar05}. However the problem we
consider here is simpler, namely incremental processing of changes in
demand.

An active demand of type $A$ is modelled as a \emph{trie}
\cite{hinze00} which, for any value of type $A$, specifies the portion
of that value which would be consumed by the client system. Demands
are partially ordered, giving rise to the notion of a
\emph{differential} demand, a pair of demands $(\sigma, \sigma')$
where $\sigma$ is below $\sigma'$. Differential demands model changes
in neededness, and we define an incremental operational semantics
which interprets an increase in demand on the output as a
corresponding increase in the demand on the program, computed by
precisely the additional steps required to compute the extra output.

For compositionality, we extend the standard notion of trie with support
for variable binding. This permits both \ttt{match}...\ttt{as}
expressions and functions defined by pattern-matching to be represented
as tries from (partial) values to expressions, capturing the demand on
the scrutinee or argument implied by the patterns. In the operational
semantics each step in the computation is indexed by the demand active
at that step.

\subsection{Examples}

Languages with pattern-matching typically employ a \emph{match-first}
rule, meaning that the first clause with a pattern which matches is
chosen. Match-first permits the following definition of
$\ttt{parity}$, adapted from \cite{cockx14}, which tests for equality
modulo 2:

\begin{lstlisting}
fun parity : nat $\times$ nat $\to$ bool {
   (zero, zero) $\to$ true;
   (zero, suc zero) $\to$ false;
   (zero, suc (suc n)) $\to$ parity zero n;
   (suc zero, zero) $\to$ false;
   (suc (suc n), zero) $\to$ parity n zero;
   (suc n, suc m) $\to$ parity n m
}
\end{lstlisting}


In order to explain the semantics of patterns in terms of eliminators,
the final $\ttt{(suc n, suc m)}$ clause is conceptually elaborated
into the following two clauses:

\begin{lstlisting}
   (suc zero, suc m) $\to$ parity zero m
   (suc (suc n), suc m) $\to$ parity (suc n) m
\end{lstlisting}

\noindent For simplicity, we assume here that patterns are already in
a \emph{sequential} form, meaning that there is no subsumption of one
clause by another (where a variable subsumes any constructor). In the
definition above the pattern $\ttt{suc n}$ subsumes $\ttt{suc zero}$
and $\ttt{suc (suc n)}$, and so the clauses are ill-formed. But by
replacing the $\ttt{(suc n, suc m)}$ clause by the two clauses above
and then simplifying to unify the two $\ttt{suc (suc n)}$ cases, we
can write $\ttt{parity}$ as follows:

\begin{lstlisting}
fun parity : nat $\times$ nat $\to$ bool {
   (zero, zero) $\to$ true;
   (zero, suc zero) $\to$ false;
   (zero, suc (suc n)) $\to$ parity zero n;
   (suc zero, zero) $\to$ false;
   (suc zero, suc n) $\to$ parity zero n;
   (suc (suc n), m) $\to$ parity n m
}
\end{lstlisting}


\noindent In the absence of subsumption, every clause is reachable,
regardless of the order in which the patterns are tested for matching.

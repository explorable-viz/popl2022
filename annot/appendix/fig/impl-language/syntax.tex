\begin{figure}
\begin{syntaxfig}
\mbox{Annotation}
&
\alpha
&
::=
&
\true
&
\text{true}
\\
&&&
\false
&
\text{false}
\\[2mm]
\mbox{Type}
&
A, B
&
::=
&
\tyInt
&
\text{integers}
\\
&&&
\tyList
&
\text{list of integers}
\\
&&&
\tyProd{A}{B}
&
\text{pairs}
\\
&&&
\tyFun{A}{B}
&
\text{function}
\\[2mm]
\mbox{Eliminator}
&
\sigma, \tau
&
::=
&
\elimVar{x}{e}
&
\text{variable}
\\
&&&
\elimList{\branchNil{e}}{\branchCons{x}{y}{e'}}
&
\text{list}
\\
&&&
\elimPair{x}{y}{e}
&
\text{pair}
\\[2mm]
\mbox{Term}
&
e
&
::=
&
\annot{r}{\alpha}
\\[2mm]
\mbox{Raw term}
&
r
&
::=
&
x
&
\text{variable}
\\
&&&
\exInt{n}
&
\text{integer}
\\
&&&
\exPair{e}{e'}
&
\text{pair}
\\
&&&
\exRec{f}{\sigma}
&
\text{recursive function}
\\
&&&
\exApp{e}{e'}
&
\text{application}
\\
&&&
\exAdd{e}{e'}
&
\text{addition}
\\
&&&
\exNil
&
\text{nil}
\\
&&&
\exCons{e}{e'}
&
\text{cons}
\\
&&&
\exLet{x}{e}{e'}
&
\text{let}
\end{syntaxfig}
\caption{Syntax for the implementation language}
\label{fig:impl-language:syntax}
\end{figure}

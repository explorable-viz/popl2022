\section{Galois slicing for evaluation}

\begin{figure}[H]
\begin{syntaxfig}
\mbox{Match}
&
w
&
::=
&
\matchVarNew{x}
&
\text{variable}
\\
&&&
\matchTrueNew \mid \matchFalseNew
&
\text{Boolean}
\\
&&&
\matchPair{w}{w'}
&
\text{pair}
\\
&&&
\matchNilNew
&
\text{nil}
\\
&&&
\matchConsNew{w}{w'}
&
\text{cons}
\\[2mm]
\mbox{Trace}
&
T, U
&
::=
&
\trVar{x}
&
\text{variable}
\\
&&&
\trInt{n}
&
\text{integer}
\\
&&&
\trPair{T}{U}
&
\text{pair}
\\
&&&
\trNil
&
\text{nil}
\\
&&&
\trCons{T}{U}
&
\text{cons}
\\
&&&
\trMatrix{T_{i,j}}{x}{y}{i}{j}{U}{i',j'}
&
\text{matrix}
\\
&&&
\trMatrixAccess{T}{i',j'}{U}{i,j}
&
\text{matrix lookup}
\\
&&&
\trMatrixDims{T}{i,j}
&
\text{matrix dimensions}
\\
&&&
\trApp{T}{U}{\matchPlug{w}{T'}}
&
\text{apply}
\\
&&&
\trAppPrim{T}{\phi}{U}{\exInt{n}}
&
\text{apply primitive}
\\
&&&
\trBinaryApp{T}{n}{\primOp}{\phi}{U}{m}
&
\text{binary application}
\\
&&&
\trLet{x}{T}{U}
&
\text{let}
\\
&&&
\trLetRec{f}{\sigma}{T}
&
\text{recursive function}
\end{syntaxfig}
\caption{Matches and traces}
\end{figure}

% \begin{figure}
\flushleft \shadebox{$\explType{\Gamma}{T}{A}$}
\begin{smathpar}
\inferrule*
{
   \envType{\Gamma}{\rho}
}
{
   \explType{\Gamma}{\sub{\trEmpty}{\rho}}{A}
}
%
\and
%
\inferrule*[right={$x : A \in \Gamma$}]
{
   \envType{\Gamma}{\rho}
}
{
   \explType{\Gamma}{\trVarTwo{x}{\rho}}{A}
}
%
\and
%
\inferrule*
{
   \explType{\Gamma}{T}{(\tyFun{A}{B})}
   \\
   \explType{\Gamma}{T'}{A}
   \\
   \elimType{\xi}{K}{A}{\Gamma'}
   \\
   \explType{\Gamma'}{U}{B}
}
{
   \explType
      {\Gamma}
      {\trApp{T}{T'}{\matchPlug{\xi}{U}}}
      {B}
}
\end{smathpar}
\caption{Typing rules for explanations}
\end{figure}


\begin{definition}
   \label{def:eval}
   \figref{evaluation:eval} defines a big-step, call-by-value operational
   semantics.
\end{definition}

For any expression $e$ of type $A$ in $\Gamma$ the judgement $\rho, e \eval{} v$
asserts that $e$, when evaluated in environment $\rho$ for $\Gamma$, yields a
value $v$.

\begin{definition}
   \label{def:uneval}
   \figref{evaluation:uneval} defines reverse evaluation.
\end{definition}

\begin{figure}
{\small \flushleft \shadebox{$\evalFwd{\rho}{e}{\alpha}{T}{v}$}%
\hfill \textbfit{$\rho$ and $e$, with argument availability $\alpha$, forward-analyse along $T$ to $v$}}
\begin{smathpar}
   \mprset{center}
   \inferrule*[
      lab={\ruleName{$\evalFwdS$-eq}}
   ]
   {
      e \eq e'
      \\
      \evalFwd{\rho}{e'}{\alpha}{T}{v}
   }
   {
      \evalFwd{\rho}{e}{\alpha}{T}{v}
   }
   %
   \and
   %
   \inferrule*[
      lab={\ruleName{$\evalFwdS$-var}},
      right={$\envLookup{\rho}{x}{v}$}
   ]
   {
      \strut
   }
   {
      \evalFwd{\rho}{\exVar{x}}{\alpha}{\trVar{x}{\rho}}{v}
   }
   %
   \and
   %
   \inferrule*[lab={\ruleName{$\evalFwdS$-lambda}}]
   {
      \strut
   }
   {
      \evalFwd{\rho}
              {\exLambda{\sigma}}
              {\alpha}
              {\trLambda{\sigma'}}
              {\exClosure{\rho}{\seqEmpty}{\sigma}}
   }
   %
   \and
   %
   \inferrule*[lab={\ruleName{$\evalFwdS$-int}}]
   {
      \strut
   }
   {
      \evalFwd{\rho}
              {\annInt{n}{\alpha}}
              {\alpha'}
              {\trInt{n}{\rho}}
              {\annInt{n}{\alpha \meet \alpha'}}
   }
   %
   \and
   %
   \inferrule*[lab={\ruleName{$\evalR$-record}}]
   {
      \evalFwd{\rho}{e_i}{\alpha'}{T_i}{v_i}
      \quad
      (\forall i \numleq \length{\vec{x}})
   }
   {
      \evalFwd{\rho}
              {\annRec{\vec{\bind{x}{e}}}{\alpha}}
              {\alpha'}
              {\trRec{\vec{\bind{x}{T}}}}
              {\annRec{\vec{\bind{x}{v}}}{\alpha \meet \alpha'}}
   }
   %
   \and
   %
   \inferrule*[
      lab={\ruleName{$\evalR$-project}},
      right={$i \numleq \length{\vec{x}}$}
   ]
   {
      \evalFwd{\rho}{e}{\alpha}{T}{\exRec{\vec{\bind{x}{v}}}}
   }
   {
      \evalFwd{\rho}
              {\exRecProj{e}{x}}
              {\alpha}
              {\trRecProj{T}{\vec{x}}{x}}
              {v_i}
   }
   %
   \and
   %
   \inferrule*[lab={\ruleName{$\evalFwdS$-nil}}]
   {
      \strut
   }
   {
      \evalFwd{\rho}
              {\annNil{\alpha}}
              {\alpha'}
              {\trNil{\rho}}
              {\annNil{\alpha \meet \alpha'}}
   }
   %
   \and
   %
   \inferrule*[
      lab={\ruleName{$\evalFwdS$-cons}},
   ]
   {
      \evalFwd{\rho}{e}{\alpha'}{T}{v}
      \\
      \evalFwd{\rho}{e'}{\alpha'}{U}{v'}
   }
   {
      \evalFwd{\rho}
              {\annCons{e}{e'}{\alpha}}
              {\alpha'}
              {\trCons{T}{U}}
              {\annCons{v}{v'}{\alpha \meet \alpha'}}
   }
   %
   \and
   %
   \inferrule*[
      lab={\ruleName{$\evalFwdS$-apply-prim}}
   ]
   {
      \evalFwdEq{\rho}{e}{\alpha}{T}{\exPrimOp{\phi}{\vec{v}}}
      \\
      \evalFwd{\rho}{e'}{\alpha}{U}{u}
   }
   {
      \evalFwd{\rho}
              {\exApp{e}{e'}}
              {\alpha}
              {\trAppPrim{T}{\phi}{\vec{n}}{U}{\exInt{m}}}
              {\primFwd{\phi}{\vec{n} \concat m}(\vec{v} \concat u)}
   }
   %
   \and
   %
   \inferrule*[
      lab={\ruleName{$\evalFwdS$-apply}},
      width={3in}
   ]
   {
      \evalFwdEq{\rho}{e}{\alpha}{T}{\exClosure{\rho_1}{h}{\sigma}}
      \\
      \rho_1, h \closeDefsR \rho_2
      \\
      \evalFwd{\rho}{e'}{\alpha}{U}{v}
      \\
      \matchFwd{v}{\sigma}{w}{\rho_3}{e^\twoPrime}{\beta}
      \\
      \evalFwd{\rho_1 \concat \rho_2 \concat \rho_3}{e^\twoPrime}{\beta}{T'}{v'}
   }
   {
      \evalFwd{\rho}{\exApp{e}{e'}}{\alpha}{\trApp{T}{U}{w}{T'}}{v'}
   }
   %
   \and
   %
   \inferrule*[lab={
      \ruleName{$\evalFwdS$-let-rec}}
   ]
   {
      \rho, h' \closeDefsR \rho'
      \\
      \evalFwd{\rho \concat \rho'}{e}{\alpha}{T}{v}
   }
   {
      \evalFwd{\rho}{\exLetRecMutual{h'}{e}}{\alpha}{\trLetRecMutual{h}{T}}{v}
   }
\end{smathpar}
\vspace{1mm}

{\small \flushleft \shadebox{$\matchFwd{v}{\sigma}{w}{\rho}{\kappa}{\alpha}$}%
\hfill \textbfit{$v$ and $\sigma$ forward-analyse along $w$ to $\rho$ and $\kappa$, with argument availability $\alpha$}}
\begin{smathpar}
   \inferrule*[
      lab={\ruleName{$\matchFwdS$-eq}}
   ]
   {
      (v,\sigma) \eq (v',\sigma')
      \\
      \matchFwd{v'}{\sigma'}{w}{\rho}{\kappa}{\alpha}
   }
   {
      \matchFwd{v}{\sigma}{w}{\rho}{\kappa}{\alpha}
   }
   %
   \and
   %
   \inferrule*[
      lab={\ruleName{$\matchFwdS$-var}}
   ]
   {
      \strut
   }
   {
      \matchFwd{v}{\elimVar{x}{\kappa}}{\matchVar{x}}{\bind{x}{v}}{\kappa}{\TT}
   }
   %
   \and
   %
   \inferrule*[
      lab={\ruleName{$\matchFwdS$-true}}
   ]
   {
      \strut
   }
   {
      \matchFwd{\annTrue{\alpha}}
               {\elimBool{\kappa}{\kappa'}}
               {\matchTrue}
               {\seqEmpty}{\kappa}{\alpha}
   }
   %
   \and
   %
   \inferrule*[
      lab={\ruleName{$\matchFwdS$-false}}
   ]
   {
      \strut
   }
   {
      \matchFwd{\annFalse{\alpha}}
               {\elimBool{\kappa}{\kappa'}}
               {\matchFalse}
               {\seqEmpty}{\kappa'}{\alpha}
   }
   %
   \and
   %
   \inferrule*[
      lab={\ruleName{$\matchFwdS$-unit}}
   ]
   {
      \strut
   }
   {
      \matchFwd{\annot{\exRecEmpty}{\alpha}}
               {\elimRecEmpty{\kappa}}
               {\matchRecEmpty}
               {\seqEmpty}
               {\kappa}
               {\alpha}
   }
   %
   \and
   %
   \inferrule*[
      lab={\ruleName{$\matchFwdS$-record}}
   ]
   {
      \matchFwd{\annRec{\vec{\bind{x}{v}}}{\FF}}
               {\elimRec{\vec{x}}{\sigma}}
               {\matchRec{\vec{\bind{x}{w}}}}
               {\rho}
               {\sigma'}
               {\beta}
      \\
      \matchFwd{u}{\sigma'}{w}{\rho'}{\kappa}{\beta'}
   }
   {
      \matchFwd{\annRec{\vec{\bind{x}{v}} \concat \bind{y}{u}}{\alpha}}
               {\elimRec{\vec{x} \concat y}{\sigma}}
               {\matchRec{\vec{\bind{x}{w}} \concat \bind{y}{w'}}}
               {\rho \concat \rho'}
               {\kappa}
               {\alpha \meet \beta \meet \beta'}
   }
   %
   \and
   %
   \inferrule*[
      lab={\ruleName{$\matchFwdS$-nil}}
   ]
   {
      \strut
   }
   {
      \matchFwd{\annNil{\alpha}}
               {\elimList{\kappa}{\sigma'}}
               {\matchNil}
               {\seqEmpty}{\kappa}{\alpha}
   }
   %
   \and
   %
   \inferrule*[
      lab={\ruleName{$\matchFwdS$-cons}}
   ]
   {
      \matchFwd{v}{\sigma}{w}{\rho}{\tau}{\beta}
      \\
      \matchFwd{v'}{\tau}{w'}{\rho'}{\kappa}{\beta'}
   }
   {
      \matchFwd{\annCons{v}{v'}{\alpha}}
               {\elimList{\kappa}{\sigma}}
               {\matchCons{w}{w'}}
               {\rho \concat \rho'}{\kappa}{\alpha \meet \beta \meet \beta'}
   }
\end{smathpar}
\caption{Forward data dependency (Boolean cases for $\evalFwdR{T}$ omitted)}
\label{fig:data-dependencies:fwd}
\end{figure}

\begin{figure}
{\small \flushleft \shadebox{$\evalBwd{v}{T}{\rho}{e}{\alpha}$}%
\hfill \textbfit{$v$ backward-analyses along $T$ to $\rho$ and $e$, with argument demand $\alpha$}}
\begin{smathpar}
   \inferrule*[
      lab={\ruleName{$\evalBwdS$-eq}}
   ]
   {
      v \eq v'
      \\
      \evalBwd{v'}{T}{\rho}{e}{\alpha}
   }
   {
      \evalBwd{v}{T}{\rho}{e}{\alpha}
   }
   %
   \and
   %
   \inferrule*[
      lab={\ruleName{$\evalBwdS$-var}},
      right={$\envLookupBwd{\rho'}{\rho}{\bind{x}{v}}$}
   ]
   {
      \strut
   }
   {
      \evalBwd{v}{\trVar{x}{\rho}}{\rho'}{\exVar{x}}{\FF}
   }
   %
   \and
   %
   \inferrule*[
      lab={\ruleName{$\evalBwdS$-lambda}}
   ]
   {
      \strut
   }
   {
      \evalBwd{\exClosure{\rho}{\seqEmpty}{\sigma}}
              {\trLambda{\sigma'}}
              {\rho}
              {\exLambda{\sigma}}
              {\FF}
   }
   %
   \and
   %
   \inferrule*[
      lab={\ruleName{$\evalBwdS$-int}}
   ]
   {
      \strut
   }
   {
      \evalBwd{\annInt{n}{\alpha}}
              {\trInt{n}{\rho}}
              {\hole_{\rho}}
              {\annInt{n}{\alpha}}
              {\alpha}
   }
   %
   \and
   %
   \inferrule*[lab={\ruleName{$\evalBwdS$-record}}]
   {
      \evalBwd{v_i}{T_i}{\rho_i}{e_i}{\alpha_i'}
      \quad
      (\forall i \numleq \length{\vec{x}})
   }
   {
      \evalBwd{\annRec{\vec{\bind{x}{v}}}{\alpha}}
              {\trRec{\vec{\bind{x}{T}}}}
              {\bigjoin\vec{\rho}}
              {\annRec{\vec{\bind{x}{e}}}{\alpha}}
              {\alpha \join \bigjoin\vec{\alpha}'}
   }
   %
   \and
   %
   \inferrule*[lab={\ruleName{$\evalBwdS$-project}}]
   {
      \evalBwd{\annRec{\mapUpdate{\vec{\bind{x}{\hole}}}{x_i}{v}}{\FF}}
              {T}
              {\rho}
              {e}
              {\alpha}
   }
   {
      \evalBwd{v}
              {\trRecProj{T}{\vec{x}}{x_i}}
              {\rho}
              {\exRecProj{e}{x_i}}
              {\alpha}
   }
   %
   \and
   %
   \inferrule*[
      lab={\ruleName{$\evalBwdS$-nil}}
   ]
   {
      \strut
   }
   {
      \evalBwd{\annNil{\alpha}}
              {\trNil{\rho}}
              {\hole_{\rho}}
              {\annNil{\alpha}}
              {\alpha}
   }
   %
   \and
   %
   \inferrule*[
      lab={\ruleName{$\evalBwdS$-cons}}
   ]
   {
      \evalBwd{v}{T}{\rho}{e}{\alpha}
      \\
      \evalBwd{v'}{U}{\rho'}{e'}{\alpha'}
   }
   {
      \evalBwd{\annCons{v}{v'}{\beta}}
              {\trCons{T}{U}}
              {\rho \join \rho'}
              {\annCons{e}{e'}{\beta}}
              {\beta \join \alpha \join \alpha'}
   }
   %
   \and
   %
   \inferrule*[
      lab={\ruleName{$\evalBwdS$-apply-prim}},
      right={$\primBwd{\phi}{\vec{n} \concat m}(v) = \vec{u} \concat u'$}
   ]
   {
      \evalBwd{\exPrimOp{\phi}{\vec{u}}}{T}{\rho}{e_1}{\alpha}
      \\
      \evalBwd{u'}{U}{\rho'}{e_2}{\alpha'}
   }
   {
      \evalBwd{v}
              {\trAppPrim{T}{\phi}{\vec{n}}{U}{\exInt{m}}}
              {\rho \join \rho'}
              {\exApp{e_1}{e_2}}
              {\alpha \join \alpha'}
   }
   %
   \and
   %
   \inferrule*[
      lab={\ruleName{$\evalBwdS$-apply}},
      width={3.25in}
   ]
   {
      \evalBwd{v}{T'}{\rho_1 \concat \rho_2 \concat \rho_3}{e}{\beta}
      \\
      \matchBwd{\rho_3}{e}{\beta}{w}{v'}{\sigma}
      \\
      \evalBwd{v'}{U}{\rho}{e_2}{\alpha}
      \\
      \rho_2 \closeDefsBwdR \rho_1', h
      \\
      \evalBwd{\exClosure{\rho_1 \join \rho_1'}{h}{\sigma}}{T}{\rho'}{e_1}{\alpha'}
   }
   {
      \evalBwd{v}{\trApp{T}{U}{w}{T'}}{\rho \join \rho'}{\exApp{e_1}{e_2}}{\alpha \join \alpha'}
   }
   %
   \and
   %
   \inferrule*[
      lab={\ruleName{$\evalBwdS$-let-rec}}
   ]
   {
      \evalBwd{v}{T}{\rho \concat \rho_1}{e}{\alpha}
      \\
      \rho_1 \closeDefsBwdR \rho', h'
   }
   {
      \evalBwd{v}{\trLetRecMutual{h}{T}}{\rho \join \rho'}{\exLetRecMutual{h'}{e}}{\alpha}
   }
\end{smathpar}
\vspace{1mm}

{\small \flushleft \shadebox{$\matchBwd{\rho}{\kappa}{\alpha}{w}{v}{\sigma}$}%
\hfill \textbfit{$\rho$ and $\kappa$, with argument demand $\alpha$, backward-analyse along $w$ to $v$ and $\sigma$}}
\begin{smathpar}
   \inferrule*[lab={\ruleName{$\matchBwdS$-var}}]
   {
      \strut
   }
   {
      \matchBwd{\bind{x}{v}}{\kappa}{\alpha}{\matchVar{x}}{v}{\elimVar{x}{\kappa}}
   }
   %
   \and
   %
   \inferrule*[lab={\ruleName{$\matchBwdS$-unit}}]
   {
      \strut
   }
   {
      \matchBwd{\seqEmpty}
               {\kappa}
               {\alpha}
               {\matchRecEmpty}
               {\annot{\exRecEmpty}{\alpha}}
               {\elimRecEmpty{\kappa}}
   }
   %
   \and
   %
   \inferrule*[lab={\ruleName{$\matchBwdS$-nil}}]
   {
      \strut
   }
   {
      \matchBwd{\seqEmpty}{\kappa}{\alpha}{\matchNil}{\annNil{\alpha}}{\elimList{\kappa}{\hole}}
   }
   %
   \and
   %
   \inferrule*[lab={\ruleName{$\matchBwdS$-record}}]
   {
      \matchBwd{\rho'}{\kappa}{\alpha}{w'}{u}{\sigma}
      \\
      \matchBwd{\rho}
               {\sigma}
               {\alpha}
               {\matchRec{\vec{\bind{x}{w}}}}
               {\annRec{\vec{\bind{x}{v}}}{\beta}}
               {\tau}
   }
   {
      \matchBwd{\rho \concat \rho'}
               {\kappa}
               {\alpha}
               {\matchRec{\vec{\bind{x}{w}} \concat \bind{y}{w'}}}
               {\annRec{\vec{\bind{x}{v}} \concat \bind{y}{u}}{\alpha}}
               {\elimRec{\vec{x} \concat y}{\tau}}
   }
   %
   \and
   %
   \inferrule*[lab={\ruleName{$\matchBwdS$-cons}}]
   {
      \matchBwd{\rho'}{\kappa}{\alpha}{w'}{v'}{\sigma}
      \\
      \matchBwd{\rho}{\sigma}{\alpha}{w}{v}{\tau}
   }
   {
      \matchBwd{\rho \concat \rho'}
               {\kappa}
               {\alpha}
               {\matchCons{w}{w'}}
               {\annCons{v}{v'}{\alpha}}
               {\elimList{\hole}{\tau}}
   }
   %
   \and
   %
   \inferrule*[lab={\ruleName{$\matchBwdS$-true}}]
   {
      \strut
   }
   {
      \matchBwd{\seqEmpty}{\kappa}{\alpha}{\matchTrue}{\annTrue{\alpha}}{\elimBool{\kappa}{\hole}}
   }
   %
   \and
   %
   \inferrule*[lab={\ruleName{$\matchBwdS$-false}}]
   {
      \strut
   }
   {
      \matchBwd{\seqEmpty}{\kappa}{\alpha}{\matchFalse}{\annFalse{\alpha}}{\elimBool{\hole}{\kappa}}
   }
\end{smathpar}
\vspace{1mm}

{\small\flushleft \shadebox{$\envLookupBwd{\rho'}{\rho}{\bind{x}{v}}$}%
\hfill \textbfit{$\bind{x}{v}$ backward-analyses along $\rho$ to $\rho'$}}
\begin{smathpar}
   \inferrule*[
      lab={\ruleName{$\envLookupBwdS$-head}}
   ]
   {
      \strut
   }
   {
      \envLookupBwd{(\hole_{\rho} \concat \bind{x}{u})}
                   {\rho \concat \bind{x}{v}}
                   {\bind{x}{u}}
   }
   %
   \and
   %
   \inferrule*[
      lab={\ruleName{$\envLookupBwdS$-tail}},
      right={$x \neq y$}
   ]
   {
      \envLookupBwd{\rho'}{\rho}{\bind{x}{u}}
   }
   {
      \envLookupBwd{(\rho' \concat \bind{y}{\hole})}
                   {\rho \concat \bind{y}{v}}
                   {\bind{x}{u}}
   }
\end{smathpar}
\vspace{1mm}

{\small\flushleft\shadebox{$\rho \closeDefsBwdR \rho', h$}%
\hfill \textbfit{$\rho$ backward-analyses to $\rho'$ and $h$}}
\begin{salign}
   \vec{\bind{x}{v}}
   &\closeDefsBwdR
   (\bigjoin\vec{\rho}, \vec{\bind{x}{\sigma}} \join {\bigjoin\vec{h}})
   &
   \text{ where }
   v_i = \exClosure{\rho_i}{h_i}{\sigma_i}
\end{salign}

\caption{Backward data dependency (Boolean cases for $\evalBwdR{T}$ omitted)}
\label{fig:data-dependencies:bwd}
\end{figure}

\begin{figure}[p]
\flushleft \shadebox{$\rho, e \eval{} T, u$}
\begin{smathpar}
   \inferrule*[left={\ruleName{$\eval{}$-annot}}]
   {
      \rho, r
      \eval{}
      T, \annot{v}{\alpha'}
   }
   {
      \rho,
      \annot{r}{\alpha}
      \eval{}
      T, \annot{v}{\alpha \wedge \alpha'}
   }
\end{smathpar}
\\
\flushleft \shadebox{$\rho, r \eval{} T, u$}
\begin{smathpar}
   \inferrule*[
      left={\ruleName{$\eval{}$-var}},
      right={\textnormal{$\exThunkVar{x}{u} \in \rho$}}
   ]
   {
      \strut
   }
   {
      \rho, \exVar{x} \eval{} \trVarTwo{x}{\rho}, u
   }
   %
   \and
   %
   \inferrule*[left={\ruleName{$\eval{}$-const}}]
   {
      \strut
   }
   {
      \rho,
      \exConst{k}
      \eval{}
      \sub{\trEmpty}{\rho}, \exConst{k}
   }
   %
   \and
   %
   \inferrule*[left={\ruleName{$\eval{}$-op}}]
   {
      \strut
   }
   {
      \rho,
      \phi
      \eval{}
      \sub{\trEmpty}{\rho}, \phi
   }
   %
   \and
   %
   \inferrule*[left={\ruleName{$\eval{}$-fun}}]
   {
      \strut
   }
   {
      \rho,
      \exFun{\sigma}
      \eval{}
      \sub{\trEmpty}{\rho}, \exClosureNew{\rho}{\seqEmpty}{\exFun{\sigma}}
   }
   %
   \and
   %
   \inferrule*[left={\ruleName{$\eval{}$-app}}]
   {
      \rho, e \eval{} T, \annot{\exClosureNew{\rho_1}{\delta}{\exFun{\sigma}}}{\alpha}
      \\
      \rho_1, \delta \closeDefs \rho_2
      \\
      \rho, e' \eval{} T', u
      \\
      u, \sigma \lookupR \rho_3, \matchPlug{\xi}{e^\twoPrime}, \alpha'
      \\
      \rho_1 \concat \rho_2 \concat \rho_3, e^\twoPrime
      \eval{}
      U, \annot{v}{\alpha^\twoPrime}
   }
   {
      \rho,
      \exApp{e}{e'}
      \eval{}
      \trApp{T}{T'}{\matchPlug{\xi}{U}},
      \annot{v}{\alpha \wedge \alpha' \wedge \alpha^\twoPrime}
   }
   %
   \and
   %
   \inferrule*[
      left={\ruleName{$\eval{}$-app-unary}},
      right={\textnormal{$k \in \dom{\interpret{\phi}}$}}
   ]
   {
      \rho, e \eval{} \explVal{T}{\annot{\phi}{\alpha}}
      \\
      \rho, e' \eval{} \explVal{T'}{\annot{k}{\alpha'}}
   }
   {
      \rho,
      \exApp{e}{e'}
      \eval{}
      \trUnaryApp{(\explVal{T}{\phi})}{(\explVal{T'}{\exConst{k}})},
      \annot{\interpret{\phi}(k)}{\alpha \wedge \alpha'}
   }
   %
   \and
   %
   \inferrule*[
      left={\ruleName{$\eval{}$-app-binary}},
      right={\textnormal{$(k, k') \in \dom{\interpret{\primOp}}$
      }}
   ]
   {
      \rho, e \eval{} \explVal{T}{\annot{k}{\alpha'}}
      \\
      \rho, e' \eval{} \explVal{T'}{\annot{k'}{\smash{\alpha^\twoPrime}}}
   }
   {
      \rho,
      \exPrimApp{e}{\annot{\primOp}{\alpha}}{e'}
      \eval{}
      \trPrimApp{(\explVal{T}{k})}{\primOp}{(\explVal{T'}{k'})},
      \annot{\interpret{\primOp}(k, k')}{\smash{\alpha \wedge \alpha' \wedge \alpha^\twoPrime}}
   }
   %
   \and
   %
   \inferrule*[left={\ruleName{$\eval{}$-constr}}]
   {
      \rho,
      \vec{e}
      \eval{}
      \vec{\explVal{T}{u}}
   }
   {
      \rho,
      \exConstr{d}{\vec{e}}
      \eval{}
      \explVal{\sub{\trEmpty}{\rho}}{\exConstr{d}{\vec{\explVal{T}{u}}}}
   }
   %
   \and
   %
   \inferrule*[left={\ruleName{$\eval{}$-match}}]
   {
      \rho, e \eval{} \explVal{T}{u}
      \\
      u, \sigma \lookupR \rho_1, \matchProd{\xi}{e'}, \alpha
      \\
      \rho \concat \rho_1, e' \eval{} \explVal{T'}{\annot{v}{\alpha'}}
   }
   {
      \rho,
      \exMatch{e}{\sigma}
      \eval{}
      \explVal{\trMatch{T}{\matchPlug{\xi}{T'}}}{\annot{v}{\alpha \wedge \alpha'}}
   }
   %
   \and
   %
   \inferrule*[left={\ruleName{$\eval{}$-let}}]
   {
      \rho, e \eval{} \explVal{T}{u}
      \\
      \envExtend{\rho}{x}{u}, e' \eval{} \explVal{U}{u'}
   }
   {
      \rho, \exLet{x}{e}{e'}
      \eval{}
      \explVal{\trLet{x}{\explVal{T}{u}}{U}}{u'}
   }
   %
   \and
   %
   \inferrule*[left={\ruleName{$\eval{}$-letrec}}]
   {
      \rho, \delta \closeDefs \rho_1
      \\
      \rho \concat \rho_1, e \eval{} \explVal{T}{u}
   }
   {
      \rho, \exLetrec{\delta}{e}
      \eval{}
      \explVal{\trLetrec{\delta}{T}}{u}
   }
   %
   \and
   %
   \inferrule*[left={\ruleName{$\eval{}$-defs}}]
   {
      \rho, \vec{d} \closeDefs \rho_1, \vec{z}
      \\
      \rho \concat \rho_1, e \eval{} \explVal{T}{u}
   }
   {
      \rho, \exDefs{\vec{d}}{e}
      \eval{}
      \explVal{\trDefs{\vec{z}}{T}}{u}
   }
\end{smathpar}
\vspace{5pt}

\flushleft \shadebox{$\rho, \vec{e} \eval{} \vec{\explVal{T}{u}}$}
\begin{smathpar}
   \inferrule*
   {
      \strut
   }
   {
      \rho,
      \exUnit
      \eval{}
      \seqEmpty
   }
   \and
   %
   \inferrule*
   {
      \rho, e \eval{} \explVal{T}{u}
      \\
      \rho, \vec{e} \eval{} \vec{\explVal{T}{u}}
   }
   {
      \rho,
      e \concat \vec{e}
      \eval{}
      (\explVal{T}{u}) \concat (\vec{\explVal{T}{u}})
   }
\end{smathpar}
\caption{Evaluation for terms and term sequences}
\end{figure}
 % redo as a function from derivation trees?
\begin{figure}
\flushleft \shadebox{$T, u \uneval \rho, e$}
\begin{smathpar}
   \inferrule*[left={\ruleName{$\uneval$-var}}]
   {
      \strut
   }
   {
      {\trVarTwo{x}{\rho}}, \annot{v}{\alpha}
      \uneval
      \envExtend{\sub{\bot}{\rho}}{x}{\annot{v}{\alpha}},
      \annot{\exVar{x}}{\alpha}
   }
   %
   \and
   %
   \inferrule*[left={\ruleName{$\uneval$-const}}]
   {
      \strut
   }
   {
      \explVal{\sub{\trEmpty}{\rho}}{\annot{\exConst{k}}{\alpha}}
      \uneval
      \sub{\bot}{\rho},
      \annot{\exConst{k}}{\alpha}
   }
   %
   \and
   %
   \inferrule*[left={\ruleName{$\uneval$-fun}}]
   {
      \strut
   }
   {
      \sub{\trEmpty}{\rho}, \annot{\exClosureNew{\rho'}{\seqEmpty}{\exFun{\sigma}}}{\alpha}
      \uneval
      \rho',
      \annot{(\exFun{\sigma})}{\alpha}
   }
   %
   \and
   %
   \inferrule*[left={\ruleName{$\uneval$-app}}]
   {
      U, \annot{v}{\alpha} \uneval \rho_1' \concat \rho_2 \concat \rho_3, e^\twoPrime
      \\
      \rho_3, \matchPlug{\xi}{e^\twoPrime}, \alpha \unlookupR{} u, \sigma
      \\
      T', u \uneval \rho, e'
      \\
      \rho_2 \uncloseDefs \rho_1^\twoPrime, \delta'
      \\
      T, \annot{\exClosureNew{\rho_1' \vee \rho_1^\twoPrime}{\delta'}{\exFun{\sigma}}}{\alpha}
      \uneval
      \rho', e
   }
   {
      \trApp{T}{T'}{\matchPlug{\xi}{U}}, \annot{v}{\alpha}
      \uneval
      \rho \vee \rho',
      \annot{(\exApp{e}{e'})}{\alpha}
   }
   %
   \and
   %
   \inferrule*[
      left={\ruleName{$\uneval$-app-unary}}
   ]
   {
      T', \annot{c}{\alpha} \uneval \rho', e'
      \\
      T, \annot{\phi}{\alpha} \uneval \rho, e
   }
   {
      \trApp{(\explVal{T}{\phi})}{(\explVal{T'}{c})}{\sub{\trEmpty}{\envEmpty}},
      \annot{v}{\alpha}
      \uneval
      \rho \vee \rho',
      \annot{(\exApp{e}{e'})}{\alpha}
   }
   %
   \and
   %
   \inferrule*[
      left={\ruleName{$\uneval$-app-binary}}
   ]
   {
      T', \annot{c'}{\alpha} \uneval \rho', e'
      \\
      T, \annot{c}{\alpha} \uneval \rho, e
   }
   {
      \trPrimApp{(\explVal{T}{c})}{\primOp}{(\explVal{T'}{c'})},
      \annot{v}{\alpha}
      \uneval
      \rho \vee \rho',
      \annot{(\exPrimApp{e}{\annot{\primOp}{\alpha}}{e'})}{\alpha}
   }
   %
   \and
   %
   \inferrule*[left={\ruleName{$\uneval$-constr}}]
   {
      \vec{T}, \vec{u} \uneval \rho', \vec{e}
   }
   {
      \trConstr{c}{\vec{T}}, \annot{\exConstr{c}{\vec{u}}}{\alpha}
      \uneval
      \rho',
      \annot{\exConstr{c}{\vec{e}}}{\alpha}
   }
   %
   \and
   %
   \inferrule*[left={\ruleName{$\uneval$-match}}]
   {
      \explVal{T'}{\annot{v}{\alpha}} \uneval \rho' \concat \rho_1, e'
      \\
      \rho_1, \matchPlug{\xi}{e'}, \alpha \unlookupR{} u, \sigma
      \\
      \explVal{T}{u} \uneval \rho, e
   }
   {
      \explVal{\trMatch{T}{\matchPlug{\xi}{T'}}}{\annot{v}{\alpha}}
      \uneval
      \rho \vee \rho',
      \annot{(\exMatch{e}{\sigma})}{\alpha}
   }
   %
   \and
   %
   \inferrule*[left={\ruleName{$\uneval$-let}}]
   {
      \explVal{U}{\annot{v}{\alpha}} \uneval \envExtend{\rho}{x}{u'}, e'
      \\
      \explVal{T}{u'} \uneval \rho', e
   }
   {
      \explVal{\trLet{x}{\explVal{T}{u}}{U}}{\annot{v}{\alpha}}
      \uneval
      \rho \vee \rho', \annot{(\exLet{x}{e}{e'})}{\alpha}
   }
   %
   \and
   %
   \inferrule*[left={\ruleName{$\uneval$-letrec}}]
   {
      \explVal{T}{\annot{v}{\alpha}} \uneval \rho \concat \rho_1, e
      \\
      \rho_1 \uncloseDefs \rho', \delta'
   }
   {
      \explVal{\trLetrec{\delta}{T}}{\annot{v}{\alpha}}
      \uneval
      \rho \vee \rho', \annot{(\exLetrec{\delta'}{e})}{\alpha}
   }
\end{smathpar}
\\
\flushleft \shadebox{$\vec{T}, \vec{u} \uneval \rho, \vec{e}$}
\begin{smathpar}
   \inferrule*
   {
      \strut
   }
   {
      \annot{\seqEmpty}{\rho}, \seqEmpty
      \uneval
      \sub{\bot}{\rho},
      \exUnit
   }
   \and
   %
   \inferrule*
   {
      \vec{T}, \vec{u} \uneval \rho_1, \vec{e}
      \\
      T, u \uneval \rho_1, e
   }
   {
      T \concat \vec{T},
      u \concat \vec{u}
      \uneval
      \rho_1 \vee \rho_2,
      e \concat \vec{e}
   }
\end{smathpar}
\caption{Reverse evaluation for terms and term sequences}
\end{figure}


\subsection{Galois connection}

\begin{theorem}
\label{thm:gc-eval}

Suppose $\rho, e \eval{} T, u$. Write $\eval{}_{\rho, e}$ to mean $\eval{}$
domain-restricted to $\Ann{\rho, e}$ and $\uneval_{\rho, e}$ to mean $\uneval$
domain-restricted to $\Ann{T, u}$. Then $\eval{}_{\rho, e}$ and $\uneval_{\rho,
e}$ form a Galois connection: \[{\eval{}_{\rho, e}} \adjoint {\uneval_{\rho,
e}}: \Ann{\rho, e} \to \Ann{T, u}\]

\end{theorem}

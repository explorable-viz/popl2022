\section{Galois slicing for evaluation}

\begin{figure}[H]
{\small
\begingroup
\renewcommand*{\arraystretch}{1}
\begin{minipage}[t]{0.5\textwidth}
\begin{tabularx}{\textwidth}{rL{2cm}L{3cm}}
&\textbfit{Trace}&
\\
$T, U ::=$
&
$\trVar{x}{\rho}$
&
variable
\\
&
$\trTrue{\rho} \mid \trFalse{\rho}$
&
Boolean
\\
&
$\trInt{n}{\rho}$
&
integer
\\
&
$\trRec{\vec{\bind{x}{T}}}$
&
record
\\
&
$\trRecProj{T}{\vec{x}}{x}$
&
record projection
\\
&
$\trNil{\rho}$
&
nil
\\
&
$\trCons{T}{U}$
&
cons
\\
&
$\trLambda{\sigma}$
&
anonymous function
\\
&
$\trApp{T}{U}{w}{T'}$
&
application
\end{tabularx}
\end{minipage}%
\begin{minipage}[t]{0.5\textwidth}
\begin{tabularx}{\textwidth}{rL{2.5cm}L{3cm}}
\\
&
$\trAppPrimNew{\phi}{U}{\exInt{n}}$
&
primitive application
\\
&
$\trLetRecMutual{h}{T}$
&
recursive let
\\[2mm]
&\textbfit{Match}&
\\
$w ::=$
&
$\matchVar{x}$
&
variable
\\
&
$\matchTrue \mid \matchFalse$
&
Boolean
\\
&
$\matchRec{\vec{\bind{x}{w}}}$
&
record
\\
&
$\matchNil$
&
nil
\\
&
$\matchCons{w}{w'}$
&
cons
\end{tabularx}
\end{minipage}
\endgroup
}
\caption{Syntax of traces and matches}
\label{fig:core-language:syntax-trace}
\end{figure}

% \begin{figure}
\flushleft \shadebox{$\explType{\Gamma}{T}{A}$}
\begin{smathpar}
\inferrule*
{
   \strut
}
{
   \explType{\Gamma}{\trEmpty}{A}
}
%
\and
%
\inferrule*[right={$x : A \in \Gamma$}]
{
   \explType{\Gamma'}{T}{A}
}
{
   \explType{\Gamma}{\trVar{x}{\Gamma'}{T}}{A}
}
%
\and
%
\inferrule*
{
   \explType{\Gamma}{T}{(\tyFun{A}{B})}
   \\
   \explType{\Gamma}{T'}{A}
   \\
   \elimType{\xi}{K}{A}{\Gamma'}
   \\
   \explType{\Gamma'}{U}{B}
}
{
   \explType
      {\Gamma}
      {\trApp{T}{T'}{\matchPlug{\xi}{U}}}
      {B}
}
%
\and
%
\inferrule*[right={\textnormal{$\interpret{\primOp} \in \BinaryOp{C}{C'}{A}$}}]
{
   \explValType{\Gamma}{\explVal{T}{c}}{C}
   \\
   \explValType{\Gamma}{\explVal{T'}{c'}}{C'}
}
{
   \explType
      {\Gamma}
      {\trPrimApp
         {(\explVal{T}{c})}
         {\exPrimOp}
         {(\explVal{T'}{c'})}}
      {A}
}
%
\and
%
\inferrule*
{
   \explType{\Gamma}{T}{A}
   \\
   \elimType{\xi}{\ExprPartial{B}}{A}{\Gamma'}
   \\
   \explType{\Gamma \concat \Gamma'}{T'}{B}
}
{
   \explType
      {\Gamma}
      {\trMatch{T}{\matchPlug{\xi}{T'}}}
      {B}
}
%
\and
%
\inferrule*
{
   \explValType{\Gamma}{\explVal{T}{u}}{A}
   \\
   \explType{\cxtExtend{\Gamma}{x}{A}}{U}{B}
}
{
   \explType
      {\Gamma}
      {\trLet{x}{\explVal{T}{u}}{\indexed{U}}}
      {B}
}
%
\and
%
\inferrule*
{
   \Gamma \vdash \delta: \Delta
   \\
   \explType{\Gamma \concat \Delta}{T}{A}
}
{
   \explType
      {\Gamma}
      {\trLetrec{\delta}{\indexed{T}}}
      {A}
}
\end{smathpar}
\vspace{5pt}

\flushleft \shadebox{$\explValType{\Gamma}{\vec{\explVal{T}{u}}}{\vec{A}}$}
\begin{smathpar}
\inferrule*
{
   \strut
}
{
   \explValType{\Gamma}{\seqEmpty}{\seqEmpty}
}
%
\and
%
\inferrule*
{
   \explValType{\Gamma}{\explVal{T}{u}}{A}
   \\
   \explValType{\Gamma}{\vec{\explVal{T}{u}}}{\vec{A}}
}
{
   \explValType{\Gamma}{(\explVal{T}{u}) \concat (\vec{\explVal{T}{u}})}{(A \concat \vec{A})}
}
\end{smathpar}
\caption{Typing rules for explanations}
\end{figure}


\begin{definition}
   \label{def:eval}
   \figref{evaluation:eval} defines a big-step, call-by-value operational
   semantics.
\end{definition}

For any expression $e$ of type $A$ in $\Gamma$ the judgement $\rho, e \eval{} v$
asserts that $e$, when evaluated in environment $\rho$ for $\Gamma$, yields a
value $v$.

\begin{definition}
   \label{def:uneval}
   \figref{evaluation:uneval} defines reverse evaluation.
\end{definition}

\begin{figure}
\flushleft\shadebox{$\rho, \delta \closeDefs \rho'$}
\begin{salign}
   \rho, \delta
   &\closeDefs
   [f: \exClosure{\rho}{\delta}{\sigma} \mid \exFun{f}{\sigma} \in \delta]
\end{salign}
\\[5mm]
\flushleft \shadebox{$\explVal{T}{\rho, e \eval v}$}
\begin{smathpar}
   \inferrule*[lab={\ruleName{$\eval$-lambda}}]
   {
      \strut
   }
   {
      \explVal
         {\trLambda{\sigma}}
         {\rho, \exLambda{\sigma} \eval \exClosure{\rho}{\seqEmpty}{\sigma}}
   }
   %
   \and
   %
   \inferrule*[lab={\ruleName{$\eval$-apply}}]
   {
      \explVal{T}{\rho, e \eval \exClosure{\rho_1}{\delta}{\sigma}}
      \\
      \rho_1, \delta \closeDefs \rho_2
      \\
      \explVal{U}{\rho, e' \eval v}
      \\
      \explVal{\xi}{v, \sigma \match \rho_3, e^\twoPrime}
      \\
      \explVal{T'}{\rho_1 \concat \rho_2 \concat \rho_3, e^\twoPrime \eval v'}
   }
   {
      \explVal{\trApp{T}{U}{\matchPlug{\xi}{T'}}}{\rho, \exApp{e}{e'} \eval v'}
   }
   %
   \and
   %
   \inferrule*[lab={\ruleName{$\eval$-let-struct}}]
   {
      \explVal{T}{\rho, e \eval v}
      \\
      \explVal{\xi}{v, \sigma \match \rho', e'}
      \\
      \explVal{U}{\rho \concat \rho', e' \eval v'}
   }
   {
      \explVal
         {\trLetStructured{\singleton{\xi}{T}}{U}}
         {\rho, \exLetStructured{\singleton{\sigma}{e}}{e'} \eval v'}
   }
\end{smathpar}
\caption{Additional evaluation rules}
\end{figure}

\begin{figure}[p]
\flushleft \shadebox{$\rho, e \eval{} T, u$}
\begin{smathpar}
   \inferrule*[lab={\ruleName{$\eval{}$-annot}}]
   {
      \rho, r
      \eval{}
      T, \annot{v}{\alpha'}
   }
   {
      \rho,
      \annot{r}{\alpha}
      \eval{}
      T, \annot{v}{\alpha \wedge \alpha'}
   }
\end{smathpar}
\\
\flushleft \shadebox{$\rho, r \eval{} T, u$}
\begin{smathpar}
   \inferrule*[
      lab={\ruleName{$\eval{}$-var}},
      right={\textnormal{$\exThunkVar{x}{u} \in \rho$}}
   ]
   {
   }
   {
      \rho, \exVar{x} \eval{} \trVarTwo{x}{\rho}, u
   }
   %
   \and
   %
   \inferrule*[lab={\ruleName{$\eval{}$-const}}]
   {
   }
   {
      \rho,
      \exConst{k}
      \eval{}
      \sub{\trEmpty}{\rho}, \exConst{k}
   }
   %
   \and
   %
   \inferrule*[lab={\ruleName{$\eval{}$-fun}}]
   {
   }
   {
      \rho,
      \exFun{\sigma}
      \eval{}
      \sub{\trEmpty}{\rho}, \exClosureNew{\rho}{\seqEmpty}{\exFun{\sigma}}
   }
   %
   \and
   %
   \inferrule*[lab={\ruleName{$\eval{}$-app}}]
   {
      \rho, e \eval{} T, \annot{\exClosureNew{\rho_1}{\delta}{\exFun{\sigma}}}{\alpha}
      \\
      \rho_1, \delta \closeDefs \rho_2
      \\
      \rho, e' \eval{} T', u
      \\\\
      u, \sigma \lookupR \rho_3, \matchPlug{\xi}{e^\twoPrime}, \alpha'
      \\
      \rho_1 \concat \rho_2 \concat \rho_3, e^\twoPrime
      \eval{}
      U, \annot{v}{\alpha^\twoPrime}
   }
   {
      \rho,
      \exApp{e}{e'}
      \eval{}
      \trApp{T}{T'}{\matchPlug{\xi}{U}},
      \annot{v}{\alpha \wedge \alpha' \wedge \alpha^\twoPrime}
   }
   %
   \and
   %
   \inferrule*[
      lab={\ruleName{$\eval{}$-app-unary}},
      right={\textnormal{$k \in \dom{\interpret{\phi}}$}}
   ]
   {
      \rho, e \eval{} T, \annot{\phi}{\alpha}
      \\
      \rho, e' \eval{} T', \annot{k}{\alpha'}
   }
   {
      \rho,
      \exApp{e}{e'}
      \eval{}
      \trUnaryApp{(\explVal{T}{\phi})}{(\explVal{T'}{\exConst{k}})},
      \annot{\interpret{\phi}(k)}{\alpha \wedge \alpha'}
   }
   %
   \and
   %
   \inferrule*[
      lab={\ruleName{$\eval{}$-app-binary}},
      right={\textnormal{$(k, k') \in \dom{\interpret{\primOp}}$
      }}
   ]
   {
      \rho, e \eval{} T, \annot{k}{\alpha'}
      \\
      \rho, e' \eval{} T', \annot{k'}{\smash{\alpha^\twoPrime}}
   }
   {
      \rho,
      \exPrimApp{e}{\annot{\primOp}{\alpha}}{e'}
      \eval{}
      \trPrimApp{(\explVal{T}{k})}{\primOp}{(\explVal{T'}{k'})},
      \annot{\interpret{\primOp}(k, k')}{\smash{\alpha \wedge \alpha' \wedge \alpha^\twoPrime}}
   }
   %
   \and
   %
   \inferrule*[lab={\ruleName{$\eval{}$-constr}}]
   {
      \rho,
      \vec{e}
      \eval{}
      \vec{T}, \vec{u}
   }
   {
      \rho,
      \exConstr{c}{\vec{e}}
      \eval{}
      \trConstr{c}{\vec{T}}, \exConstr{c}{\vec{u}}
   }
   %
   \and
   %
   \inferrule*[lab={\ruleName{$\eval{}$-match}}]
   {
      \rho, e \eval{} T, u
      \\
      u, \sigma \lookupR \rho_1, \matchProd{\xi}{e'}, \alpha
      \\
      \rho \concat \rho_1, e' \eval{} T', \annot{v}{\alpha'}
   }
   {
      \rho,
      \exMatch{e}{\sigma}
      \eval{}
      \trMatch{T}{\matchPlug{\xi}{T'}}, \annot{v}{\alpha \wedge \alpha'}
   }
   %
   \and
   %
   \inferrule*[lab={\ruleName{$\eval{}$-defs}}]
   {
      \rho, \vec{d} \closeDefs \rho_1, \vec{z}
      \\
      \rho \concat \rho_1, e \eval{} T, u
   }
   {
      \rho, \exDefs{\vec{d}}{e}
      \eval{}
      \trDefs{\vec{z}}{T}, u
   }
\end{smathpar}
\vspace{5pt}

\flushleft \shadebox{$\rho, \vec{e} \eval{} \vec{T}, \vec{u}$}
\begin{smathpar}
   \inferrule*
   {
      \strut
   }
   {
      \rho,
      \seqEmpty
      \eval{}
      \sub{\seqEmpty}{\rho},
      \seqEmpty
   }
   \and
   %
   \inferrule*
   {
      \rho, e \eval{} T, u
      \\
      \rho, \vec{e} \eval{} \vec{T}, \vec{u}
   }
   {
      \rho,
      e \concat \vec{e}
      \eval{}
      T \concat \vec{T},
      u \concat \vec{u}
   }
\end{smathpar}
\caption{Evaluation for terms and term sequences}
\end{figure}
 % redo as a function from derivation trees?
\begin{figure}
\flushleft \shadebox{$T, u \uneval \rho, e$}
\mprset{flushleft}
\begin{smathpar}
   \inferrule*[lab={\ruleName{$\uneval$-var}}]
   {
      \strut
   }
   {
      \trVarTwo{x}{\rho}, \annot{v}{\alpha}
      \uneval
      \envExtend{\sub{\bot}{\rho}}{x}{\annot{v}{\alpha}},
      \annot{\exVar{x}}{\alpha}
   }
   %
   \and
   %
   \inferrule*[lab={\ruleName{$\uneval$-const}}]
   {
      \strut
   }
   {
      \sub{\trEmpty}{\rho}, \annot{\exConst{k}}{\alpha}
      \uneval
      \sub{\bot}{\rho},
      \annot{\exConst{k}}{\alpha}
   }
   %
   \and
   %
   \inferrule*[lab={\ruleName{$\uneval$-fun}}]
   {
      \strut
   }
   {
      \sub{\trEmpty}{\rho}, \annot{\exClosureNew{\rho'}{\seqEmpty}{\exFun{\sigma}}}{\alpha}
      \uneval
      \rho',
      \annot{(\exFun{\sigma})}{\alpha}
   }
   %
   \and
   %
   \inferrule*[lab={\ruleName{$\uneval$-app}}]
   {
      U, \annot{v}{\alpha} \uneval \rho_1' \concat \rho_2 \concat \rho_3, e^\twoPrime
      \\
      \rho_3, \matchPlug{\xi}{e^\twoPrime}, \alpha \unlookupR{} u, \sigma
      \\\\
      T', u \uneval \rho, e'
      \\
      \rho_2 \uncloseDefs \rho_1^\twoPrime, \delta'
      \\
      T, \annot{\exClosureNew{\rho_1' \vee \rho_1^\twoPrime}{\delta'}{\exFun{\sigma}}}{\alpha}
      \uneval
      \rho', e
   }
   {
      \trApp{T}{T'}{\matchPlug{\xi}{U}}, \annot{v}{\alpha}
      \uneval
      \rho \vee \rho',
      \annot{(\exApp{e}{e'})}{\alpha}
   }
   %
   \and
   %
   \inferrule*[
      lab={\ruleName{$\uneval$-app-unary}}
   ]
   {
      T', \annot{c}{\alpha} \uneval \rho', e'
      \\
      T, \annot{\phi}{\alpha} \uneval \rho, e
   }
   {
      \trApp{(\explVal{T}{\phi})}{(\explVal{T'}{c})}{\sub{\trEmpty}{\envEmpty}},
      \annot{v}{\alpha}
      \uneval
      \rho \vee \rho',
      \annot{(\exApp{e}{e'})}{\alpha}
   }
   %
   \and
   %
   \inferrule*[
      lab={\ruleName{$\uneval$-app-binary}}
   ]
   {
      T', \annot{c'}{\alpha} \uneval \rho', e'
      \\
      T, \annot{c}{\alpha} \uneval \rho, e
   }
   {
      \trPrimApp{(\explVal{T}{c})}{\primOp}{(\explVal{T'}{c'})},
      \annot{v}{\alpha}
      \uneval
      \rho \vee \rho',
      \annot{(\exPrimApp{e}{\annot{\primOp}{\alpha}}{e'})}{\alpha}
   }
   %
   \and
   %
   \inferrule*[lab={\ruleName{$\uneval$-constr}}]
   {
      \vec{T}, \vec{u} \uneval \rho', \vec{e}
   }
   {
      \trConstr{c}{\vec{T}}, \annot{\exConstr{c}{\vec{u}}}{\alpha}
      \uneval
      \rho',
      \annot{\exConstr{c}{\vec{e}}}{\alpha}
   }
   %
   \and
   %
   \inferrule*[lab={\ruleName{$\uneval$-match}}]
   {
      T', \annot{v}{\alpha} \uneval \rho' \concat \rho_1, e'
      \\
      \rho_1, \matchPlug{\xi}{e'}, \alpha \unlookupR{} u, \sigma
      \\
      T, u \uneval \rho, e
   }
   {
      \trMatch{T}{\matchPlug{\xi}{T'}}, \annot{v}{\alpha}
      \uneval
      \rho \vee \rho',
      \annot{(\exMatch{e}{\sigma})}{\alpha}
   }
   %
   \and
   %
   \inferrule*[lab={\ruleName{$\uneval$-let}}]
   {
      U, \annot{v}{\alpha} \uneval \envExtend{\rho}{x}{u'}, e'
      \\
      T, u' \uneval \rho', e
   }
   {
      \trLet{x}{T}{U}, \annot{v}{\alpha}
      \uneval
      \rho \vee \rho', \annot{(\exLet{x}{e}{e'})}{\alpha}
   }
   %
   \and
   %
   \inferrule*[lab={\ruleName{$\uneval$-letrec}}]
   {
      T, \annot{v}{\alpha} \uneval \rho \concat \rho_1, e
      \\
      \rho_1 \uncloseDefs \rho', \delta'
   }
   {
      \trLetrec{\delta}{T}, \annot{v}{\alpha}
      \uneval
      \rho \vee \rho', \annot{(\exLetrec{\delta'}{e})}{\alpha}
   }
\end{smathpar}
\\[2mm]
\flushleft \shadebox{$\vec{T}, \vec{u} \uneval \rho, \vec{e}$}
\begin{smathpar}
   \inferrule*
   {
      \strut
   }
   {
      \annot{\seqEmpty}{\rho}, \seqEmpty
      \uneval
      \sub{\bot}{\rho},
      \seqEmpty
   }
   \and
   %
   \inferrule*
   {
      \vec{T}, \vec{u} \uneval \rho_1, \vec{e}
      \\
      T, u \uneval \rho_1, e
   }
   {
      T \concat \vec{T},
      u \concat \vec{u}
      \uneval
      \rho_1 \vee \rho_2,
      e \concat \vec{e}
   }
\end{smathpar}
\caption{Reverse evaluation for terms and term sequences}
\end{figure}


\subsection{Galois connection}

\begin{theorem}
\label{thm:gc-eval}

Suppose $\rho, e \eval{} T, u$. Write $\eval{}_{\rho, e}$ to mean $\eval{}$
domain-restricted to $\Ann{\rho, e}$ and $\uneval_{\rho, e}$ to mean $\uneval$
domain-restricted to $\Ann{T, u}$. Then $\eval{}_{\rho, e}$ and $\uneval_{\rho,
e}$ form a Galois connection: \[{\eval{}_{\rho, e}} \adjoint {\uneval_{\rho,
e}}: \Ann{\rho, e} \to \Ann{T, u}\]

\end{theorem}

\section{Galois slicing for pattern-matching}

\begin{figure}
\begin{syntaxfig}
\mbox{Match}
&
\xi
&
::=
&
\matchVar{x}{\matchHole{}}
&
\text{variable}
\\
&&&
\matchUnit{\matchHole{}}
&
\text{unit}
\\
&&&
\matchSum{\xi}{\sigma}
&
\text{inject left}
\\
&&&
\matchSumLL{\xi}
&
\\
&&&
\matchSumRR{\xi}
&
\\
&&&
\matchSum{\sigma}{\xi}
&
\text{inject right}
\\
&&&
\matchProd{\xi}{\xi'}
&
\text{product}
\\
&&&
\matchRoll{\xi}
&
\text{roll}
\end{syntaxfig}
\caption{Syntax of matches}
\end{figure}

% \begin{figure}
\flushleft \shadebox{$\matchType{\Gamma}{\matchPlug{\xi}{\kappa}}{K}{A}{\Gamma'}$}
\begin{smathpar}
\inferrule*[right={$\kappa \in K_{\Gamma \concat \Gamma'}$}]
{
   \matchType{\Gamma}{\xi}{K}{A}{\Gamma'}
}
{
   \matchType{\Gamma}{\matchPlug{\xi}{\kappa}}{K}{A}{\Gamma'}
}
\end{smathpar}
\\[3mm]
\flushleft \shadebox{$\matchType{\Gamma}{\xi}{K}{A}{\Gamma'}$}
\begin{smathpar}
\inferrule*
{
   \strut
}
{
   \matchType{\Gamma}{\matchVar{x}{\matchHole{}}}{K}{A}{\cxtExtend{\cxtEmpty}{x}{A}}
}
%
\and
%
\inferrule*
{
   \strut
}
{
   \matchType{\Gamma}{\matchUnit{\matchHole{}}}{K}{\tyUnit}{\cxtEmpty}
}
%
\and
%
\inferrule*
{
   \matchType{\Gamma}{\xi}{K}{A}{\Gamma'}
   \\
   \trieType{\sigma}{K}{B}{\Gamma}
}
{
   \matchType{\Gamma}{\matchSumL{\xi}{\sigma}}{K}{(\tySum{A}{B})}{\Gamma'}
}
%
\and
%
\inferrule*
{
   \trieType{\sigma}{K}{A}{\Gamma}
   \\
   \matchType{\Gamma}{\xi}{K}{B}{\Gamma'}
}
{
   \matchType{\Gamma}{\matchSumR{\sigma}{\xi}}{K}{(\tySum{A}{B})}{\Gamma'}
}
%
\and
%
\inferrule*
{
   \matchType{\Gamma}{\explVal{T}{\xi}}{K'}{A}{\Delta}
   \\
   \matchType{\Gamma \concat \Delta}{\explVal{U}{\xi'}}{K}{B}{\Gamma'}
}
{
   \matchType{\Gamma}{\matchProd{\xi}{\xi'}}{K}{(\tyProd{A}{B})}{\Gamma'}
}
%
\and
%
\inferrule*
{
   \matchType{\Gamma}{\xi}{K}{\subst{A}{\tyRec{\alpha}{A}}{\alpha}}{\Gamma'}
}
{
   \matchType{\Gamma}{\matchRoll{\xi}}{K}{(\tyRec{\alpha}{A})}{\Gamma'}
}
\end{smathpar}
\vspace{5pt}

\flushleft \shadebox{$\explMatchType{\Gamma}{\explVal{T}{\xi}}{K}{A}{\Gamma'}$}
\begin{smathpar}
\inferrule*
{
   \explType{\Gamma}{T}{A}
   \\
   \matchType{\Gamma}{\xi}{T}{A}{\Gamma'}
}
{
   \explMatchType{\Gamma}{\explVal{T}{\xi}}{K}{A}{\Gamma'}
}
\end{smathpar}
\caption{Typing rules for matches}
\end{figure}


\begin{definition}
   \label{def:match}
   \figref{pattern-matching:match} defines pattern matching.
\end{definition}

Pattern-matching consumes the least precise approximant of the value of $e$
which is precise enough to match a unique entry $\kappa$ of $\sigma$. When $e$
is the argument to a function, $\kappa$ will be the body of the selected branch;
if $\sigma$ is a product trie, then $\kappa$ will be another trie, representing
the demand on the second component.

\begin{definition}
   \label{def:unmatch}
   \figref{pattern-matching:unmatch} defines reverse pattern matching.
\end{definition}

\begin{figure}
\flushleft \shadebox{$v, \sigma \match \rho, \matchPlugLow{\xi}{e}$}
\begin{smathpar}
   \inferrule*[lab={\ruleName{$\match$-var}}]
   {
      \strut
   }
   {
      v, \elimVar{x}{e} \match \envExtend{\envEmpty}{x}{v}, \matchVar{x}{e}
   }
   \\
   %
   \and
   %
   \inferrule*[lab={\ruleName{$\match$-true}}]
   {
      \strut
   }
   {
      \exTrue, \elimBool{e}{e'}
      \match
      \envEmpty, \matchTrueLow{e}{e'}
   }
   %
   \and
   %
   \inferrule*[lab={\ruleName{$\match$-false}}]
   {
      \strut
   }
   {
      \exFalse, \elimBool{e}{e'}
      \match
      \envEmpty, \matchFalseLow{e}{e'}
   }
   %
   \and
   %
   \inferrule*[lab={\ruleName{$\matchFwd$-pair}}]
   {
      \strut
   }
   {
      \exPair{v}{v'}, \elimPair{x}{y}{e}
      \matchFwd
      \envExtend{\envExtend{\envEmpty}{x}{v}}{y}{v'}, e, \top
   }
   %
   \and
   %
   \inferrule*[lab={\ruleName{$\matchFwd$-pair-del}}]
   {
      \strut
   }
   {
      \exPairDel{v}{v'}, \elimPair{x}{y}{e}
      \matchFwd
      \envExtend{\envExtend{\envEmpty}{x}{v}}{y}{v'}, e, \bot
   }
   %
   \and
   %
   \inferrule*[lab={\ruleName{$\match$-nil}}]
   {
      \strut
   }
   {
      \exNil, \elimList{\branchNil{e}}{\branchCons{x}{y}{e'}}
      \match
      \envEmpty, \matchNilLow{e}{x}{y}{e'}
   }
   %
   \and
   %
   \inferrule*[lab={\ruleName{$\match$-cons}}]
   {
      \strut
   }
   {
      \exCons{v}{v'}, \elimList{\branchNil{e}}{\branchCons{x}{y}{e'}}
      \match
      \envExtend{\envExtend{\envEmpty}{x}{v}}{y}{v'}, \matchConsLow{e}{x}{y}{e'}
   }
\end{smathpar}
\caption{Pattern-matching}
\end{figure}

\begin{figure}
\flushleft \shadebox{$\rho, \matchPlug{\xi}{\kappa}, \alpha \unlookupR{} u, \sigma$}
\begin{smathpar}
   \inferrule*[left={\ruleName{$\unlookupR{}$-var}}]
   {
     \strut
   }
   {
     \envExtend{\envEmpty}{x}{u}, \matchVar{x}{\matchHole{\kappa}}, \alpha
     \unlookupR{}
     u, \trieVar{x}{\kappa}
   }
   %
   \and
   %
   \inferrule*[left={\ruleName{$\unlookupR{}$-unit}}]
   {
     \strut
   }
   {
     \envEmpty, \matchUnit{\matchHole{\kappa}}, \alpha
     \unlookupR{}
     \annot{\exUnit}{\alpha}, \trieUnit{\kappa}
   }
   %
   \and
   %
   \inferrule*[left={\ruleName{$\unlookupR{}$-inl}}]
   {
     \rho, \matchPlug{\xi}{\kappa}, \alpha \unlookupR{} u, \sigma
   }
   {
     \rho, \matchSumL{\matchPlug{\xi}{\kappa}}{\tau}, \alpha
     \unlookupR{}
     \annot{(\exInl{u})}{\alpha}, \trieSum{\sigma}{\bot_{\tau}}
   }
   %
   \and
   %
   \inferrule*[left={\ruleName{$\unlookupR{}$-inr}}]
   {
     \rho, \matchPlug{\xi}{\kappa}, \alpha \unlookupR{} u, \tau
   }
   {
     \rho, \matchSumR{\sigma}{\matchPlug{\xi}{\kappa}}, \alpha
     \unlookupR{}
     \annot{(\exInr{u})}{\alpha}, \trieSum{\bot_{\sigma}}{\tau}
   }
   %
   \and
   %
   \inferrule*[
      left={\ruleName{$\unlookupR{}$-pair}}
   ]
   {
     \rho_2, \matchPlug{\xi'}{\kappa}, \alpha \unlookupR{} u_2, \tau
     \\
     \rho_1, \matchPlug{\xi}{\tau}, \alpha \unlookupR{} u_1, \sigma
   }
   {
     \rho_1 \concat \rho_2, \matchProd{\xi}{\matchPlug{\xi'}{\kappa}}, \alpha
     \unlookupR{}
     \annot{\exPair{u_1}{u_2}}{\alpha}, \trieProd{\sigma}
   }
   %
   \and
   %
   \inferrule*[left={\ruleName{$\unlookupR{}$-roll}}]
   {
     \rho, \matchPlug{\xi}{\kappa}, \alpha \unlookupR{} u, \sigma
   }
   {
     \rho, \matchRoll{\matchPlug{\xi}{\kappa}}, \alpha
     \unlookupR{}
     \annot{(\exRoll{u})}{\alpha}, \trieRoll{\sigma}
   }
\end{smathpar}
\caption{Reverse pattern-matching}
\label{fig:pattern-matching:unmatch}
\end{figure}


\subsection{Galois connection}

\begin{theorem}
\label{thm:gc-pattern-match}

Write $\lookupR_{u, \sigma}$ for unannotated $u$, $\sigma$ to mean $\lookupR$
domain-restricted to $\Ann{u, \sigma}$ and suppose $u, \sigma \lookupR \rho,
\kappa$. Then $\lookupR_{u, \sigma}$ and $\unlookupR{u, \sigma}$ form a Galois
connection: \[{\lookupR_{u, \sigma}} \adjoint {\unlookupR{u, \sigma}}: \Ann{u,
\sigma} \to \Ann{\rho, \kappa}\]

\end{theorem}

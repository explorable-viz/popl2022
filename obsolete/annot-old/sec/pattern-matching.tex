\section{Galois slicing for pattern-matching}

\begin{figure}
\begin{syntaxfig}
\mbox{Match}
&
\xi
&
::=
&
\matchVar{x}{\matchHole{}}
&
\text{variable}
\\
&&&
\matchUnit{\matchHole{}}
&
\text{unit}
\\
&&&
\matchSumL{\xi}{\sigma}
&
\text{inject left}
\\
&&&
\matchSumR{\sigma}{\xi}
&
\text{inject right}
\\
&&&
\matchProd{\xi}{\xi'}
&
\text{product}
\\
&&&
\matchRoll{\xi}
&
\text{roll}
\end{syntaxfig}
\caption{Syntax of matches}
\end{figure}

% \begin{figure}
\flushleft \shadebox{$\matchType{\Gamma}{\xi}{K}{A}{\Gamma'}$}
\begin{smathpar}
\inferrule*[right={$\kappa \in K_{\cxtExtend{\Gamma'}{x}{A}}$}]
{
   \strut
}
{
   \matchType{\Gamma}{\matchedVar{x}{\kappa}}{K}{A}{\Gamma'}
}
%
\and
%
\inferrule*[right={$\kappa \in K_{\Gamma'} \wedge \interpret{c} \in C$}]
{
   \strut
}
{
   \matchType{\Gamma}{\matchedGround{c}{\kappa}}{K}{C}{\Gamma'}
}
%
\and
%
\inferrule*[right={$\kappa \in K_{\Gamma'}$}]
{
   \valType{\exClosure{\rho}{\exFun{\sigma}}}{\tyFun{A}{B}}
}
{
   \matchType{\Gamma}{\matchedFun{\exClosure{\rho}{\exFun{\sigma}}}{\kappa}}{K}{(\tyFun{A}{B})}{\Gamma'}
}
%
\and
%
\inferrule*[right={\textnormal{$\dom{\interpret{D}} = \dom{\Xi} = \set{d} \cup \tilde{d}$}}]
{
   \matchType{\Gamma}{\Xi(d)}{K}{\interpret{D}(d)}{\Gamma'}
   \\
   \elimType{\Xi(d')}{K}{\interpret{D}(d')}{\Gamma'}
   \\
   (\forall d' \in \tilde{d})
}
{
   \matchType{\Gamma}{\matchedConstr{\Xi}}{K}{\tyData{D}}{\Gamma'}
}
\end{smathpar}
\vspace{5pt}

\flushleft \shadebox{$\explMatchType{\Gamma}{\Psi}{K}{\vec{A}}{\Gamma'}$}
\begin{smathpar}
\inferrule*[right={$\kappa \in K_{\Gamma'}$}]
{
   \strut
}
{
   \explMatchType{\Gamma}{\matchedUnit{\kappa}}{K}{\seqEmpty}{\Gamma'}
}
%
\and
%
\inferrule*
{
   \explMatchType{\Gamma}{\explVal{T}{\xi}}{\matchTypePartial{K}{\vec{A}}}{A}{\Gamma'}
}
{
   \explMatchType{\Gamma}{\matchedProd{\explVal{T}{\xi}}}{K}{A \concat \vec{A}}{\Gamma'}
}
\end{smathpar}
\caption{Typing rules for matches and match products}
\end{figure}


\begin{definition}
   \label{def:match}
   \figref{pattern-matching:match} defines pattern matching.
\end{definition}

Pattern-matching consumes the least precise approximant of the value of $e$
which is precise enough to match a unique entry $\kappa$ of $\sigma$. When $e$
is the argument to a function, $\kappa$ will be the body of the selected branch;
if $\sigma$ is a product trie, then $\kappa$ will be another trie, representing
the demand on the second component.

\begin{definition}
   \label{def:unmatch}
   \figref{pattern-matching:unmatch} defines reverse pattern matching.
\end{definition}

\begin{figure}[H]
\flushleft \shadebox{$p, \sigma \matchp \kappa$}
\begin{smathpar}
   \inferrule*[lab={\ruleName{$\matchp$-var}}]
   {
      \strut
   }
   {
      x, \elimVar{x}{\kappa}
      \matchp
      \kappa
   }
   %
   \and
   %
   \inferrule*[lab={\ruleName{$\matchp$-true}}]
   {
      \strut
   }
   {
      \exTrue, \elimBool{\kappa}{\kappa'}
      \matchp
      \kappa
   }
   %
   \and
   %
   \inferrule*[lab={\ruleName{$\matchp$-false}}]
   {
      \strut
   }
   {
      \exFalse, \elimBool{\kappa}{\kappa'}
      \matchp
      \kappa'
   }
   %
   \and
   %
   \inferrule*[lab={\ruleName{$\matchp$-pair}}]
   {
      p_1, \sigma \matchp, \tau
      \\
      p_2, \tau \matchp \kappa
   }
   {
      \exPair{p_1}{p_2}, \elimProd{\sigma}
      \matchp
      \kappa
   }
   %
   \and
   %
   \inferrule*[lab={\ruleName{$\matchp$-nil}}]
   {
      \strut
   }
   {
      \exNil, \elimList{\branchNil{\kappa}}{\branchCons{\sigma}}
      \matchp
      \kappa
   }
   %
   \and
   %
   \inferrule*[lab={\ruleName{$\matchp$-cons}}]
   {
      p_1, \sigma \matchp \tau
      \\
      p_2, \tau \matchp \kappa'
   }
   {
      \exCons{p_1}{p_2}, \elimList{\branchNil{\kappa}}{\branchCons{\sigma}}
      \matchp
      \kappa'
   }
\end{smathpar}
\caption{Pattern-matching (by pattern)}
\end{figure}

\begin{figure}
\flushleft \shadebox{$\rho, \matchPlug{\xi}{\kappa}, \alpha \unlookupR{} u, \sigma$}
\begin{smathpar}
   \inferrule*[lab={\ruleName{$\unlookupR{}$-var}}]
   {
   }
   {
      \envExtend{\envEmpty}{x}{u}, \matchVar{x}{\matchHole{\kappa}}, \alpha
      \unlookupR{}
      u, \trieVar{x}{\kappa}
   }
   %
   \and
   %
   \inferrule*[lab={\ruleName{$\unlookupR{}$-constr}}]
   {
      \rho, \matchPlug{\Psi}{\kappa}, \alpha \unlookupR{} \vec{u}, \kappa'
   }
   {
      \rho, \envExtend{\Sigma}{c}{\matchPlug{\Psi}{\kappa}}, \alpha
      \unlookupR{}
      \annot{\exConstr{c}{\vec{u}}}{\alpha}, \envExtend{\sub{\bot}{\Sigma}}{c}{\kappa'}
   }
\end{smathpar}
\\[2mm]
\flushleft \shadebox{$\rho, \matchPlug{\Psi}{\kappa}, \alpha \unlookupR{} \vec{u}, \kappa$}
\begin{smathpar}
   \inferrule*[lab={\ruleName{$\unlookupR{}$-args-nil}}]
   {
   }
   {
      \envEmpty, \matchHole{\kappa}, \alpha \unlookupR{} \seqEmpty, \kappa
   }
   %
   \and
   %
   \inferrule*[
      lab={\ruleName{$\unlookupR{}$-args-cons}}
   ]
   {
      \rho_2, \matchPlug{\Psi}{\kappa}, \alpha \unlookupR{} \vec{u}, \kappa'
      \\
      \rho_1, \matchPlug{\xi}{\kappa'}, \alpha \unlookupR{} u, \sigma
   }
   {
      \rho_1 \concat \rho_2, \matchProd{\xi}{\matchPlug{\Psi}{\kappa}}, \alpha
      \unlookupR{}
      u \concat \vec{u}, \trieProd{\sigma}
   }
\end{smathpar}
\caption{Reverse pattern-matching}
\label{fig:impl-language:unmatch}
\end{figure}


\subsection{Galois connection}

\begin{theorem}
\label{thm:gc-pattern-match}

Write $\lookupR_{u, \sigma}$ for unannotated $u$, $\sigma$ to mean $\lookupR$
domain-restricted to $\Ann{u, \sigma}$ and suppose $u, \sigma \lookupR \rho,
\kappa$. Then $\lookupR_{u, \sigma}$ and $\unlookupR{u, \sigma}$ form a Galois
connection: \[{\lookupR_{u, \sigma}} \adjoint {\unlookupR{u, \sigma}}: \Ann{u,
\sigma} \to \Ann{\rho, \kappa}\]

\end{theorem}

\documentclass[usenames,dvipsnames,preprint]{sigplanconf}

\usepackage{float}
\usepackage{mathpartir}
\usepackage{amsthm}
\usepackage{amsmath}
\usepackage{stmaryrd}
\usepackage{graphicx}
\usepackage{soul}
\usepackage[usenames,dvipsnames]{color}
\usepackage{MnSymbol}
\usepackage{xspace}
\usepackage{mathtools}
\usepackage[hyphens]{url}
\usepackage{hyperref}
\usepackage{breakurl}
\usepackage[T1]{fontenc}
\usepackage{libertine}
\usepackage[scaled=0.8]{beramono}
\usepackage{upgreek}
\usepackage{pifont}
\usepackage{framed}
\usepackage{subdepth}
\usepackage{tikz}
\usepackage{dashundergaps}


\usetikzlibrary{shapes,chains,positioning}

% Macros which I'd like to be shared/stable across projects.

% Housekeeping.
\newcommand{\todo}[1]{\textcolor{red}{\textit{#1}}}

% Formatting.
\newcommand{\ttt}[1]{\texttt{#1}}
\newcommand{\kw}[1]{{\text{\tt{#1}}}} % is this different from \ttt?

% Misc math.
\newcommand{\dom}{\mathrm{dom}}
\newcommand{\ran}{\mathrm{ran}}
\newcommand{\Pow}[1]{\mathcal{P}({#1})}
\newcommand{\eqdef}{\stackrel{\text{\tiny def}}{=}}

% Option/Maybe.
\newcommand{\some}[1]{\textsf{some}\;{#1}}
\newcommand{\none}[0]{\textsf{none}}

% Lattices.
\newcommand{\sqgeq}{\sqsupseteq}
\newcommand{\sqleq}{\sqsubseteq}
\newcommand{\sqgt}{\sqsupset}
\newcommand{\sqlt}{\sqsubset}
\newcommand{\join}{\sqcup}
\newcommand{\meet}{\sqcap}
\newcommand{\bigjoin}{\bigsqcup}
\newcommand{\bigmeet}{\bigsqcap}

% Definitions, theorems, etc.
\newtheorem{theorem}{Theorem}
\newtheorem{corollary}{Corollary}
\newtheorem{lemma}{Lemma}
\theoremstyle{definition}
\newtheorem{definition}{Definition}
\newtheorem{remark}{Remark}
\newtheorem{example}{Example}

% Cross-referencing.
\newcommand{\corref}[1]{Corollary~\ref{cor:#1}}
\newcommand{\thmref}[1]{Theorem~\ref{thm:#1}}
\newcommand{\thmrefbrev}[1]{Thm.~\ref{thm:#1}}
\newcommand{\thmrefTwo}[2]{Theorems~\ref{thm:#1} and \ref{thm:#2}}
\newcommand{\lemref}[1]{Lemma~\ref{lem:#1}}
\newcommand{\lemrefTwo}[2]{Lemmas~\ref{lem:#1} and \ref{lem:#2}}
\newcommand{\eqnref}[1]{Equation~\ref{eqn:#1}}
\newcommand{\secref}[1]{Section~\ref{sec:#1}}
\newcommand{\secrefTwo}[2]{Sections~\ref{sec:#1} and \ref{sec:#2}}
\newcommand{\figref}[1]{Figure~\ref{fig:#1}}
\newcommand{\appref}[1]{Appendix~\ref{app:#1}}
\newcommand{\defref}[1]{Definition~\ref{def:#1}}

% Math mode.
\newenvironment{nop}{}{}
\newenvironment{sdisplaymath}{
\begin{nop}\small\begin{displaymath}}{
\end{displaymath}\end{nop}\ignorespacesafterend}
\newenvironment{smathpar}{
\begin{nop}\small\begin{mathpar}}{
\end{mathpar}\end{nop}\ignorespacesafterend}

% Figures.
\newenvironment{mathfig}{\begin{sdisplaymath}}{\end{sdisplaymath}}
\newenvironment{syntaxfig}{\begin{mathfig}\begin{array}{@{}l@{\quad}r@{~~}c@{\quad}ll}}{\end{array}\end{mathfig}}

% Sequences.
\renewcommand{\vec}[1]{\overrightarrow{#1}}

% Sub-floats.
\makeatletter
\newbox\sf@box
\newenvironment{SubFloat}[2][]%
  {\def\sf@one{#1}%
   \def\sf@two{#2}%
   \setbox\sf@box\hbox
     \bgroup}%
  { \egroup
   \ifx\@empty\sf@two\@empty\relax
     \def\sf@two{\@empty}
   \fi
   \ifx\@empty\sf@one\@empty\relax
     \subfloat[\sf@two]{\box\sf@box}%
   \else
     \subfloat[\sf@one][\sf@two]{\box\sf@box}%
   \fi}
\makeatother

\newcommand\grayuline{\bgroup\markoverwith{\textcolor{gray}{\rule[-0.9ex]{2pt}{0.4pt}}}\ULon}
% \newcommand*{\del}[1]{\grayuline{\textcolor{gray}{\smash{#1}}}}
\newcommand*{\del}[1]{\textcolor{red}{#1}}
\newcommand*{\sel}[1]{\textcolor{blue}{#1}}
% \let\oldtop\top
% \renewcommand{\top}{\sel{\oldtop}}
\newcommand*{\Sel}[1]{[#1]}
\newcommand*{\TT}{\mathsf{tt}}
\newcommand*{\FF}{\mathsf{ff}}

\renewcommand*{\ldots}{...} % \ldots sucks with acmart
\renewcommand*{\ruleName}[1]{\textcolor{gray}{\textnormal{\textsf{#1}}}}

% Section symbol.
\renewcommand*{\secref}{\Secref}
\renewcommand*{\secrefTwo}{\SecrefTwo}

% Use this in place of \mathsf.
\newcommand*{\mathSf}[1]{\textup{\textsf{#1}}}
\newcommand*{\Set}[1]{\mathSf{#1}}

\newcommand*{\Int}{\mathbb{Z}}
\newcommand*{\Pow}[1]{\MnSymbolpowerset({#1})}
\newcommand*{\id}{\textsf{id}}
\newcommand*{\adjoint}{\dashv}
\newcommand*{\inj}{\hookrightarrow}

% Vectors.
\newcommand*{\override}{\reloverrideleft}
\newcommand*{\seqEmpty}{\varepsilon}

% Misc.
\newcommand*{\envExtendTwo}[2]{{#1}[#2]}
\newcommand*{\Dagger}[1]{\smash{#1^\dagger}}
\newcommand*{\DDagger}[1]{\smash{#1^\ddagger}}
\newcommand*{\closeDefs}{\mathrel{\rightharpoonup}}
\newcommand*{\closeDefsFwd}{\mathrel{\rotatebox[origin=c]{45}{$\rightharpoonup$}}}
\newcommand*{\closeDefsBwd}{\mathrel{\rotatebox[origin=c]{-45}{$\rightharpoonup$}}}

% Misc. syntax.
\newcommand*{\asMap}[1]{\underline{#1}}
\newcommand*{\trace}[2]{\textcolor{gray}{#1}\mathrel{\textcolor{gray}{::}}{#2}}

% Relations.
\newcommand*{\numLabel}[1]{\textcolor{blue}{\mathSf{#1}}}

% Primitives.
\newcommand*{\primOp}{\oplus}

% Annotations.
\newcommand*{\annot}[2]{#1_{#2}}

\newcommand*{\sym}[1]{\textbf{\kw{#1}}}
\newcommand*{\symCons}{\sym{:}}

\newcommand*{\arrayLBrack}{\langle}
\newcommand*{\arrayRBrack}{\rangle}

% Expressions.
%\newcommand*{\exAdd}[2]{{#1} + {#2}}
\newcommand*{\exApp}[2]{{#1}\,{#2}}
\newcommand*{\exArray}[4]{\arrayLBrack{#4}\;\sym{$\vert$}\;\exPair{#2}{#3}\;\sym{$\leq$}\;{#1}\arrayRBrack}
\newcommand*{\exArrayAccess}[2]{{#1}\,\sym{!}\,{#2}}
\newcommand*{\exArraySize}[1]{\kw{size}\;{#1}}
\newcommand*{\exBinaryApp}[3]{{#1} \mathbin{#2} {#3}}
\newcommand*{\exCons}[2]{{#1}\,\symCons\,{#2}}
\newcommand*{\exClosure}[3]{\kw{cls}({#1},{#2},{#3})}
\newcommand*{\exClosureRec}[3]{\kw{cls}({#1},{#2},{#3})}
\newcommand*{\exFalse}{\kw{false}}
\newcommand*{\exFun}[2]{#1\,#2}
\newcommand*{\sExList}[2]{\kw{[}\;#1 \ #2}
\newcommand*{\exList}[1]{\kw{[}\;{#1}\;\kw{]}}
\newcommand*{\exListSeq}[2]{\kw{[}\;{#1}\;\kw{..}\;{#2}\;\kw{]}}
\newcommand*{\exListComp}[2]{\kw{[}\;{#1}\;\sym{$\vert$}\;{#2}\;\kw{]}}
\newcommand*{\exInt}[1]{#1}
\newcommand*{\exLambda}[1]{\lambda{#1}}
\newcommand*{\exLet}[3]{\kw{let}\;{#1}\equal{#2}\;\kw{in}\;{#3}}
\newcommand*{\sExLet}[3]{\kw{let}\;{#1}\;{#2}\;\kw{in}\;{#3}}
\newcommand*{\exLetRec}[3]{\kw{let}\,\exRec{#1}{#2}\;\kw{in}\;{#3}}
\newcommand*{\exLetRecMutual}[2]{\kw{let}\;{#1}\;\kw{in}\;{#2}}
\newcommand*{\exIfThenElse}[3]{\kw{if}\;{#1}\;\kw{then}\;{#2}\;\kw{else}\;{#3}}
\newcommand*{\exLetStructured}[2]{\kw{let}\;{#1}\;\kw{in}\;{#2}}
\newcommand*{\exMatch}[2]{\kw{match}~{#1}~\kw{as}\;{#2}}
\newcommand*{\exNil}{\kw{[]}}
\newcommand*{\exOp}[1]{(#1)}
\newcommand*{\exPair}[2]{(#1,#2)}
\newcommand*{\exPrim}[1]{#1}
\newcommand*{\exRec}[2]{{#1}{#2}}
\newcommand*{\exTrue}{\kw{true}}
\newcommand*{\exVar}[1]{#1}

% Branches
\newcommand*{\branchCons}[1]{(\symCons)\mapsto{#1}}
\newcommand*{\branchNil}[1]{\exNil\mapsto{#1}}
\newcommand*{\branchTrue}[1]{\exTrue\mapsto{#1}}
\newcommand*{\branchFalse}[1]{\exFalse\mapsto{#1}}

% List rest
\newcommand*{\sExNil}{\kw{]}}
\newcommand*{\sExCons}[2]{\comma #1 \ #2}

% Clause
\newcommand*{\clause}[2]{\vec{{#1}}\equal{#2}}
\newcommand*{\clausenv}[2]{{#1}\equal{#2}}

% Qualifiers
\newcommand*{\qualGuard}[1]{\kw{if}\;{#1}}
\newcommand*{\qualDeclaration}[2]{\kw{let}\;{#1}\equal{#2}}
\newcommand*{\qualGenerator}[2]{{#1}\;\kw{$\leftarrow$}\;{#2}}

% Patterns
\newcommand*{\pattVar}[1]{#1}
\newcommand*{\pattTrue}{\exTrue}
\newcommand*{\pattFalse}{\exFalse}
\newcommand*{\pattCons}[2]{\exCons{#1}{#2}}
\newcommand*{\pattNil}{\exNil}
\newcommand*{\pattList}[2]{\sExList{#1}{#2}}
\newcommand*{\pattSCons}[2]{\sExCons{#1}{#2}}
\newcommand*{\pattSNil}{\sExNil}
\newcommand*{\pattPair}[2]{\langle#1,#2\rangle}

% Eliminators.
\newcommand*{\elimBool}[2]{\{\exTrue \mapsto {#1}, \exFalse \mapsto {#2}\}}
\newcommand*{\elimBoolTrue}[1]{\{\exTrue \mapsto {#1}\}}
\newcommand*{\elimBoolFalse}[1]{\{\exFalse \mapsto {#1}\}}
\newcommand*{\elimListSingleton}[1]{\{{#1}\}}
\newcommand*{\elimFun}[1]{\elimFunNoRHS \mapsto {#1}}
\newcommand*{\elimList}[2]{\{{#1},{#2}\}}
\newcommand*{\elimProd}[1]{\kw{prod}\;#1}
\newcommand*{\elimVar}[2]{{#1} \mapsto {#2}}

\newcommand*{\singleton}[2]{#1[#2]}

\renewcommand*{\lowlight}[1]{\textcolor{gray}{#1}}

% Matches.
\newcommand*{\matchCons}[3]{\elimList{\branchNil{#1}}{\exCons{#2}{#3}}}
\newcommand*{\matchConsNew}[2]{\exCons{#1}{#2}}
\newcommand*{\matchFalse}[1]{\elimBool{#1}{\matchHole}}
\newcommand*{\matchFalseNew}{\exFalse}
\newcommand*{\matchHole}{\square}
\newcommand*{\matchNil}[1]{\elimList{\branchNil{\matchHole}}{\branchCons{#1}}}
\newcommand*{\matchNilNew}{\exNil}
\newcommand*{\matchPairNew}[2]{\exPair{#1}{#2}}
\newcommand*{\matchPlug}[2]{{#1} \mapsto {#2}}
\newcommand*{\matchTrueNew}{\exTrue}
\newcommand*{\matchTrue}[1]{\elimBool{\matchHole}{#1}}
\newcommand*{\matchVar}[1]{\elimVar{#1}{\matchHole}}
\newcommand*{\matchVarNew}[1]{#1}

\newcommand*{\interpret}[1]{\hat{#1}}

% Types.
\newcommand*{\tyArray}[1]{\kw{Array}\;{#1}}
\newcommand*{\tyBool}[0]{\kw{Bool}}
\newcommand*{\tyInt}[0]{\kw{Int}}
\newcommand*{\tyList}[1]{\kw{List}\;{#1}}
\newcommand*{\tyProd}[2]{{#1}\times{#2}}
\newcommand*{\tyFun}[2]{{#1}\rightarrow{#2}}

% Sets
\newcommand*{\Ann}[1]{\Set{Ann}(#1)}
\newcommand*{\Below}[1]{{\downarrow{#1}}}
\newcommand*{\Cxt}{\Set{Cxt}}
\newcommand*{\Env}[1]{\Set{Env}\,#1}
\newcommand*{\Expl}[1]{\Set{Expl}\,#1}
\newcommand*{\ExplVal}[2]{\ExplValPartial{#1}\:#2}
\newcommand*{\ExplValPartial}[1]{\Set{ExplVal}\,#1}
\newcommand*{\Expr}[2]{\ExprPartial{#1}\,#2}
\newcommand*{\ExprPartial}[1]{\Set{Term}\,#1}
\newcommand*{\Match}[4]{\Set{Match}\,#1\,#2\,#3\,#4}
\newcommand*{\Prim}[3]{\Set{Prim}\,#1\,#2\,#3}
\newcommand*{\BinaryOp}[3]{\Set{BinaryOp}\,#1\,#2\,#3}
\newcommand*{\UnaryOp}[2]{\Set{UnaryOp}\,#1\,#2}
\newcommand*{\Val}[1]{\Set{Val}\,#1}
\newcommand*{\ValExpl}[2]{\Set{Val}\,#1\:#2}

% Misc.
\newcommand*{\rcons}{\cdot} % can't use \snoc as a macro name inside mathpar
\newcommand*{\cons}{::}
\newcommand*{\unionWith}[1]{\cup_{#1}}
\renewcommand*{\join}{\vee}
\renewcommand*{\meet}{\wedge}
\newcommand*{\joinWith}[1]{\join_{#1}}
\newcommand*{\eqJoin}{\mathbin{!}}

% Variable contexts.
\newcommand*{\cxtBot}{\exBot}
\newcommand*{\cxtEmpty}{\varepsilon}
\newcommand*{\cxtExtend}[3]{{#1},{#2}: {#3}}
\newcommand*{\cxtLookup}[2]{{#1}[{x}]}

% Environments.
\newcommand*{\subst}[3]{{#1}[{#2}/{#3}]} % still needed for recursive types..
\newcommand*{\envEmpty}{\varepsilon}
\newcommand*{\envExtend}[3]{{#1}[{#2}: {#3}]}
\newcommand*{\envLookup}[3]{{#2}: {#3} \in {#1}}
\newcommand*{\envUpdate}[3]{{#1}\smalltriangleleft[{#2}: {#3}]}
\newcommand*{\env}[1]{\mathsf{env}(#1)}

% Evaluation.
\newcommand*{\eval}{\mathrel{\Rightarrow}}
% \newcommand*{\fwdSlice}{\nearrow}
% \newcommand*{\bwdSlice}{\searrow}
\newcommand*{\fwdSlice}{\mathrel{\rotatebox[origin=c]{45}{$\eval$}}}
\newcommand*{\bwdSlice}{\mathrel{\rotatebox[origin=c]{-45}{$\eval$}}}
\newcommand*{\match}{\rightsquigarrow}
\newcommand*{\matchFwd}{\mathrel{\rotatebox[origin=c]{45}{$\rightsquigarrow$}}}
\newcommand*{\matchBwd}{\mathrel{\rotatebox[origin=c]{-45}{$\rightsquigarrow$}}}
\newcommand*{\proj}{\downarrow}
\newcommand*{\nproj}{\not\downarrow}
\newcommand*{\val}[1]{\mathsf{val}(#1)}

\newcommand*{\seq}{\mathrel{\smalltriangleright}}

% Traces.
\newcommand*{\trApp}[3]{\exApp{#1}{#2}\seq{#3}}
\newcommand*{\trAppPrim}[2]{\exApp{#1}{#2}}
\newcommand*{\trArray}[6]{\exArray{{#6}}{#2:#4}{#3:#5}{#1}}
\newcommand*{\trArrayAccess}[2]{\exArrayAccess{#1}{#2}}
\newcommand*{\trArraySize}[1]{\exArraySize{#1}}
\newcommand*{\trBinaryApp}[4]{\exBinaryApp{#1}{#2_{#3}}{#4}}
\newcommand*{\trCons}[2]{\exCons{#1}{#2}}
\newcommand*{\trFalse}{\exFalse}
\newcommand*{\trInt}[1]{\exInt{#1}}
\newcommand*{\trLambda}[1]{\exLambda{#1}}
\newcommand*{\trLet}[3]{\exLet{#1}{#2}{#3}}
\newcommand*{\trLetRec}[3]{\exLetRec{#1}{#2}{#3}}
\newcommand*{\trLetRecMutual}[2]{\exLetRecMutual{#1}{#2}}
\newcommand*{\trLetStructured}[2]{\exLetStructured{#1}{#2}}
\newcommand*{\trMatch}[2]{\exMatch{#1}{#2}}
\newcommand*{\trNil}{\exNil}
\newcommand*{\trOp}[1]{\exOp{#1}}
\newcommand*{\trPair}[2]{\exPair{#1}{#2}}
\newcommand*{\trRec}[2]{\kw{rec}\,{#1}{#2}}
\newcommand*{\trTrue}{\exTrue}
\newcommand*{\trVar}[1]{#1}

\newcommand*{\explVal}[2]{{\textcolor{gray}{#1}}\mathrel{::}{#2}}

% Desugaring.
\newcommand*{\tarrow}{\mathrel{\Rightarrow}}
\newcommand*{\darrow}{\mathrel{\twoheadrightarrow}}
\newcommand*{\dfwdarrow}{\mathrel{{\rotatebox[origin=c]{45}{$\twoheadrightarrow$}}}}
\newcommand*{\dbwdarrow}{\mathrel{{\rotatebox[origin=c]{-45}{$\twoheadrightarrow$}}}}
\newcommand*{\totalise}[1]{\mathsf{totalise}({#1})}
\newcommand*{\untotalise}[2]{\mathsf{untotalise}({#1}, {#2})}
\newcommand*{\branch}[2]{\kw{branch}({#1}, {#2})}
\newcommand*{\range}[2]{\kw{range} \ {#1}\,{#2}}
\newcommand*{\desugar}[1]{\kw{desugar}\;({#1})}
\newcommand*{\comma}{\textbf{\kw{,}}\;}
\newcommand*{\equal}{\mathrel{\textbf{\kw{=}}}}
\newcommand*{\zipWith}[3]{\mathsf{zipWith}\;\;{#1}\;\;{#2}\;\;{#3}}
\newcommand*{\where}{\kw{where}}
\newcommand*{\lambdasyn}{\mathrel{\textbf{\kw{\lambda}}}}


\setulcolor{blue}

\makeatletter
  \let\@copyrightspace\relax
\makeatother

\begin{document}

\title{Demand-indexed computation}

\authorinfo{}{}{}{}

\maketitle

\begin{figure}
\begin{syntaxfig}
\mbox{Type}
&
\sigma, \tau
&
::=
&
\tyVar{\alpha}
\mid
\tyZero
\mid
\tyUnit
\mid
\tySum{\tau_1}{\tau_2}
\mid
\tyProd{\tau_1}{\tau_2}
\mid
\tyFun{\tau_1}{\tau_2}
\mid
\\
&&&
\tyRec{\alpha}{\tau}
\\[1mm]
\mbox{Variable context}
&
\Gamma
&
::=
&
\cxtEmpty
\mid
\cxtExtend{\Gamma}{x}{\tau}
\\[1mm]
\mbox{Pattern context}
&
\Delta
&
::=
&
\cxtEmpty
\mid
\cxtExtend{\Delta}{p}{\tau}
\\[1mm]
\mbox{Pattern}
&
p
&
::=
&
x
\mid
\exUnit
\mid
\exInl{p}
\mid
\exInr{p}
\mid
\exPair{p_1}{p_2}
\mid
\exRoll{p}
\\[1mm]
\mbox{Term}
&
e
&
::=
&
r
\mid
\exInl{e}
\mid
\exInr{e}
\mid
\exMatch{r}{\vec{\exBranch{p}{e}}}
\mid
\\
&&&
\exPair{e_1}{e_2}
\mid
\exFun{f}{\vec{\exBranch{p}{e}}}
\mid
\exRoll{e}
\mid
\\[1mm]
\mbox{Synthesis term}
&
r
&
::=
&
\exAnnot{e}{\tau}
\mid
x
\mid
\exUnit
\mid
\exFst{r}
\mid
\exSnd{r}
\mid
\exApp{r}{e}
\mid
\\
&&&
\exUnroll{r}
\\[1mm]
\mbox{Value}
&
v
&
::=
&
\exSuspend{e}
\mid
\exUnit
\mid
\exInl{v}
\mid
\exInr{v}
\mid
\exPair{v_1}{v_2}
\mid
\\
&&&
\exFun{f}{\vec{\exBranch{p}{e}}}
\mid
\exRoll{v}
\\[1mm]
\mbox{Demand}
&
u
&
::=
&
\bot
\mid
\exUnit
\mid
\exInlr{u_1}{u_2}
\mid
\exPair{u_1}{u_2}
\mid
\exRecDemand
\mid
\\
&&&
\exRoll{u}
\end{syntaxfig}
\caption{Abstract syntax}
\label{fig:syntax:values}
\end{figure}

\begin{figure}
\fbox{$\Gamma \vdash e \tycheck \tau$}
\begin{smathpar}
  \inferrule*
  {
    \Gamma \vdash r \synth \tau
  }
  {
    \Gamma \vdash r \tycheck \tau
  }
%
\and
%
  \inferrule*
  {
    \Gamma \vdash e \tycheck \tau_1
  }
  {
    \Gamma \vdash \exInl{e} \tycheck \tySum{\tau_1}{\tau_2}
  }
%
\and
%
  \inferrule*
  {
    \Gamma \vdash r \synth \sigma
    \\
    \Gamma;\vec{p} \tycheck \sigma \vdash \vec{e} \tycheck \tau
  }
  {
    \Gamma
    \vdash \exMatch{r}{\vec{\exBranch{p}{e}}} \tycheck \tau
  }
%
\and
%
  \inferrule*
  {
    \Gamma \vdash e_1 \tycheck \tau_1
    \\
    \Gamma \vdash e_2 \tycheck \tau_2
  }
  {
    \Gamma
    \vdash \exPair{e_1}{e_2} \tycheck \tyProd{\tau_1}{\tau_2}
  }
%
\and
%
  \inferrule*
  {
    \cxtExtend{\Gamma}{f}{\tyFun{\sigma}{\tau}};\vec{p} \tycheck \sigma \vdash \vec{e} \tycheck \tau
  }
  {
    \Gamma \vdash \exFun{f}{\vec{\exBranch{p}{e}}}
    \tycheck \tyFun{\sigma}{\tau}
  }
%
\and
%
  \inferrule*
  {
    \Gamma \vdash e \tycheck \subst{\tyRec{\alpha}{\tau}}{\alpha}{\tau}
  }
  {
    \Gamma \vdash \exRoll{e} \tycheck \tyRec{\alpha}{\tau}
  }
\end{smathpar}
\\
\fbox{$\Gamma \vdash r \synth \tau$}
\begin{smathpar}
  \inferrule*
  {
    \Gamma \vdash e \tycheck \tau
  }
  {
    \Gamma \vdash \exAnnot{e}{\tau} \synth \tau
  }
%
\and
%
  \inferrule*[right={$x : \tau \in \Gamma$}]
  {
    \strut
  }
  {
    \Gamma \vdash x \synth \tau
  }
%
\and
%
  \inferrule*
  {
    \strut
  }
  {
    \Gamma \vdash \exUnit \synth \tyUnit
  }
%
\and
%
  \inferrule*
  {
    \Gamma \vdash r \synth \tyProd{\tau_1}{\tau_2}
  }
  {
    \Gamma \vdash \exFst{r} \synth \tau_1
  }
%
\and
%
  \inferrule*
  {
    \Gamma \vdash r \synth \tyFun{\sigma}{\tau}
    \\
    \Gamma \vdash e \tycheck \sigma
  }
  {
    \Gamma \vdash \exApp{r}{e} \synth \tau
  }
%
\and
%
  \inferrule*
  {
    \Gamma \vdash r \synth \tyRec{\alpha}{\tau}
  }
  {
    \Gamma \vdash \exUnroll{r} \synth \subst{\tyRec{\alpha}{\tau}}{\alpha}{\tau}
  }
\end{smathpar}
\caption{Expression typing}
\label{fig:typing}
\end{figure}

\begin{figure}
\fbox{$\Gamma;\vec{p} \tycheck \sigma \vdash \vec{e} \tycheck \tau$}
\begin{smathpar}
  \inferrule*
  {
    \Gamma;\cxtExtend{\cxtEmpty}{p_1}{\sigma}
    \vdash e_1 \tycheck \tau
    \quad \ldots \quad
    \Gamma;\cxtExtend{\cxtEmpty}{p_n}{\sigma}
    \vdash e_n \tycheck \tau
  }
  {
    \Gamma;\vec{p} \tycheck \sigma
    \vdash \vec{e} \tycheck \tau
  }
\end{smathpar}
\\
\fbox{$\Gamma;\Delta \vdash e \tycheck \tau$}
\begin{smathpar}
  \inferrule*
  {
    \Gamma \vdash e \tycheck \tau
  }
  {
    \Gamma; \cxtEmpty \vdash e \tycheck \tau
  }
%
\and
%
  \inferrule*
  {
    \cxtExtend{\Gamma}{x}{\sigma}; \Delta
    \vdash e \tycheck \tau
  }
  {
    \Gamma; \cxtExtend{\Delta}{x}{\sigma}
    \vdash e \tycheck \tau
  }
%
\and
%
  \inferrule*
  {
    \Gamma;\Delta' \vdash e \tycheck \tau
  }
  {
    \Gamma; \cxtExtend{\Delta}{\exUnit}{\tyUnit}
    \vdash e \tycheck \tau
  }
%
\and
%
  \inferrule*
  {
    \Gamma;\cxtExtend{\Delta}{p}{\tau_1}
    \vdash e \tycheck \tau
  }
  {
    \Gamma;\cxtExtend{\Delta}{\exInl{p}}{\tySum{\tau_1}{\tau_2}}
    \vdash e \tycheck \tau
  }
%
\and
%
  \inferrule*
  {
    \Gamma;\cxtExtend{\cxtExtend{\Delta}{p_1}{\tau_1}}{p_2}{\tau_2}
    \vdash e \tycheck \tau
  }
  {
    \Gamma;\cxtExtend{\Delta}{\exPair{p_1}{p_2}}{\tyProd{\tau_1}{\tau_2}}
    \vdash e \tycheck \tau
  }
%
\and
%
  \inferrule*
  {
    \Gamma; \cxtExtend{\Delta}{p}{\subst{\tyRec{\alpha}{\sigma}}{\alpha}{\sigma}}
    \vdash e \tycheck \tau
  }
  {
    \Gamma;\cxtExtend{\Delta}{\exRoll{p}}{\tyRec{\alpha}{\sigma}}
    \vdash e \tycheck \tau
  }
\end{smathpar}
\caption{Pattern typing}
\label{fig:typing:patterns}
\end{figure}

\begin{figure}
\fbox{$\vdash v \tycheck \tau$}
\begin{smathpar}
  \inferrule*
  {
    \cxtEmpty \vdash e \tycheck \tau
  }
  {
    \vdash \exSuspend{e} \tycheck \tau
  }
%
\and
%
  \inferrule*
  {
    \strut
  }
  {
    \vdash \exUnit \tycheck \tyUnit
  }
%
\and
%
  \inferrule*
  {
    \vdash v \tycheck \tau_1
  }
  {
    \vdash \exInl{v} \tycheck \tySum{\tau_1}{\tau_2}
  }
%
\and
%
  \inferrule*
  {
    \vdash v_1 \tycheck \tau_1
    \\
    \vdash v_2 \tycheck \tau_2
  }
  {
    \vdash \exPair{v_1}{v_2} \tycheck \tyProd{\tau_1}{\tau_2}
  }
%
\and
%
  \inferrule*
  {
    \cxtExtend{\cxtEmpty}{f}{\tyFun{\sigma}{\tau}};\vec{p}\tycheck\sigma \vdash \vec{e} \tycheck \tau
  }
  {
    \vdash \exFun{f}{\vec{\exBranch{p}{e}}} \tycheck \tyFun{\sigma}{\tau}
  }
%
\and
%
  \inferrule*
  {
    \vdash v \tycheck \subst{\tyRec{\alpha}{\tau}}{\alpha}{\tau}
  }
  {
    \vdash \exRoll{v} \tycheck \tyRec{\alpha}{\tau}
  }
\end{smathpar}
\caption{Value typing}
\label{fig:typing:values}
\end{figure}

\begin{figure}
\fbox{$e \eval_{u} v$}
\begin{smathpar}
  \inferrule*
  {
    r
    \seval_{u}
    v \synth \tau
  }
  {
    r
    \eval_{u}
    v
  }
%
\and
%
  \inferrule*
  {
    \strut
  }
  {
    e
    \eval_{\bot}
    \exSuspend{e}
  }
%
\and
%
  \inferrule*
  {
    e \eval_{u_1} v
  }
  {
    \exInl{e}
    \eval_{\exInlr{u_1}{u_2}}
    \exInl{v}
  }
%
\and
%

  \inferrule*[right={$u \sqgt \bot$,
                      \textnormal{$\match{v:\sigma}{\vec{\exBranch{p}{e}}} \equal e'$}}]
  {
    r \seval_{\demand{\vec{p}}} v \synth \sigma
    \\
    e' \eval_{u} v'
  }
  {
    \exMatch{r}{\vec{\exBranch{p}{e}}}
    \eval_{u}
    v'
  }
%
\and
%
  \inferrule*
  {
    e_1 \eval_{u_1} v_1
    \\
    e_2 \eval_{u_2} v_2
  }
  {
    \exPair{e_1}{e_2}
    \eval_{\exPair{u_1}{u_2}}
    \exPair{v_1}{v_2}
  }
%
\and
%
  \inferrule*
  {
    \strut
  }
  {
    \exFun{f}{\vec{\exBranch{p}{e}}}
    \eval_{\exRecDemand}
    \exFun{f}{\vec{\exBranch{p}{e}}}
  }
%
\and
%
  \inferrule*
  {
    e \eval_{u} v
  }
  {
    \exRoll{e}
    \eval_{\tyRoll{u}}
    \exRoll{v}
  }
\end{smathpar}
\\
\fbox{$r \seval_{u} v \synth \tau$}
\begin{smathpar}
  \inferrule*[right={$u \sqgt \bot$}]
  {
    e
    \eval_{u}
    v
  }
  {
    \exAnnot{e}{\tau}
    \seval_{u}
    v \synth \tau
  }
%
\and
%
  \inferrule*[right={$\cxtEmpty \vdash r \synth \tau$}]
  {
    \strut
  }
  {
    r
    \seval_{\bot}
    \exSuspend{r} \synth \tau
  }
%
\and
%
  \inferrule*
  {
    \strut
  }
  {
    \exUnit
    \seval_{\exUnit}
    \exUnit \synth \tyUnit
  }
%
\and
%
  \inferrule*[right={$u \sqgt \bot$}]
  {
    r
    \seval_{\exPair{u}{\bot}}
    \exPair{v_1}{v_2}
    \synth \tyProd{\tau_1}{\tau_2}
  }
  {
    \exFst{r}
    \seval_{u}
    v_1 \synth \tau_1
  }
%
\and
%
  \inferrule*[right={$u \sqgt \bot$}]
  {
    r
    \seval_{\exPair{\bot}{u}}
    \exPair{v_1}{v_2}
    \synth \tyProd{\tau_1}{\tau_2}
  }
  {
    \exSnd{r}
    \seval_{u}
    v_2 \synth \tau_2
  }
%
\and
%
  \inferrule*[right={$u \sqgt \bot$,
                      \textnormal{$\match{v:\sigma}{\vec{\exBranch{p}{e'}}} \equal e''$}}]
  {
    r \seval_{\exRecDemand} \exFun{f}{\vec{\exBranch{p}{e'}}} \synth \tyFun{\sigma}{\tau}
    \\
    e \eval_{\demand{\vec{p}}} v
    \\
    \subst{\exAnnot{\exFun{f}{\vec{\exBranch{p}{e'}}}}{\tyFun{\sigma}{\tau}}}{f}
          {e''}
    \eval_{u} v'
  }
  {
    \exApp{r}{e}
    \seval_{u}
    v' \synth \tau
  }
%
\and
%
  \inferrule*[right={$u \sqgt \bot$}]
  {
    r
    \seval_{\exRoll{u}}
    \exRoll{v} \synth \tyRec{\alpha}{\tau}
  }
  {
    \exUnroll{r}
    \seval_{u}
    v \synth \subst{\tyRec{\alpha}{\tau}}{\alpha}{\tau}
  }
\end{smathpar}
\caption{Demand-indexed evaluation}
\label{fig:eval}
\end{figure}

\begin{figure}
\fbox{$\demand{\vec{p}}$}
\small{
\begin{align*}
\demand{\vec{p}}
&=
\demand{p_1} \join \quad \ldots \quad \join \demand{p_n}
\intertext{\fbox{$\demand{p}$}}
\demand{x}
&=
\bot
\\
\demand{\exUnit}
&=
\exUnit
\\
\demand{\exInl{p}}
&=
\exInlr{\demand{p}}{\bot}
\\
\demand{\exRoll{p}}
&=
\exRoll{\demand{p}}
\\
\demand{\exPair{p_1}{p_2}}
&=
\exPair{\demand{p_1}}{\demand{p_2}}
\end{align*}
}
\caption{Patterns as demands}
\end{figure}

\begin{figure*}
\fbox{$\match{v:\tau}{\vec{\exBranch{p}{e}}}$}
\begin{align*}
\match{v:\tau}{\vec{\exBranch{p}{e}}}
&=
\catMaybes{\match{v:\tau}{\exBranch{p_1}{e_1}}, \ldots, \match{v:\tau}{\exBranch{p_n}{e_n}}}
\\
\intertext{where}
\catMaybes{\some{x}, \vec{y}}
&=
x, \catMaybes{\vec{y}}
\\
\catMaybes{\none, \vec{y}}
&=
\catMaybes{\vec{y}}
\end{align*}
\\
\fbox{$\match{v:\tau}{\exBranch{p}{e}}$}
\begin{align*}
\match{\exSuspend{e'}:\tau}{\exBranch{x}{e}}
&=
\some{
\begin{cases}
\;\subst{r}{x}{e}
&
\text{if }e' = r
\\
\;\subst{\exAnnot{e'}{\tau}}{x}{e}
& \text{otherwise}
\end{cases}
}
\\
\match{\exUnit:\tyUnit}{\exBranch{\exUnit}{e}}
&=
\some{e}
\\
\match{\exInl{v}:\tySum{\tau_1}{\tau_2}}{\exBranch{\exInl{p}}{e}}
&=
\match{v:\tau_1}{\exBranch{p}{e}}
\\
\match{\exInl{v}:\tySum{\tau_1}{\tau_2}}{\exBranch{\exInr{p}}{e}}
&=
\none
\\
\match{\exInr{v}:\tySum{\tau_1}{\tau_2}}{\exBranch{\exInr{p}}{e}}
&=
\match{v:\tau_2}{\exBranch{p}{e}}
\\
\match{\exInr{v}:\tySum{\tau_1}{\tau_2}}{\exBranch{\exInl{p}}{e}}
&=
\none
\\
\match{\exPair{v_1}{v_2}:\tyProd{\tau_1}{\tau_2}}{\exBranch{\exPair{p_1}{p_2}}{e}}
&=
\match{v_1:\tau_1}{\exBranch{p_1}{e}} \bind \match{v_2:\tau_2}{\exBranch{p_2}{-}}
\\
\match{\exRoll{v}:\tyRec{\alpha}{\tau}}{\exBranch{\exRoll{p}}{e}}
&=
\match{v:\subst{\tyRec{\alpha}{\tau}}{\alpha}{\tau}}{\exBranch{p}{e}}
\\
\intertext{where}
\some{x} \bind f
&= f x
\\
\none \bind f
&= \none
\end{align*}
\caption{Branch matching}
\label{fig:match}
\end{figure*}


\section{Typing}

\begin{enumerate}
\item We omit the rules for $\kw{snd}$ and $\kw{inr}$ throughout.
\item The set $D_{\tau}$ of ``demands'' of type $\tau$ form a Boolean
  algebra $(D_{\tau},\bot,\top,\meet,\join,\neg)$.
\item Moding: for all judgements, contexts and expressions are
  inputs. For types, input polarity is indicated by $\tycheck$ and
  output by $\synth$.
\item As an example of a disjunctive demand, $\ttt{Nil} \join
  \ttt{Cons}(\bot,\ttt{Nil})$ would desugar to
  $\exRoll{(\exInlr{\exUnit}{\exPair{\bot}{\exRoll{\exInl{\exUnit}}}})}$. Forcing
  a list computation with this demand would cause it to get stuck if
  it tried to produce a two-element list.
\end{enumerate}

\section{Evaluation}

\begin{lemma}[Determinism]
\label{lem:eval:deterministic}
\item
\begin{enumerate}
\item If $e \eval_{u} v$ and $e \eval_{u} v'$ then $v = v'$.
\item If $r \seval_{u} v \synth \tau$ and $r \seval_{u} v' \synth
  \tau'$ then $v = v'$ and $\tau = \tau'$.
\end{enumerate}
\end{lemma}
\begin{proof}
  Simultaneous induction, relying on the $u \sqgt \bot$
  side-conditions in \figref{eval}.
\end{proof}

In the first judgement, when $e$ is of the form $r$ and the demand is
$\bot$, there are two possibilities that both lead to the same
result. (We can suspend $e$ directly, or go via the other judgement.)
In the second judgement, when $r$ is of the form $\exAnnot{e}{\tau}$
and the demand is $\bot$, we need the $u \sqgt \bot$ side-condition to
ensure that we obtain we want a suspension of an annotated term, not
an annotated suspension.

The evaluation judgement $e \seval_{u} v \synth \tau$ for synthesis
terms returns the type $\tau$ of $v$. When matching unpacks an
expression $e$ from a suspension $\exSuspend{e}$ in order to
substitute it, $e$ can be coerced into a synthesis form by annotating
it with $\tau$. If it is already in synthesis form $r$ then we
substitute $r$ directly.

Substitution of a closed synthesis term for a variable is
type-preserving.

\begin{lemma}[Substitution]
\label{lem:substitution:sound}
Suppose $\cxtEmpty \vdash r \synth \tau$.
\begin{enumerate}
\item If $\cxtExtend{\Gamma}{x}{\tau} \vdash e \tycheck \tau'$ then
  $\Gamma \vdash \subst{r}{x}{e} \tycheck \tau'$.
\item If $\cxtExtend{\Gamma}{x}{\tau} \vdash r \synth \tau'$ then
  $\Gamma \vdash \subst{r}{x}{r} \synth \tau'$.
\end{enumerate}
\end{lemma}

Matching, if it succeeds with a unique expression, is also
type-preserving.

\begin{lemma}[Matching]
\label{lem:matching:sound}
  Suppose $\vdash v \tycheck \sigma$. If $\Gamma;\vec{p} \tycheck
  \sigma \vdash \vec{e} \tycheck \tau$ and
  $\match{v}{\vec{\exBranch{p}{e}}} = e'$, then $\Gamma \vdash e'
  \tycheck \tau$.
\end{lemma}

\begin{lemma}[Soundness of typing]
\item
\begin{enumerate}
\item If $\cxtEmpty \vdash e \tycheck \tau$ and $e \eval_{u} v$ then
  $\vdash v \tycheck \tau$.
\item If $\cxtEmpty \vdash r \synth \tau$ and $r \seval_{u} v \synth
  \tau'$ then $\tau' = \tau$ and $\vdash v \tycheck \tau$.
\end{enumerate}
\end{lemma}
\begin{proof}
  Simultaneous induction. In the suspension rule for synthesis
  evaluation, we need that typing is deterministic. For the case rule
  and app rules we need that substitution and matching preserve types
  (\lemrefTwo{substitution:sound}{matching:sound}).
\end{proof}

If a program terminates for demand $u$, then the evaluation relation
is total if we restrict it to that program and to the prefixes of
$u$. It is also monotone.

\begin{theorem}[Total and monotonic]
  If $e \eval_{u} v$ and $u' \sqleq u$ then $e \eval_{u'} v'$ with $v'
  \sqleq v$.
\end{theorem}

\section{Applications}

\begin{enumerate}
\item Interactively change the demand on the output of a program to
  see more of the output. Can already be achieved by running a
  suspended part of a lazy result as a new top-level computation, but
  with our approach there is an explicit notion of precisely how much
  demand is being placed on the entire computation.
\item We can use this demand specification to force a computation to a
  certain depth without having to repeat the interactions. E.g., for
  restoring the state of a computation.
\item If the type of a terminating program specifies that it can yield
  a result of a certain size, then it will also terminate for any
  smaller demand, with a smaller output.
\end{enumerate}

\pagebreak
\begin{figure}
\fbox{$u \join u'$}
\small{
\begin{align*}
\bot
&\join
u
&=
u
\\
u
&\join
\bot
&=
u
\\
\exUnit
&\join
\exUnit
&=
\exUnit
\\
\exInlr{u_1}{u_2}
&\join
\exInlr{{u_1}'}{{u_2}'}
&=
\exInlr{u_1 \join {u_1}'}{u_2 \join {u_2}'}
\\
\exPair{u_1}{u_2}
&\join
\exPair{{u_1}'}{{u_2}'}
&=
\exPair{u_1 \join {u_1}'}{u_2 \join {u_2}'}
\\
(\exRoll{u})
&\join
(\exRoll{u'})
&=
\exRoll{(u \join u')}
\\
\exRecDemand
&\join
\exRecDemand
&=
\exRecDemand
\end{align*}
}
\caption{Binary join for demands}
\end{figure}


{
\bibliographystyle{plain}
\small
\bibliography{bib}
}

\end{document}

\begin{figure}
\fbox{$\rho, e \trEval_{u} v$}
\begin{smathpar}
  \inferrule*
  {
    \strut
  }
  {
    \rho, e
    \trEval_{\hole}
    \exThunk{\rho}{e}
  }
%
\and
%
  \inferrule*[right={$u \neq \hole$, $\rho(x) = \exThunk{\rho'}{e}$}]
  {
    \rho', e \trEval_{u} v
  }
  {
    \rho, x
    \trEval_{u}
    v
  }
%
\and
%
  \inferrule*
  {
    \strut
  }
  {
    \rho, \exUnit
    \trEval_{\hnf}
    \exUnit
  }
%
\and
%
  \inferrule*
  {
    \strut
  }
  {
    \rho, \exConst
    \trEval_{\hnf}
    \exConst
  }
%
\and
%
  \inferrule*
  {
    \rho, e_1 \trEval_{\hnf} c_1
    \\
    \rho, e_2 \trEval_{\hnf} c_2
  }
  {
    \rho, \exOp{e_1}{e_2}
    \trEval_{\hnf}
    c_1 \mathrel{\hat{\primOp}} c_2
  }
%
\and
%
  \inferrule*
  {
    \strut
  }
  {
    \rho, \exFun{f}{x}{e}
    \trEval_{\hnf}
    \exClosure{\rho}{f}{x}{e}
  }
%
\and
%
  \inferrule*[right={$u \neq \hole$, $v_1 = \exClosure{\rho'}{f}{x}{e}$}]
  {
    \rho, e_1 \trEval_{\hnf} v_1
    \\
    \envExtend{\envExtend{\rho'}{f}{v_1}}{x}{\exThunk{\rho}{e_2}}, e
    \trEval_{u}
    v
  }
  {
    \rho, \exApp{e_1}{e_2}
    \trEval_{u}
    v
  }
%
\and
%
  \inferrule*
  {
    \rho, e_1 \trEval_{u_1} v_1
    \\
    \rho, e_2 \trEval_{u_2} v_2
  }
  {
    \rho, \exPair{e_1}{e_2}
    \trEval_{\exPair{u_1}{u_2}}
    \exPair{v_1}{v_2}
  }
%
\and
%
  \inferrule*[right={$u \neq \hole$}]
  {
    \rho, e \trEval_{\exPair{u}{\hole}} \exPair{v_1}{v_2}
  }
  {
    \rho, \exFst{e}
    \trEval_{u}
    v_1
  }
%
\and
%
  \inferrule*[right={$u \neq \hole$}]
  {
    \rho, e \trEval_{\exPair{\hole}{u}} \exPair{v_1}{v_2}
  }
  {
    \rho, \exSnd{e}
    \trEval_{u}
    v_2
  }
%
\and
%
  \inferrule*
  {
    \rho, e \trEval_{u} v
  }
  {
    \rho, \exInl{e}
    \trEval_{\exInl{u}}
    \exInl{v}
  }
%
\and
%
  \inferrule*
  {
  }
  {
    \rho, \exInl{e}
    \trEval_{\hnf}
    \exInl{\exThunk{\rho}{e}}
  }
%
\and
%
  \inferrule*
  {
    \rho, e \trEval_{u} v
  }
  {
    \rho, \exInr{e}
    \trEval_{\exInr{u}}
    \exInr{v}
  }
%
\and
%
  \inferrule*
  {
  }
  {
    \rho, \exInr{e}
    \trEval_{\hnf}
    \exInr{\exThunk{\rho}{e}}
  }
%
\and
%
  \inferrule*[right={$u \neq \hole$}]
  {
    \rho, e \trEval_{\hnf} \exInl{v_1}
    \\
    \envExtend{\rho}{x_1}{v_1}, e_1 \trEval_{u} v
  }
  {
    \rho, \exCase{e}{x_1}{e_1}{x_2}{e_2}
    \trEval_{u}
    v
  }
%
\and
%
  \inferrule*[right={$u \neq \hole$}]
  {
    \rho, e \trEval_{\hnf} \exInr{v_2}
    \\
    \envExtend{\rho}{x_2}{v_2}, e_2 \trEval_{u} v
  }
  {
    \rho, \exCase{e}{x_1}{e_1}{x_2}{e_2}
    \trEval_{u}
    v
  }
%
\and
%
  \inferrule*
  {
    \rho, e \trEval_{u} v
  }
  {
    \rho, \exRoll{e}
    \trEval_{\exRoll{u}}
    \exRoll{v}
  }
%
\and
%
  \inferrule*[right={$u \neq \hole$}]
  {
    \rho, e \trEval_{\exRoll{u}} \exRoll{v}
  }
  {
    \rho, \exUnroll{e}
    \trEval_{u}
    v
  }
\end{smathpar}
\caption{Demand-indexed call-by-name (big-step)}
\end{figure}

\begin{figure}
\begin{syntaxfig}
\mbox{Eval context}
&
C
&
::=
&
[]
\mid
e_{\hole}
\mid
{\exApp{C}{e}}
\mid
{\exOp{C}{e}}
\mid
{\exOp{c}{C}}
\mid
\exFst{C}
\mid
\exSnd{C}
\mid
\\
&&
\mid
&
\exRoll{C}
\mid
\exUnroll{C}
\mid
\exPair{C}{e}
\mid
\exPair{v}{C}
\mid
\exInl{C}
\\
&&
\mid
&
\exInr{C}
\mid
\exCase{C}{x_1}{e_1}{x_2}{e_2}
\end{syntaxfig}
\caption{Evaluation contexts}
\end{figure}

\begin{figure*}
\begin{align*}
\focus{C[e]_{\hole}}
& =
C[e]
\\
\focus{C[\exUnit]_{\hnf}}
& =
C[\exUnit]
\\
\focus{C[\exConst]_{\hnf}}
& =
C[\exConst]
\\
\focus{C[\exOp{e_1}{e_2}]_{\hnf}}
& =
\focus{C \circ \exOp{[e_1]_{\hnf}}{e_2}}
\\
\focus{C[\exOp{c}{e}]_{\hnf}}
& =
\focus{C \circ \exOp{c}{[e]_{\hnf}}}
\\
\focus{C[\exOp{c_1}{c_2}]_{\hnf}}
& =
C[\exOp{c_1}{c_2}]_{\hnf}
\\
\focus{C[\exFun{f}{x}{e}]_{\hnf}}
& =
C[\exFun{f}{x}{e}]
\\
\focus{C[\exApp{(\exFun{f}{x}{e})}{e_2}]_{w}}
& =
C[\exApp{e_1}{e_2}]_{w}
\\
\focus{C[\exApp{e_1}{e_2}]_{w}}
& =
\focus{C \circ \exApp{[e_1]_{\hnf}}{e_2}}
\\
\focus{C[\exPair{e_1}{e_2}]_{\exPair{w}{u}}}
& =
\focus{C \circ \exPair{[e_1]_{w}}{e_2}}
\\
\focus{C[\exPair{v}{e_2}]_{\exPair{u}{w}}}
& =
\focus{C \circ \exPair{v}{[e_2]_{w}}}
\\
\focus{C[\exPair{e_1}{e_2}]_{\exPair{\hole}{\hole}}}
& =
C[\exPair{e_1}{e_2}]
\\
\focus{C[\exFst{\exPair{e_1}{e_2}}]_{w}}
& =
C[\exFst{e}]_{w}
\\
\focus{C[\exFst{e}]_{w}}
& =
\focus{C \circ \exFst{[e]_{\exPair{w}{\hole}}}}
\\
\focus{C[\exSnd{\exPair{e_1}{e_2}}]_{w}}
& =
C[\exSnd{e}]_{w}
\\
\focus{C[\exSnd{e}]_{w}}
& =
\focus{C \circ \exSnd{[e]_{\exPair{\hole}{w}}}}
\\
\focus{C[\exInl{e}]_{\hnf}}
& =
C[\exInl{e}]
\\
\focus{C[\exInl{e}]_{\exInl{w}}}
& =
\focus{C \circ \exInl{[e]_{w}}}
\\
\focus{C[\exInr{e}]_{\hnf}}
& =
C[\exInr{e}]
\\
\focus{C[\exInr{e}]_{\exInr{w}}}
& =
\focus{C \circ \exInr{[e]_{w}}}
\\
\focus{C[\exCase{\exInl{e'}}{x_1}{e_1}{x_2}{e_2}]_{w}}
& =
C[\exCase{e}{x_1}{e_1}{x_2}{e_2}]
\\
\focus{C[\exCase{\exInr{e'}}{x_1}{e_1}{x_2}{e_2}]_{w}}
& =
C[\exCase{e}{x_1}{e_1}{x_2}{e_2}]
\\
\focus{C[\exCase{e}{x_1}{e_1}{x_2}{e_2}]_{w}}
& =
\focus{C \circ \exCase{[e]_{\hnf}}{x_1}{e_1}{x_2}{e_2}}
\\
\focus{C[\exRoll{e}]_{\hnf}}
& =
C[\exRoll{e}]
\\
\focus{C[\exRoll{e}]_{\exRoll{w}}}
& =
\focus{C \circ \exRoll{[e]_{w}}}
\\
\focus{C[\exUnroll{(\exRoll{e'})}]_{w}}
& = 
C[\exUnroll{e}]_{w}
\\
\focus{C[\exUnroll{e}]_{w}}
& = 
\focus{C \circ \exUnroll{[e]_{\exRoll{w}}}}
\end{align*}
\caption{Demand-indexed focusing}
\end{figure*}

\begin{figure*}
\begin{align*}
C[\exApp{(\exFun{f}{x}{e})}{e'}]
&
\rightarrow
C[\subst{\exFun{f}{x}{e}}{f}{\subst{e'}{x}{e}}]
\\
C[\exCase{\exInl{e}}{x_1}{e_1}{x_2}{e_2}]
&
\rightarrow
C[\subst{e}{x_1}{e_1}]
\\
C[\exCase{\exInr{e
}}{x_1}{e_1}{x_2}{e_2}]
&
\rightarrow
C[\subst{e}{x_2}{e_2}]
\\
C[\exFst{\exPair{v_1}{v_2}}]
&
\rightarrow
C[v_1]
\\
C[\exSnd{\exPair{v_1}{v_2}}]
&
\rightarrow
C[v_2]
\\
C[\exOp{{c_1}}{{c_2}}]
&
\rightarrow
C[c_1\;\hat{\primOp}\;c_2]
\\
C[\exUnroll{(\exRoll{v})}]
&
\rightarrow
C[v]
\end{align*}
\caption{Contraction rules}
\end{figure*}

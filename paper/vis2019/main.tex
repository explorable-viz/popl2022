\documentclass{vgtc}
% other options: review, preprint, widereview, electronic (for hyperrefs)
% IEEE Vis 2019 does not require anonymous submissions

%% Figures should be in CMYK or Grey scale format, otherwise, colour
%% shifting may occur during the printing process.

%% These few lines make a distinction between latex and pdflatex calls and they
%% bring in essential packages for graphics and font handling.
%% Note that due to the \DeclareGraphicsExtensions{} call it is no longer necessary
%% to provide the the path and extension of a graphics file:
%% \includegraphics{diamondrule} is completely sufficient.
%%
\ifpdf%                                % if we use pdflatex
  \pdfoutput=1\relax                   % create PDFs from pdfLaTeX
  \pdfcompresslevel=9                  % PDF Compression
  \pdfoptionpdfminorversion=7          % create PDF 1.7
  \ExecuteOptions{pdftex}
  \usepackage{graphicx}                % allow us to embed graphics files
  \DeclareGraphicsExtensions{.pdf,.png,.jpg,.jpeg} % for pdflatex we expect .pdf, .png, or .jpg files
\else%                                 % else we use pure latex
  \ExecuteOptions{dvips}
  \usepackage{graphicx}                % allow us to embed graphics files
  \DeclareGraphicsExtensions{.eps}     % for pure latex we expect eps files
\fi%

%% If you are submitting a paper to a conference for review with a double
%% blind reviewing process, please replace the value ``0'' below with your
%% OnlineID. Otherwise, you may safely leave it at ``0''.
\onlineid{0}

%% declare the category of your paper, only shown in review mode
\vgtccategory{Research}

%% allow for this line if you want the electronic option to work properly
\vgtcinsertpkg

%% In preprint mode you may define your own headline.
%\preprinttext{To appear in an IEEE VGTC sponsored conference.}

\newcommand\hmmax{0} % http://tex.stackexchange.com/questions/3676
\newcommand\bmmax{0}

\usepackage{float}
\usepackage{mathpartir}
\usepackage{bm}
\usepackage{soul}
\usepackage{xspace}
\usepackage{mathtools}
\usepackage{hyperref}
\usepackage{breakurl}
\usepackage[scaled=0.75]{beramono}
\usepackage{pifont}
\usepackage{framed}
\usepackage{subdepth}
\usepackage{dashundergaps}
\usepackage{listings}
\usepackage{mleftright}
\usepackage{extarrows}
\usepackage{placeins}
\usepackage[a]{esvect}
\usepackage{colortbl}
\usepackage{tabularx}
\usepackage{multirow}
\usepackage{subcaption}
\usepackage{enumitem}
\usepackage{sfmath}
\usepackage{tikz}
\usetikzlibrary{calc}

\newcommand*{\derivationWidth}{0.6\textwidth}
\input{tex-common/proof-helpers}
\input{tex-common/localref}
\input{tex-common/column-types}
\input{tex-common/relational-override} % no longer need this, but must sync up proofs first


\title{Explorable Data Visualisations}

\author{Roly Perera}
\orcid{0000-0001-9249-9862}
\affiliation{University of Glasgow}
\affiliation{University of Edinburgh}
\email{roly.perera@glasgow.ac.uk}
\email{roly.perera@ed.ac.uk}

\abstract{
   Linked brushing is a popular interactive visualisation technique for
   coordinating multiple views of a single underlying dataset. Selecting
   (brushing) elements in one view automatically selects the corresponding
   elements in other views, allowing the user to easily explore how things are
   related. While such techniques for view coordination are common, their
   application is typically limited to particular predefined views and to
   certain visual attributes of those views, such as the appearance of a point
   in a scatterplot.

   In this paper, we present a framework for authoring and rendering
   visualisations with automatic support for linking been visualisations, code
   and data. We adapt a dynamic program analysis technique called \emph{program
   slicing} to the visual domain, allowing us to track how individual graphical
   properties of visual components (say the colour used to fill a shape, or the
   height of a bar in a histogram) relate both to specific source data and
   fragments of program text. We give two examples of the kind of linked
   brushing scenarios that our framework supports automatically and leave a more
   formal evaluation of our approach to future work.
}


\CCScatlist{
  \CCScatTwelve{Human-centered computing}{Visualization}{Visualization systems and tools}{Visualization toolkits};
  \CCScatTwelve{Human-centered computing}{Visualization}{Visualization application domains}{Scientific visualization}
}

%% Copyright enabled by default; disabled by 'review' option or via \nocopyrightspace.
\begin{document}
\maketitle

\section{Introduction}

Data visualisation is essential to data science and science communication, but
is open to both misinterpretation and misuse: patterns in raw data can be
obscured, statistical assumptions hidden, and effect sizes misrepresented
\cite{weissgerber15}. These concerns can be addressed in part through improved
statistical practices and better plot and chart designs \cite{allen19}, but also
by making visualisations themselves more open and explorable
\cite{dragicevic19}.

Understanding a visualisation requires grasping how it relates to the underlying
data and other visualisations. For example, geoscientists often work with
multiple layered views. To show how these are related, spatial analytics
applications like GeoDa \cite{anselin06} can automatically select the relevant
part of one view as the user changes the selection in a related view, say a
choropleth map. However, this feature is available only if it was specifically
anticipated by the application or library developer; if the geoscientist uses
custom libraries or wants other views linked that the developer did not
consider, they are out of luck.

\includegraphics[scale=0.35]{image/chart-fwd}

In this paper we present a framework for authoring visualisations where support
for linking, between data, code, and visualisations is built in, making this
powerful comprehension feature automatic.

% \cite{perera16d,ricciotti17}

\section{Another section}

Our framework provides infrastructure for ``linking'', or multiple-coordinated
views \cite{tobiasz09}, in an application-independent way. The theoretical
foundation of the work is our own prior work on dynamic dependency analysis and
provenance \cite{perera16d, ricciotti17}; in contrast to prior work on
provenance in data visualisation \cite{callahan06}, our approach is much more
fine-grained and is able to associate specific parts of the data or code with
parts of a visualisation in a precise way. While the theoretical technique is
proven, this approach has never been applied to data visualisation before.

\includegraphics[scale=0.35]{image/chart-bwd}

\section{Conclusions and future work}

There are a number of challenges associated with making our approach practical
and appealing to actual data scientists. A central usability challenge is
visualising these complex relationships between the various parts of a
visualisation and the relevant data and/or visualisation code. This is
essentially a higher-order visualisation problem: visualising information about
the provenance of visualisations. Ideas from temporal data visualisation
\cite{bach16}, data-driven storytelling \cite{bach18}, and ``literate''
visualisation \cite{wood19} may inform our efforts here.

\acknowledgments{}
\bibliographystyle{tex/abbrv-doi}
% Other options: abbrv, abbrv-doi-narrow, abbrv-doi-hyperref, abbrv-doi-hyperref-narrow
\bibliography{../../tex-common/bib}
\end{document}

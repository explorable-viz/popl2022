\section{Introduction and motivation}

Data visualisation is an essential component of modern data science and science
communication. Unfortunately, its power as a communicative tool means leaves it
open to misinterpretation and misuse: patterns in raw data can be obscured,
statistical assumptions hidden, and effect sizes misrepresented
\cite{weissgerber15}. These concerns can be addressed in part through improved
statistical practices and novel plot and chart designs \cite{allen19}, but also
by making visualisations themselves more open and explorable
\cite{dragicevic19}.

One well-established technique for explorability is \emph{linked brushing}
\cite{fisherkeller75,becker87,buja91}, which allows the user to explore
interactively how a visualisation relates to other concurrent visualisations of
the same data. For example, geoscientists often work with multiple layered
views. To show how these are related, spatial analytics applications like GeoDa
\cite{anselin06} can automatically select the relevant part of one view as the
user changes the selection in a related view, say a choropleth map. Such linking
is an important navigational tool, allowing the user to switch contexts in one
view and have related views synchronise automatically. With complex multivariate
data, linking can also be an essential sense-making aid, helping the user see
relationships in the data more readily \cite{he18}.

However, linked brushing and similar features usually have to be specifically
anticipated by library developer and carefully designed in, so that the
visualisation author can make use of them. If the author uses a custom library
or wants other views or visual attributes linked that the library developer did
not consider, they are out of luck. For example, contemporary visualisation
libraries for the web like Bokeh \cite{jolly18} and D3 \cite{bostock11} provide
advanced linking and brushing features, but only for specific visual elements,
such as the glyphs used to render the points in a chart.

\begin{figure}[h]
\includegraphics[scale=0.35]{image/chart-fwd}
\caption{Forward linking from code and data to visualisation}
\end{figure}

Moroever, it is our research hypothesis that, during the development of a
visualisation, it is useful to see how the visualisation relates to the
underlying data. For example, one might select a column or cell in a tabular
view of the data in order to explore how particular data elements contribute to
various parts of the view; equally, one might select a visual element or
attribute in order to see all the data involved in computing it. Roberts and
Wright identify the potential utility of ``ubiquitous brushing'' for visual
elements other than plots, such as data tables and legends \cite{roberts06}; in
this paper, we present a framework that extends linking not only to all visual
attributes but also to data and source code. Visualisations authored in our
framework have fine-grained ubiquitous linking built in, making this powerful
comprehension feature automatic.

We formulate the problem more precisely in \Secref{problem-overview} and
describe our solution in \Secref{framework}. \Secref{related-work} discusses
related work in more detail and \Secref{conclusion} concludes with some
directions for future work.

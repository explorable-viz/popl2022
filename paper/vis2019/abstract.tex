\abstract{
   Linked brushing is a popular interactive visualisation technique for
   coordinating multiple views of the same underlying dataset. Selecting
   (brushing) elements in one view automatically selects the corresponding
   elements in other views, allowing the user to quickly understand how things
   are related. While such techniques for view coordination are common, their
   implementation is somewhat ad hoc: typically only certain views can be
   coordinated, and only certain aspects of those views, such as the appearance
   of a point in a scatterplot.

   In this paper, we present a framework for authoring and rendering
   visualisations with automatic support for linking been visualisations, code
   and data. Individual graphical properties of visual components (the colour
   used to fill a shape, or the height of a bar in a histogram) are related both
   to specific source data and program fragments that contributed to them. By
   providing this . We give two examples of the kind of linked brushing
   scenarios that our framework supports automatically and leave a more formal
   evaluation of our approach to future work.
}

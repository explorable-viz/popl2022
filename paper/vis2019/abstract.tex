\abstract{
   Linked brushing is a powerful interactive visualisation technique for
   coordinating multiple views of the same underlying dataset. Selecting
   (brushing) elements in one view automatically selects the corresponding
   elements in other views, allowing the user to quickly test hypotheses about
   how things are related. While such techniques for view coordination are
   common, their implementation is somewhat ad hoc: typically only certain views
   can be coordinated, and only certain aspects of those views, such as the
   appearance of a point in a scatterplot.

   Moreover analogous linking to the
   underlying data and code used to generate the visualisation is not possible.

   In this paper, we present a framework for authoring and rendering
   visualisations [...]. Each graphical property of a visual component (for
   example the colour used to fill a shape or the height of a bar in a
   histogram) We present two representative interaction scenarios that our
   framework supports and leave a more formal evaluation of our approach to
   future work.
}

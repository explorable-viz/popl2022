\abstract{
   Linked brushing is a popular interactive visualisation technique for
   coordinating multiple views of a dataset. Selecting (brushing) visual
   elements in one view automatically selects the corresponding elements in
   other views, allowing the user to easily explore how things are related.
   While such coordinated views are common, their implementation is somewhat ad
   hoc, limiting their application to certain predefined charts or widgets, and
   particular visual attributes, such as the appearance of points in a
   scatterplot.

   In this paper, we present a framework for authoring and rendering
   visualisations with built-in support for fine-grained linking been visual
   attributes, code and data. Our approach employs a dynamic program analysis
   technique called \emph{program slicing} to track how individual visual
   attributes (say the colour used to fill a shape or the height of a bar in a
   histogram) depend on individual data values and on specific bits of
   visualisation code. Linking between two views involves a round-trip traversal
   of these dependencies: backwards from the first view to select the required
   code and data, and then forwards to determine the relevant parts of the
   second view. We give two examples of the kinds of view coordination scenarios
   that our framework supports, and contrast our approach to existing solutions.
}

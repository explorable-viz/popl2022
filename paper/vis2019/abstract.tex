\abstract{
   Linked brushing is a popular interactive visualisation technique for
   coordinating multiple views of a single underlying dataset. Selecting
   (brushing) elements in one view automatically selects the corresponding
   elements in other views, allowing the user to easily explore how things are
   related. While such techniques for view coordination are common, their
   application is typically limited to particular predefined views and to
   certain visual attributes of those views, such as the appearance of a point
   in a scatterplot.

   In this paper, we present a framework for authoring and rendering
   visualisations with automatic support for linking been visualisations, code
   and data. We adapt a dynamic program analysis technique called \emph{program
   slicing} to the visual domain, allowing us to track how individual graphical
   properties of visual components (say the colour used to fill a shape, or the
   height of a bar in a histogram) relate both to specific source data and
   fragments of program text. We give two examples of the kind of linked
   brushing scenarios that our framework supports automatically and leave a more
   formal evaluation of our approach to future work.
}

\abstract{
   Linked brushing is a popular interactive visualisation technique for
   coordinating multiple views of a single dataset. Selecting (brushing) visual
   elements in one view automatically selects the corresponding elements in
   other views, allowing the user to easily explore how things are related.
   While such coordinated views are common, their implementation is somewhat ad
   hoc, limiting their application to particular predefined views or to certain
   visual attributes, such as the appearance of a point in a scatterplot.

   In this paper, we present a framework for authoring and rendering
   visualisations with built-in support for fine-grained linking been visual
   attributes, code and data. Our approach employs a dynamic program analysis
   technique called \emph{program slicing} to track how individual visual
   attributes (say the colour used to fill a shape or the height of a bar in a
   histogram) relate to specific source data and to individual expressions in
   the visualisation code. We give two examples of the kind of view coordination
   scenarios that our framework supports, and discuss opportunities and
   challenges.
}

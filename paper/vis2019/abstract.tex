\abstract{
   Linking and brushing is a powerful interactive visualisation technique for
   coordinating disparate views of the same underlying dataset. Selecting
   (brushing) elements in one view automatically selects the corresponding
   elements in other views, allowing the user to quickly test hypotheses about
   how things are related. While such techniques for view coordination are
   common, their implementation is somewhat ad hoc: typically only certain views
   can be coordinated in this way

   In this paper, we present a framework for authoring and rendering
   visualisations [...]. Each graphical property of a visual component (for
   example the colour used to fill a shape or the height of a bar in a
   histogram) 

   Our proof-of-concept implementation consists of a
   simple programming language and library for authoring visualisations, a
   language runtime that provides the dependency-tracking infrastructure to
   support linkage. We present a couple of representative interaction scenarios
   and leave a more formal evaluation of our approach to future work.
}

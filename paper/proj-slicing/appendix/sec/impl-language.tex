\section{Implementation language}

We now describe a more realistic implementation language with \emph{data types}
(\secref{impl-language:data-types}), which replace recursive types, binary
sums and binary products by a single construct for named recursive
sums-of-products; primitive operations
(\secref{impl-language:primitives}), \ttt{match}...\ttt{as}
(\secref{impl-language:match-as}), and mutually recursive functions
(\secref{impl-language:recursion}).

\subsection{Data types}
\label{sec:impl-language:data-types}

\figref{impl-language:syntax} includes all the syntax associated with data
types, including data type names $D$, constructors $c$, constructor expressions
$\exConstr{c}{\vec{e}}$ where $\vec{e}$ is a vector of arguments, constructor
(or \emph{sum}) eliminators $\Sigma$ of the form $\family{\kappa_c}{c}{\tilde{c}}$
where $\tilde{c}$ is a set of constructors. Every data type name $D$ has an
interpretation $\interpret{D}$ as a family of vectors of types
$\family{\vec{A}_c}{c}{\tilde{c}}$.

\subsection{Primitives}
\label{sec:impl-language:primitives}

\begin{figure}
{\small
\begingroup
\renewcommand*{\arraystretch}{1}
\begin{minipage}[t]{0.5\textwidth}
\begin{tabularx}{\textwidth}{rL{2.3cm}L{3cm}}
%\rowcolor{verylightgray}
&\textbfit{Identifier}&
\\
$x, y ::=$
&
$\ldots$
&
\\
&
$\primOp$
&
operator name
\\[2mm]
&\textbfit{Surface term}&
\\
$s, t ::=$
&
$\ldots$
&
\\
&
$\exOp{\primOp}$
&
first-class operator
\\
&
$\exBinaryApp{s}{\primOp}{s'}$
&
binary application
\\
&
$\exLetRecMutual{\vec{g}}{s}$
&
recursive functions
\\
&
$\exIfThenElse{s}{s}{s}$
&
if
\\
&
$\exMatch{s}{\vec{\clauseUncurried{p}{s}}}$
&
match
\\
&
$\exLet{p}{s}{s}$
&
structured let
\\
&
$\annList{s}{r}{\alpha}$
&
non-empty list
\\
&
$\exListEnum{s}{s}$
&
list enum
\\
&
$\annListComp{s}{\vec{q}}{\alpha}$
&
list comprehension
\\[2mm]
&\textbfit{List rest term}&
\\
$r ::=$
&
$\annListEnd{\alpha}$
&
end
\\
&
$\annListNext{s}{r}{\alpha}$
&
cons
\\
\\
\end{tabularx}
\end{minipage}%
\begin{minipage}[t]{0.5\textwidth}
\begin{tabularx}{\textwidth}{rL{2.8cm}L{2.9cm}}
&\textbfit{Recursive function}&
\\
$g ::=$
&
$\bind{x}{\vec{c}}$
&
\\[2mm]
&\textbfit{Clause}&
\\
$c ::=$
&
$\clause{\vec{p}}{s}$
&
\\[2mm]
&\textbfit{Pattern}&
\\
$p ::=$
&
$\pattVar{x}$
&
variable
\\
&
$\pattRec{\vec{\bind{x}{p}}}$
&
record
\\
&
$\pattNil$
&
nil
\\
&
$\pattCons{p}{p}$
&
cons
\\
&
$\pattList{p}{o}$
&
non-empty list
\\[2mm]
&\textbfit{List rest pattern}&
\\
$o ::=$
&
$\pattListEnd$
&
end
\\
&
$\pattListNext{p}{o}$
&
cons
\\[2mm]
&\textbfit{Qualifier}&
\\
$q ::=$
&
$\qualGuard{s}$
&
guard
\\
&
$\qualDeclaration{p}{s}$
&
declaration
\\
&
$\qualGenerator{p}{s}$
&
generator
%$x, y$
%&
%&
%identifier
%\\
%$i, j$
%&&
%positive integer
%\\[2mm]
\end{tabularx}
\end{minipage}
\endgroup
}
\caption{Syntax of surface language}
\label{fig:surface-language:syntax}
\end{figure}

\begin{figure}
\begin{syntaxfig}
\mbox{Value}
&
v
&
::=
&
\bot
&
\text{absent}
\\
&&&
\exTrue \mid \exFalse
&
\text{Boolean}
\\
&&&
\exInt{n}
&
\text{integer}
\\
&&&
\exClosureRec{\rho}{f}{\sigma}
&
\text{closure}
\\
&&&
\exPair{v}{v'}
&
\text{pair}
\\
&&&
\exPairDel{v}{v'}
\\
&&&
\exNil
&
\text{nil}
\\
&&&
\exCons{v}{v'}
&
\text{cons}
\\
&&&
\exConsDel{v}{v'}
\\[2mm]
\mbox{Environment}
&
\rho
&
::=
&
\envEmpty
&
\text{empty}
\\
&&&
\envExtend{\rho}{x_{\alpha}}{v}
&
\text{extend}
\end{syntaxfig}
\caption{Values and environments}
\end{figure}


\figref{impl-language:syntax} includes all the syntax associated with primitive
operations, including pairwise-disjoint ground types $\tyGround{C}$, constants
$\exConst{k}$ of ground type, unary primitives $\phi$ which are first-class
functions, and binary primitives $\primOp$ which are infix and not first-class.
It is straightforward to add Haskell-style sections which convert binary
operators $\primOp$ into first-class values.

Each constant $\exConst{k}$ has a unique ground type $\typeof{\exConst{k}}$.
Each unary primitive $\phi$ has a unique type $\typeof{\phi}$ of the form $C \to
A$ and interpretation $\interpret{\phi} \in \Val{C} \to \Val{A}$. Unary
primitives have no expression form; they are introduced by
$\exPrimDef{\exVar{x}}$ definitions, which bind $\exVar{x}$ to its
interpretation $\interpret{\exVar{x}}$ as a unary primitive, if it has one.
(This rather complex setup is to allow each first-class primitive to be
associated with a specific bit of \emph{syntax} for slicing purposes.) Each
binary operator $\primOp$ has a unique type $\typeof{\primOp}$ and
interpretation $\interpret{\primOp} \in \Val{C_1} \times \Val{C_2} \to \Val{A}$.

\begin{figure}[H]
\flushleft \shadebox{$\Gamma \vdash e: A$}
\begin{smathpar}
\inferrule*[right={$x : A \in \Gamma$}]
{
   \strut
}
{
   \Gamma \vdash \exVar{x}: A
}
%
\and
%
\inferrule*[right={$\primOp: A \in \Gamma$}]
{
   \strut
}
{
   \Gamma \vdash \exOp{\primOp}: A
}
%
\and
%
\inferrule*
{
   \strut
}
{
   \Gamma \vdash n: \tyInt
}
%
\and
%
\inferrule*
{
   \strut
}
{
   \Gamma \vdash \exTrue: \tyBool
}
%
\and
%
\inferrule*
{
   \strut
}
{
   \Gamma \vdash \exFalse: \tyBool
}
%
\and
%
\inferrule*
{
   \Gamma \vdash u: A
   \\
   \Gamma \vdash v: B
}
{
   \Gamma \vdash \exPair{u}{v}: \tyProd{A}{B}
}
%
\and
%
\inferrule*[right={$\primOp: \tyFun{\tyInt}{\tyFun{\tyInt}{\tyInt}} \in \Gamma$}]
{
   \Gamma \vdash e: \tyInt
   \\
   \Gamma \vdash e': \tyInt
}
{
   \Gamma \vdash \exBinaryApp{e}{\primOp}{e'}: \tyInt
}
%
\and
%
\inferrule*
{
   \Gamma \vdash \sigma : \tyFun{A}{B}
}
{
   \Gamma \vdash \exLambda{\sigma} : \tyFun{A}{B}
}
%
\and
%
\inferrule*
{
   \Gamma \vdash e: \tyFun{A}{B}
   \\
   \Gamma \vdash e': A
}
{
   \Gamma \vdash \exApp{e}{e'}: B
}
%
\and
%
\inferrule*
{
   \strut
}
{
   \Gamma \vdash \exNil: \tyList{A}
}
%
\and
%
\inferrule*
{
   \Gamma \vdash u: A
   \\
   \Gamma \vdash v: \tyList{A}
}
{
   \Gamma \vdash (\exCons{u}{v}): \tyList{A}
}
%
\and
%
\inferrule*
{
   \Gamma \vdash e': \tyProd{\tyInt}{\tyInt}
   \\
   \Gamma \concat \bind{x}{\tyInt} \concat \bind{y}{\tyInt} \vdash e: A
}
{
   \Gamma \vdash \exMatrix{e}{x}{y}{e'}: \tyMatrix{A}
}
%
\and
%
\inferrule*
{
   \Gamma \vdash e: \tyMatrix{A}
   \\
   \Gamma \vdash e': \tyProd{\tyInt}{\tyInt}
}
{
   \Gamma \vdash \exMatrixAccess{e}{e'}: A
}
%
\and
%
\inferrule*
{
   \Gamma \vdash e: \tyMatrix{A}
}
{
   \Gamma \vdash \exMatrixDims{e}: \tyProd{\tyInt}{\tyInt}
}
%
\and
%
\inferrule*
{
   \Gamma \vdash h: \Delta
   \\
   \Gamma \concat \Delta \vdash e: A
}
{
   \Gamma \vdash \exLetRecMutual{h}{e}: A
}
\end{smathpar}

\vspace{5pt}
\flushleft \shadebox{$\Gamma \vdash \sigma: \tyFun{A}{B}$}
\begin{smathpar}
\inferrule*
{
   \Gamma \concat \bind{x}{A} \vdash \kappa: B
}
{
   \Gamma \vdash (\elimVar{x}{\kappa}): \tyFun{A}{B}
}
%
\and
%
\inferrule*
{
   \Gamma \vdash \kappa: A
   \\
   \Gamma \vdash \kappa': A
}
{
   \Gamma \vdash \elimBool{\kappa}{\kappa'}: \tyFun{\tyBool}{A}
}
%
\and
%
\inferrule*
{
   \Gamma \vdash \sigma: \tyFun{A}{\tyFun{A'}{B}}
}
{
   \Gamma \vdash \elimProd{\sigma}: \tyFun{\tyProd{A}{A'}}{B}
}
%
\and
%
\inferrule*
{
   \Gamma \vdash \kappa: B
   \\
   \Gamma \vdash \sigma: \tyFun{A}{\tyFun{\tyList{A}}{B}}
}
{
   \Gamma \vdash \elimList{\kappa}{\sigma}: \tyFun{\tyList{A}}{B}
}
\end{smathpar}
\caption{Typing rules for terms and eliminators}
\end{figure}

\begin{figure}
\flushleft \shadebox{$\Gamma \vdash \sigma: \tyFun{A}{B}$}
\begin{smathpar}
\inferrule*
{
   \cxtExtend{\Gamma}{x}{A} \vdash e: B
}
{
   \Gamma \vdash \elimVar{x}{e}: \tyFun{A}{B}
}
%
\and
%
\inferrule*
{
   \Gamma \vdash e: A
   \\
   \Gamma \vdash e': A
}
{
   \Gamma \vdash \elimBool{e}{e'}: \tyFun{\tyBool}{A}
}
%
\and
%
\inferrule*
{
   \cxtExtend{\cxtExtend{\Gamma}{x}{A}}{y}{A'} \vdash e: B
}
{
   \Gamma \vdash \elimPair{x}{y}{e}: \tyFun{\tyProd{A}{A'}}{B}
}
%
\and
%
\inferrule*
{
   \Gamma \vdash e: A
   \\
   \cxtExtend{\cxtExtend{\Gamma}{x}{\tyInt}}{y}{\tyList{\tyInt}} \vdash e': A
}
{
   \Gamma \vdash \elimList{\branchNil{e}}{\branchCons{x}{y}{e'}}: \tyFun{\tyList{\tyInt}}{A}
}
%
\and
%
\inferrule*
{
   \Gamma \vdash e: A
   \\
   \cxtExtend{\cxtExtend{\Gamma}{x}{\bot}}{y}{\tyListProj{\tyInt}} \vdash e: A
}
{
   \Gamma \vdash \elimList{\branchNil{e}}{\branchCons{x}{y}{e'}}: \tyFun{\tyListProj{\tyInt}}{A}
}
\end{smathpar}
\caption{Typing rules for eliminators}
\end{figure}

\begin{figure}
\flushleft \shadebox{$\valType{v}{A}$}
\begin{smathpar}
\inferrule*
{
   \valType{u}{A}
}
{
   \valType{\annot{u}{\alpha}}{A}
}
\end{smathpar}
\\[3mm]
\flushleft \shadebox{$\valType{u}{A}$}
\begin{smathpar}
\inferrule*
{
   \strut
}
{
   \valType{\exUnit}{\tyUnit}
}
%
\and
%
\inferrule*
{
   \valType{v}{A}
}
{
   \valType{\exInl{v}}{\tySum{A}{B}}
}
%
\and
%
\inferrule*
{
   \valType{v}{B}
}
{
   \valType{\exInr{v}}{\tySum{A}{B}}
}
%
\and
%
\inferrule*
{
   \valType{v_1}{A}
   \\
   \valType{v_2}{B}
}
{
   \valType{\exPair{v_1}{v_2}}{\tyProd{A}{B}}
}
%
\and
%
\inferrule*
{
   \envType{\Gamma}{\rho}
   \\
   \trieType{\sigma}{B}{A}{\Gamma}
}
{
   \valType{\exClosure{\rho}{\exFun{\sigma}}}{\tyFun{A}{B}}
}
%
\and
%
\inferrule*
{
   \valType{v}{\subst{A}{\tyRec{\tyVar{X}}{A}}{\alpha}}
}
{
   \valType{\exRoll{v}}{\tyRec{\tyVar{X}}{A}}
}
\end{smathpar}
\vspace{5pt}

\noindent \shadebox{$\envType{\Gamma}{\rho}$}
\begin{smathpar}
\inferrule*
{
   \strut
}
{
   \envType{\cxtEmpty}{\envEmpty}
}
%
\and
%
\inferrule*
{
   \envType{\Gamma}{\rho}
   \\
   \valType{u}{A}
}
{
   \envType{\cxtExtend{\Gamma}{x}{A}}{\envExtend{\rho}{x}{u}}
}
\end{smathpar}
\caption{Typing rules for values and environments}
\label{fig:demand-indexed:typing-value}
\end{figure}

% \begin{figure}
\flushleft \shadebox{$\explType{\Gamma}{T}{A}$}
\begin{smathpar}
\inferrule*
{
   \strut
}
{
   \explType{\Gamma}{\trEmpty}{A}
}
%
\and
%
\inferrule*[right={$x : A \in \Gamma$}]
{
   \explType{\Gamma'}{T}{A}
}
{
   \explType{\Gamma}{\trVar{x}{\Gamma'}{T}}{A}
}
%
\and
%
\inferrule*
{
   \explType{\Gamma}{T}{(\tyFun{A}{B})}
   \\
   \explType{\Gamma}{T'}{A}
   \\
   \elimType{\xi}{K}{A}{\Gamma'}
   \\
   \explType{\Gamma'}{U}{B}
}
{
   \explType
      {\Gamma}
      {\trApp{T}{T'}{\matchPlug{\xi}{U}}}
      {B}
}
%
\and
%
\inferrule*[right={\textnormal{$\interpret{\primOp} \in \BinaryOp{C}{C'}{A}$}}]
{
   \explValType{\Gamma}{\explVal{T}{c}}{C}
   \\
   \explValType{\Gamma}{\explVal{T'}{c'}}{C'}
}
{
   \explType
      {\Gamma}
      {\trPrimApp
         {(\explVal{T}{c})}
         {\exPrimOp}
         {(\explVal{T'}{c'})}}
      {A}
}
%
\and
%
\inferrule*
{
   \explType{\Gamma}{T}{A}
   \\
   \elimType{\xi}{\ExprPartial{B}}{A}{\Gamma'}
   \\
   \explType{\Gamma \concat \Gamma'}{T'}{B}
}
{
   \explType
      {\Gamma}
      {\trMatch{T}{\matchPlug{\xi}{T'}}}
      {B}
}
%
\and
%
\inferrule*
{
   \explValType{\Gamma}{\explVal{T}{u}}{A}
   \\
   \explType{\cxtExtend{\Gamma}{x}{A}}{U}{B}
}
{
   \explType
      {\Gamma}
      {\trLet{x}{\explVal{T}{u}}{\indexed{U}}}
      {B}
}
%
\and
%
\inferrule*
{
   \Gamma \vdash \delta: \Delta
   \\
   \explType{\Gamma \concat \Delta}{T}{A}
}
{
   \explType
      {\Gamma}
      {\trLetrec{\delta}{\indexed{T}}}
      {A}
}
\end{smathpar}
\vspace{5pt}

\flushleft \shadebox{$\explValType{\Gamma}{\vec{\explVal{T}{u}}}{\vec{A}}$}
\begin{smathpar}
\inferrule*
{
   \strut
}
{
   \explValType{\Gamma}{\seqEmpty}{\seqEmpty}
}
%
\and
%
\inferrule*
{
   \explValType{\Gamma}{\explVal{T}{u}}{A}
   \\
   \explValType{\Gamma}{\vec{\explVal{T}{u}}}{\vec{A}}
}
{
   \explValType{\Gamma}{(\explVal{T}{u}) \concat (\vec{\explVal{T}{u}})}{(A \concat \vec{A})}
}
\end{smathpar}
\caption{Typing rules for explanations}
\end{figure}


\begin{figure}
\flushleft \shadebox{$v, \sigma \match \rho, \matchPlugLow{\xi}{e}$}
\begin{smathpar}
   \inferrule*[lab={\ruleName{$\match$-var}}]
   {
      \strut
   }
   {
      v, \elimVar{x}{e} \match \envExtend{\envEmpty}{x}{v}, \matchVar{x}{e}
   }
   \\
   %
   \and
   %
   \inferrule*[lab={\ruleName{$\match$-true}}]
   {
      \strut
   }
   {
      \exTrue, \elimBool{e}{e'}
      \match
      \envEmpty, \matchTrueLow{e}{e'}
   }
   %
   \and
   %
   \inferrule*[lab={\ruleName{$\match$-false}}]
   {
      \strut
   }
   {
      \exFalse, \elimBool{e}{e'}
      \match
      \envEmpty, \matchFalseLow{e}{e'}
   }
   %
   \and
   %
   \inferrule*[lab={\ruleName{$\matchFwd$-pair}}]
   {
      \strut
   }
   {
      \exPair{v}{v'}, \elimPair{x}{y}{e}
      \matchFwd
      \envExtend{\envExtend{\envEmpty}{x}{v}}{y}{v'}, e, \top
   }
   %
   \and
   %
   \inferrule*[lab={\ruleName{$\matchFwd$-pair-del}}]
   {
      \strut
   }
   {
      \exPairDel{v}{v'}, \elimPair{x}{y}{e}
      \matchFwd
      \envExtend{\envExtend{\envEmpty}{x}{v}}{y}{v'}, e, \bot
   }
   %
   \and
   %
   \inferrule*[lab={\ruleName{$\match$-nil}}]
   {
      \strut
   }
   {
      \exNil, \elimList{\branchNil{e}}{\branchCons{x}{y}{e'}}
      \match
      \envEmpty, \matchNilLow{e}{x}{y}{e'}
   }
   %
   \and
   %
   \inferrule*[lab={\ruleName{$\match$-cons}}]
   {
      \strut
   }
   {
      \exCons{v}{v'}, \elimList{\branchNil{e}}{\branchCons{x}{y}{e'}}
      \match
      \envExtend{\envExtend{\envEmpty}{x}{v}}{y}{v'}, \matchConsLow{e}{x}{y}{e'}
   }
\end{smathpar}
\caption{Pattern-matching}
\end{figure}

\begin{figure}
\flushleft \shadebox{$\rho, \matchPlug{\xi}{\kappa}, \alpha \unlookupR{} u, \sigma$}
\begin{smathpar}
   \inferrule*[left={\ruleName{$\unlookupR{}$-var}}]
   {
     \strut
   }
   {
     \envExtend{\envEmpty}{x}{u}, \matchVar{x}{\matchHole{\kappa}}, \alpha
     \unlookupR{}
     u, \trieVar{x}{\kappa}
   }
   %
   \and
   %
   \inferrule*[left={\ruleName{$\unlookupR{}$-unit}}]
   {
     \strut
   }
   {
     \envEmpty, \matchUnit{\matchHole{\kappa}}, \alpha
     \unlookupR{}
     \annot{\exUnit}{\alpha}, \trieUnit{\kappa}
   }
   %
   \and
   %
   \inferrule*[left={\ruleName{$\unlookupR{}$-inl}}]
   {
     \rho, \matchPlug{\xi}{\kappa}, \alpha \unlookupR{} u, \sigma
   }
   {
     \rho, \matchSumL{\matchPlug{\xi}{\kappa}}{\tau}, \alpha
     \unlookupR{}
     \annot{(\exInl{u})}{\alpha}, \trieSum{\sigma}{\bot_{\tau}}
   }
   %
   \and
   %
   \inferrule*[left={\ruleName{$\unlookupR{}$-inr}}]
   {
     \rho, \matchPlug{\xi}{\kappa}, \alpha \unlookupR{} u, \tau
   }
   {
     \rho, \matchSumR{\sigma}{\matchPlug{\xi}{\kappa}}, \alpha
     \unlookupR{}
     \annot{(\exInr{u})}{\alpha}, \trieSum{\bot_{\sigma}}{\tau}
   }
   %
   \and
   %
   \inferrule*[
      left={\ruleName{$\unlookupR{}$-pair}}
   ]
   {
     \rho_2, \matchPlug{\xi'}{\kappa}, \alpha \unlookupR{} u_2, \tau
     \\
     \rho_1, \matchPlug{\xi}{\tau}, \alpha \unlookupR{} u_1, \sigma
   }
   {
     \rho_1 \concat \rho_2, \matchProd{\xi}{\matchPlug{\xi'}{\kappa}}, \alpha
     \unlookupR{}
     \annot{\exPair{u_1}{u_2}}{\alpha}, \trieProd{\sigma}
   }
   %
   \and
   %
   \inferrule*[left={\ruleName{$\unlookupR{}$-roll}}]
   {
     \rho, \matchPlug{\xi}{\kappa}, \alpha \unlookupR{} u, \sigma
   }
   {
     \rho, \matchRoll{\matchPlug{\xi}{\kappa}}, \alpha
     \unlookupR{}
     \annot{(\exRoll{u})}{\alpha}, \trieRoll{\sigma}
   }
\end{smathpar}
\caption{Reverse pattern-matching}
\label{fig:pattern-matching:unmatch}
\end{figure}

\begin{figure}[p]
\flushleft \shadebox{$\rho, e \eval{} T, u$}
\begin{smathpar}
   \inferrule*[lab={\ruleName{$\eval{}$-annot}}]
   {
      \rho, r
      \eval{}
      T, \annot{v}{\alpha'}
   }
   {
      \rho,
      \annot{r}{\alpha}
      \eval{}
      T, \annot{v}{\alpha \wedge \alpha'}
   }
\end{smathpar}
\\
\flushleft \shadebox{$\rho, r \eval{} T, u$}
\begin{smathpar}
   \inferrule*[
      lab={\ruleName{$\eval{}$-var}},
      right={\textnormal{$\exThunkVar{x}{u} \in \rho$}}
   ]
   {
   }
   {
      \rho, \exVar{x} \eval{} \trVarTwo{x}{\rho}, u
   }
   %
   \and
   %
   \inferrule*[lab={\ruleName{$\eval{}$-const}}]
   {
   }
   {
      \rho,
      \exConst{k}
      \eval{}
      \sub{\trEmpty}{\rho}, \exConst{k}
   }
   %
   \and
   %
   \inferrule*[lab={\ruleName{$\eval{}$-fun}}]
   {
   }
   {
      \rho,
      \exFun{\sigma}
      \eval{}
      \sub{\trEmpty}{\rho}, \exClosureNew{\rho}{\seqEmpty}{\exFun{\sigma}}
   }
   %
   \and
   %
   \inferrule*[lab={\ruleName{$\eval{}$-app}}]
   {
      \rho, e \eval{} T, \annot{\exClosureNew{\rho_1}{\delta}{\exFun{\sigma}}}{\alpha}
      \\
      \rho_1, \delta \closeDefs \rho_2
      \\
      \rho, e' \eval{} T', u
      \\\\
      u, \sigma \lookupR \rho_3, \matchPlug{\xi}{e^\twoPrime}, \alpha'
      \\
      \rho_1 \concat \rho_2 \concat \rho_3, e^\twoPrime
      \eval{}
      U, \annot{v}{\alpha^\twoPrime}
   }
   {
      \rho,
      \exApp{e}{e'}
      \eval{}
      \trApp{T}{T'}{\matchPlug{\xi}{U}},
      \annot{v}{\alpha \wedge \alpha' \wedge \alpha^\twoPrime}
   }
   %
   \and
   %
   \inferrule*[
      lab={\ruleName{$\eval{}$-app-unary}},
      right={\textnormal{$k \in \dom{\interpret{\phi}}$}}
   ]
   {
      \rho, e \eval{} T, \annot{\phi}{\alpha}
      \\
      \rho, e' \eval{} T', \annot{k}{\alpha'}
   }
   {
      \rho,
      \exApp{e}{e'}
      \eval{}
      \trUnaryApp{(\explVal{T}{\phi})}{(\explVal{T'}{\exConst{k}})},
      \annot{\interpret{\phi}(k)}{\alpha \wedge \alpha'}
   }
   %
   \and
   %
   \inferrule*[
      lab={\ruleName{$\eval{}$-app-binary}},
      right={\textnormal{$(k, k') \in \dom{\interpret{\primOp}}$
      }}
   ]
   {
      \rho, e \eval{} T, \annot{k}{\alpha'}
      \\
      \rho, e' \eval{} T', \annot{k'}{\smash{\alpha^\twoPrime}}
   }
   {
      \rho,
      \exPrimApp{e}{\annot{\primOp}{\alpha}}{e'}
      \eval{}
      \trPrimApp{(\explVal{T}{k})}{\primOp}{(\explVal{T'}{k'})},
      \annot{\interpret{\primOp}(k, k')}{\smash{\alpha \wedge \alpha' \wedge \alpha^\twoPrime}}
   }
   %
   \and
   %
   \inferrule*[lab={\ruleName{$\eval{}$-constr}}]
   {
      \rho,
      \vec{e}
      \eval{}
      \vec{T}, \vec{u}
   }
   {
      \rho,
      \exConstr{c}{\vec{e}}
      \eval{}
      \trConstr{c}{\vec{T}}, \exConstr{c}{\vec{u}}
   }
   %
   \and
   %
   \inferrule*[lab={\ruleName{$\eval{}$-match}}]
   {
      \rho, e \eval{} T, u
      \\
      u, \sigma \lookupR \rho_1, \matchProd{\xi}{e'}, \alpha
      \\
      \rho \concat \rho_1, e' \eval{} T', \annot{v}{\alpha'}
   }
   {
      \rho,
      \exMatch{e}{\sigma}
      \eval{}
      \trMatch{T}{\matchPlug{\xi}{T'}}, \annot{v}{\alpha \wedge \alpha'}
   }
   %
   \and
   %
   \inferrule*[lab={\ruleName{$\eval{}$-defs}}]
   {
      \rho, \vec{d} \closeDefs \rho_1, \vec{z}
      \\
      \rho \concat \rho_1, e \eval{} T, u
   }
   {
      \rho, \exDefs{\vec{d}}{e}
      \eval{}
      \trDefs{\vec{z}}{T}, u
   }
\end{smathpar}
\vspace{5pt}

\flushleft \shadebox{$\rho, \vec{e} \eval{} \vec{T}, \vec{u}$}
\begin{smathpar}
   \inferrule*
   {
      \strut
   }
   {
      \rho,
      \seqEmpty
      \eval{}
      \sub{\seqEmpty}{\rho},
      \seqEmpty
   }
   \and
   %
   \inferrule*
   {
      \rho, e \eval{} T, u
      \\
      \rho, \vec{e} \eval{} \vec{T}, \vec{u}
   }
   {
      \rho,
      e \concat \vec{e}
      \eval{}
      T \concat \vec{T},
      u \concat \vec{u}
   }
\end{smathpar}
\caption{Evaluation for terms and term sequences}
\end{figure}

\begin{figure}
\flushleft \shadebox{$T, u \uneval \rho, e$}
\mprset{flushleft}
\begin{smathpar}
   \inferrule*[lab={\ruleName{$\uneval$-var}}]
   {
      \strut
   }
   {
      \trVarTwo{x}{\rho}, \annot{v}{\alpha}
      \uneval
      \envExtend{\sub{\bot}{\rho}}{x}{\annot{v}{\alpha}},
      \annot{\exVar{x}}{\alpha}
   }
   %
   \and
   %
   \inferrule*[lab={\ruleName{$\uneval$-const}}]
   {
      \strut
   }
   {
      \sub{\trEmpty}{\rho}, \annot{\exConst{k}}{\alpha}
      \uneval
      \sub{\bot}{\rho},
      \annot{\exConst{k}}{\alpha}
   }
   %
   \and
   %
   \inferrule*[lab={\ruleName{$\uneval$-fun}}]
   {
      \strut
   }
   {
      \sub{\trEmpty}{\rho}, \annot{\exClosureNew{\rho'}{\seqEmpty}{\exFun{\sigma}}}{\alpha}
      \uneval
      \rho',
      \annot{(\exFun{\sigma})}{\alpha}
   }
   %
   \and
   %
   \inferrule*[lab={\ruleName{$\uneval$-app}}]
   {
      U, \annot{v}{\alpha} \uneval \rho_1' \concat \rho_2 \concat \rho_3, e^\twoPrime
      \\
      \rho_3, \matchPlug{\xi}{e^\twoPrime}, \alpha \unlookupR{} u, \sigma
      \\\\
      T', u \uneval \rho, e'
      \\
      \rho_2 \uncloseDefs \rho_1^\twoPrime, \delta'
      \\
      T, \annot{\exClosureNew{\rho_1' \vee \rho_1^\twoPrime}{\delta'}{\exFun{\sigma}}}{\alpha}
      \uneval
      \rho', e
   }
   {
      \trApp{T}{T'}{\matchPlug{\xi}{U}}, \annot{v}{\alpha}
      \uneval
      \rho \vee \rho',
      \annot{(\exApp{e}{e'})}{\alpha}
   }
   %
   \and
   %
   \inferrule*[
      lab={\ruleName{$\uneval$-app-unary}}
   ]
   {
      T', \annot{c}{\alpha} \uneval \rho', e'
      \\
      T, \annot{\phi}{\alpha} \uneval \rho, e
   }
   {
      \trApp{(\explVal{T}{\phi})}{(\explVal{T'}{c})}{\sub{\trEmpty}{\envEmpty}},
      \annot{v}{\alpha}
      \uneval
      \rho \vee \rho',
      \annot{(\exApp{e}{e'})}{\alpha}
   }
   %
   \and
   %
   \inferrule*[
      lab={\ruleName{$\uneval$-app-binary}}
   ]
   {
      T', \annot{c'}{\alpha} \uneval \rho', e'
      \\
      T, \annot{c}{\alpha} \uneval \rho, e
   }
   {
      \trPrimApp{(\explVal{T}{c})}{\primOp}{(\explVal{T'}{c'})},
      \annot{v}{\alpha}
      \uneval
      \rho \vee \rho',
      \annot{(\exPrimApp{e}{\annot{\primOp}{\alpha}}{e'})}{\alpha}
   }
   %
   \and
   %
   \inferrule*[lab={\ruleName{$\uneval$-constr}}]
   {
      \vec{T}, \vec{u} \uneval \rho', \vec{e}
   }
   {
      \trConstr{c}{\vec{T}}, \annot{\exConstr{c}{\vec{u}}}{\alpha}
      \uneval
      \rho',
      \annot{\exConstr{c}{\vec{e}}}{\alpha}
   }
   %
   \and
   %
   \inferrule*[lab={\ruleName{$\uneval$-match}}]
   {
      T', \annot{v}{\alpha} \uneval \rho' \concat \rho_1, e'
      \\
      \rho_1, \matchPlug{\xi}{e'}, \alpha \unlookupR{} u, \sigma
      \\
      T, u \uneval \rho, e
   }
   {
      \trMatch{T}{\matchPlug{\xi}{T'}}, \annot{v}{\alpha}
      \uneval
      \rho \vee \rho',
      \annot{(\exMatch{e}{\sigma})}{\alpha}
   }
   %
   \and
   %
   \inferrule*[lab={\ruleName{$\uneval$-let}}]
   {
      U, \annot{v}{\alpha} \uneval \envExtend{\rho}{x}{u'}, e'
      \\
      T, u' \uneval \rho', e
   }
   {
      \trLet{x}{T}{U}, \annot{v}{\alpha}
      \uneval
      \rho \vee \rho', \annot{(\exLet{x}{e}{e'})}{\alpha}
   }
   %
   \and
   %
   \inferrule*[lab={\ruleName{$\uneval$-letrec}}]
   {
      T, \annot{v}{\alpha} \uneval \rho \concat \rho_1, e
      \\
      \rho_1 \uncloseDefs \rho', \delta'
   }
   {
      \trLetrec{\delta}{T}, \annot{v}{\alpha}
      \uneval
      \rho \vee \rho', \annot{(\exLetrec{\delta'}{e})}{\alpha}
   }
\end{smathpar}
\\[2mm]
\flushleft \shadebox{$\vec{T}, \vec{u} \uneval \rho, \vec{e}$}
\begin{smathpar}
   \inferrule*
   {
      \strut
   }
   {
      \annot{\seqEmpty}{\rho}, \seqEmpty
      \uneval
      \sub{\bot}{\rho},
      \seqEmpty
   }
   \and
   %
   \inferrule*
   {
      \vec{T}, \vec{u} \uneval \rho_1, \vec{e}
      \\
      T, u \uneval \rho_1, e
   }
   {
      T \concat \vec{T},
      u \concat \vec{u}
      \uneval
      \rho_1 \vee \rho_2,
      e \concat \vec{e}
   }
\end{smathpar}
\caption{Reverse evaluation for terms and term sequences}
\end{figure}


% \begin{figure}
\begin{syntaxfig}
\mbox{Match}
&
\xi
&
::=
&
\matchVar{x}{\matchHole{}}
&
\text{variable}
\\
&&&
\matchUnit{\matchHole{}}
&
\text{unit}
\\
&&&
\matchSum{\xi}{\sigma}
&
\text{inject left}
\\
&&&
\matchSumLL{\xi}
&
\\
&&&
\matchSumRR{\xi}
&
\\
&&&
\matchSum{\sigma}{\xi}
&
\text{inject right}
\\
&&&
\matchProd{\xi}{\xi'}
&
\text{product}
\\
&&&
\matchRoll{\xi}
&
\text{roll}
\end{syntaxfig}
\caption{Syntax of matches}
\end{figure}

% \begin{figure}
\flushleft \shadebox{$\matchType{\Gamma}{\matchPlug{\xi}{\kappa}}{K}{A}{\Gamma'}$}
\begin{smathpar}
\inferrule*[right={$\kappa \in K_{\Gamma \concat \Gamma'}$}]
{
   \matchType{\Gamma}{\xi}{K}{A}{\Gamma'}
}
{
   \matchType{\Gamma}{\matchPlug{\xi}{\kappa}}{K}{A}{\Gamma'}
}
\end{smathpar}
\\[3mm]
\flushleft \shadebox{$\matchType{\Gamma}{\xi}{K}{A}{\Gamma'}$}
\begin{smathpar}
\inferrule*
{
   \strut
}
{
   \matchType{\Gamma}{\matchVar{x}{\matchHole{}}}{K}{A}{\cxtExtend{\cxtEmpty}{x}{A}}
}
%
\and
%
\inferrule*
{
   \strut
}
{
   \matchType{\Gamma}{\matchUnit{\matchHole{}}}{K}{\tyUnit}{\cxtEmpty}
}
%
\and
%
\inferrule*
{
   \matchType{\Gamma}{\xi}{K}{A}{\Gamma'}
   \\
   \trieType{\sigma}{K}{B}{\Gamma}
}
{
   \matchType{\Gamma}{\matchSumL{\xi}{\sigma}}{K}{(\tySum{A}{B})}{\Gamma'}
}
%
\and
%
\inferrule*
{
   \trieType{\sigma}{K}{A}{\Gamma}
   \\
   \matchType{\Gamma}{\xi}{K}{B}{\Gamma'}
}
{
   \matchType{\Gamma}{\matchSumR{\sigma}{\xi}}{K}{(\tySum{A}{B})}{\Gamma'}
}
%
\and
%
\inferrule*
{
   \matchType{\Gamma}{\explVal{T}{\xi}}{K'}{A}{\Delta}
   \\
   \matchType{\Gamma \concat \Delta}{\explVal{U}{\xi'}}{K}{B}{\Gamma'}
}
{
   \matchType{\Gamma}{\matchProd{\xi}{\xi'}}{K}{(\tyProd{A}{B})}{\Gamma'}
}
%
\and
%
\inferrule*
{
   \matchType{\Gamma}{\xi}{K}{\subst{A}{\tyRec{\alpha}{A}}{\alpha}}{\Gamma'}
}
{
   \matchType{\Gamma}{\matchRoll{\xi}}{K}{(\tyRec{\alpha}{A})}{\Gamma'}
}
\end{smathpar}
\vspace{5pt}

\flushleft \shadebox{$\explMatchType{\Gamma}{\explVal{T}{\xi}}{K}{A}{\Gamma'}$}
\begin{smathpar}
\inferrule*
{
   \explType{\Gamma}{T}{A}
   \\
   \matchType{\Gamma}{\xi}{T}{A}{\Gamma'}
}
{
   \explMatchType{\Gamma}{\explVal{T}{\xi}}{K}{A}{\Gamma'}
}
\end{smathpar}
\caption{Typing rules for matches}
\end{figure}

% \begin{figure}
{\small
\begin{align*}
   \fmapTwo{f}{g}{\matchedInj{\sigma}}
   &=
   \matchedInj{\fmap{g}{\sigma}}
   \\
   \fmapTwo{f}{g}{(\matchedVar{x}{\kappa})}
   &=
   \matchedVar{x}{f\kappa}
   \\
   \fmapTwo{f}{g}{(\matchedUnit{\kappa})}
   &=
   \matchedUnit{f\kappa}
   \\
   \fmapTwo{f}{g}{\matchedSumL{\explVal{T}{\xi}}{\sigma}}
   &=
   \matchedSumL{\explVal{T}{\fmapTwo{f}{g}{\xi}}}{\fmap{g}{\sigma}}
   \\
   \fmapTwo{f}{g}{\matchedSumR{\sigma}{\explVal{T}{\xi}}}
   &=
   \matchedSumR{\fmap{g}{\sigma}}{\explVal{T}{\fmapTwo{f}{g}{\xi}}}
   \\
   \fmapTwo{f}{g}{\matchedProd{\explVal{T}{\xi}}}
   &=
   \matchedProd{\explVal{T}{\fmapTwo{(\fmapTwoPartial{f}{g})}{(\fmapTwoPartial{g}{g})}{\xi}}}
   \\
   \fmapTwo{f}{g}{(\matchedRoll{\explVal{T}{\xi}})}
   &=
   \matchedRoll{\explVal{T}{\fmapTwo{f}{g}{\xi}}}
   \\[2mm]
   \fmapTwo{f}{g}{(\explVal{T}{\xi})}
   &=
   \explVal{T}{\fmapTwo{f}{g}{\xi}}
\end{align*}}
\caption{Mapping families of functions $f_{\Gamma}, g_{\Gamma}: K_{\Gamma} \to K'_{\Gamma}$ to an (explained) match}
\end{figure}

% \begin{figure}
\begin{align*}
\exprTr{\exVar{x}}
&=
\explVal{\trVar{x}{}{\trBot}}{\exBot}
\\
\exprTr{\exConst{c}}
&=
\explVal{\trEmpty}{\exConst{c}}
\\
\exprTr{\exConst{\phi}}
&=
\explVal{\trEmpty}{\exConst{\phi}}
\\
\exprTr{\exFun{\sigma}}
&=
\explVal{\trEmpty}{\exClosure{\envBot}{\exFun{\exprTr{\sigma}}}}
\\
\exprTr{\exApp{e}{e'}}
&=
\explVal{\trApp{\exprTr{e}}{\exprTr{e'}}{\trBot}}{\exBot}
\\
\exprTr{\exPrimApp{e_1}{\primOp}{e_2}}
&=
\explVal{\trPrimApp{\exprTr{e_1}}{\primOp}{\exprTr{e_2}}}{\exBot}
\\
\exprTr{\exConstr{c}{\vec{e}}}
&=
\explVal{\trEmpty}{\exConstr{c}{\exprTr{\vec{e}}}}
\\
\exprTr{\exMatch{e}{\sigma}}
&=
\explVal{\trMatch{\exprTr{e}}{\exprTr{\sigma}}{}{\exBot}}{\exBot}
\\
\exprTr{\exLetrec{\delta}{e}}
&=
\explVal{\trLetrec{\exprTr{\delta}}{T}}{\exBot}
\text{ where }
\explVal{T}{v} = \exprTr{e}
\end{align*}
\caption{Embedding of expressions into explained values}
\end{figure}


% \begin{figure}
\flushleft \shadebox{$\rho, \indexed{e} \eval{\sigma} \explVal{\indexed{U}}{\indexed{u}}, \rho', \kappa$}
\begin{smathpar}
\inferrule*
{
   \strut
}
{
   \rho,
   \indexed{e}
   \eval{\elimVar{x}{\kappa}}
   \explVal{\indexed{e}}{\exBot},
   \envExtend{\envEmpty}{x}{\recEnvEntry{\rho}{\envEmpty}{\indexed{e}}},
   \kappa
}
%
\and
%
\inferrule*[right={$
   \exThunkVar{x}{\recEnvEntry{\rho'}{\delta}{\indexed{e}}} \in \rho \wedge
   \envType{\Gamma}{\rho' \concat \closeDefs{\delta}{\rho'}}
$}]
{
   \rho' \concat \closeDefs{\delta}{\rho'},
   \shift{\closeDefs{\delta}{\rho'}}{\indexed{e}}
   \eval{\tau}
   \explVal{\indexed{T}}{\indexed{v}},
   \rho^\twoPrime,
   \kappa
}
{
   \rho,
   \raw{\exVar{x}}{\idx{i}}
   \eval{\tau}
   \explVal{\raw{(\trVar{x}{\Gamma}{\indexed{T}})}{\idx{i}}}{\indexed{v}},
   \rho^\twoPrime,
   \kappa
}
%
\and
%
\inferrule*[right={$\interpret{\exConst{c}} \in \tyGround{C}$}]
{
   \strut
}
{
   \rho,
   \raw{\exConst{c}}{\idx{i}}
   \eval{\elimGround{C}{\kappa}}
   \explVal
      {\raw{\trEmpty}{\idx{i}}}
      {\raw{\exConst{c}}{\idx{i}}},
   \envEmpty,
   \kappa
}
%
\and
%
\inferrule*
{
   \strut
}
{
   \rho,
   \raw{\phi}{\idx{i}}
   \eval{\elimFun{\kappa}}
   \explVal
      {\raw{\trEmpty}{\idx{i}}}
      {\raw{\phi}{\idx{i}}},
   \envEmpty,
   \kappa
}
%
\and
%
\inferrule*
{
   \strut
}
{
   \rho,
   \raw{\exFun{\sigma}}{\idx{i}}
   \eval{\elimFun{\kappa}}
   \explVal
      {\raw{\trEmpty}{\idx{i}}}
      {\raw{\exClosure{\rho}{\exFun{\sigma}}}{\idx{i}}},
   \envEmpty,
   \kappa
}
%
\and
%
   \inferrule*[right={$\envType{\Gamma}{\rho_1 \concat \rho_2}$}]
   {
      \rho, \indexed{e} \eval{\elimFun{\metaunit}}
      \explVal{\indexed{T}}{\raw{\exClosure{\rho_1}{\exFun{\sigma}}}{\idx{j}}}, \envEmpty, \metaunit
      \\
      \rho, \indexed{e}' \eval{\sigma} \explVal{\indexed{T}'}{\indexed{u}}, \rho_2, \indexed{e}^\twoPrime
      \\
      \rho_1 \concat \rho_2,
      \shift{\rho_2}{\indexed{e}^\twoPrime}
      \eval{\tau}
      \explVal{\indexed{T}^\twoPrime}{\indexed{v}}, \rho', \kappa
   }
   {
      \rho,
      \raw{(\exApp{\indexed{e}}{\indexed{e}'})}{\idx{i}}
      \eval{\tau}
      \explVal
         {\raw
            {\smash{(\trApp{(\explVal{\indexed{T}}{\raw{\exClosure{\rho_1}{\exFun{\sigma}}}{\idx{j}}})}
                           {(\explVal{\indexed{T'}}{\indexed{u}})}
                           {\indexed{T^\twoPrime}})}}
            {\idx{i}}}
         {\indexed{v}},
      \rho',
      \kappa
   }
%
\and
%
\inferrule*[right={\textnormal{
   $\interpret{\phi} \in \UnaryOp{C}{A} \wedge \interpret{\phi}(c, \tau) = (v, \kappa)$
}}]
{
   \rho, \indexed{e} \eval{\elimFun{\metaunit}} \explVal{\indexed{T}}{\raw{\phi}{\idx{j}}}, \envEmpty, \metaunit
   \\
   \rho, \indexed{e}' \eval{\elimGround{C}{\metaunit}} \explVal{\indexed{T}'}{\raw{c}{\idx{k}}}, \envEmpty, \metaunit
}
{
   \rho,
   \raw{(\exApp{\indexed{e}}{\indexed{e}'})}{\idx{i}}
   \eval{\tau}
   \explVal
      {\raw{\smash{(\trApp{(\explVal{\indexed{T}}{\raw{\phi}{\idx{j}}})}
                          {(\explVal{\indexed{T}'}{\raw{c}{\idx{k}}})}
                          {\trBot})}}
           {\idx{i}}}
      {\raw{v}{\idx{i}}},
   \envEmpty,
   \kappa
}
%
\and
%
\inferrule*[right={\textnormal{
   $\interpret{\primOp} \in \BinaryOp{C}{C'}{A} \wedge \interpret{\primOp}(c, c', \tau) = (v, \kappa)$
}}]
{
   \rho, e \eval{\elimGround{C}{\metaunit}} \explVal{\indexed{T}}{\raw{c}{\idx{j}}}, \envEmpty, \metaunit
   \\
   \rho, e' \eval{\elimGround{C'}{\metaunit}} \explVal{\indexed{T}'}{\raw{\smash{c'}}{\idx{k}}}, \envEmpty, \metaunit
}
{
   \rho,
   \raw{(\exPrimApp{\indexed{e}}{\primOp}{\indexed{e}'})}{\idx{i}}
   \eval{\tau}
   \explVal
      {\raw{\smash{\trPrimApp{(\explVal{\indexed{T}}{\raw{c}{\idx{j}}})}
                             {\primOp}
                             {(\explVal{\indexed{T}'}{\raw{\smash{c'}}{\idx{k}}})}}}{\idx{i}}}
      {\raw{v}{\idx{i}}},
   \envEmpty,
   \kappa
}
%
\and
%
\inferrule*
{
   \rho,
   \vec{\indexed{e}}
   \eval{\Sigma(d)}
   \vec{\explVal{\indexed{T}}{\indexed{u}}},
   \rho',
   \kappa
}
{
   \rho,
   \raw{\exConstr{d}{\vec{\indexed{e}}}}{\idx{i}}
   \eval{\Sigma}
   \explVal
      {\raw{\trEmpty}{\idx{i}}}
      {\raw{\exConstr{d}{\vec{\explVal{\indexed{T}}{\indexed{u}}}}}{\idx{i}}},
   \rho',
   \kappa
}
%
\and
%
\inferrule*
{
   \rho, \indexed{e} \eval{\sigma} \explVal{\indexed{U}}{\indexed{u}}, \rho', \indexed{e}'
   \\
   \rho \concat \rho',
   \shift{\rho'}{\indexed{e}'}
   \eval{\tau}
   \explVal{\indexed{T}}{\indexed{v}},
   \rho^\twoPrime,
   \kappa
}
{
   \rho,
   \raw{(\exMatch{\indexed{e}}{\sigma})}{\idx{i}}
   \eval{\tau}
   \explVal
      {\raw{(\trMatch{\explVal{\indexed{U}}{u}}{\sigma}{\Gamma}{\indexed{T}})}{\idx{i}}}
      {\indexed{v}},
   \rho^\twoPrime,
   \kappa
}
%
\and
%
\inferrule*
{
   \rho \concat \closeDefs{\delta}{\rho},
   \shift{\closeDefs{\delta}{\rho}}{\indexed{e}}
   \eval{\tau}
   \explVal{\indexed{T}}{\indexed{v}},
   \rho',
   \kappa
}
{
   \rho,
   \raw{(\exLetrec{\delta}{\indexed{e}})}{\idx{i}}
   \eval{\tau}
   \explVal
      {\raw{(\trLetrec{\delta}{\indexed{T}})}{\idx{i}}}
      {\indexed{v}},
   \rho',
   \kappa
}
\end{smathpar}
\caption{Evaluation for persistent terms}
\end{figure}


\subsection{\ttt{match}...\ttt{as}}
\label{sec:impl-language:match-as}

Pattern-matching expressions, which are essentially generalised $\kw{let}$
forms, are easily added as syntactic sugar for a beta-redex. Specifically,
$\exMatch{e}{\sigma}$ desugars to $\exApp{\exFun{\sigma'}}{e'}$ where $\sigma'$
and $e'$ are the desugaring of $\sigma$ and $e$. Alternatively, the rules shown
are derivable and give the same static and dynamic semantics as the desugaring.
Similarly, $\exLet{x}{e}{e'}$ can also be expressed as a desugaring to
$\exMatch{e}{\elimVar{x}{e^\twoPrime}}$ where $e^\twoPrime$ is the desugaring of
$e'$.

\subsection{Mutual recursion}
\label{sec:impl-language:recursion}

\begin{figure}
\flushleft \shadebox{$\rho, \vec{d} \closeDefs \rho', \vec{z}$}
\begin{smathpar}
\inferrule*
{
   \strut
}
{
   \rho, \seqEmpty \closeDefs \envEmpty, \seqEmpty
}
%
\and
%
\inferrule*
{
   \rho, d \closeDefs \rho_1, z
   \\
   \rho \concat \rho_1, \vec{d} \closeDefs \rho_2, \vec{z}
}
{
   \rho, d \concat \vec{d}
   \closeDefs
   \rho_1 \concat \rho_2, z \concat \vec{z}
}
\end{smathpar}
\flushleft \shadebox{$\rho, \vec{z} \uncloseDefs \rho', \vec{d}$}
\begin{smathpar}
\inferrule*
{
   \strut
}
{
   \rho, \seqEmpty \uncloseDefs \envEmpty, \seqEmpty
}
%
\and
%
\inferrule*
{
   \rho_2, \vec{z} \uncloseDefs \rho \concat \rho_1', \vec{d}
   \\
   \rho_1', z \uncloseDefs \rho^\dagger, d
}
{
   \rho_1 \concat \rho_2, z \concat \vec{z}
   \uncloseDefs
   \rho \vee \rho^\dagger,
   d \concat \vec{d}
}
\end{smathpar}
\flushleft \shadebox{$\rho, d \closeDefs \rho', z$}
\begin{smathpar}
   \inferrule*
   {
      \rho, e \eval{} T, u
   }
   {
      \rho, \exLetDef{x}{e}
      \closeDefs
      \envExtend{\envEmpty}{x}{u}, \trLetDef{x}{T}
   }
   %
   \and
   %
   \inferrule*
   {
      \strut
   }
   {
      \rho, \exLetrecDef{\delta}
      \closeDefs
      \set{
         f \mapsto \annot{\exClosureNew{\rho}{\delta}{\exFun{\sigma}}}{\alpha}
         \mid
         f \mapsto \annot{(\exFun{\sigma})}{\alpha} \in \delta
      },
      \trLetrecDef{\delta}
   }
   %
   \and
   %
   \inferrule*
   {
      \strut
   }
   {
      \rho, \exPrimDef{x}
      \closeDefs
      \envExtend{\envEmpty}{x}{\annot{\phi}{\top}}, \trPrimDef{\sub{x}{\rho}}
   }
\end{smathpar}
\flushleft \shadebox{$\rho, d \uncloseDefs \rho', z$}
\begin{smathpar}
   \inferrule*
   {
      T, u \uneval \rho, e
   }
   {
      \envExtend{\envEmpty}{x}{u},
      \trLetDef{x}{T}
      \uncloseDefs
      \rho, \exLetDef{x}{e}
   }
   %
   \and
   %
   \inferrule*
   {
      \strut
   }
   {
      \begin{aligned}
         \set{
            \annot{\exClosureNew{\rho_f}{\delta_f}{\exFun{\sigma_f}}}{\alpha_f}
         }_{f \in F},
         \exLetrecDef{\delta'}
         &\uncloseDefs
         \bigvee_{f \in F}\rho_f,
         \exLetrecDef{
            \set{f \mapsto \annot{(\exFun{\sigma_f})}{\alpha_f} \mid f \in F} \vee \bigvee_{f \in F}\delta_f
         }
      \end{aligned}
   }
   %
   \and
   %
   \inferrule*
   {
      \strut
   }
   {
      \envExtend{\envEmpty}{x}{\annot{\phi}{\alpha}},
      \trPrimDef{\sub{x}{\rho}}
      \uncloseDefs
      \sub{\bot}{\rho}, \exPrimDef{x}
   }
\end{smathpar}
\caption{Bindings for a sequence of definitions}
\label{fig:impl-language:closedefs}
\end{figure}


\begin{definition}
\label{def:closedefs}
   \figref{impl-language:closedefs} defines the deterministic relation $\closeDefs$.
\end{definition}

\begin{definition}
\label{def:unclosedefs}
   \figref{impl-language:closedefs} defines the deterministic relation $\uncloseDefs$.
\end{definition}

\begin{lemma}
\label{lem:gc-closedefs}

Suppose $\rho, \exLetrecDef{\delta} \closeDefs \rho', \exLetrecDef{\delta}$
where $\rho$, $\delta$ and $\rho'$ are unannotated, and write $\closeDefs_{\rho,
\delta}$ for $\closeDefs$ domain-restricted to $\Ann{\rho,
\exLetrecDef{\delta}}$ and $\uncloseDefs_{\rho, \delta}$ for $\uncloseDefs$
domain-restricted to $\Ann{\rho',\exLetrecDef{\delta}}$. Then $\closeDefs_{\rho,
\delta}$ and $\uncloseDefs_{\rho, \delta}$ form a Galois injection:

\[
   {\closeDefs_{\rho, \delta}} \adjoint {\uncloseDefs_{\rho, \delta}}:
   \Ann{\rho, \exLetrecDef{\delta}} \inj \Ann{\rho',\exLetrecDef{\delta}}
\]

\end{lemma}

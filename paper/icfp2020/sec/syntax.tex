\section{Core calculus}

\subsection{Syntax}

\figref{demand-indexed:syntax} gives the syntax of types, contexts, tries, and
terms. Types are conventional sums of products with recursion. Terms are
parameterised over a lattice $\mathcal{A} = \langle \mathcal{A}, \top, \bot,
\wedge, \vee\rangle$; we write $\leq$ for the partial order of $\mathcal{A}$.
The lattice $\mathcal{A}$ extends to the set $\Ann{e}$ of all possible
annotations of the (unannotated) term $e$ by defining $\top_e$, $\bot_e$,
$\wedge_e$ and $\vee_e$ to apply the corresponding operation of $\mathcal{A}$ at
every subterm of $e$. For example $\annot{(\exInl{e'})}{\alpha}
\wedge_{\exInl{e}} \annot{(\exInl{e^\twoPrime})}{\alpha'}$, where $e, e' \in
\Ann{e}$, is defined recursively as $\annot{(\exInl{(e' \wedge_e
e^\twoPrime)})}{\alpha \wedge \alpha'}$. The subscripts on the lattice
operations for $\Ann{e}$, shown here for clarity, are usually omitted.

\begin{figure}
\begin{syntaxfig}
\mbox{Eliminator}
&
\sigma, \tau
&
::=
&
\ldots
\\
&&&
\elimBoolTrue{\kappa}
\\
&&&
\elimBoolFalse{\kappa}
\\
&&&
\elimListSingleton{\branchNil{\kappa}}
\\
&&&
\elimListSingleton{\branchCons{\sigma}}
\\[2mm]
\mbox{Raw term}
&
r
&
::=
&
\ldots
\\
&&&
\exLambda{\sigma}
&
\text{anonymous function}
\end{syntaxfig}
\caption{Additional syntax}
\end{figure}


Tries \cite{hinze00} and terms are somewhat non-standard and are best explained
with reference to the typing rules in
\figrefTwo{demand-indexed:typing-trie}{demand-indexed:typing-term}. We start
with tries, which are a particular syntactic representation of functions over
values. In the judgement $\trieType{\sigma}{K}{A}{\Gamma}$, the domain of
$\sigma$ is the set of values of type $A$, and the codomain or
\emph{continuation type} $K$ is a function from contexts to sets, written either
$A$, denoting the function $\Gamma \mapsto \exprTypeJudge{\Gamma}{A}$, or
$\trieTypePartial{K}{A}$, denoting the function $\Gamma \mapsto
\trieTypeJudge{K}{A}{\Gamma}$. If $v$ is in the domain of $\sigma$, then
$\lookup{\sigma}{v}$ is an element of $K_{\Gamma\concat\Delta}$, where $\Delta$
extends $\Gamma$ with additional bindings for the parts of the value which were
matched by variables in the trie. The $\trieArrow$ arrow associates to the
right.

\begin{figure}
\flushleft \shadebox{$\trieType{\sigma}{K}{A}{\Gamma}$}
\begin{smathpar}
\inferrule*[right={$\kappa \in K_{\cxtExtend{\Gamma}{x}{A}}$}]
{
   \strut
}
{
   \trieType{\trieVar{x}{\kappa}}{K}{A}{\Gamma}
}
%
\and
%
\inferrule*[right={\textnormal{$\dom{\interpret{D}} = \dom{\Sigma}$}}]
{
   \trieType{\Sigma(d)}{K}{\interpret{D}(d)}{\Gamma}
   \\
   (\forall d \in \dom{\interpret{D}})
}
{
   \trieType{\trieConstr{\Sigma}}{K}{\tyData{D}}{\Gamma}
}
\end{smathpar}
\vspace{5pt}
\flushleft \shadebox{$\trieType{\kappa}{K}{\vec{A}}{\Gamma}$}
\begin{smathpar}
\inferrule*[right={$\kappa \in K_{\Gamma}$}]
{
   \strut
}
{
   \trieType{\kappa}{K}{\seqEmpty}{\Gamma}
}
%
\and
%
\inferrule*
{
   \trieType{\sigma}{\trieTypePartial{K}{\vec{A}}}{A}{\Gamma}
}
{
   \trieType{\sigma}{K}{A \concat \vec{A}}{\Gamma}
}
\end{smathpar}
\caption{Typing rules for tries and trie products}
\end{figure}

\begin{figure}
\flushleft \shadebox{$\Gamma \vdash e: A$}
\begin{smathpar}
\inferrule*[right={$x : A \in \Gamma$}]
{
   \strut
}
{
   \Gamma \vdash \exVar{x}: A
}
%
\and
%
\inferrule*[right={$\primOp: A \in \Gamma$}]
{
   \strut
}
{
   \Gamma \vdash \exOp{\primOp}: A
}
%
\and
%
\inferrule*
{
   \strut
}
{
   \Gamma \vdash n: \tyInt
}
%
\and
%
\inferrule*
{
   \strut
}
{
   \Gamma \vdash \exTrue: \tyBool
}
%
\and
%
\inferrule*
{
   \strut
}
{
   \Gamma \vdash \exFalse: \tyBool
}
%
\and
%
\inferrule*
{
   \Gamma \vdash u: A
   \\
   \Gamma \vdash v: B
}
{
   \Gamma \vdash \exPair{u}{v}: \tyProd{A}{B}
}
%
\and
%
\inferrule*[right={$\primOp: \tyFun{\tyInt}{\tyFun{\tyInt}{\tyInt}} \in \Gamma$}]
{
   \Gamma \vdash e: \tyInt
   \\
   \Gamma \vdash e': \tyInt
}
{
   \Gamma \vdash \exBinaryApp{e}{\primOp}{e'}: \tyInt
}
%
\and
%
\inferrule*
{
   \cxtExtend{\Gamma}{f}{\tyFun{A}{B}} \vdash \sigma: \tyFun{A}{B}
}
{
   \Gamma \vdash \exRec{f}{\sigma}: \tyFun{A}{B}
}
%
\and
%
\inferrule*
{
   \Gamma \vdash e: \tyFun{A}{B}
   \\
   \Gamma \vdash e': A
}
{
   \Gamma \vdash \exApp{e}{e'}: B
}
%
\and
%
\inferrule*
{
   \Gamma \vdash e: A
   \\
   \cxtExtend{\Gamma}{x}{A} \vdash e': B
}
{
   \Gamma \vdash \exLet{x}{e}{e'}: B
}
%
\and
%
\inferrule*
{
   \strut
}
{
   \Gamma \vdash \exNil: \tyList{\tyInt}
}
%
\and
%
\inferrule*
{
   \Gamma \vdash u: \tyInt
   \\
   \Gamma \vdash v: \tyList{\tyInt}
}
{
   \vdash \exCons{u}{v}: \tyList{\tyInt}
}
\end{smathpar}
\caption{Typing rules for terms}
\end{figure}


The variable trie $\trieVar{x}{\kappa}$ maps \emph{any} value of type $A$ to
$\kappa$, which is typed in a context which extends $\Gamma$ with a new binding
of type $A$. The unit trie $\trieUnit{\kappa}$ maps the single value of type
$\tyUnit$ to $\kappa$, without introducing a new variable.

Of the non-leaf cases, the trie $\trieRoll{\sigma}$ for a recursive type
$\tyRec{\tyVar{X}}{A}$, is the simplest: it simply injects $\sigma$, which is a
trie over the one-step unrolling of $\tyRec{\tyVar{X}}{A}$, into
$\tyRec{\tyVar{X}}{A}$. Sum tries $\trieSum{\sigma}{\tau}$ are slightly more
interesting, allowing a trie over $A$ and a trie over $B$ to be combined into a
trie over $\tySum{A}{B}$ as long they share a codomain. Product tries
$\trieProd{\sigma}$ rely on currying, allowing a trie over $A$ which returns a
trie over $B$ to be treated as a trie over $\tyProd{A}{B}$.

\subsection{Values and environments}

\begin{figure}
\begin{syntaxfig}
\mbox{Value}
&
u, v
&
::=
&
\exTrue \mid \exFalse
&
\text{Boolean}
\\
&&&
\exTrueSel \mid \exFalseSel
\\
&&&
\exInt{n}
\\
&&&
\exIntSel{n}
&
\text{integer}
\\
&&&
\exClosureRec{\rho}{f}{\sigma}
&
\text{closure}
\\
&&&
\exPair{u}{v}
&
\text{pair}
\\
&&&
\exPairSel{u}{v}
\\
&&&
\exNil
\\
&&&
\exNilSel
&
\text{nil}
\\
&&&
\exCons{u}{v}
&
\text{cons}
\\
&&&
\exConsSel{u}{v}
\\[2mm]
\mbox{Environment}
&
\rho
&
::=
&
\envEmpty
&
\text{empty}
\\
&&&
\envExtend{\rho}{x}{v}
&
\text{extend}
\end{syntaxfig}
\caption{Values and environments}
\end{figure}

\begin{figure}
\flushleft \shadebox{$\Gamma \vdash \vec{d}: \Delta$}
\begin{smathpar}
   \inferrule*
   {
      \strut
   }
   {
      \Gamma \vdash \seqEmpty: \cxtEmpty
   }
   %
   \and
   %
   \inferrule*
   {
      \Gamma \vdash d: \Delta
      \\
      \Gamma \concat \Delta \vdash \vec{d}: \Delta'
   }
   {
      \Gamma \vdash d \concat \vec{d}: \Delta \concat \Delta'
   }
\end{smathpar}
\vspace{5pt}

\flushleft \shadebox{$\Gamma \vdash d: \Delta$}
\begin{smathpar}
   \inferrule*
   {
      \exprType{e}{\Gamma}{A}
   }
   {
      \Gamma \vdash \exLetDef{x}{e}: \cxtExtend{\cxtEmpty}{x}{A}
   }
   %
   \and
   %
   \inferrule*[right={
      \textnormal{$\Delta = \set{f \mapsto A_f \mid f \in \dom{\delta}}$}
   }]
   {
      \exprType{\exFun{\sigma_f}}{\Gamma \concat \Delta}{A_f}
      \quad
      (\forall f \mapsto \exFun{\sigma_f} \in \delta)
   }
   {
      \Gamma \vdash \exLetrecDef{\delta}: \Delta
   }
   %
   \and
   %
   \inferrule*
   {
      \interpret{x} \in \interpret{A}
   }
   {
      \Gamma \vdash \exPrimDef{x}: \cxtExtend{\cxtEmpty}{x}{A}
   }
\end{smathpar}
\vspace{5pt}

\flushleft \shadebox{$\valExplTypeNew{u}{A}$}
\begin{smathpar}
\inferrule*[right={$\interpret{\exConst{c}} \in C$}]
{
   \strut
}
{
   \valExplTypeNew{\exConst{k}}{\tyGround{C}}
}
%
\and
%
\inferrule*[right={\textnormal{$\interpret{\phi} \in \interpret{C} \to \interpret{A}$}}]
{
   \strut
}
{
   \valExplTypeNew{\phi}{\tyFun{C}{A}}
}
%
\and
%
\inferrule*
{
   \envType{\Gamma}{\rho}
   \\
   \Gamma \vdash \exLetrecDef{\delta}: \Delta
   \\
   \exprType{\exFun{\sigma}}{\Gamma \concat \Delta}{\tyFun{A}{B}}
}
{
   \valExplTypeNew{\exClosureNew{\rho}{\delta}{\exFun{\sigma}}}{\tyFun{A}{B}}
}
%
\and
%
\inferrule*[right={\textnormal{$c \in \dom{\interpret{D}}$}}]
{
   \valExplTypeNew{\vec{v}}{\interpret{D}(c)}
}
{
   \valExplTypeNew{\exConstr{c}{\vec{v}}}{\tyData{D}}
}
\end{smathpar}
\vspace{5pt}

\flushleft \shadebox{$\envType{\Gamma}{\rho}$}
\begin{smathpar}
\inferrule*
{
   \strut
}
{
   \envType{\cxtEmpty}{\envEmpty}
}
%
\and
%
\inferrule*
{
   \envType{\Gamma}{\rho}
   \\
   \valExplType{\Gamma'}{u}{A}
}
{
   \envType{\cxtExtend{\Gamma}{x}{A}}{\envExtend{\rho}{x}{u}}
}
\end{smathpar}
\caption{Typing rules for partial values, environments and recursive definitions}
\end{figure}


The syntax of values and environments is given in
\figref{demand-indexed:syntax-value}. Concatenation of sequences is denoted by
the $\concat$ operator.

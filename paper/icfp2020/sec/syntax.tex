\section{Core calculus}

\subsection{Syntax}

\figref{demand-indexed:syntax} gives the syntax of types, contexts, tries, and
terms. Types are conventional sums of products with recursion. Terms are
parameterised over a lattice $\mathcal{A} = \langle \mathcal{A}, \top, \bot,
\wedge, \vee\rangle$; we write $\leq$ for the partial order of $\mathcal{A}$.
The lattice $\mathcal{A}$ extends to the set $\Ann{e}$ of all possible
annotations of the (unannotated) term $e$ by defining $\top_e$, $\bot_e$,
$\wedge_e$ and $\vee_e$ to apply the corresponding operation of $\mathcal{A}$ at
every subterm of $e$. For example $\annot{(\exInl{e'})}{\alpha}
\wedge_{\exInl{e}} \annot{(\exInl{e^\twoPrime})}{\alpha'}$, where $e, e' \in
\Ann{e}$, is defined recursively as $\annot{(\exInl{(e' \wedge_e
e^\twoPrime)})}{\alpha \wedge \alpha'}$. The subscripts on the lattice
operations for $\Ann{e}$, shown here for clarity, are usually omitted.

\begin{figure}
{\small
\begingroup
\renewcommand*{\arraystretch}{1}
\begin{minipage}[t]{0.5\textwidth}
\begin{tabularx}{\textwidth}{rL{2.3cm}L{3cm}}
%\rowcolor{verylightgray}
&\textbfit{Identifier}&
\\
$x, y ::=$
&
$\ldots$
&
\\
&
$\primOp$
&
operator name
\\[2mm]
&\textbfit{Surface term}&
\\
$s, t ::=$
&
$\ldots$
&
\\
&
$\exOp{\primOp}$
&
first-class operator
\\
&
$\exBinaryApp{s}{\primOp}{s'}$
&
binary application
\\
&
$\exLetRecMutual{\vec{g}}{s}$
&
recursive functions
\\
&
$\exIfThenElse{s}{s}{s}$
&
if
\\
&
$\exMatch{s}{\vec{\clauseUncurried{p}{s}}}$
&
match
\\
&
$\exLet{p}{s}{s}$
&
structured let
\\
&
$\annList{s}{r}{\alpha}$
&
non-empty list
\\
&
$\exListEnum{s}{s}$
&
list enum
\\
&
$\annListComp{s}{\vec{q}}{\alpha}$
&
list comprehension
\\[2mm]
&\textbfit{List rest term}&
\\
$r ::=$
&
$\annListEnd{\alpha}$
&
end
\\
&
$\annListNext{s}{r}{\alpha}$
&
cons
\\
\\
\end{tabularx}
\end{minipage}%
\begin{minipage}[t]{0.5\textwidth}
\begin{tabularx}{\textwidth}{rL{2.8cm}L{2.9cm}}
&\textbfit{Recursive function}&
\\
$g ::=$
&
$\bind{x}{\vec{c}}$
&
\\[2mm]
&\textbfit{Clause}&
\\
$c ::=$
&
$\clause{\vec{p}}{s}$
&
\\[2mm]
&\textbfit{Pattern}&
\\
$p ::=$
&
$\pattVar{x}$
&
variable
\\
&
$\pattRec{\vec{\bind{x}{p}}}$
&
record
\\
&
$\pattNil$
&
nil
\\
&
$\pattCons{p}{p}$
&
cons
\\
&
$\pattList{p}{o}$
&
non-empty list
\\[2mm]
&\textbfit{List rest pattern}&
\\
$o ::=$
&
$\pattListEnd$
&
end
\\
&
$\pattListNext{p}{o}$
&
cons
\\[2mm]
&\textbfit{Qualifier}&
\\
$q ::=$
&
$\qualGuard{s}$
&
guard
\\
&
$\qualDeclaration{p}{s}$
&
declaration
\\
&
$\qualGenerator{p}{s}$
&
generator
%$x, y$
%&
%&
%identifier
%\\
%$i, j$
%&&
%positive integer
%\\[2mm]
\end{tabularx}
\end{minipage}
\endgroup
}
\caption{Syntax of surface language}
\label{fig:surface-language:syntax}
\end{figure}


Tries \cite{hinze00} and terms are somewhat non-standard and are best explained
with reference to the typing rules in
\figrefTwo{demand-indexed:typing-trie}{demand-indexed:typing-term}. We start
with tries, which are a particular syntactic representation of functions. The
$\trieArrow$ arrow associates to the right. The variable trie
$\trieVar{x}{\kappa}$ maps \emph{any} value of type $A$ to $\kappa$, which is
typed in a context which extends $\Gamma$ with a new binding of type $A$. The
unit trie $\trieUnit{\kappa}$ maps the single value of type $\tyUnit$ to
$\kappa$, without introducing a new variable.

\begin{figure}
\flushleft \shadebox{$\trieType{\sigma}{B}{A}{\Gamma}$}
\begin{smathpar}
\inferrule*
{
   \contType{\kappa}{B}{\cxtExtend{\Gamma}{x}{A}}
}
{
   \trieType{\trieVar{x}{\kappa}}{B}{A}{\Gamma}
}
%
\and
%
\inferrule*
{
   \contType{\kappa}{B}{\Gamma}
}
{
   \trieType{\trieUnit{\kappa}}{B}{\tyUnit}{\Gamma}
}
%
\and
%
\inferrule*
{
   \trieType{\sigma_1}{B}{A}{\Gamma}
   \\
   \trieType{\sigma_2}{B}{A'}{\Gamma}
}
{
   \trieType{\trieSum{\sigma_1}{\sigma_2}}{B}{\tySum{A}{A'}}{\Gamma}
}
%
\and
%
\inferrule*
{
   \trieType{\sigma}{\trieTypePartial{B}{A'}}{A}{\Gamma}
}
{
   \trieType{\trieProd{\sigma}}{B}{\tyProd{A}{A'}}{\Gamma}
}
%
\and
%
\inferrule*
{
   \trieType{\sigma}{B}{\subst{A}{\tyRec{\tyVar{X}}{A}}{\alpha}}{\Gamma}
}
{
   \trieType{\trieRoll{\sigma}}{B}{(\tyRec{\tyVar{X}}{A})}{\Gamma}
}
\end{smathpar}
\caption{Typing rules for tries}
\label{fig:demand-indexed:typing-trie}
\end{figure}

\begin{figure}[H]
\flushleft \shadebox{$\Gamma \vdash e: A$}
\begin{smathpar}
\inferrule*[right={$x : A \in \Gamma$}]
{
   \strut
}
{
   \Gamma \vdash \exVar{x}: A
}
%
\and
%
\inferrule*[right={$\primOp: A \in \Gamma$}]
{
   \strut
}
{
   \Gamma \vdash \exOp{\primOp}: A
}
%
\and
%
\inferrule*
{
   \strut
}
{
   \Gamma \vdash n: \tyInt
}
%
\and
%
\inferrule*
{
   \strut
}
{
   \Gamma \vdash \exTrue: \tyBool
}
%
\and
%
\inferrule*
{
   \strut
}
{
   \Gamma \vdash \exFalse: \tyBool
}
%
\and
%
\inferrule*
{
   \Gamma \vdash u: A
   \\
   \Gamma \vdash v: B
}
{
   \Gamma \vdash \exPair{u}{v}: \tyProd{A}{B}
}
%
\and
%
\inferrule*[right={$\primOp: \tyFun{\tyInt}{\tyFun{\tyInt}{\tyInt}} \in \Gamma$}]
{
   \Gamma \vdash e: \tyInt
   \\
   \Gamma \vdash e': \tyInt
}
{
   \Gamma \vdash \exBinaryApp{e}{\primOp}{e'}: \tyInt
}
%
\and
%
\inferrule*
{
   \Gamma \vdash \sigma : \tyFun{A}{B}
}
{
   \Gamma \vdash \exLambda{\sigma} : \tyFun{A}{B}
}
%
\and
%
\inferrule*
{
   \Gamma \vdash e: \tyFun{A}{B}
   \\
   \Gamma \vdash e': A
}
{
   \Gamma \vdash \exApp{e}{e'}: B
}
%
\and
%
\inferrule*
{
   \strut
}
{
   \Gamma \vdash \exNil: \tyList{A}
}
%
\and
%
\inferrule*
{
   \Gamma \vdash u: A
   \\
   \Gamma \vdash v: \tyList{A}
}
{
   \Gamma \vdash (\exCons{u}{v}): \tyList{A}
}
%
\and
%
\inferrule*
{
   \Gamma \vdash e': \tyProd{\tyInt}{\tyInt}
   \\
   \Gamma \concat \bind{x}{\tyInt} \concat \bind{y}{\tyInt} \vdash e: A
}
{
   \Gamma \vdash \exMatrix{e}{x}{y}{e'}: \tyMatrix{A}
}
%
\and
%
\inferrule*
{
   \Gamma \vdash e: \tyMatrix{A}
   \\
   \Gamma \vdash e': \tyProd{\tyInt}{\tyInt}
}
{
   \Gamma \vdash \exMatrixAccess{e}{e'}: A
}
%
\and
%
\inferrule*
{
   \Gamma \vdash e: \tyMatrix{A}
}
{
   \Gamma \vdash \exMatrixDims{e}: \tyProd{\tyInt}{\tyInt}
}
%
\and
%
\inferrule*
{
   \Gamma \vdash h: \Delta
   \\
   \Gamma \concat \Delta \vdash e: A
}
{
   \Gamma \vdash \exLetRecMutual{h}{e}: A
}
\end{smathpar}

\vspace{5pt}
\flushleft \shadebox{$\Gamma \vdash \sigma: \tyFun{A}{B}$}
\begin{smathpar}
\inferrule*
{
   \Gamma \concat \bind{x}{A} \vdash \kappa: B
}
{
   \Gamma \vdash (\elimVar{x}{\kappa}): \tyFun{A}{B}
}
%
\and
%
\inferrule*
{
   \Gamma \vdash \kappa: A
   \\
   \Gamma \vdash \kappa': A
}
{
   \Gamma \vdash \elimBool{\kappa}{\kappa'}: \tyFun{\tyBool}{A}
}
%
\and
%
\inferrule*
{
   \Gamma \vdash \sigma: \tyFun{A}{\tyFun{A'}{B}}
}
{
   \Gamma \vdash \elimProd{\sigma}: \tyFun{\tyProd{A}{A'}}{B}
}
%
\and
%
\inferrule*
{
   \Gamma \vdash \kappa: B
   \\
   \Gamma \vdash \sigma: \tyFun{A}{\tyFun{\tyList{A}}{B}}
}
{
   \Gamma \vdash \elimList{\kappa}{\sigma}: \tyFun{\tyList{A}}{B}
}
\end{smathpar}
\caption{Typing rules for terms and eliminators}
\end{figure}


Of the non-leaf cases, the trie $\trieRoll{\sigma}$ for a recursive type
$\tyRec{\tyVar{X}}{A}$, is the simplest: it simply injects $\sigma$, which is a
trie over the one-step unrolling of $\tyRec{\tyVar{X}}{A}$, into
$\tyRec{\tyVar{X}}{A}$. Sum tries $\trieSum{\sigma}{\tau}$ are slightly more
interesting, allowing a trie over $A$ and a trie over $B$ to be combined into a
trie over $\tySum{A}{B}$ as long they share a codomain. Product tries
$\trieProd{\sigma}$ rely on currying, allowing a trie over $A$ which returns a
trie over $B$ to be treated as a trie over $\tyProd{A}{B}$.

\subsection{Values and environments}

\begin{figure}
\begin{syntaxfig}
\mbox{Value}
&
v
&
::=
&
\bot
&
\text{absent}
\\
&&&
\exTrue \mid \exFalse
&
\text{Boolean}
\\
&&&
\exInt{n}
&
\text{integer}
\\
&&&
\exClosureRec{\rho}{f}{\sigma}
&
\text{closure}
\\
&&&
\exPair{v}{v'}
&
\text{pair}
\\
&&&
\exPairDel{v}{v'}
\\
&&&
\exNil
&
\text{nil}
\\
&&&
\exCons{v}{v'}
&
\text{cons}
\\
&&&
\exConsDel{v}{v'}
\\[2mm]
\mbox{Environment}
&
\rho
&
::=
&
\envEmpty
&
\text{empty}
\\
&&&
\envExtend{\rho}{x_{\alpha}}{v}
&
\text{extend}
\end{syntaxfig}
\caption{Values and environments}
\end{figure}

\begin{figure}
\flushleft \shadebox{$\valType{v}{A}$}
\begin{smathpar}
\inferrule*
{
   \valType{u}{A}
}
{
   \valType{\annot{u}{\alpha}}{A}
}
\end{smathpar}
\\[3mm]
\flushleft \shadebox{$\valType{u}{A}$}
\begin{smathpar}
\inferrule*
{
   \strut
}
{
   \valType{\exUnit}{\tyUnit}
}
%
\and
%
\inferrule*
{
   \valType{v}{A}
}
{
   \valType{\exInl{v}}{\tySum{A}{B}}
}
%
\and
%
\inferrule*
{
   \valType{v}{B}
}
{
   \valType{\exInr{v}}{\tySum{A}{B}}
}
%
\and
%
\inferrule*
{
   \valType{v_1}{A}
   \\
   \valType{v_2}{B}
}
{
   \valType{\exPair{v_1}{v_2}}{\tyProd{A}{B}}
}
%
\and
%
\inferrule*
{
   \envType{\Gamma}{\rho}
   \\
   \trieType{\sigma}{B}{A}{\Gamma}
}
{
   \valType{\exClosure{\rho}{\exFun{\sigma}}}{\tyFun{A}{B}}
}
%
\and
%
\inferrule*
{
   \valType{v}{\subst{A}{\tyRec{\tyVar{X}}{A}}{\alpha}}
}
{
   \valType{\exRoll{v}}{\tyRec{\tyVar{X}}{A}}
}
\end{smathpar}
\vspace{5pt}

\noindent \shadebox{$\envType{\Gamma}{\rho}$}
\begin{smathpar}
\inferrule*
{
   \strut
}
{
   \envType{\cxtEmpty}{\envEmpty}
}
%
\and
%
\inferrule*
{
   \envType{\Gamma}{\rho}
   \\
   \valType{u}{A}
}
{
   \envType{\cxtExtend{\Gamma}{x}{A}}{\envExtend{\rho}{x}{u}}
}
\end{smathpar}
\caption{Typing rules for values and environments}
\label{fig:demand-indexed:typing-value}
\end{figure}


The syntax of values and environments is given in
\figref{demand-indexed:syntax-value}. Concatenation of sequences is denoted by
the $\concat$ operator.

\begin{figure}
\begin{syntaxfig}
\mbox{Type}
&
A, B
&
::=
&
\tyInt
&
\text{integers}
\\
&&&
\tyList
&
\text{list of integers}
\\
&&&
\tyProd{A}{B}
&
\text{pairs}
\\
&&&
\tyFun{A}{B}
&
\text{function}
\\[2mm]
\mbox{Eliminator}
&
\sigma, \tau
&
::=
&
\elimVar{x}{e}
&
\text{variable}
\\
&&&
\elimConstr{\Sigma}
&
\text{constructor}
\\[2mm]
\mbox{Constructor eliminator}
&
\Sigma
&
::=
&
\family{\kappa_c}{c}{\tilde{c}}
\\[2mm]
\mbox{Term}
&
r
&
::=
&
x
&
\text{variable}
\\
&&&
\exConst{k}
&
\text{constant}
\\
&&&
\phi
&
\text{first-class primitive}
\\
&&&
\exFun{\sigma}
&
\text{function}
\\
&&&
\exApp{e}{e'}
&
\text{application}
\\
&&&
\exPrimApp{e}{\exPrimOp}{e'}
&
\text{binary infix application}
\\
&&&
\exConstr{c}{\vec{e}}
&
\text{constructor}
\\
&&&
\exMatch{e}{\sigma}
&
\text{pattern-match}
\\
&&&
\exDefs{\vec{d}}{e}
&
\text{definitions}
\\[2mm]
\mbox{Definition}
&
d
&
::=
&
\exLetDef{x}{e}
&
\text{let}
\\
&&&
\exLetrecDef{\delta}
&
\text{mutual recursion}
\\
&&&
\exPrimDef{x}
&
\text{primitive}
\\[2mm]
\mbox{Recursive functions}
&
\delta
&
::=
&
\set{\annot{(\exFun{\sigma_f})}{\alpha_f}}_{f \in F}
\end{syntaxfig}
\caption{Syntax for the implementation language}
\label{fig:impl-language:syntax}
\end{figure}

We see this in visualisation libraries like Bokeh \cite{jolly18} which enable
interaction with data, calls for research papers to support multiverse analyses
for testing and playing with assumptions \cite{dragicevic19}, and new journals
like Distill which focus on so-called ``explorable explanations'' in AI
\cite{goertler19}.

\secbreak

We will also seek to generalise our analysis technique to allow scientists to
explore alternative data or modelling scenarios interactively, obtaining visual
feedback on precisely how a chart or model would behave under different
assumptions or counterfactual inputs.

A more theoretical challenge will be to extend our existing dynamic analysis
technique to an account of how changes to programs lead to changes in their
outputs. This is related to so-called incremental computation, but considers
code changes as well as data changes. We do not expect to fully address this
difficult challenge in this work package, but we will need to make sufficient
progress to support a practical implementation. Here we will draw on relevant
expertise of Cheney, Malik and Perera in incremental computation.

\secbreak

\paragraph{Wrattler integration.} The Wrattler platform offers the best
opportunity for this, since it will allow our framework to leverage the R,
Python (and eventually Julia) interoperability provided by Wrattler.

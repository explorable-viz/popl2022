\section{Conclusion}
\label{sec:conclusion}

Our research was motived by the goal of making computational outputs which are automatically able to reveal how they relate to data in a fine-grained way. A casual reader who wants to understand or fact-check a chart, or a scientist evaluating another's work, should be able to do so by interacting directly with an output. Recent work by \citeauthor{walny19} suggests that developers would also benefit from such a feature while implementing visualisations, for example to check whether a quantity is represented by diameter or area in a bubble chart \cite{walny19}.

Galois connections provide an appealing setting for this problem because of their elegant round-tripping properties. However,  existing dynamic analysis techniques based on Galois connections do not lend themselves to richly structured outputs like visualisations and matrices. We presented an approach that allows focusing on arbitrary substructures, which also means data selections can be inverted. This enables linking not just of outputs to data, but of outputs to other outputs, providing a mathematical basis for a widely used (but so far ad hoc) feature in data visualisation. We implemented our approach in \href{https://github.com/explorable-viz/fluid}{\OurLanguage}, a realistic high-level functional programming language.

\subsection{Other Related Work and Future Directions}
\label{sec:conclusion:other-related-work}

We close by considering some limitations and opportunities in the context of other related work. Galois slicing \cite{perera12a,ricciotti17,perera16d} was considered in \secref{de-morgan:galois-slicing}.

\emph{Executable slicing.} Executable slices \cite{hall95} are programs with some parts removed, but which are still executable. Our approach computes data selections, not executable slices, but such a notion has obvious relevance in data science: ``explaining'' part of a result should (arguably) entail being able to recompute it. \emph{Expression provenance} \cite{acar12} explains how primitive values are computed using only primitive operations; however, this still omits crucial information, and does not obviously generalise to structured outputs. Work on executable slicing in term rewriting \cite{field98} could perhaps be adapted to structured data and combined with dependency tracking for higher-order data (\secref{data-dependencies:lattices-of-selections}).

\emph{Dynamic program analysis.} Dynamic analysis techniques like dataflow analysis \cite{chen88} and taint tracking \cite{reps95} tend to focus on variables, rather than parts of structured values, and lack round-tripping properties; Galois dependencies have a clear advantage here. A limitation of dynamic techniques which is shared by our approach is that they can usually only reveal \emph{that} certain dependencies arise, not \emph{why}, which requires analysing path conditions \cite{hammer06}. In a data science setting this would clearly be valuable too, and it would be interesting to see if the benefits of the Galois framework can be extended to techniques for computing dynamic path conditions.

\emph{Brushing and linking.} Brushing and linking has been extensive studied in the data visualisation community~\cite{mcdonald82,becker87}, but although \citet{roberts06} argued it should be ubiquitous, no automated method of implementation has been proposed to date. Geospatial applications like GeoDa \cite{anselin06} hard-code view coordination features into specific views, and libraries like d3.js and Plotly support ad hoc linking mechanisms, with varying degrees of programmer effort required. No existing approach provides automation or round-tripping guarantees, or is able to provide data selections explaining why visual selections are linked.

\emph{Data provenance in data visualisation.} A recent vision paper by \citet{psallidas18} is the only work we are aware of that proposes that brushing and linking, and related view cooordination features like cross-filtering, can be understood in terms of data provenance. In a relational (query processing) setting, where the relevant notion of provenance is lineage, they propose backward-analysing to data, and then forward-analysing to another view, although again without the round-tripping features of Galois connections. Moreover theirs is primarily a concept paper, proposing a research programme, rather than solving a specific problem.

\paragraph{Acknowledgements}

Perera and Petricek were supported by The UKRI Strategic Priorities Fund under EPSRC Grant EP/T001569/1, particularly the \emph{Tools, Practices and Systems} theme within that grant, and by The Alan Turing Institute under EPSRC grant EP/N510129/1. Wang was supported by \emph{Expressive High-Level Languages for Bidirectional Transformations}, EPSRC Grant EP/T008911/1.
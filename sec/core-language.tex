\section{Core language}
\label{sec:core-language}

The core calculus which provides the setting for the rest of the paper is a mostly standard call-by-value functional language with datatypes and records. The main unusual feature is the use of \emph{eliminators}, a trie-like construct that provides a uniform syntax and semantics for pattern-matching; this allows us to assume that incomplete or overlapping patterns and other syntactic considerations have been dealt in the surface language. (In \secref{surface-language} we show how familiar pattern-matching features like case expressions and piecewise function definitions easily desugar into eliminators.) We give a big-step environment-based semantics, which is easier for the backward and forward dependency analyses in \secref{data-dependencies}, and introduce a compact (term-like) representation of derivation trees in the operational semantics, called \emph{traces}, which we will use to define the analyses over a fixed execution. Mutual recursion requires some care for the backwards analysis, so we also treat that as a core language feature.

\subsection{Syntax and typing}
\label{sec:core-language:syntax-typing}

Although our implementation is untyped, types help describe the structure of the core language. \figref{core-language:syntax} introduces the types $A, B$ which include $\tyBool$, $\tyInt$ and function types $\tyFun{A}{B}$, but also lists $\tyList{A}$ and records $\tyRec{\vec{\bind{x}{A}}}$ which exemplify the two kinds of structured data which are of interest: recursive datatypes with varying structure, and tabular data with a fixed shape. As usual the notation $\bind{x}{A}$ denotes the binding of $x$ to $A$ (understood formally as a pair); $\seq{\bind{x}{A}}$ denotes the sequence of bindings that results from zipping same-length sequences $\vec{x}$ and $\vec{A}$. In a record type $\tyRec{\vec{\bind{x}{A}}}$ the field names in $\vec{x}$ are required to be unique.

\begin{figure}
{\small
\begingroup
\renewcommand*{\arraystretch}{1}
\begin{minipage}[t]{0.5\textwidth}
\begin{tabularx}{\textwidth}{rL{2cm}L{3cm}}
%\rowcolor{verylightgray}
&\textbfit{Type}&
\\
$A, B ::=$
&
$\tyBool$
&
Booleans
\\
&
$\tyInt$
&
integers
\\
&
$\tyRec{\vec{\bind{x}{A}}}$
&
records
\\
&
$\tyList{A}$
&
lists
\\
&
$\tyFun{A}{B}$
&
functions
\\[2mm]
$\Gamma, \Delta ::=$
&
$\seq{\bind{x}{A}}$
&
typing context
\\[2mm]
&\textbfit{Term}&
\\
$e ::=$
&
$\exTrue \mid \exFalse$
&
Boolean
\\
&
$\exInt{n}$
&
integer
\\
&
$\exVar{x}$
&
variable
\\
&
$\exAppPrim{\phi}{\vec{e}}$
&
primitive application
\\
&
$\exApp{e}{e'}$
&
application
\\
&
$\exNil \mid \exCons{u}{v}$
&
list
\\
&
$\exRec{\vec{\bind{x}{e}}}$
&
record
\\
&
$\exRecProj{e}{x}$
&
record projection
\\
&
$\exLambda{\sigma}$
&
anonymous function
\\
&
$\exLetRecMutual{h}{e}$
&
recursive let
\\[2mm]
$h ::=$
&
$\seq{\bind{x}{\sigma}}$
&
recursive functions
\end{tabularx}
\end{minipage}%
\begin{minipage}[t]{0.5\textwidth}
\begin{tabularx}{\textwidth}{rL{2.6cm}L{2.9cm}}
&\textbfit{Continuation type}&
\\
$K ::=$
&
$A$
&
term
\\
&
$\elimTy{A}{K}$
&
eliminator
\\[2mm]
&\textbfit{Continuation}&
\\
$\kappa ::=$
&
$e$
&
term
\\
&
$\sigma$
&
eliminator
\\[2mm]
&\textbfit{Eliminator}&
\\
$\sigma, \tau ::=$
&
$\elimVar{x}{\kappa}$
&
variable
\\
&
$\elimBool{\kappa}{\kappa'}$
&
Boolean
\\
&
$\elimRec{\vec{x}}{\kappa}$
&
record
\\
&
$\elimList{\kappa}{\sigma}$
&
list
\\[2mm]
&\textbfit{Value}&
\\
$u, v ::=$
&
$\exTrue \mid \exFalse$
&
Boolean
\\
&
$\exInt{n}$
&
integer
\\
&
$\exNil \mid \exCons{u}{v}$
&
list
\\
&
$\exRec{\vec{\bind{x}{v}}}$
&
record
\\
&
$\exClosure{\rho}{h}{\sigma}$
&
closure
\\[2mm]
$\rho ::=$
&
$\seq{\bind{x}{v}}$
&
environment
\\[2mm]
%$x, y$
%&
%&
%identifier
%\\
%$i, j$
%&&
%positive integer
%\\[2mm]
\end{tabularx}
\end{minipage}
\endgroup
}
\caption{Syntax of core language}
\label{fig:core-language:syntax}
\end{figure}


The terms $e$ of the language are defined in \figref{core-language:syntax}. These include Boolean constants $\exTrue$ and $\exFalse$, integers $n$, variables $x$, and applications $\exApp{e}{e'}$. Primitives are not first-class; the expression $\exAppPrim{\phi}{\vec{e}}$ is the fully saturated application of $\phi$ to a sequence of arguments. (First-class and infix primitives are provided by desugarings in \secref{surface-language}). We also provide list constructors nil $\exNil$ and cons $\exCons{e}{e'}$, record construction $\exRec{\vec{\bind{x}{e}}}$ and record projection $\exRecProj{e}{x}$. The final two term forms, anonymous functions $\exLambda{\sigma}$ and recursive let-bindings $\exLetRecMutual{h}{e}$ where $h$ is of the form $\vec{\bind{x}{\sigma}}$, are explained below after we introduce the pattern-matching construct $\sigma$ (\emph{eliminator}). The typing rules for terms are given in \figref{core-language:typing}, and are intended only to help a reader understand the language; therefore the rules are simple and do not include features such as polymorphism. The main typing rules of interest are the ones which involve eliminators.

\subsection{Eliminators}
\label{sec:core-language:syntax-eliminator}

Eliminators $\sigma, \tau$ are also defined in \figref{core-language:syntax}, and are inspired by \citeauthor{connelly95}'s notion of generalised trie \cite{connelly95}, elaborated further by \citet{hinze00}. Eliminators specify how to match an initial part of a value and select a continuation $\kappa$ for further execution, where $\kappa$ is either a term $e$ or another eliminator $\sigma$. The Boolean eliminator $\elimBool{\kappa}{\kappa'}$ selects either $\kappa$ or $\kappa'$ depending on whether a Boolean value is $\exTrue$ or $\exFalse$. The record eliminator $\elimRec{\seq{x}}{\kappa}$ matches a record with fields $\seq{x}$ and then returns $\kappa$ with the variables $\seq{x}$ bound to the components of the record. The list eliminator $\elimList{\kappa}{\sigma}$ maps the empty list to $\kappa$ and a non-empty list to another eliminator $\sigma$ which specifies how the head and tail of the list are to be matched. The variable eliminator $\elimVar{x}{\kappa}$ extends the usual notion of trie, matching any value, binding it to $x$, and then returning $\kappa$. Eliminators resemble the ``case trees'' commonly used as an intermediate form when compiling languages with pattern-matching \cite{graf20}, and can serve as an elaboration target for more advanced features such as the piecewise definitions described in \secref{surface-language}.

\begin{figure}
\small{\flushleft \shadebox{$\Gamma \vdash e: A$}%
\hfill \textbfit{$e$ has type $A$ under $\Gamma$}}
\begin{smathpar}
\inferrule*[right={$x : A \in \Gamma$}]
{
   \strut
}
{
   \Gamma \vdash \exVar{x}: A
}
%
\and
%
\inferrule*
{
   \strut
}
{
   \Gamma \vdash \exInt{n}: \tyInt
}
%
\and
%
\inferrule*
{
   \strut
}
{
   \Gamma \vdash \exTrue: \tyBool
}
%
\and
%
\inferrule*
{
   \strut
}
{
   \Gamma \vdash \exFalse: \tyBool
}
%
\and
%
\inferrule*
{
   \Gamma \vdash e_i: A_i
   \quad
   (\forall i \numleq \length{\vec{x}})
}
{
   \Gamma \vdash \exRec{\vec{\bind{x}{e}}}: \tyRec{\vec{\bind{x}{A}}}
}
%
\and
%
\inferrule*[
   right={$i \numleq \length{\vec{x}}$}
]
{
   \Gamma \vdash e: \tyRec{\vec{\bind{x}{A}}}
}
{
   \Gamma \vdash \exRecProj{e}{x_i}: A_i
}
%
\and
%
\inferrule*
{
   \Gamma \vdash \sigma : \elimTy{A}{B}
}
{
   \Gamma \vdash \exLambda{\sigma} : \tyFun{A}{B}
}
%
\and
%
\inferrule*[
   right={$\phi \in \Int^j \to \Int$}
]
{
   \Gamma \vdash e_i: \tyInt
   \quad
   (\forall i \numleq j)
}
{
   \Gamma \vdash \exAppPrim{\phi}{\vec{e}}: \tyInt
}
%
\and
%
\inferrule*
{
   \Gamma \vdash e: \tyFun{A}{B}
   \\
   \Gamma \vdash e': A
}
{
   \Gamma \vdash \exApp{e}{e'}: B
}
%
\and
%
\inferrule*
{
   \strut
}
{
   \Gamma \vdash \exNil: \tyList{A}
}
%
\and
%
\inferrule*
{
   \Gamma \vdash e: A
   \\
   \Gamma \vdash e': \tyList{A}
}
{
   \Gamma \vdash (\exCons{e}{e'}): \tyList{A}
}
%
\and
%
\inferrule*
{
   \Gamma \vdash h: \Delta
   \\
   \Gamma \concat \Delta \vdash e: A
}
{
   \Gamma \vdash \exLetRecMutual{h}{e}: A
}
\end{smathpar}
{\small \flushleft \shadebox{$\Gamma \vdash \sigma: \elimTy{A}{K}$}%
\hfill \textbfit{$\sigma$ has type $\elimTy{A}{K}$ under $\Gamma$}}
\begin{smathpar}
\inferrule*
{
   \Gamma \concat \bind{x}{A} \vdash \kappa: K
}
{
   \Gamma \vdash (\elimVar{x}{\kappa}): \elimTy{A}{K}
}
%
\and
%
\inferrule*
{
   \Gamma \vdash \kappa: K
   \\
   \Gamma \vdash \kappa': K
}
{
   \Gamma \vdash \elimBool{\kappa}{\kappa'}: \elimTy{\tyBool}{K}
}
%
\and
%
\inferrule*
{
   \Gamma \vdash \kappa: K
}
{
   \Gamma
   \vdash
   \elimRecEmpty{\kappa}: \elimTy{\tyRecEmpty}{K}
}
%
\and
%
\inferrule*
{
   \Gamma \vdash \elimRec{\vec{x}}{\sigma}: \elimTy{\tyRec{\vec{\bind{x}{A}}}}{\elimTy{B}{K}}
}
{
   \Gamma
   \vdash
   \elimRec{\vec{x} \concat y}{\sigma}: \elimTy{\tyRec{\vec{\bind{x}{A}} \concat \bind{y}{B}}}{K}
}
%
\and
%
\inferrule*
{
   \Gamma \vdash \kappa: K
   \\
   \Gamma \vdash \sigma: \elimTy{A}{\elimTy{\tyList{A}}{K}}
}
{
   \Gamma \vdash \elimList{\kappa}{\sigma}: \elimTy{\tyList{A}}{B}
}
\end{smathpar}
\\[2mm]
{\small \flushleft \shadebox{$\vdash v: A$}%
\hfill \textbfit{$v$ has type $A$}}
\begin{smathpar}
\inferrule*
{
   \strut
}
{
   \vdash \exInt{n}: \tyInt
}
%
\and
%
\inferrule*
{
   \strut
}
{
   \vdash \exTrue: \tyBool
}
%
\and
%
\inferrule*
{
   \strut
}
{
   \vdash \exFalse: \tyBool
}
%
\and
%
\inferrule*
{
   \vdash v_i: A_i
   \quad
   (\forall i \numleq \length{\vec{x}})
}
{
   \vdash \exRec{\vec{\bind{x}{v}}}: \tyRec{\vec{\bind{x}{A}}}
}
%
\and
%
\inferrule*
{
   \strut
}
{
   \vdash \exNil: \tyList{A}
}
%
\and
%
\inferrule*
{
   \vdash u: A
   \\
   \vdash v: \tyList{A}
}
{
   \vdash (\exCons{u}{v}): \tyList{A}
}
%
\and
%
\inferrule*
{
   \vdash \rho: \Gamma
   \\
   \Gamma \vdash h: \Delta
   \\
   \Gamma \concat \Delta \vdash \sigma: \elimTy{A}{B}
}
{
   \vdash \exClosure{\rho}{h}{\sigma}: \tyFun{A}{B}
}
\end{smathpar}
\\[2mm]
\begin{minipage}[t]{0.4\textwidth}%
{\small \flushleft \shadebox{$\vdash \rho: \Gamma$}%
\hfill \textbfit{$\rho$ has type $\Gamma$}}
\begin{smathpar}
\inferrule*
{
   \vdash v_i: A_i
   \quad
   (\forall i \numleq \length{\vec{x}})
}
{
   \vdash \vec{\bind{x}{v}}: \vec{\bind{x}{A}}
}
\end{smathpar}
\end{minipage}%
\hspace{1.5cm}
\begin{minipage}[t]{0.48\textwidth}%
{\small \flushleft \shadebox{$\Gamma \vdash h: \Delta$}%
\hfill \textbfit{$h$ has type $\Delta$ under $\Gamma$}}
\begin{smathpar}
   \inferrule*[right={
      \textnormal{$\Delta = \vec{\bind{x}{\tyFun{A}{B}}}$}
   }]
   {
      \Gamma \concat \Delta \vdash \sigma_i: \elimTy{A_i}{B_i}
      \quad
      (\forall i \numleq \length{\vec{x}})
   }
   {
      \Gamma \vdash \vec{\bind{x}{\sigma}}: \Delta
   }
\end{smathpar}
\end{minipage}
\caption{Typing rules for core language}
\label{fig:core-language:typing}
\end{figure}


The use of nested eliminators to match sub-values will become clearer if we consider the typing judgement $\Gamma \vdash \sigma: \elimTy{A}{B}$ for eliminators given in \figref{core-language:typing}. Because eliminators always have a function-like type, this should be read as a four-place relation, with $\elimTyArrow$ being part of the notation. (The definition delegates to an auxiliary judgement $\Gamma \vdash \kappa: K$ which we define to be the union of the typing relations for terms and eliminators.) The typing rule for variable eliminators reveals the connection between eliminators and functions: it converts a continuation $\kappa$ which can be assigned type $K$ under the assumption that $x$ is of type $A$ into an eliminator of type $\elimTy{A}{K}$. The typing rule for Boolean eliminators says that to make an eliminator of type $\elimTy{\tyBool}{K}$, we simply need continuations $\kappa$ and $\kappa'$ of type $K$. The rule for the empty record states that to make an eliminator of type $\elimTy{\tyRecEmpty}{K}$, we simply need a continuation $\kappa$ of type $K$. The rule for non-empty records allows us to transform a ``curried'' eliminator of type $\elimTy{\tyRec{\vec{\bind{x}{A}}}}{\elimTy{B}{K}}$ into one of type $\elimTy{\tyRec{\vec{\bind{x}{A}} \concat \bind{y}{B}}}{K}$, analogous to the isomorphism between $\tyFun{\tyFun{A}{B}}{C}$ and $\tyFun{\tyProd{A}{B}}{C}$ \cite{hinze00}. (Formalising eliminators precisely requires \citeauthor{bird98}-style nested datatypes \cite{bird98} and polymorphic recursion, but these details need not concern us here.)

The typing rule for list eliminators $\elimList{\kappa}{\sigma}$ combines some of the flavour of record and Boolean eliminators. To make an eliminator of type $\elimTy{\tyList{A}}{K}$, we need a continuation of type $K$ for the empty case, and for the non-empty case, an eliminator of type $\elimTy{A}{\elimTy{\tyList{A}}{K}}$ which can be treated as an eliminator of type $\elimTy{\tyProd{A}{\tyList{A}}}{K}$ in order to process the head and tail.

\subsubsection{Eliminators as functions}

We can now revisit the term forms $\exLambda{\sigma}$ and $\exLetRecMutual{h}{e}$. If $\sigma$ is an eliminator of type $\elimTy{A}{B}$, then $\exLambda{\sigma}$ is an anonymous function of type $\tyFun{A}{B}$. If $h$ is of the form $\vec{\bind{x}{\sigma}}$, then $\exLetRecMutual{h}{e}$ introduces a sequence of mutually recursive functions which are in scope in $e$. The typing rule for $\exLetRecMutual{h}{e}$ uses an auxiliary typing judgement $\Gamma \vdash h : \Delta$ which assigns to every $x$ in $\Delta$ the function type $\tyFun{A}{B}$ if the $\sigma$ to which $x$ is bound in $h$ has the eliminator type $\elimTy{A}{B}$.

\subsubsection{Values}
Values $v, u$, and environments $\rho$ are also defined in \figref{core-language:syntax}, and are standard for call-by-value. (Environments are more convenient than substitution for tracking variable usage.) To support mutual recursion, the closure form  $\exClosure{\rho}{h}{\sigma}$ captures the (possibly empty) sequence $h$ of functions with which the function was mutually defined, in addition to the ambient environment $\rho$. For the typing judgements $\vdash \rho: \Gamma$ and $\vdash v: A$ for environments and values (\figref{core-language:typing}), only the closure case is worth noting, which delegates to the typing rules for recursive definitions and eliminators.

\subsubsection{Evaluation}
\label{sec:core-language:eval}

\figref{core-language:semantics} gives the operational semantics of the core language. In \secref{data-dependencies} we will define forward and backward analyses over a single execution; in anticipation of that use case, we treat the operational semantics as an inductive data type, following the ``proved transitions'' approach of \citeauthor{boudol89} for reversible CCS~\cite{boudol89}. The inhabitants of this data type are derivation trees explaining how a result was computed, and the analyses will be defined by structural recursion over these trees. Expressed in terms of inference rules, these trees can become quite cumbersome, so we introduce an equivalent but more term-like syntax for them, called a \emph{trace} (\figref{core-language:syntax-trace}), similar to the approach taken by \citeauthor{perera16d} for $\pi$-calculus~\cite{perera16d}.

\begin{figure}[H]
{\small
\begingroup
\renewcommand*{\arraystretch}{1}
\begin{minipage}[t]{0.5\textwidth}
\begin{tabularx}{\textwidth}{rL{2cm}L{3cm}}
&\textbfit{Trace}&
\\
$T, U ::=$
&
$\trVar{x}{\rho}$
&
variable
\\
&
$\trTrue{\rho} \mid \trFalse{\rho}$
&
Boolean
\\
&
$\trInt{n}{\rho}$
&
integer
\\
&
$\trRec{\vec{\bind{x}{T}}}$
&
record
\\
&
$\trRecProj{T}{\vec{\bind{x}{v}}}{y}$
&
record projection
\\
&
$\trNil{\rho} \mid \trCons{T}{U}$
&
list
\\
&
$\trLambda{\sigma}$
&
anonymous function
\\
&
$\trApp{T}{U}{w}{T'}$
&
application
\end{tabularx}
\end{minipage}%
\begin{minipage}[t]{0.5\textwidth}
\begin{tabularx}{\textwidth}{rL{2.5cm}L{3cm}}
\\
&
$\trAppPrimNew{\phi}{U}{\exInt{n}}$
&
primitive application
\\
&
$\trLetRecMutual{h}{T}$
&
recursive let
\\[2mm]
&\textbfit{Match}&
\\
$w ::=$
&
$\matchVar{x}$
&
variable
\\
&
$\matchTrue \mid \matchFalse$
&
Boolean
\\
&
$\matchRec{\vec{\bind{x}{w}}}$
&
record
\\
&
$\matchNil \mid \matchCons{w}{w'}$
&
list
\\[4mm]
\end{tabularx}
\end{minipage}
\endgroup
}
\caption{Syntax of traces and matches}
\label{fig:core-language:syntax-trace}
\end{figure}


\begin{figure}
   \begin{figure}
{\small \flushleft \shadebox{$\evalFwd{\rho}{e}{\alpha}{T}{v}$}%
\hfill \textbfit{$\rho$ and $e$, with argument availability $\alpha$, forward-analyse along $T$ to $v$}}
\begin{smathpar}
   \mprset{center}
   \inferrule*[
      lab={\ruleName{$\evalFwdS$-eq}}
   ]
   {
      e \eq e'
      \\
      \evalFwd{\rho}{e'}{\alpha}{T}{v}
   }
   {
      \evalFwd{\rho}{e}{\alpha}{T}{v}
   }
   %
   \and
   %
   \inferrule*[
      lab={\ruleName{$\evalFwdS$-var}},
      right={$\envLookup{\rho}{x}{v}$}
   ]
   {
      \strut
   }
   {
      \evalFwd{\rho}{\exVar{x}}{\alpha}{\trVar{x}{\rho}}{v}
   }
   %
   \and
   %
   \inferrule*[lab={\ruleName{$\evalFwdS$-lambda}}]
   {
      \strut
   }
   {
      \evalFwd{\rho}
              {\exLambda{\sigma}}
              {\alpha}
              {\trLambda{\sigma'}}
              {\exClosure{\rho}{\seqEmpty}{\sigma}}
   }
   %
   \and
   %
   \inferrule*[lab={\ruleName{$\evalFwdS$-int}}]
   {
      \strut
   }
   {
      \evalFwd{\rho}
              {\annInt{n}{\alpha}}
              {\alpha'}
              {\trInt{n}{\rho}}
              {\annInt{n}{\alpha \meet \alpha'}}
   }
   %
   \and
   %
   \inferrule*[lab={\ruleName{$\evalR$-record}}]
   {
      \evalFwd{\rho}{e_i}{\alpha'}{T_i}{v_i}
      \quad
      (\forall i \numleq \length{\vec{x}})
   }
   {
      \evalFwd{\rho}
              {\annRec{\vec{\bind{x}{e}}}{\alpha}}
              {\alpha'}
              {\trRec{\vec{\bind{x}{T}}}}
              {\annRec{\vec{\bind{x}{v}}}{\alpha \meet \alpha'}}
   }
   %
   \and
   %
   \inferrule*[
      lab={\ruleName{$\evalR$-project}},
      right={$i \numleq \length{\vec{x}}$}
   ]
   {
      \evalFwd{\rho}{e}{\alpha}{T}{\exRec{\vec{\bind{x}{v}}}}
   }
   {
      \evalFwd{\rho}
              {\exRecProj{e}{x}}
              {\alpha}
              {\trRecProj{T}{\vec{x}}{x}}
              {v_i}
   }
   %
   \and
   %
   \inferrule*[lab={\ruleName{$\evalFwdS$-nil}}]
   {
      \strut
   }
   {
      \evalFwd{\rho}
              {\annNil{\alpha}}
              {\alpha'}
              {\trNil{\rho}}
              {\annNil{\alpha \meet \alpha'}}
   }
   %
   \and
   %
   \inferrule*[
      lab={\ruleName{$\evalFwdS$-cons}},
   ]
   {
      \evalFwd{\rho}{e}{\alpha'}{T}{v}
      \\
      \evalFwd{\rho}{e'}{\alpha'}{U}{v'}
   }
   {
      \evalFwd{\rho}
              {\annCons{e}{e'}{\alpha}}
              {\alpha'}
              {\trCons{T}{U}}
              {\annCons{v}{v'}{\alpha \meet \alpha'}}
   }
   %
   \and
   %
   \inferrule*[
      lab={\ruleName{$\evalFwdS$-apply-prim}}
   ]
   {
      \evalFwdEq{\rho}{e}{\alpha}{T}{\exPrimOp{\phi}{\vec{v}}}
      \\
      \evalFwd{\rho}{e'}{\alpha}{U}{u}
   }
   {
      \evalFwd{\rho}
              {\exApp{e}{e'}}
              {\alpha}
              {\trAppPrim{T}{\phi}{\vec{n}}{U}{\exInt{m}}}
              {\primFwd{\phi}{\vec{n} \concat m}(\vec{v} \concat u)}
   }
   %
   \and
   %
   \inferrule*[
      lab={\ruleName{$\evalFwdS$-apply}},
      width={3in}
   ]
   {
      \evalFwdEq{\rho}{e}{\alpha}{T}{\exClosure{\rho_1}{h}{\sigma}}
      \\
      \rho_1, h \closeDefsR \rho_2
      \\
      \evalFwd{\rho}{e'}{\alpha}{U}{v}
      \\
      \matchFwd{v}{\sigma}{w}{\rho_3}{e^\twoPrime}{\beta}
      \\
      \evalFwd{\rho_1 \concat \rho_2 \concat \rho_3}{e^\twoPrime}{\beta}{T'}{v'}
   }
   {
      \evalFwd{\rho}{\exApp{e}{e'}}{\alpha}{\trApp{T}{U}{w}{T'}}{v'}
   }
   %
   \and
   %
   \inferrule*[lab={
      \ruleName{$\evalFwdS$-let-rec}}
   ]
   {
      \rho, h' \closeDefsR \rho'
      \\
      \evalFwd{\rho \concat \rho'}{e}{\alpha}{T}{v}
   }
   {
      \evalFwd{\rho}{\exLetRecMutual{h'}{e}}{\alpha}{\trLetRecMutual{h}{T}}{v}
   }
\end{smathpar}
\vspace{1mm}

{\small \flushleft \shadebox{$\matchFwd{v}{\sigma}{w}{\rho}{\kappa}{\alpha}$}%
\hfill \textbfit{$v$ and $\sigma$ forward-analyse along $w$ to $\rho$ and $\kappa$, with argument availability $\alpha$}}
\begin{smathpar}
   \inferrule*[
      lab={\ruleName{$\matchFwdS$-eq}}
   ]
   {
      (v,\sigma) \eq (v',\sigma')
      \\
      \matchFwd{v'}{\sigma'}{w}{\rho}{\kappa}{\alpha}
   }
   {
      \matchFwd{v}{\sigma}{w}{\rho}{\kappa}{\alpha}
   }
   %
   \and
   %
   \inferrule*[
      lab={\ruleName{$\matchFwdS$-var}}
   ]
   {
      \strut
   }
   {
      \matchFwd{v}{\elimVar{x}{\kappa}}{\matchVar{x}}{\bind{x}{v}}{\kappa}{\TT}
   }
   %
   \and
   %
   \inferrule*[
      lab={\ruleName{$\matchFwdS$-true}}
   ]
   {
      \strut
   }
   {
      \matchFwd{\annTrue{\alpha}}
               {\elimBool{\kappa}{\kappa'}}
               {\matchTrue}
               {\seqEmpty}{\kappa}{\alpha}
   }
   %
   \and
   %
   \inferrule*[
      lab={\ruleName{$\matchFwdS$-false}}
   ]
   {
      \strut
   }
   {
      \matchFwd{\annFalse{\alpha}}
               {\elimBool{\kappa}{\kappa'}}
               {\matchFalse}
               {\seqEmpty}{\kappa'}{\alpha}
   }
   %
   \and
   %
   \inferrule*[
      lab={\ruleName{$\matchFwdS$-unit}}
   ]
   {
      \strut
   }
   {
      \matchFwd{\annot{\exRecEmpty}{\alpha}}
               {\elimRecEmpty{\kappa}}
               {\matchRecEmpty}
               {\seqEmpty}
               {\kappa}
               {\alpha}
   }
   %
   \and
   %
   \inferrule*[
      lab={\ruleName{$\matchFwdS$-record}}
   ]
   {
      \matchFwd{\annRec{\vec{\bind{x}{v}}}{\FF}}
               {\elimRec{\vec{x}}{\sigma}}
               {\matchRec{\vec{\bind{x}{w}}}}
               {\rho}
               {\sigma'}
               {\beta}
      \\
      \matchFwd{u}{\sigma'}{w}{\rho'}{\kappa}{\beta'}
   }
   {
      \matchFwd{\annRec{\vec{\bind{x}{v}} \concat \bind{y}{u}}{\alpha}}
               {\elimRec{\vec{x} \concat y}{\sigma}}
               {\matchRec{\vec{\bind{x}{w}} \concat \bind{y}{w'}}}
               {\rho \concat \rho'}
               {\kappa}
               {\alpha \meet \beta \meet \beta'}
   }
   %
   \and
   %
   \inferrule*[
      lab={\ruleName{$\matchFwdS$-nil}}
   ]
   {
      \strut
   }
   {
      \matchFwd{\annNil{\alpha}}
               {\elimList{\kappa}{\sigma'}}
               {\matchNil}
               {\seqEmpty}{\kappa}{\alpha}
   }
   %
   \and
   %
   \inferrule*[
      lab={\ruleName{$\matchFwdS$-cons}}
   ]
   {
      \matchFwd{v}{\sigma}{w}{\rho}{\tau}{\beta}
      \\
      \matchFwd{v'}{\tau}{w'}{\rho'}{\kappa}{\beta'}
   }
   {
      \matchFwd{\annCons{v}{v'}{\alpha}}
               {\elimList{\kappa}{\sigma}}
               {\matchCons{w}{w'}}
               {\rho \concat \rho'}{\kappa}{\alpha \meet \beta \meet \beta'}
   }
\end{smathpar}
\caption{Forward data dependency (Boolean cases for $\evalFwdR{T}$ omitted)}
\label{fig:data-dependencies:fwd}
\end{figure}

\begin{figure}
{\small \flushleft \shadebox{$\evalBwd{v}{T}{\rho}{e}{\alpha}$}%
\hfill \textbfit{$v$ backward-analyses along $T$ to $\rho$ and $e$, with argument demand $\alpha$}}
\begin{smathpar}
   \inferrule*[
      lab={\ruleName{$\evalBwdS$-eq}}
   ]
   {
      v \eq v'
      \\
      \evalBwd{v'}{T}{\rho}{e}{\alpha}
   }
   {
      \evalBwd{v}{T}{\rho}{e}{\alpha}
   }
   %
   \and
   %
   \inferrule*[
      lab={\ruleName{$\evalBwdS$-var}},
      right={$\envLookupBwd{\rho'}{\rho}{\bind{x}{v}}$}
   ]
   {
      \strut
   }
   {
      \evalBwd{v}{\trVar{x}{\rho}}{\rho'}{\exVar{x}}{\FF}
   }
   %
   \and
   %
   \inferrule*[
      lab={\ruleName{$\evalBwdS$-lambda}}
   ]
   {
      \strut
   }
   {
      \evalBwd{\exClosure{\rho}{\seqEmpty}{\sigma}}
              {\trLambda{\sigma'}}
              {\rho}
              {\exLambda{\sigma}}
              {\FF}
   }
   %
   \and
   %
   \inferrule*[
      lab={\ruleName{$\evalBwdS$-int}}
   ]
   {
      \strut
   }
   {
      \evalBwd{\annInt{n}{\alpha}}
              {\trInt{n}{\rho}}
              {\hole_{\rho}}
              {\annInt{n}{\alpha}}
              {\alpha}
   }
   %
   \and
   %
   \inferrule*[lab={\ruleName{$\evalBwdS$-record}}]
   {
      \evalBwd{v_i}{T_i}{\rho_i}{e_i}{\alpha_i'}
      \quad
      (\forall i \numleq \length{\vec{x}})
   }
   {
      \evalBwd{\annRec{\vec{\bind{x}{v}}}{\alpha}}
              {\trRec{\vec{\bind{x}{T}}}}
              {\bigjoin\vec{\rho}}
              {\annRec{\vec{\bind{x}{e}}}{\alpha}}
              {\alpha \join \bigjoin\vec{\alpha}'}
   }
   %
   \and
   %
   \inferrule*[lab={\ruleName{$\evalBwdS$-project}}]
   {
      \evalBwd{\annRec{\mapUpdate{\vec{\bind{x}{\hole}}}{x_i}{v}}{\FF}}
              {T}
              {\rho}
              {e}
              {\alpha}
   }
   {
      \evalBwd{v}
              {\trRecProj{T}{\vec{x}}{x_i}}
              {\rho}
              {\exRecProj{e}{x_i}}
              {\alpha}
   }
   %
   \and
   %
   \inferrule*[
      lab={\ruleName{$\evalBwdS$-nil}}
   ]
   {
      \strut
   }
   {
      \evalBwd{\annNil{\alpha}}
              {\trNil{\rho}}
              {\hole_{\rho}}
              {\annNil{\alpha}}
              {\alpha}
   }
   %
   \and
   %
   \inferrule*[
      lab={\ruleName{$\evalBwdS$-cons}}
   ]
   {
      \evalBwd{v}{T}{\rho}{e}{\alpha}
      \\
      \evalBwd{v'}{U}{\rho'}{e'}{\alpha'}
   }
   {
      \evalBwd{\annCons{v}{v'}{\beta}}
              {\trCons{T}{U}}
              {\rho \join \rho'}
              {\annCons{e}{e'}{\beta}}
              {\beta \join \alpha \join \alpha'}
   }
   %
   \and
   %
   \inferrule*[
      lab={\ruleName{$\evalBwdS$-apply-prim}},
      right={$\primBwd{\phi}{\vec{n} \concat m}(v) = \vec{u} \concat u'$}
   ]
   {
      \evalBwd{\exPrimOp{\phi}{\vec{u}}}{T}{\rho}{e_1}{\alpha}
      \\
      \evalBwd{u'}{U}{\rho'}{e_2}{\alpha'}
   }
   {
      \evalBwd{v}
              {\trAppPrim{T}{\phi}{\vec{n}}{U}{\exInt{m}}}
              {\rho \join \rho'}
              {\exApp{e_1}{e_2}}
              {\alpha \join \alpha'}
   }
   %
   \and
   %
   \inferrule*[
      lab={\ruleName{$\evalBwdS$-apply}},
      width={3.25in}
   ]
   {
      \evalBwd{v}{T'}{\rho_1 \concat \rho_2 \concat \rho_3}{e}{\beta}
      \\
      \matchBwd{\rho_3}{e}{\beta}{w}{v'}{\sigma}
      \\
      \evalBwd{v'}{U}{\rho}{e_2}{\alpha}
      \\
      \rho_2 \closeDefsBwdR \rho_1', h
      \\
      \evalBwd{\exClosure{\rho_1 \join \rho_1'}{h}{\sigma}}{T}{\rho'}{e_1}{\alpha'}
   }
   {
      \evalBwd{v}{\trApp{T}{U}{w}{T'}}{\rho \join \rho'}{\exApp{e_1}{e_2}}{\alpha \join \alpha'}
   }
   %
   \and
   %
   \inferrule*[
      lab={\ruleName{$\evalBwdS$-let-rec}}
   ]
   {
      \evalBwd{v}{T}{\rho \concat \rho_1}{e}{\alpha}
      \\
      \rho_1 \closeDefsBwdR \rho', h'
   }
   {
      \evalBwd{v}{\trLetRecMutual{h}{T}}{\rho \join \rho'}{\exLetRecMutual{h'}{e}}{\alpha}
   }
\end{smathpar}
\vspace{1mm}

{\small \flushleft \shadebox{$\matchBwd{\rho}{\kappa}{\alpha}{w}{v}{\sigma}$}%
\hfill \textbfit{$\rho$ and $\kappa$, with argument demand $\alpha$, backward-analyse along $w$ to $v$ and $\sigma$}}
\begin{smathpar}
   \inferrule*[lab={\ruleName{$\matchBwdS$-var}}]
   {
      \strut
   }
   {
      \matchBwd{\bind{x}{v}}{\kappa}{\alpha}{\matchVar{x}}{v}{\elimVar{x}{\kappa}}
   }
   %
   \and
   %
   \inferrule*[lab={\ruleName{$\matchBwdS$-unit}}]
   {
      \strut
   }
   {
      \matchBwd{\seqEmpty}
               {\kappa}
               {\alpha}
               {\matchRecEmpty}
               {\annot{\exRecEmpty}{\alpha}}
               {\elimRecEmpty{\kappa}}
   }
   %
   \and
   %
   \inferrule*[lab={\ruleName{$\matchBwdS$-nil}}]
   {
      \strut
   }
   {
      \matchBwd{\seqEmpty}{\kappa}{\alpha}{\matchNil}{\annNil{\alpha}}{\elimList{\kappa}{\hole}}
   }
   %
   \and
   %
   \inferrule*[lab={\ruleName{$\matchBwdS$-record}}]
   {
      \matchBwd{\rho'}{\kappa}{\alpha}{w'}{u}{\sigma}
      \\
      \matchBwd{\rho}
               {\sigma}
               {\alpha}
               {\matchRec{\vec{\bind{x}{w}}}}
               {\annRec{\vec{\bind{x}{v}}}{\beta}}
               {\tau}
   }
   {
      \matchBwd{\rho \concat \rho'}
               {\kappa}
               {\alpha}
               {\matchRec{\vec{\bind{x}{w}} \concat \bind{y}{w'}}}
               {\annRec{\vec{\bind{x}{v}} \concat \bind{y}{u}}{\alpha}}
               {\elimRec{\vec{x} \concat y}{\tau}}
   }
   %
   \and
   %
   \inferrule*[lab={\ruleName{$\matchBwdS$-cons}}]
   {
      \matchBwd{\rho'}{\kappa}{\alpha}{w'}{v'}{\sigma}
      \\
      \matchBwd{\rho}{\sigma}{\alpha}{w}{v}{\tau}
   }
   {
      \matchBwd{\rho \concat \rho'}
               {\kappa}
               {\alpha}
               {\matchCons{w}{w'}}
               {\annCons{v}{v'}{\alpha}}
               {\elimList{\hole}{\tau}}
   }
   %
   \and
   %
   \inferrule*[lab={\ruleName{$\matchBwdS$-true}}]
   {
      \strut
   }
   {
      \matchBwd{\seqEmpty}{\kappa}{\alpha}{\matchTrue}{\annTrue{\alpha}}{\elimBool{\kappa}{\hole}}
   }
   %
   \and
   %
   \inferrule*[lab={\ruleName{$\matchBwdS$-false}}]
   {
      \strut
   }
   {
      \matchBwd{\seqEmpty}{\kappa}{\alpha}{\matchFalse}{\annFalse{\alpha}}{\elimBool{\hole}{\kappa}}
   }
\end{smathpar}
\vspace{1mm}

{\small\flushleft \shadebox{$\envLookupBwd{\rho'}{\rho}{\bind{x}{v}}$}%
\hfill \textbfit{$\bind{x}{v}$ backward-analyses along $\rho$ to $\rho'$}}
\begin{smathpar}
   \inferrule*[
      lab={\ruleName{$\envLookupBwdS$-head}}
   ]
   {
      \strut
   }
   {
      \envLookupBwd{(\hole_{\rho} \concat \bind{x}{u})}
                   {\rho \concat \bind{x}{v}}
                   {\bind{x}{u}}
   }
   %
   \and
   %
   \inferrule*[
      lab={\ruleName{$\envLookupBwdS$-tail}},
      right={$x \neq y$}
   ]
   {
      \envLookupBwd{\rho'}{\rho}{\bind{x}{u}}
   }
   {
      \envLookupBwd{(\rho' \concat \bind{y}{\hole})}
                   {\rho \concat \bind{y}{v}}
                   {\bind{x}{u}}
   }
\end{smathpar}
\vspace{1mm}

{\small\flushleft\shadebox{$\rho \closeDefsBwdR \rho', h$}%
\hfill \textbfit{$\rho$ backward-analyses to $\rho'$ and $h$}}
\begin{salign}
   \vec{\bind{x}{v}}
   &\closeDefsBwdR
   (\bigjoin\vec{\rho}, \vec{\bind{x}{\sigma}} \join {\bigjoin\vec{h}})
   &
   \text{ where }
   v_i = \exClosure{\rho_i}{h_i}{\sigma_i}
\end{salign}

\caption{Backward data dependency (Boolean cases for $\evalBwdR{T}$ omitted)}
\label{fig:data-dependencies:bwd}
\end{figure}
\vspace{1mm}
   \begin{figure}[H]
\flushleft \shadebox{$p, \sigma \matchp \kappa$}
\begin{smathpar}
   \inferrule*[lab={\ruleName{$\matchp$-var}}]
   {
      \strut
   }
   {
      x, \elimVar{x}{\kappa}
      \matchp
      \kappa
   }
   %
   \and
   %
   \inferrule*[lab={\ruleName{$\matchp$-true}}]
   {
      \strut
   }
   {
      \exTrue, \elimBool{\kappa}{\kappa'}
      \matchp
      \kappa
   }
   %
   \and
   %
   \inferrule*[lab={\ruleName{$\matchp$-false}}]
   {
      \strut
   }
   {
      \exFalse, \elimBool{\kappa}{\kappa'}
      \matchp
      \kappa'
   }
   %
   \and
   %
   \inferrule*[lab={\ruleName{$\matchp$-pair}}]
   {
      p_1, \sigma \matchp, \tau
      \\
      p_2, \tau \matchp \kappa
   }
   {
      \exPair{p_1}{p_2}, \elimProd{\sigma}
      \matchp
      \kappa
   }
   %
   \and
   %
   \inferrule*[lab={\ruleName{$\matchp$-nil}}]
   {
      \strut
   }
   {
      \exNil, \elimList{\branchNil{\kappa}}{\branchCons{\sigma}}
      \matchp
      \kappa
   }
   %
   \and
   %
   \inferrule*[lab={\ruleName{$\matchp$-cons}}]
   {
      p_1, \sigma \matchp \tau
      \\
      p_2, \tau \matchp \kappa'
   }
   {
      \exCons{p_1}{p_2}, \elimList{\branchNil{\kappa}}{\branchCons{\sigma}}
      \matchp
      \kappa'
   }
\end{smathpar}
\caption{Pattern-matching (by pattern)}
\end{figure}

   \begin{figure}[H]
\small
$\begin{array}{lllll}
&&&
\textit{where}
\\
\primAppF{\phi}(\vec{v})
&=&
\exPrimOp{\phi}{\vec{v}}
&
\arity{\phi} \numgt \length{\vec{v}}
\\
\primAppF{\phi}(\vec{v})
&=&
\primFwd{\phi}(\vec{v})
&
\arity{\phi} = \length{\vec{n}}
\end{array}$
\caption{Auxiliary functions: forward and backward slicing}
\end{figure}

   \caption{Operational semantics}
   \label{fig:core-language:semantics}
\end{figure}

The judgement $T :: \rho, e \evalR v$ defined at the top of \figref{core-language:semantics} states that term $e$ under environment $\rho$ evaluates to value $v$, and that $T$ is a proof term that witness that fact. (In the figure, the traces appear in grey, to reinforce the idea that they are not part of the definition of $\evalR$ but rather a notation for its inhabitants.) The rules for Booleans, integers and lists are standard and have unsurprising trace forms. For variables, we give an explicit inductive definition of the environment lookup relation $\envLookupR$ at the bottom of the figure, again so that later we can perform analysis over a proof that an environment contains a binding. The lambda rule is standard except that we specify $\seqEmpty$ for the sequence of definitions being simultaneously defined, since a lambda is not recursive. For record construction, the trace form contains a subtrace $T_i$ for each field, and for record projection, which also uses the lookup relation $\envLookupR$, the trace form $\trRecProj{T}{\vec{\bind{x}{v}}}{y}$ records both the record $\vec{\bind{x}{v}}$ and the field name $y$ that was selected.

The rule for (mutually) recursive functions $\exLetRecMutual{h}{e}$, where $h$ is a sequence $\vec{\bind{x}{\sigma}}$ of function definitions, makes use of the auxiliary relation $\rho, h \closeDefsR \rho'$ at the bottom of \figref{core-language:semantics} which turns $h$ into an environment $\rho'$ binding each function name $x_i$ to a closure $\exClosure{\rho}{h}{\sigma_i}$ capturing $\rho$ and a copy of $h$. For primitive applications, the trace records the values of the arguments which were passed to the operation $\phi$. The rule for application $\exApp{e}{e'}$ is slightly non-standard, because it must deal with both mutual recursion and pattern-matching. First we unpack the recursive definitions $h$ from the closure $\exClosure{\rho_1}{h}{\sigma}$ computed by $e$, and again use the auxiliary relation $\closeDefsR$ to promote this into an environment $\rho_2$ of closures. We then use the relation $\matchR$ explained below to match $v$ against the eliminator $\sigma$, obtaining the branch $e^\twoPrime$ of the function to be executed and parameter bindings $\rho_3$. In addition to subtraces $T$ and $U$ for the function and argument, the application trace $\trApp{T}{U}{w}{T'}$ also has subtraces $w$ for the pattern-match and $T'$ for the selected branch.

\subsubsection{Pattern matching}
\label{sec:core-language:pattern-match}

The judgement $w :: v, \sigma \matchR \rho, \kappa$ also defined in \figref{core-language:semantics} states that eliminator $\sigma$ can match $v$ and produce environment $\rho$ and continuation $\kappa$, with $\rho$ containing the variable bindings that arose during the match. \emph{Matches} $w$ are a compact notation for proof terms for the $\matchR$ relation, analogous to traces for the $\evalR$ relation, and again appear in grey in the figure.

Variable eliminators $\elimVar{x}{\kappa}$ match any value, returning the singleton environment $\bind{x}{v}$ and continuation $\kappa$. Boolean eliminators match any Boolean value, returning the appropriate branch and empty environment $\seqEmpty$. List eliminators $\elimList{\kappa}{\sigma}$ match any list. The nil case is analogous to the handling of Booleans; the cons case depends on the fact that the nested eliminator $\sigma$ for the cons branch has the curried type $\elimTy{A}{\tyFun{\tyList{A}}{K}}$. First, we recursively match the head $v$ of type $A$ using $\sigma$, obtaining bindings $\rho$ and eliminator $\tau: \elimTy{\tyList{A}}{K}$ as the continuation. Then the tail $v'$ is matched using $\tau$ to yield additional bindings $\rho'$ and final continuation $\kappa'$ of type K. Record matching is similar: the empty record case resembles the nil case, and the non-empty case relies on the nested eliminator having curried type $\elimTy{\tyRec{\vec{\bind{x}{A}}}}{\elimTy{B}{K}}$. The initial part $\vec{\bind{x}{v}}$ of the record is matched using $\sigma$, returning another eliminator $\sigma'$ of type $\elimTy{B}{K}$. Then the last field $\bind{y}{u}$ is matched using $\sigma'$ to yield final continuation $\kappa$ of type $K$.
